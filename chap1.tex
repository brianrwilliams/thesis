\documentclass[10pt]{amsart}

\usepackage{macros,slashed}

\linespread{1.25}

\usepackage{tikz}
\usetikzlibrary{arrows,shapes}
\usetikzlibrary{trees}
\usetikzlibrary{matrix,arrows}
\usetikzlibrary{positioning}
\usetikzlibrary{calc,through}
\usetikzlibrary{decorations.pathreplacing}
\usepackage{pgffor}

\title{Holomorphic theories and renormalization}

\def\brian{\textcolor{blue}{BW: }\textcolor{blue}}


\begin{document}
\maketitle
\tableofcontents

Our main objective in this chapter is two-fold. 
First we will define the concept of a holomorphic field theory and set up notation and terminology that we will use throughout the text. 
Our next goal is more technical, but will provide the backbone for much of the analysis throughout the remainder of this thesis.
We will show how certain holomorphic theories are surprisingly well-behaved when it comes to the problem of renormalization. 

In \cite{CostelloRenormalization} Costello has provided a mathematical formulation of the Wilsonian approach to quantum field theory.
The main takeaway is that to construct a full quantum field theory it suffices to define the theory at each energy (or length) scale and to ask that these descriptions be compatible as we vary the scale.
The infamous infinities of quantum field theory arise due to studying behavior of theories at arbitrarily high energies (or small lengths). 
In physics this is called the ultra-violet (UV) divergence. 

A classical theory is described by a local functional $I$ on the space of fields....\brian{not sure how much to review. This might go in the overview section of the thesis.}

Summarizing, there are two main steps to construct a QFT in our formalism.
\begin{itemize}
\item[{\bf Renormalization:}] For each scale $L$ and regulator $\epsilon > 0$ consider the RG flow from scale $\epsilon$ to $L$:
\be
W(P_{\epsilon < L} , I) .
\ee
In general, the limit $\epsilon \to 0$ will not be defined, but by Costello's main result there exists counterterms $I^{CT}(\epsilon)$ such that the $\epsilon \to 0$ limit of 
\ben
W(P_{\epsilon<L} , I - I^{CT}(\epsilon))
\een
is well-defined. 
Denote this limit by $I[L]$.
The family $\{I[L]\}$ defines a prequantization.
\item[{\bf Gauge consistency:}] We then ask if the family $\{I[L]\}$ defines a consistent quantization.
For each $L$ we require that $I[L]$ satisfy the scale $L$ quantum master equation....
\end{itemize}

In this section we are concerned with the first step, that of renormalization. 
The complication here is that even very natural field theories can have a very complicated collections of counterterms. 
For instance, the naive quantization of Chern-Simons theory on a three-manifold has counterterms even at one-loop. 
For holomorphic theories, however, we will show how the situation becomes much simpler at least at the level of one-loop.  

\begin{thm}
Let $\sE$ be the fields of a holomorphic theory on $\CC^d$ with classical interaction $I \in \oloc(\sE)$.  
Then, there exists a one-loop prequantization $\{I[L] \; | \; L > 0\}$ of $I$ involving no counterterms. 
That is, we can find a propagator $P_{\epsilon < L}$ for which the $\epsilon \to 0$ limit of
\ben
W(P_{\epsilon<L} , I) \mod \hbar^2
\een
exisits.
Moreover, if $I$ is holomorphically translation invariant we can pick the family $\{I[L]\}$ to be holomorphically translation invariant as well.
\end{thm}

We will use this result repeatedly throughout this thesis. 
For instance, knowing the one-loop behavior of a field theory is enough to study the possible anomalies to quantization. 
We will leverage this to formulate and prove index theorems in the context of holomorphic QFT.

One surprising aspect of this comes from thinking about holomorphic theories in a different way. 
Any supercharge $Q$ of a supersymmetric theory satisfying $Q^2 = 0$ allows one to construct a ``twist". 
In some cases, where clifford multiplication with $Q$ spans all translations such a twist becomes a topological theory (in the weak sense). 
In any case, however, such a $Q$ defines a ``holomorphic twist", which results in the type of holomorphic theories we consider.
Regularization in supersymmetric theories, especially gauge theories, is notoriously difficult. 
Our result implies that after twisting the analytic difficulties become much easier to deal with. 
Consequently, phenomena such as anomalies can be cast in a more algebraic framework.
We will see such an example of this in the case of the holomorphic $\sigma$-model in the next chapter. 

Already, in \cite{LiVertex} Li has used a complex one-dimensional version of this fact to all orders in $\hbar$. 
He uses this to give an elegant interpretation of the quantum master equation for two-dimensional chiral conformal field theories using vertex algebras.
We do not make any statements in this work past one-loop
quantizations, but the higher loop behavior remains a very interesting and subtle problem that we hope to return to.

\section{The definition of a quantum field theory}

\subsection{Classical field theory}

\subsection{Renormalization}

In the BV formalism, as developed in \cite{CosRenormalization,CG}, 
a quantum BV theory consists of a space of fields and an effective action functional $\{S[L]\}_{L \in (0,\infty)}$,
which is a family of non-local functionals on the fields that are parametrized by a length scale $L$ 
and satisfy
\begin{enumerate}[(a)]
\item an exact renormalization group (RG) flow equation,
\item the scale $L$ quantum master equation (QME) at every length scale $L$,~and
\item as $L \to 0$, the functional $S[L]$ has an asymptotic expansion that is local.
\end{enumerate}
The first condition ensures that the scale $L$ action functional $S[L]$ determines the functional at every other scale.
The second can be interpreted as saying that we have a proper path integral measure at scale $L$ 
(i.e., the QME can be seen as a definition of the measure).
The third condition implies that the effective action is a quantization of a classical field theory,
since a defining property of a classical theory is that its action functional is local.
(A full definition is available in Section 8.2 of \cite{CG}.)

\begin{rmk}
The length scale is associated with a choice of Riemannian metric on the underlying manifold,
but the formalism of \cite{CosBook} keeps track of how the space of quantum BV theories depends upon such a choice 
(and other choices that might go into issues like renormalization).
Hence, when the choices should not be essential --- such as with a topological field theory --- one can typically show rigorously that different choices give equivalent answers.
The length scale is also connected with the use of heat kernels in \cite{CosBook},
but one can work with more general parametrices (and hence more general notions of ``scale''),
as explained in Chapter 8 of \cite{CG}.
We use a natural length scale in this section; 
when it becomes relevant, in the context of factorization algebras, we switch to general parametrices.
\end{rmk}

\subsection{The quantum master equation}

\section{Holomorphic field theories}

The goal of this section is to define the notion of a holomorphic field theory. 

\subsection{The definition of a holomorphic theory}

We give a general definition of a classical holomorphic theory on a general complex manifold $X$ of complex dimension $d$.
We start with the definition of a {\em free} holomorphic field theory. 
After that we will go on to define what an interacting holomorphic theory is.

\subsubsection{Free holomorphic theories}

The fields of any theory are always expressed as sections of some $\ZZ$-graded vector bundle.
Here, the $\ZZ$-grading is the cohomological, or BRST, grading of the theory.
For a holomorphic theory we take this graded vector bundle to be holomorphic.  
By a {\em holomorphic} $\ZZ$-graded vector bundle we mean a $\ZZ$-graded vector bundle $
V = \oplus_i V^i$ such that each graded piece $V^i$ is a holomorphic vector bundle. 
Thus, the data we start with is the following:

\begin{itemize}
\item[(1)] a $\ZZ$-graded holomorphic vector bundle $V^* = \oplus_i V^i [-i]$, so that the finite dimensional holomorphic vector bundle $V^i$ is in cohomological degree $i$. 
\end{itemize}

A free classical theory is made up of a space of fields as above together with the data of a linearized BRST differential $Q^{BRST}$ and a symplectic pairing. 
Ordinarily, the BRST operator is a differential operator on the vector bundle defining the fields. 
For the class of theories we are considering, we want this operator to be holomorphic. 

If $E$ and $F$ are two holomorphic vector bundles on $X$, we can speak of holomorphic differential operators between $E$ and $F$. 
First, note that the Hom-bundle ${\rm Hom}(E,F)$ inherits a natural holomorphic structure. 
By definition, a holomorphic differential operator of order $m$ is a linear map
\ben
D : \Gamma^{hol}(X ; E) \to \Gamma^{hol}(X ; F)
\een
such that, with respect to a holomorphic coordinate chart $\{z_i\}$ on $X$, $D$ can be written as
\be\label{local holomorphic}
D|_{\{z_i\}} = \sum_{|I| \leq m} a_I (z) \frac{\partial^{|I|}}{\partial z_I}
\ee
where $a_I(z)$ is a local holomorphic section of ${\rm Hom}(E,F)$.
Here, the sum is over all multi-indices $I = (i_1,\ldots, i_d)$ and 
\ben
\frac{\partial^{|I|}}{\partial z_I} := \frac{\partial^{i_k}}{\partial z_k^{i_k}} . 
\een 
The length is defined by $|I| := i_1 + \cdots + i_d$. 

The most basic example of a holomorphic differential operator is the $\partial$ operator which, for each $1 \leq \ell \leq d$, is a holomorphic differential operator from $E = \wedge^\ell T^{1,0*}X$ to $F = \wedge^{\ell+1} T^{1,0*}X$. 
Locally, of course, it has the form
\ben
\partial = \sum_{i = 1}^{d} (\d z_i \wedge (-)) \frac{\partial}{\partial z_i},
\een
where $\d z_i \wedge (-)$ is the vector bundle homomorphism $\wedge^\ell T^{1,0*}X \to \wedge^{\ell+1} T^{1,0*}X$ sending $\alpha \mapsto \d z_i \wedge \alpha$. 

The next piece of data we fix is:
\begin{itemize}
\item[(2)] a square zero holomorphic differential operator 
\ben
Q^{hol} : V \to V[-1]
\een
of cohomological degree $+1$. 
\end{itemize}

Finally, to define a free theory we need the data of a symplectic pairing. 
For reasons to become clear in a moment, we must choose this pairing to have a strange cohomological degree. 
The last piece of data we fix is:
\begin{itemize}
\item[(3)] an invertible bundle map
\ben
(-,-)_V : V \times V \to \Omega^{d,hol}_X[d-1]
\een
Here, $\Omega^{d,hol}_X$ is the holomorphic canonical bundle on $X$. 
\end{itemize}

The definition of the fields of an ordinary field theory are the {\em smooth} sections of the vector bundle $V$. 
In our situation this is a silly thing to do since we lose all of the data of the complex structure we used to define the objects above. 
The more natural thing to do is take the {\em holomorphic} sections of the vector bundle $V$. 
By construction, the operator $Q^{hol}$ and the pairing $(-,-)_V$ are defined on holomorphic sections, so on the surface this seems reasonable. 
\brian{what should I say the problem is with doing things in the analytic category?}

The solution to this problem is in the existence of a resolution for the holomorphic sections of a vector bundle by smooth sections of bundles. 
Given any holomorphic vector bundle $E$ we can define its {\em Dolbeualt complex} $\Omega^{0,*}(X , E)$ with it's Dolbeualt operator 
\ben
\dbar : \Omega^{0,p}(X, E) \to \Omega^{0,p+1}(X, E) .
\een
Here, $\Omega^{0,p}(X, E)$ denotes smooth sections of the vector bundle
\ben
\Wedge^p T^{0,1*} X \tensor E
\een
and $\dbar$ is defined in the usual way \brian{recall this?}.

Using this construction, we take the fields of our free theory to be the complex
\ben
\sE_V = \left(\Omega^{0,*}(X, E), \dbar + Q^{hol}\right) .
\een
The operator $\dbar + Q^{hol}$ is the total linearized BRST operator.
By assumption we have $\dbar Q^{hol} = Q^{hol} \dbar^*$ so that $(\dbar + Q^{hol})^2 = 0$ and so the fields still define a complex. 
The $(-1)$-shifted symplectic pairing is obtained by composition of the pairing $(-,-)_V$ with integration on $\Omega^{d,hol}_X$. 
The thing to observe here is that $(-,-)_V$ extends to the Dolbeualt complex in a natural way: we simply combine the wedge product of forms with the pairing on $V$.
The $(-1)$-shifted pairing $\omega$ on $\sE$ is defined by the diagram
\ben
\xymatrix{
\sE_V \tensor \sE_V \ar[r]^-{(-,-)_V} \ar@{.>}[dr]_-{\omega_V} & \Omega^{0,*}(X , \Omega^{d,hol}_X) [d-1] \ar[d]^-{\int_X} \\
& \CC[-1] .
}
\een

We arrive at the following definition. 

\begin{dfn/lem}\label{dfn hol free theory}
A {\em free holomorphic theory} on a complex manifold $X$ is the data $(V, Q^{hol}, (-,-)_V)$ as in (1), (2), (3) above such that $Q^{hol}$ is a square zero elliptic differential operator that is graded skew self-adjoint for the pairing $(-,-)_V$.
The triple $(\sE_V, Q_V = \dbar + Q^{hol}, \omega_V)$ defines a free BV theory in the usual sense.
\end{dfn/lem}

The usual prescription for writing down the associated action functional holds in this case.
If $\varphi \in \Omega^{0,*}(X , V)$ denotes a field the action is
\ben
S(\varphi) = \int_X \left(\varphi, (\dbar + Q^{hol}) \varphi)\right)_V .
\een

The first example we explain is related to the subject of Chapter \ref{chap holsig} and will serve as the fundamental example of a holomorphic theory. 

\begin{eg}\label{eg bg affine} {\em The free $\beta\gamma$ system}.
Suppose that 
\ben
V = \ul{\CC} \oplus \Omega^{d,hol}_X [d - 1] .
\een
Let $(-,-)_V$ be the pairing
\ben
(\ul{\CC} \oplus \Omega^{d,hol}_X) \tensor (\ul{\CC} \oplus \Omega^{d,hol}_X) \to \Omega^{d,hol}_X \oplus \Omega^{d,hol}_X \to \Omega^{d,hol}_X 
\een 
sending $(\lambda, \mu) \tensor (\lambda',\mu') \mapsto (\lambda \mu', \lambda'\mu) \mapsto \lambda\mu' + \lambda' \mu$.
In this example we set $Q^{hol} = 0$. 
One immediately checks that this is a holomorphic free theory as above.
The space of fields can be written as
\ben
\sE_V = \Omega^{0,*}(X) \oplus \Omega^{d,*}(X)[d - 1] .
\een 
We write $\gamma \in \Omega^{0,*}(X)$ for a field in the first component, and $\beta \in \Omega^{d,*}(X)[d - 1]$ for a field in the second component. 
The action functional reads
\ben
S(\gamma + \beta, \gamma'+\beta') = \int_{X} \beta \wedge \dbar \gamma' + \beta' \wedge \dbar \gamma .
\een 
When $d = 1$ this reduces to the ordinary chiral $\beta\gamma$ system from conformal field theory \brian{ref}. 
We will discuss this higher dimensional version further in Section \brian{}.
For instance, we will see how this theory is the starting block for constructing general holomorphic $\sigma$-models. 
\end{eg}

\begin{eg}
{\em The free chiral scalar}.
Another basic example is the free chiral scalar. 
Let $X$ be a complex manifold with Hermitian metric $g$. 
Let $V = \ul{\CC}$, the trivial vector bundle. 
\brian{do this}
\end{eg}

\subsubsection{Interacting holomorphic theories}

\def\olochol{\sO_{\rm loc}^{hol}}

We now define what an interacting holomorphic theory is.
In general, an interacting field theory on a manifold $M$ is prescribed by the data of a free theory plus a local functional $I \in \oloc(\sE)$ that satisfies the classical master equation. 
Recall, the sheaf of local functionals on $\sE = \Gamma(E)$ is defined as the sheaf of Lagrangian densities
\ben
\oloc(\sE) = {\rm Dens}_M \tensor_{D_M} \sO_{red}(JE) .
\een
In the expression above $JE$ stands for the sheaf of smooth sections of the $\infty$-jet bundle ${\rm Jet}(E)$ which has the structure of a $D_X$-module.

If $V$ is a holomorphic vector bundle let $J^{hol} E$ denote the sheaf of holomorphic sections of the holomorphic bundle of holomorphic $\infty$-jets ${\rm Jet}^{hol}(V)$. 
The fibers of the infinite rank vector bundle ${\rm Jet}^{hol}(V)$ are isomorphic to 
\ben
{\rm Jet}^{hol}(V)|_w = V_w \times \CC[[z_1,\ldots,z_d]] 
\een
where $w \in X$.
The sheaf $J^{hol}V$ has the natural structure of a $D_X^{hol}$-module, that is, it is equipped with a holomorphic flat connection $\nabla^{hol}$.
This is completely analogous to the smooth case.
Locally, the holomorphic flat connection is of the form
\ben
\nabla^{hol} |_w = \sum_{i=1}^d \d w_i \left(\frac{\partial}{\partial w_i} - \frac{\partial}{\partial z_i}\right),
\een
where $\{w_i\}$ is the local coordinate on $X$ near $w$ and $z_i$ is the fiber coordinate labeling the holomorphic jet expansion.
Using holomorphic jets we can make a completely analogous definition in our setting.

\begin{dfn}\label{dfn hol lag}
Let $V$ be a vector bundle. 
The space of {\em holomorphic Lagrangian densities} on $V$ is
\ben
\sO_{red}^{hol}(V) = \prod_{n > 0} {\rm Hom} ({\rm Jet}(V)^{\times n} , K_X) .
\een
Equivalently, a holomorphic Lagrangian density is of the form $F = \sum_n F_n \in \sO_{red}^{hol}(V)$ where each $F_n$ is a holomorphic polydifferential operator 
\ben
F_n : V \tensor \cdots \tensor V \to K_X .
\een
\end{dfn}

\begin{dfn}
Suppose $V$ is a graded holomorphic vector bundle.
We define the sheaf of {\em holomorphic} local functionals on $V$ by
\ben
\olochol(V) = \Omega^{d,hol}_X \tensor_{D^{hol}_X} \sO_{red}(J^{hol}V) [d]
\een
\end{dfn}

Suppose that $V$ is part of the data of a free holomorphic theory $(V, Q^{hol},(-,-)_V)$.
The pairing $(-,-)_V$ endows the space of local functionals with a bracket of cohomological degree $+1$ that we denote by $\{-,-\}^{hol}$. 
We can now state the definition of a classical holomorphic theory. 

\begin{dfn}
A {\em classical holomorphic theory} on a complex manifold $X$ is the data of a free holomorphic theory $(V, Q^{hol}, (-,-)_V)$ plus a functional
\ben
I^{hol} \in \olochol(V)^+
\een
of cohomological degree zero such that $Q^{hol} I^{hol} + \{I^{hol}, I^{hol}\}^{hol} = 0$.
\end{dfn} 

\begin{dfn/lem}
Let $(V, Q^{hol}, (-,-)_V, I^{hol})$ be the data of an interacting holomorphic theory. 
Then $Q^{hol} + \{I^{hol},-\}$ equips $\olochol(V)$ with the structure of a sheaf of cochain complexes that we will denote
\ben
\Def^{hol}_{V} := \left(\olochol(V), Q^{hol} + \{I^{hol}, -\}^{hol}\right) .
\een
\end{dfn/lem}

Just as in the free case, we see that classical holomorphic theories are special cases of ordinary classical BV theories.
The underlying space of fields, as we have already seen is $\sE_V = \Omega^{0,*}(X , V)$. 
Now, we know that $I^{hol}$ is a $\Omega^{d,hol}_X$-valued functional that is a linear combination of functionals of the form 
\ben
(\varphi_1,\ldots,\varphi_k) \mapsto D_1(\varphi_1)\cdots D_k(\varphi_k) \in \Omega^{d,hol}_X
\een
where $\varphi_i$ is a section of $V^{hol}$ and $D_i$ is a holomorphic differential operator on $V$.
Now, every holomorphic differential operator on the holomorphic vector bundle $V$ extends to a differential operator on its Dolbeualt complex $\sE_V = \Omega^{0,*}(X, V)$. 
Thus, we can define the functional
\ben
(\alpha_1\tensor \varphi_1,\ldots, \alpha_k \tensor \varphi_k) \mapsto \int_X D_1(\alpha_1 \tensor \varphi_1)\cdots D_k(\alpha_k \tensor \varphi_k)
\een
where $\alpha_i \tensor \varphi_i \in \Omega^{0,*}(X, V)$. 
The symbol $\int_X$ reminds us that we are working modulo total derivatives, so that the above expression defines an element of $\oloc(\sE_V)$. 
This defines a linear map $\olochol(V) \to \oloc(\sE_V)$ that we denote $I^{hol} \mapsto I^{\Omega^{0,*}}$. 

\begin{lem} Every classical holomorphic theory $(V, Q^{hol},(-,-)_V, I^{hol})$ determines the structure of a classical BV theory.
The underlying free BV theory is given in Definition/Lemma \ref{dfn hol free theory} $(\sE_V, Q, \omega_V)$ and the interaction is $I^{\Omega^{0,*}}$. 
\end{lem}
\begin{proof}
We must show that $Q^{hol}I^{hol} + \frac{1}{2} \{I^{hol},I^{hol}\}^{hol} = 0$ implies 
\ben
\dbar I^{\Omega^{0,*}} + Q^{hol}I^{\Omega^{0,*}} + \frac{1}{2} \{I^{\Omega^{0,*}},I^{\Omega^{0,*}}\} = 0 .
\een
\end{proof}

\begin{eg} {\em Holomorphic $BF$-theory}
Let $\fg$ be a Lie algebra and $X$ any complex manifold.
Consider the following holomorphic vector bundle on $X$:
\ben
V = \ul{\fg}_X \oplus \Omega^{d,hol}_X \tensor \fg^* [d-1] .
\een
The pairing $V \tensor V \to \Omega^d_{hol}[d-1]$ is similar to the pairing for the $\beta\gamma$ system, except we use the evaluation pairing $\<-.-\>_\fg$ between $\fg$ and its dual. 
In this example, $Q^{hol} = 0$.
Write $f \in \sO^{hol}_X$ and $\beta \in \Omega^{d,hol}_X$ and consider
\ben
I^{hol} (f_1 \tensor X_1, f_2 \tensor X_2, \beta \tensor X^\vee) = f_1f_2 \beta \<X^\vee, [X_1,X_2]\> + \cdots
\een
where the $\cdots$ means that we symmetrize the inputs.
This defines an element $I^{hol} \in \olochol(V)^+$ and the Jacobi identity ensures $\{I^{hol}, I^{hol}\}^{hol} = 0$. 
The fields of the corresponding BV theory are
\ben
\sE_V = \Omega^{0,*}(X, \fg) \oplus \Omega^{d,*}(X, \fg^*) [d-1] .
\een
The induced local functional $I^{\Omega^{0,*}}$ on $\sE_V$ is
\ben
I^{\Omega^{0,*}} (\alpha, \beta) = \int_X \<\beta, [\alpha,\alpha]\>_\fg .
\een
\end{eg}

\begin{eg} {\em Topological $BF$-theory}
\end{eg}
\begin{lem}
Suppose $(V, 0, (-,-)_V, I^{hol})$ is the data of a holomorphic theory with $Q^{hol} = 0$.
Let $(\sE_V, Q = \dbar, \omega_V, I)$ be the corresponding BV theory.
Then, there is a quasi-isomorphism of sheaves
\ben
\Def^{hol}_{V} \simeq \Def_{\sE_V}  
\een
compatible with the brackets $\{-,-\}^{hol}$ and $\{-,-\}$ on both sides.
\end{lem}

\begin{rmk}
Just as in the ordinary case we can formulate the data of a classical holomorphic theory in terms of sheaves of $L_\infty$ algebras. 
We will not do that here, but hope the idea of how to do so is clear.
\end{rmk}

%We start with the definition of a {\em holomorphic local Lie algebra} on a complex manifold. 
%Enhancing this to a definition of a holomorphic classical BV theory will be immediate. 
%
%Fix a complex manifold $X$, of complex dimension $d$.
%The starting data of a local Lie algebra on $X$ is a $\ZZ$-graded vector bundle $L$ on $X$. 
%
%\begin{dfn} 
%A {\em holomorphic} local Lie algebra on a complex manifold $X$ is the data
%\begin{enumerate}
%\item A $\ZZ$-graded holomorphic vector bundle $L$ on $X$;
%\item For each $n \geq 1$ a holomorphic differential operator 
%\ben
%\ell_n : L^{\boxtimes n} \to L;
%\een
%\end{enumerate}
%such that $\{\ell_n\}$ endow the sheaf of holomorphic sections $\Gamma_{hol}(L)$ with the structure of a sheaf of $L_\infty$ algebras. 
%\end{dfn}
%
%A holomorphic local Lie algebra is {\em not} a local Lie algebra in the usual sense. 
%The problem is that, in the definition, we have utilized the space of {\em holomorphic sections}, instead of the space of all smooth sections. 
%There is a natural resolution that will allow us to turn every holomorphic local Lie algebra into an ordinary local Lie algebra. 
%Of course, given a holomorphic vector bundle $L$ we can consider its Dolbeualt complex ${\rm Dol}(L) = \Omega^{0,*}(X , L)$.
%For a holomorphic local Lie algebra $L$ this is really the object we want to consider.
%
%Applying the Dolbeault complex functor to the $L_\infty$ maps $\ell_n : L^{\boxtimes n} \to L$ we obtain maps
%\ben
%{\rm Dol}(\ell_n) : \Omega^{0,*}(X , L)^{\tensor n} \to \Omega^{0,*}(X , L) . 
%\een 
%
%\begin{lem} Suppose $L$ is a holomorphic local Lie algebra. 
%Then, ${\rm Dol}(L) = \Omega^{0,*}(X , L)$ is endowed with the structure of a local Lie algebra with structure maps given by $\ell_1 = \dbar + {\rm Dol}(\ell^L_1)$, and $\ell_n = {\rm Dol}(\ell^L_n)\}$ for $n \geq 2$. 
%\end{lem}
%\begin{proof} 
%We need to check that for each $n \geq 1$ the map
%\ben 
%{\rm Dol}(\ell^L_n) : \Omega^{0,*}(X, L)^{\tensor n} = \Omega^{0,*}(X^{\times n} , L^{\boxtimes n}) \to \Omega^{0,*}(X , L)
%\een
%is compatible with the Dolbeualt differential. 
%This is a local calculation, so it suffices to assume all operators $\ell_n^L$ are of the form (\ref{local holomorphic}). 
%Indeed, since each of the coefficients $a_I(z)$ are holomorphic we see that ${\rm Dol}(\ell_n^L)$ is compatible with $\dbar$. 
%\end{proof}

\subsection{Holomorphically translation invariant theories}

When working on affine space $\RR^n$ one can ask for a theory to be invariant with respect to translations. 
In this section, we consider the affine manifold $\CC^d = \RR^{2d}$ equipped with its standard complex structure and define what a {\em holomorphically translation invariant} theory is on it. 
It will be a very special case of a general holomorphic theory as defined above. 

Let $V$ be a holomorphic vector bundle on $\CC^n$ and suppose we fix an identification of bundles 
\ben
V \cong \CC^d \times V_0
\een
where $V_0$ is the fiber of $V$ at $0 \in \CC^d$. 
We want to consider a classical theory with space of fields given by $\Omega^{0,*}(\CC^d, V) \cong \Omega^{0,*}(\CC^d) \tensor_\CC V_0$. 
Moreover, we want this theory to be invariant with respect to the group of holomorphic translations on $\CC^d$. 
Per usual, it is best to work with the corresponding Lie algebra of translations. 
Using the complex structure, we choose a presentation for the Lie algebra of all translations given by
\ben
\CC^{2d} \cong {\rm span}_\CC \left\{\frac{\partial}{\partial z_i}, \frac{\partial}{\partial \zbar_i}\right\}_{1 \leq i \leq d}.
\een

To define a theory, we need to fix a non-degenerate pairing on $V$.
Moreover, we want this to be translation invariant. 
So, suppose
\be\label{pairing 1}
\< \;\;,\;\; \>_V : V \tensor V \to \Omega^{d,hol}_{\CC^d} [d-1]
\ee
is a skew-symmetric bundle map that is equivariant for the Lie algebra of translations. 
The shift is so that the resulting pairing on the Dolbeualt complex is of the appropriate degree.
Here, equivariance means that for sections $v,v'$ we have
\ben
\< \frac{\partial}{\partial z_i} v, v'\>_V = L_{\partial_{z_i}} \<v,v'\>_V
\een
where the right-hand side denotes the Lie derivative applied to $(v,v')_V \in \Omega^{d,hol}_{\CC^d}$. 
There is a similar relation for the anti-holomorphic derivatives. 
We obtain a $\CC$-valued pairing on $\Omega^{0,*}_c(\CC^d , V)$ via integration:
\ben
\int_{\CC^d} \circ (-,-)_V : \Omega^{0,*}_c (\CC^d , V) \tensor \Omega^{0,*}_c(\CC^d , V) \xto{\wedge \cdot (-,-)_V} \Omega^{d,*}(\CC^d) \xto{\int} \CC .
\een
The first arrow is the wedge product of forms combined with the pairing on $V$. 
The second arrow is only nonzero on forms of type $\Omega^{d,d}$. 
Clearly, integration is translation invariant, so that the composition is as well. 

This pairing $\Omega^{0,*}(\CC^d , V)$ together with the differential $\dbar$ are enough to define a free theory. 
However, it is convenient to consider a slightly generalized version of this situation. 
We want to allow deformations of the differential $\dbar$ on Dolbeault forms of the form
\ben
Q = \dbar + Q^{hol}
\een
where $Q^{hol}$ is a holomorphic differential operator of the form
\be\label{hol operator}
Q^{hol} = \sum_I \frac{\partial}{\partial z^I} \mu_I
\ee
where $I$ is some multi-index and $\mu_I : V \to V$ is a linear map of cohomological degree $+1$. 
Note that we have automatically written $Q^{hol}$ in a way that it is translation invariant.
Of course, for this differential to define a free theory there needs to be some compatibility with the pairing on $V$. 
We can summarize this in the following definition, which should be viewed as a slight modification of a free theory to this translation invariant holomorphic setting. 

\begin{dfn} A {\em holomorphically translation invariant free BV theory} is the data of a holomorphic vector bundle $V$ together with
\begin{enumerate}
\item an identification $V \cong \CC^d \times V_0$;
\item a translation invariant skew-symmetric pairing  $\<-,-\>_V$ as in (\ref{pairing 1});
\item a holomorphic differential operator $Q^{hol}$ as in (\ref{hol operator});
\end{enumerate}
such that the following conditions hold
\begin{enumerate}
\item the induced $\CC$-valued pairing $\int \circ \<-,-\>_V$ is non-degenerate;
\item the operator $Q^{hol}$ satisfies $(\dbar + Q^{hol})^2 = 0$ and
\ben
\int \<Q^{hol} v, v'\>_V = \pm \int \<v, Q^{hol} v'\> .
\een
\end{enumerate}
\end{dfn}

The first condition is required so that we obtain an actual $(-1)$-shifted symplectic structure on $\Omega^{0,*}(\CC^d, V)$. 
The second condition implies that the derivation $Q = \dbar + Q^{hol}$ defines a cochain complex
\ben
\sE_V = \left(\Omega^{0,*}(\CC^d, V), \dbar + Q^{hol}\right),
\een
and that $Q$ is skew self-adjoint for the symplectic structure. 
Thus, in particular, $\sE_V$ together with the pairing define a free BV theory in the ordinary sense. 
In the usual way, we obtain the action functional via
\ben
S(\varphi) = \int \<\varphi, (\dbar + Q^{hol}) \varphi\>_V .
\een 

Before going further, we will list some familiar examples.

\begin{eg}\label{eg bg affine} {\em The free $\beta\gamma$ system on $\CC^d$}.
Suppose that 
\ben
V = \ul{\CC} \oplus K_{\CC^d} [d - 1] .
\een
Let $\<\;,\;\>_V$ be the pairing
\ben
(\ul{\CC} \oplus K_{\CC^d}) \tensor (\ul{\CC} \oplus K_{\CC^d}) \to K_{\CC^{d}} \oplus K_{\CC^d} \to K_{\CC^d} 
\een 
sending $(\lambda, \mu) \tensor (\lambda',\mu') \mapsto (\lambda \mu', \lambda'\mu) \mapsto \lambda\mu' + \lambda' \mu$.
Finally, let $Q^{hol} = 0$. 
One immediately checks that this is a holomorphically translation invariant free theory as above.
The space of fields can be written as
\ben
\Omega^{0,*}(\CC^d) \oplus \Omega^{d,*}(\CC^d)[d - 1] .
\een 
We write $\gamma$ for a field in the first component, and $\beta$ for a field in the second component. 
The action functional reads
\ben
S(\gamma + \beta, \gamma'+\beta') = \int_{\CC^d} \beta \wedge \dbar \gamma' + \beta' \wedge \dbar \gamma .
\een 
When $d = 1$ this reduces to the ordinary chiral $\beta\gamma$ system from conformal field theory \brian{ref}. 
We will discuss this higher dimensional version further in Section \brian{}.
For instance, we will see how this theory is the starting block for constructing general holomorphic $\sigma$-models. 
\end{eg}

Of course, there are many variants of the $\beta\gamma$ system that we can consider.
For instance, if $E$ is {\em any} holomorphic vector bundle we can take 
\ben
V = E \oplus K_{\CC^d} \tensor E^\vee
\een
where $E^\vee$ is the linear dual bundle. 
The pairing is constructed as in the case above where we also use the evaluation pairing between $E$ and $E^\vee$, ${\rm ev}_E : E \tensor E^\vee \to \CC$.
In thise case, the fields are $\gamma \in \Omega^{0,*}(\CC^d, E)$ and $\beta \in \Omega^{d,*}(\CC^d, E^\vee)[d-1]$. 
The action functional is simply
\ben
S(\gamma, \beta) = \int_{\CC^d} {\rm ev}_E(\beta \wedge \dbar \gamma) .
\een

\begin{eg} {\em Topological ...}
Consider the above example with $Q^{hol} = \partial$..\brian{finish}. 
\end{eg}

%\subsubsection{Interacting holomorphic theories}
%
%It is convenient to introduce the following set of degree $-1$ derivations of $\Omega^{0,*}(\CC^d)$ given by
%\ben
%\Bar{\eta}_i := \frac{\partial}{\partial (\d \zbar_i)} .
%\een
%The right-hand side is sometimes written using the interior derivative notation $\iota_{\partial / \partial \zbar_i}$. 
%By a holomorphic version of ``Cartan's magic formula" these derivations satisfy the relation
%\ben
%L_{\frac{\partial}{\partial \zbar_i}} = \dbar \Bar{\eta}_i + \Bar{\eta}_i \dbar .
%\een
%In addition, they serve to define homotopies for the following holomorphic version of Poincar\'{e}'s lemma. 
%First, consider the algebraic case. 
%Let $\AA^d$ be the complex $d$-dimensional affine space with space of smooth algebraic functions $\sO^{alg, sm}(\AA^d) = \CC[z_1,\ldots, z_d, \zbar_1,\ldots,\zbar_d]$. 
%Then, we can build the algebraic Dolbeualt complex 
%\ben
%\Omega^{0,*}_{alg} (\AA^d) = \CC[z_1,\ldots,z_d, \zbar_1,\ldots,\zbar_d][\d \zbar_1,\ldots,\d \zbar_d]
%\een
%where the $\d \zbar_i$'s are in cohomological degree $1$. 
%The Dolbeault differential $\dbar$ is define in the same way. 
%Note that the operators $\Bar{\eta}_i$ also make sense on $\Omega^{0,*}_{alg}(\AA^d)$. 
%
%\begin{lem} 
%The map 
%\ben
%\CC[z_1,\ldots, z_d] \hookrightarrow \Omega^{0,*}_{alg}(\AA^d)
%\een
%is a quasi-isomorphism.
%\end{lem}
%\begin{proof}
%Note that
%\ben
%\Omega^{0,*}_{alg}(\AA^d) \cong \CC[z_1,\ldots, z_d] \tensor_\CC \CC[\zbar_1,\ldots,\zbar_d][\d \zbar_1,\ldots,\d \zbar_d] .
%\een
%The right-hand term is one-dimensional, concentrated in degree zero by the ordinary Poincar\'{e} lemma. 
%An explicit homotopy for a \brian{...}
%\end{proof}
%
%The analogous result holds with $\AA^d$ replaced by the formal $d$-disk $\hD^d$. 
%
%For the general case...
%
%\begin{lem} \brian{find good citation. should I just state the general Stein resut?}
%The map
%\ben
%\sO^{hol} (\CC^d) \hookrightarrow \Omega^{0,*}(\CC^d)
%\een
%is a quasi-isomorphism.
%\end{lem}
%
%\brian{Include equivalence with certain structured local Lie algebras. Namely holomorphically translation invariant local Lie algebras.}
%
%Local Lie algebras \footnote{Local Lie algebras will mean $L_\infty$...} provide a convenient language to cast the data of an interacting classical field theory. 
%In \brian{ref} it is shown that the data of a local Lie algebra together with a non-degenerate pairing of degree $-3$ is equivalent to the data of a classical interacting BV theory. 
%It will be convenient for us to formulate, under this equivalence, a Lie theoretic interpretation of holomorphically translation invariant interacting BV theories. 
%First, we introduce the following definition. 
%Recall, data of a local Lie algebra is a $\ZZ$-graded vector bundle $L$ together with poly-differential operators 
%\ben
%\ell_n : L \tensor \cdots \tensor L \to L 
%\een 
%for $n \geq 1$, satisfying some conditions. 
%The sheaf of smooth sections of this bundle will be denoted by $\sL$, which inherits from the operators $\{\ell_n\}$ the structure of a sheaf of $L_\infty$ algebras. 
%We will often refer to the local Lie algebra simply by its sheaf of sections. 
%
%\begin{dfn}
%A translation invariant local Lie algebra on $\RR^n$ is a local Lie algebra $\sL$,
%together with an identification $L = \RR^n \times L_0$ such that for each $n$ the structure map
%\ben
%\ell_n : \RR^n \times (L_0 \tensor \cdots \tensor L_0) \to \RR^n \times L_0
%\een
%is compatible with translations. 
%\end{dfn}

\subsubsection{Holomorphic translation invariant deformations}

Any local Lie algebra on a manifold endows the structure of an $L_\infty$ algebra on its fibers. 
In particular, if $L$ the graded vector bundle associated to local Lie algebra on $\CC^d$, its fiber over $0$, $L_0$, is equipped with the structure of an $L_\infty$ algebra. 

Suppose $\sL$ is a holomorphically translation invariant local Lie algebra on $\CC^d$ of the form $\Omega^{0,*}(\CC^d, L)$ where $L$ is a graded holomorphic vector bundle.
In this situation, we are interested in studying the local cochains of $\sL$ that are translation invariant.
We will use the following result over and over again throughout this work.

%\begin{lem} Let $\sL_0$ denote the fiber of $\sL$ over $0 \in \CC^d$. 
%Then, if $\ell_1 = \dbar$, there is an equivalence of $L_\infty$ algebras
%\ben
%L_0 \xto{\simeq} \sL_0
%\een 
%where $L_0$ is the fiber of the holomorphic bundle $L$ over $0 \in \CC^d$. 
%\end{lem}

\begin{prop} Suppose $\sL$ is a holomorphically translation invariant local Lie algebra on $\CC^d$ such that $\ell_1 = \dbar$.
Then, one has
\ben
\cloc^*(\sL)^{\CC^d} \simeq \CC \cdot \d^d z \tensor^{\LL}_{\CC[\frac{\partial}{\partial z_i}]} \cred^*(L_0 [[z_1,\ldots,z_d]]) [d] .
\een
\end{prop}

For instance, if $L = \ul{\fg}$ is the constant bundle on $\CC^d$ where $\fg$ is an ordinary Lie (or $L_\infty$) algebra one has $L_0 = \fg$ so that
\ben
\cloc^*(\Omega^{0,*}(\CC^d, \fg))^{\CC^d} \simeq \CC \cdot \d^d z \tensor^{\LL}_{\CC[\frac{\partial}{\partial z_i}]} \cred^*(\fg [[z_1,\ldots,z_d]]) .
\een

\subsubsection{Holomorphic deformations on an arbitrary complex manifold}

\brian{fix and move this to above}
There is a more general result that holds on an arbitrary complex manifold. 
Recall, that on a complex manifold $X$ we have introduced the notion of a holomorphic local Lie algebra $L$. 
Let $\sL = {\rm Dol}(L) = \Omega^{0,*}(X , L)$ be its associated local Lie algebra. 
Ordinarily, the local Lie algebra cohomology of a local Lie algebra is computed in terms of the Lie algebra cohomology of the associated jet bundle. 
With holomorphicity and some mild assumption, we can, up to quasi-isomorphism, exhibit a smaller complex computing this local cohomology. 

\begin{prop} 
Suppose $L$ is a holomorphic local Lie algebra with $\ell_1 = 0$, and let $\sL = \Omega^{0,*}(X, L)$ be its associated local Lie algebra.
There is a quasi-isomorphisms of sheaves of cochain complexes
\ben
\cloc^*(\sL) \simeq \Omega^{d,hol}_X \tensor^{\LL}_{D_X^{hol}} \clie^*(J \sL^{hol}) .
\een
\end{prop}
\begin{proof}
Recall, the definition of $\cloc^*(\sL)$ is given in terms of $D$-module data by
\ben
\cloc^*(\sL) = \Omega^{d,d}_{X} \tensor^{\LL}_{D_X} \clie^*(J \sL)
\een
where $J \sL$ denotes the $\infty$-jet bundle of $\Omega^{0,*}(X , L)$. 
Of course, $J\sL$ is a bundle equipped with a natural flat connection, and hence the structure of a $D_X$-module. 
The Chevalley-Eilenberg complex $\clie^*(J\sL)$ inherits this $D_X$-module structure.  

On the other hand, if we view $L$ as a holomorphic vector bundle, it makes sense to look at the {\em holomorphic} jet bundle $J^{hol} L$. 
This holomorphic vector bundle is equipped with a holomorphic flat connection, and hence is a module for the sheaf of holomorphic differential operators $D^{hol}_X$. 

\begin{lem} Let $(J^{hol} L)^{C^\infty}$ be the $D_X^{hol}$-module $J^{hol} L$ viewed, via the forgetful functor, as a $D_X$-module. 
Then, there is a quasi-isomorphism of dg $D_X$-modules $(J^{hol} L)^{C^\infty} \simeq J \Omega^{0,*}(X , L)$.
Furthermore, this quasi-isomorphism is compatible with the $L_\infty$ structures, so that we obtain a quasi-isomorphism of dg $D_X$-modules $\clie^*(J^{hol} L)^{C^\infty} \simeq \clie^*(J \sL)$. 
\end{lem}
\begin{proof}
For any holomorphic vector bundle $E$, the Dolbeualt complex of $E$ is a resolution of the sheaf of holomorphic sections of $E$.
Thus, there is an equivalence of sheaves on $X$
\ben
\Omega^{0,*}_X(E) \simeq \Gamma^{hol}_X(E) .
\een 
Thus, there is an equivalence of sheaves on $X$:
\ben
J^{hol} L \simeq J \Omega^{0,*}(X , L) .
\een
We need to see that this equivalence respects the $D_X$-module structure present on both sides....

\end{proof}

To finish the proof, we verify the following general lemma. 

\begin{lem} Suppose $V$ is a $D_X^{hol}$-module, and let $V^{C^\infty}$ denote its underlying $D_X$-module. 
Then, there is a quasi-isomorphism of sheaves of cochain complexes
\ben
\Omega^{d,hol}_X \tensor^\LL_{D_X^{hol}} V [d] \simeq \Omega^{d,d}_X \tensor_{D_X}^\LL V^{C^\infty} .
\een 
\end{lem}

\end{proof}

\section{Renormalization of holomorphic theories}

In this section we study the renormalization of holomorphically translation invariant field theories on $\CC^d$ for any $d \geq 1$. 
We start with a classical interacting holomorphic theory on $\CC^d$ and consider one-loop homotopy RG flow from some finite scale $\epsilon$ to scale $L$.
That is, we consider the sum over graphs of genus zero and one where at each vertex we place the holomorphic interaction.
To obtain a prequantization of a classical theory one must make sense of the $\epsilon \to 0$ limit of this construction. 
In general, this involves introducing a family of counterterms.
Our main result is that for a holomorphic theory no such counterterms are required, and one obtains a well-defined $\epsilon \to 0$ limit. 

We can write the fields of a holomorphic theory on $\CC^d$ as
\ben
\sE = \left(\Omega^{0,*}(\CC^d, V), \dbar + Q^{hol}\right)
\een
where $V$ is a graded holomorphic vector bundle and $Q^{hol}$ is a holomorphic differential operator.

Since the theory is holomorphically translation invariant we have an identification $\Omega^{0,*}(\CC^d , V) \cong \Omega^{0,*}(\CC^d) \tensor_\CC V_0$ where $V_0$ is the fiber of $V$ over $0 \in \CC^d$.
Further, we can write the $(-1)$-shifted symplectic structure defining the classical BV theory in the form
\ben
\omega(\alpha \tensor v, \beta \tensor w) = (v,w)_{V_0} \int \d^d z (\alpha \wedge \beta)
\een
where $(-,-)_{V_0}$ is a degree $(d-1)$-shifted \brian{check} pairing on the finite dimensional vector space $V_0$. 

\subsection{Holomorphic gauge fixing}

To begin the process of renormalization we must fix the data of a gauge fixing operator. 
A gauge fixing operator is an operator on fields
\ben
Q^{GF} : \sE \to \sE[1]
\een
of cohomological degree $-1$ such that $[Q, Q^{GF}]$ is a generalized Laplacian on $\sE$ where $Q$ is the linearized BRST operator. 
For a full definition of this see Definition ?? \ref{CG}. 

For holomorphic theories there is a convenient choice for a gauge fixing operator. 
To construct it we fix the standard flat metric on $\CC^d$. 
Doing this, we let $\dbar^*$ be the adjoint of the operator $\dbar$.
Using the coordinates on $(z_1,\ldots, z_d) \in \CC^d$ we can write this operator as
\ben
\dbar^* = \sum_{i=1}^d \frac{\partial}{\partial (\d \zbar_i)} \frac{\partial}{\partial z_i} .
\een
Equivalently $\frac{\partial}{\partial (\d \zbar_i)}$ is equal to contraction with the anti-holomorphic vector field $\frac{\partial}{\partial \zbar_i}$. 
The operator $\dbar^*$ extends to the complex of fields via the formula
\ben
Q^{GF} = \dbar^* \tensor {\rm id}_V : \Omega^{0,*}(X , V) \to \Omega^{0,*-1}(X, V),
\een
We claim that this is a gauge fixing operator for our holomorphic theory.
Indeed, since $Q^{hol}$ is a translation invariant holomorphic differential operator we have
\ben
[\dbar + Q^{hol}, Q^{GF}] = [\dbar,\dbar^*] \tensor \id_{V} .
\een
The operator $[\dbar,\dbar^*]$ is simply the Dolbeualt Laplacian on $\CC^d$, which is certainly a generalized Laplacian.
In coordinates it is
\ben
[\dbar,\dbar^*] = -\sum_{i=1}^d \frac{\partial}{\partial \zbar_i}\frac{\partial}{\partial z_i} 
\een

By definition, the heat kernel is the dual
\brian{check factor}


Pick a basis $\{e_i\}$ of $V_0$ and let 
\ben
{\bf C}_{V_0} = \sum_{i,j} \omega_{ij} (e_i \tensor e_j) \in V_0 \tensor V_0
\een
be the quadratic Casimir.
Here, $(\omega_{ij})$ is the inverse matrix to the pairing $(-,-)_{V_0}$. 
The regularized heat kernel then takes the form
\ben
K_{\epsilon}(z,w) = K^{an}(z,w) \cdot {\bf C}_{V_0}
\een

\begin{lem} 
If $\Gamma$ is a tree then $\lim_{\epsilon \to 0} W_{\Gamma}(P_{\epsilon < L}, I)$ exists.
\end{lem}

\subsection{One-loop weights}

\begin{dfn}
Let $\epsilon , L > 0$. 
In addition, fix the following data.
\begin{enumerate}
\item An integer $k \geq 1$ that will be the number of vertices of the graph.
\item For each $\alpha = 1, \ldots, k$ a sequence of integers
\ben
\vec{n}^\alpha = (n_1^\alpha, \ldots, n_d^{\alpha}) .
\een
We denote by $(\vec{n}) = (n_{i}^j)$ the corresponding $d \times k$ matrix of integers. 
\item A smooth compactly supported function $\Phi \in C_c^\infty((\CC^d)^k) = C_c^\infty(\CC^{dk})$.
\end{enumerate}
The analytic weight associated to the triple $(k, (\vec{n}), \Phi)$ is
\be\label{weight1}
W_{k,(\vec{n})}^\Phi (\epsilon, L) = \int_{(z^1,\ldots, z^k) \in (\CC^d)^k} \prod_{\alpha=1}^k \d^d z^\alpha \Phi(z^1,\ldots,z^\alpha) \prod_{\alpha = 1}^k \left(\frac{\partial}{\partial z^i}\right)^{\vec{n}^\alpha} P_{\epsilon < L}^{an}(z^i, z^{i+1}) .
\ee
In the above expression, we use the convention that $z^{k+1} = z^1$. 
\end{dfn}

We will refer to the collection of data $(k, (\vec{n}), \Phi)$ in the definition as {\em wheel data}.
The motivation for this is that the weight $W_{k,(\vec{n})}^\Phi (\epsilon, L)$ is the analytic part of the full weight $W_{\Gamma}(P_{\epsilon<L}, I)$ where $\Gamma$ is a wheel with $k$ vertices. 

We have reduced the proof of Theorem \ref{thm holo renorm} to showing that the $\epsilon \to 0$ limit of the analytic weight $W_{k,(\vec{n})}^\Phi (\epsilon, L)$ exists for any tripe of wheel data $(k, (\vec{n}), \Phi)$.
To do this, there are two steps. 
First, we show a vanishing result that says when $k \geq d$ the  weights vanish for purely algebraic reasons. 
The second part is the most technical aspect of the chapter where we show that for $k > d$ the weights have nice asymptotic behavior as a function of $\epsilon$.

\begin{lem} Let $(k, (\vec{n}), \Phi)$ be a triple of wheel data.
If the number of vertices $k$ satisfies $k \leq d$ then
\ben
W_{k, (\vec{n})}^{\Phi}(\epsilon , L) = 0
\een
for any $\epsilon,L > 0$. 
\end{lem}
\begin{proof}
In the integral expression for the weight (\ref{weight1}) there is the following factor involving the product over the edges of the propagators:
\be\label{productprops2}
\prod_{\alpha = 1}^k \left(\frac{\partial}{\partial z^i}\right)^{\vec{n}^\alpha} P_{\epsilon < L}^{an}(z^i, z^{i+1}) .
\ee
We will show that this expression is identically zero.
To simplify the expression we first make the following change of coordinates on $\CC^{dk}$:
\begin{align}
w^i & = z^{\alpha+1} - z^\alpha \;\;\; , \;\;\; 1\leq \alpha < k \label{coords1}\\
w^k & = z^k \label{coords2} .
\end{align}
Introduce the following operators
\ben
\eta^\alpha = \sum_{i=1}^{d} \wbar_i^\alpha \frac{\partial}{\partial (\d \wbar_i^\alpha)}
\een
acting on differential forms on $\CC^{dk}$.
The operator $\eta^\alpha$ lowers the anti-holomorphic Dolbuealt type by one : $\eta : (p,q) \to (p,q-1)$.
Equivalently, $\eta^\alpha$ is contraction with the anti-holomorphic Euler vector field $\wbar_i^\alpha \partial / \partial \wbar_i^\alpha$.

Once we do this, we see that the expression (\ref{productprops2}) can be written as 
\ben
\left(\left(\sum_{\alpha=1}^{k-1} \eta^\alpha \right) \prod_{i=1}^d \left(\sum_{\alpha = 1}^{k-1} \d \wbar_{i}^\alpha\right) \right) \prod_{\alpha=1}^{k-1}\left( \eta^\alpha \prod_{i=1}^d \d \wbar_i^\alpha\right) .
\een
Note that only the variables $\wbar_i^{\alpha}$ for $i=1,\ldots,d$ and $\alpha = 1,\ldots, k-1$ appear. 
Thus we can consider it as a form on $\CC^{d(k-1)}$.
As such a form it is of Dolbeualt type $(0, (d-1) + (k-1)(d-1)) = (0, (d-1)k)$. 
If $k < d$ then clearly $(d-1)k > d(k-1)$ so the form has greater degree than the dimension of the manifold and hence it vanishes. 

The case left to consider is when $k = d$.
In this case, the expression in (\ref{productprops2}) can be written as
\be\label{productprops1}
\left(\left(\sum_{\alpha=1}^{d-1} \eta^\alpha \right) \prod_{i=1}^d \left(\sum_{\alpha = 1}^{d-1} \d \wbar_{i}^\alpha\right) \right) \prod_{\alpha=1}^{d-1}\left( \eta^\alpha \prod_{i=1}^d \d \wbar_i^\alpha\right) .
\ee
Again, since only the variables $\wbar_i^{\alpha}$ for $i=1,\ldots,d$ and $\alpha = 1,\ldots, d-1$ appear, we can view this as a differential form on $\CC^{d(d-1)}$. 
Furthermore, it is a form of type $(0, d(d-1))$. 
For any vector field $X$ on $\CC^{d(d-1)}$ the interior derivative $i_X$ is a graded derivation. 
Suppose $\omega_1,\omega_2$ are two $(0,*)$ forms on $\CC^{d(d-1)}$ such that the sum of their degrees is equal to $d^2$. 
Then, $\omega_1 \iota_X \omega_2$ is a top form for any vector field on $\CC^{d(d-1)}$.
Since $\omega_1 \omega_2 = 0$ for form type reasons, we conclude that $\omega_1 \iota_X \omega_2 = \pm (i_X \omega_1) \omega_2$ with sign depending on the dimension $d$. 
Applied to the vector field $\zbar_i^1\partial / \partial \wbar_i^1$ in (\cite{productprops1}) we see that the expression can be written (up to a sign) as 
\ben
\eta^1 \left(\sum_{\alpha=1}^{d-1} \eta^\alpha \prod_{i=1}^d \left(\sum_{\alpha = 1}^{d-1} \d \wbar_{i}^\alpha\right) \right) \left(\prod_{i=1}^d \d \wbar_i^1\right) \prod_{\alpha=2}^{d-1} \left( \eta^\alpha \prod_{i=1}^d \d \wbar_i^\alpha\right) .
\een
Repeating this, for $\alpha =2,\ldots,k-1$ we can write this expression (up to a sign) as
\ben
\left(\eta_{k-1} \cdots \eta_2 \eta _1 \sum_{\alpha=1}^{k-1} \eta^\alpha \prod_{i=1}^d \left(\sum_{\alpha = 1}^{k-1} \d \wbar_{i}^\alpha\right) \right) \prod_{\alpha=1}^{k-1} \prod_{i=1}^d \d \wbar_i^\alpha 
\een
The expression inside the parentheses is zero since each term in the sum over $\alpha$ involves a term like $\eta^\beta \eta^\beta = 0$. 
This completes the proof for $k=d$. 
\end{proof}

\begin{lem}
Let $(k, (\vec{n}), \Phi)$ be a triple of wheel data such that $k > d$.
Then the $\epsilon \to 0$ limit of the analytic weight
\ben
\lim_{\epsilon \to 0} W_{k,(\vec{n})}^\Phi (\epsilon, L) 
\een
exists.
\end{lem}

\begin{proof}

We will bound the absolute value of the weight in Equation (\ref{weight1}) and show that it has a well-defined $\epsilon\to 0$ limit.
First, consider the change of coordinates as in Equations (\ref{coords1}),(\ref{coords2}).
The weight can be written as
\be\label{weight2}
\int_{w^k \in \CC^d} \d^{d} w^k \int_{(w_1,\ldots,w_{k-1}) \in (\CC^d)^{k-1}} \left(\prod_{\alpha=1}^{k-1} \d^{d} w^\alpha\right) \Phi(w^1,\ldots,w^k) \left(\prod_{\alpha=1}^{k-1} \left(\frac{\partial}{\partial w^\alpha}\right)^{\vec{n}^\alpha}P^{an}_{\epsilon < L} (w^\alpha) \right) \sum_{\alpha=1}^{k-1} \left(\frac{\partial}{\partial w^\alpha}\right)^{\vec{n}^k} P^{an} \left(\sum_{\alpha=1}^{k-1} w^\alpha\right) .
\ee
For $\alpha = 1,\ldots,k-1$ the notation 
\ben
P^{an}_{\epsilon < L} (w^\alpha) = \int_{t_\alpha=\epsilon}^L \frac{\d t_\alpha}{4\pi t_\alpha} \dbar^* \brian{FINISH}
\een
makes sense since $P^{an}_{\epsilon<L}(z^\alpha,z^{\alpha+1})$ is only a function of $w^\alpha = z^{\alpha+1}-z^\alpha$.
Similarly $P^{an}_{\epsilon<L}(z^{k+1},z^1)$ is a function of 
\ben
z^k - z^1 = \sum_{\alpha=1}^{k-1} w^\alpha . 
\een
Expanding out the propagators the weight takes the form
\ben
\begin{array}{lll}
& \displaystyle \int_{w^k \in \CC^d} \d^{2d} w^k \int_{(w_1,\ldots,w_{k-1}) \in (\CC^d)^{k-1}} \left(\prod_{\alpha=1}^{k-1} \d^{2d} w^\alpha\right) \Phi(w^1,\ldots,w^k) \int_{(t_1,\ldots,t_k) \in [\epsilon,L]^k} \prod_{\alpha=1}^k \frac{\d t_\alpha}{4 \pi t_\alpha} \\
& \displaystyle \times \sum_{i_1,\ldots,i_{k-1} =1}^d \left(\frac{\wbar^1_{i_1}}{t_1} \frac{(\wbar^1)^{n^1}}{t^{|n^1|}}\right) \cdots \left(\frac{\wbar^{k-1}_{i_{k-1}}}{t_{k-1}}\frac{(\wbar^{k-1})^{n^{k-1}}}{t^{|n^{k-1}|}}\right) \left(\sum_{\alpha=1}^{k-1} \frac{\wbar^\alpha_{i_k}}{t_k} \cdot \frac{1}{t^{|n^k|}} \left(\sum_{\alpha=1}^{k-1} \wbar^\alpha\right)^{n^k}\right) \\
& \displaystyle \times \exp\left(- \sum_{\alpha=1}^{k-1} \frac{|w^{\alpha}|^2}{t_\alpha} - \frac{1}{t_k} \left|\sum_{\alpha=1}^{k-1} w^\alpha \right|^2\right)
\end{array}
\een
The notation used above warrants some explanation. 
Recall, for each $\alpha$ the vector of integers is defined as $n^\alpha = (n^{\alpha}_1,\ldots,n^{\alpha}_d)$. 
We use the notation
\ben
(\wbar^\alpha)^{n^\alpha} = \wbar^{n^\alpha_1}_1 \cdots \wbar^{n^\alpha_d}_d .
\een
Furthermore, $|n^\alpha| = n_1^\alpha + \cdots + n_d^\alpha$. 
Each factor of the form $\frac{\wbar^\alpha_{i_\alpha}}{t_\alpha}$ comes from the application of the operator $\frac{\partial}{\partial z_i}$ in $\dbar^*$ applied to the propagator. 
The factor $\frac{(\wbar^\alpha)^{n^\alpha}}{t^{|n^\alpha|}}$ comes from applying the operator $\left(\frac{\partial}{\partial w}\right)^{n^\alpha}$ to the propagator. 
Note that $\dbar^*$ commutes with any translation invariant holomorphic differential operator, so it doesn't matter which order we do this.

To bound this integral we will recognize each of the factors
\ben
\frac{\wbar^\alpha_{i_\alpha}}{t_\alpha} \frac{(\wbar^\alpha)^{n^\alpha}}{t^{|n^\alpha|}}
\een
as coming from the application of a certain holomorphic differential operator to the exponential in the last line.
We will then integrate by parts to obtain a simple Gaussian integral which will give us the necessary bounds in the $t$-variables. 
Let us denote this Gaussian factor by
\ben
E(w,t) := \exp\left(- \sum_{\alpha=1}^{k-1} \frac{|w^{\alpha}|^2}{t_\alpha} - \frac{1}{t_k} \left|\sum_{\alpha=1}^{k-1} w^\alpha \right|^2\right)
\een

For each $\alpha,i_\alpha$ introduce the $t=(t_1,\ldots,t_k)$-dependent holomorphic differential operator
\ben
D_{\alpha, i_\alpha}(t) := \left(\frac{\partial}{\partial w^\alpha_{i_\alpha}} - \sum_{\beta = 1}^{k-1} \frac{t_\beta}{t_1+\cdots + t_k} \frac{\partial}{\partial w_{i_\alpha}^{\beta}}\right)
\prod_{j=1}^d \left(\frac{\partial}{\partial w_j^\alpha} - \sum_{\beta =1}^{k-1} \frac{t_\beta}{t_1+\cdots + t_k} \frac{\partial}{\partial w_{j}^\beta}\right)
^{n_j^\alpha} .
\een

The following lemma is an immediate calculation
\begin{lem}
One has
\ben
D_{\alpha,i_\alpha} E(w,t) = \frac{\wbar^\alpha_{i_\alpha}}{t_\alpha} \frac{(\wbar^\alpha)^{n^\alpha}}{t^{|n^\alpha|}} E(w,t) . 
\een
\end{lem}

Note that all of the $D_{\alpha,i_{\alpha}}$ operators mutually commute. 
Thus, we can integrate by parts iteratively to obtain the following expression for the weight:
\ben
\begin{array}{lll}
& \displaystyle \pm \int_{w^k \in \CC^d} \d^{2d} w^k \int_{(w_1,\ldots,w_{k-1}) \in (\CC^d)^{k-1}}\left(\prod_{\alpha=1}^{k-1} \d^{2d} w^\alpha\right) \int_{(t_1,\ldots,t_k) \in [\epsilon,L]^k} \prod_{\alpha=1}^k \frac{\d t_\alpha}{4 \pi t_\alpha}  \\ 
& \displaystyle \times\left( \sum_{i_1,\ldots, i_d} D_{1, i_1} \cdots D_{k-1,i_{k-1}} \sum_{\alpha=1}^{k-1} D_{\alpha, i_k} \Phi(w^1,\ldots,w^k) \right) \times \exp\left(- \sum_{\alpha=1}^{k-1} \frac{|w^{\alpha}|^2}{t_\alpha} - \frac{1}{t_k} \left|\sum_{\alpha=1}^{k-1} w^\alpha \right|^2\right) .
\end{array}
\een
\brian{all the differential operators $D_{\alpha, i_\alpha}$ are uniformly bounded in $t$. To make these precise I should find what the uniform bound is.}

Thus, the absolute value of the weight is bounded by 
\be\label{weight bound1}
|W_{k,(\vec{n})}^\Phi (\epsilon, L)| \leq C \int_{w^k \in \CC^d} \d^{2d} w^k \int_{(w^1,\ldots,w^{k-1}} \prod_{\alpha=1}^{k-1} \d^{2d} w^\alpha \Psi(w^1,\ldots,w^{k-1},w^k) \int_{(t_1,\ldots,t_k) \in [\epsilon,L]^k} \d t_1 \ldots \d t_k \frac{1}{(4\pi)^{dk}} \frac{1}{t^d_1\cdots t^d_k} \times E(w,t)
\ee
To compute the right hand side we will perform a Gaussian integration with respect to the variables $(w^1,\ldots,w^{k-1})$. 
To this end, notice that the exponential can be written as
\ben
E(w,t) = \exp\left(-M_{\alpha\beta} (w^\alpha, w^\beta)\right)
\een
where $(M_{\alpha\beta})$ is the $(k-1)\times (k-1)$ matrix given by
\ben
\begin{pmatrix}
a_1 & b & b & \cdots & b \\
b & a_2 & b & \cdots & b \\
b & b & a_3 & \cdots & b \\
\vdots & \vdots & \vdots &  \ddots & \vdots \\
b & b & b & \cdots & a_{k-1}
\end{pmatrix} 
\een
where $a_\alpha = t_\alpha^{-1} + t_k^{-1}$ and $b = t_k^{-1}$.
The pairing $(w^{\alpha}, w^{\beta})$ is the usual Hermitian pairing on $\CC^d$, $(w^{\alpha}, w^{\beta}) = \sum_i w^{\alpha}_i \wbar^\beta_i$.
After some straightforward linear algebra we find that 
\ben
\det(M_{\alpha\beta})^{-1} = \frac{t_1\cdots t_k}{t_1+\cdots+t_k} .
\een 
Thus, after performing the Gaussian integration over $(w^1,\ldots,w^{k-1})$ the inequality in (\ref{weight bound1}) becomes
\begin{align}\label{weight bound2}
|W_{k,(\vec{n})}^\Phi (\epsilon, L)| & \leq C' \int_{w^k \in \CC^d} \d^{2d} w^k \Psi(0, \ldots, 0, w^k) \int_{(t_1,\ldots,t_k) \in [\epsilon,L]^k} \d t_1 \ldots \d t_k \frac{1}{(4\pi)^{dk}} \frac{1}{(t_1\cdots t_k)^d}\left(\frac{t_1\cdots t_k}{t_1+\cdots+t_k}\right)^d + O(\epsilon) \\ & = C' \int_{w^k \in \CC^d} \d^{2d} w^k \Psi(0, \ldots, 0, w^k) \int_{(t_1,\ldots,t_k) \in [\epsilon,L]^k} \d t_1 \ldots \d t_k \frac{1}{(4\pi)^{dk}} \frac{1}{(t_1+\cdots+t_k)^d}
\end{align}
The expression $\Psi(0, \ldots, 0, w^k)$ means that we have evaluate the function $\Psi(w^1,\ldots, w^k)$ at $w^1=\ldots=w^{k-1} =0$ leaving it as a function only of $w^k$. 
In the original coordinates this is equivalent to setting $z^1=\cdots=z^{k-1} = z^k$.

Our goal is to show that $\epsilon \to 0$ limit exists. 
The only $\epsilon$ dependence on the right hand side of (\ref{weight bound2}) is in the integral over the regulation parameters $t_1,\ldots, t_k$. 
Thus, it suffices to show that the $\epsilon \to 0$ limit of 
\ben
\int_{(t_1,\ldots,t_k) \in [\epsilon,L]^k} \frac{\d t_1 \ldots \d t_k}{(t_1+\cdots+t_k)^d}
\een
exists.
By the AM/GM inequality we have $(t_1+\cdots+t_k)^d \geq (t_1 \cdots t_d)^{d/k}$. 
So, the integral is bounded by
\ben
\int_{(t_1,\ldots,t_k) \in [\epsilon,L]^k}\frac{\d t_1 \ldots \d t_k}{(t_1+\cdots+t_k)^d} \leq \int_{(t_1,\ldots,t_k) \in [\epsilon,L]^k}\frac{\d t_1 \ldots \d t_k }{(t_1 \cdots t_k)^{d/k}} = \frac{1}{(1-d/k)^k} \left(\epsilon^{1-d/k} - L^{1-d/k}\right)^k .
\een
By assumption, $d < k$, so the right hand side has a well-defined $\epsilon \to 0$ limit. 
This concludes the proof.

%Now, since $t_\alpha / \sum_\beta t_\beta < 1$ for each $\alpha$ we have the following bound for the operators $D_{\alpha, i_\alpha}$:
%\bestar
%\left|D_{\alpha,i_{\alpha}} \Phi\right| & \leq \left(\left|\frac{\partial}{\partial w^\alpha_{i_\alpha}} \Phi\right| +  \sum_{\beta = 1}^{k-1}\frac{\partial}{\partial w_{i_\beta}^{\beta}}\right)
%\prod_{j=1}^d \left(\frac{\partial}{\partial w_j^\alpha} - \frac{1}{k} \sum_{\beta =1}^{k-1} \frac{\partial}{\partial w_{j}^\beta}\right)
%^{n_j^\alpha} \right| 
\end{proof}

\section{Equivariant BV quantization}

Equivariant BV quantization is an enhancement of ordinary BV quantization where one takes into account the action of a group or Lie algebra. 
We will heavily rely on techniques of equivariant BV quantization throughout this thesis, notably in the construction of the holomorphic $\sigma$-model in Chapter \ref{chap hol sig} and in the proof of a local version of the Grothendieck-Riemann-Roch theorem in Chapter \ref{chap symmetries} using Feynman diagrammatic expansions.

\subsection{Classical equivariance}

The equivariance that we consider is a direct analog of symmetries in ordinary Hamiltonian mechanics, which we briefly recall. 
Suppose that $\fh$ is a Lie algebra on $(M,\omega)$ is an ordinary symplectic manifold. 
A symplectic action of $\fh$ on $X$ is a map of Lie algebras 
\ben
\rho : \fh \to {\rm SympVect}(M)
\een
where ${\rm SympVect}(M)$ is the Lie algebra of symplectic vector fields, i.e. those vector fields $X$ which preserve the symplectic form $L_X \omega = 0$.
On any symplectic manifold, the Poisson algebra of functions admits a Lie algebra map $\sO(M) \to {\rm SympVect}(M)$ sending a function $f$ to its Hamiltonian vector field $X_f = \{f,-\}$, where $\{-,-\}$ is the Poisson bracket.
An action $\rho$ is said to be {\em inner} if it lifts to a map of Lie algebras $\Tilde{\rho} : \fh \to \sO(M)$. 
Recall that on any symplectic manifold the kernel of $f \mapsto X_f$ is precisely the constant functions. 

All of our classical theories arise as $(-1)$-shifted symplectic formal moduli problems.
Hence, suppose we replace the symplectic manifold $M$ by a formal moduli problem $B \fg$, where $\fg$ is some dg Lie (or $L_\infty$ algebra). 
To give $B \fg$ the structure of a $(-1)$-shifted symplectic structure is equivalent to having a $(-3)$-shifted non-degenerate pairing on $\fg$. 
Functions on $B\fg$ are precisely the Chevalley-Eilenberg cochains $\sO(B\fg) = \clie^*(\fg)$. 
The $(-1)$-shifted symplectic structure equips $\clie^*(\fg) [-1]$ with the structure of a dg Lie algebra.
Since all symplectic vector fields are Hamiltonian in this case we see that 
\ben
{\rm SympVect}(B \fg) = \clie^*(\fg)[-1] / \CC = \cred^*(\fg)[-1]
\een
where we have taken the quotient by the constants, which by definition is the reduced cochains. 
We modify the notion of a symplectic action slightly to allow for more general maps of Lie algebras.
A symplectic action of $\fh$ on the $(-1)$-shifted symplectic formal moduli space $B \fg$ is a map of $L_\infty$ algebras, or a homotopy coherent map of dg Lie algebras
\ben
\rho : \fh \rightsquigarrow \cred^*(\fg)[-1] .
\een
Such a map $\rho$ is equivalent to a Maurer-Cartan element in the dg Lie algebra
\ben
\clie^*(\fh) \tensor \cred^*(\fg)[-1].
\een
This is a cohomological degree $+1$ element $I^\fh$ such that $\d I^\fh + \frac{1}{2}\{I^\fh, I^\fh\} = 0$.
Here $\{-,-\}$ is the bracket on $\cred^*(\fg)$ and $\d$ is the sum of the Chevalley-Eilenberg differentials on $\fh$ and $\fg$. 

\subsection{Quantum equivariance}

If we start with an $\fh$-equivariant classical BV theory with fields $\sE$ with action functional $S$ --- so that $\fh$ has an $\L8$ action on the fields that preserves the pairing and the action functional $S$ --- then we can encode the action of $\fh$ as a Maurer-Cartan element $I^\fh$ in $\clie^*(\fh) \otimes \oloc (\sE)$.
(For the formal $\beta\gamma$ system, we did this in Lemma \ref{Noether}.)
We then view the sum $S + I^\fh$ as the \emph{equivariant} action functional:
the operator $\{S+ I^\fh,-\}$ is the twisted differential on $\clie^*(\fh) \otimes \oloc(\sE)$ with $I^\fh$ as the twisting cocycle,
and this operator is square-zero because $\{S + I^\fh, S+ I^\fh\}$ is a ``constant'' (i.e., lives in $\clie^*(\fh)$ and hence is annihilated by the BV bracket).

This perspective suggests the following definition of an equivariant quantum BV theory.
The starting data is two-fold:
an $\fh$-equivariant classical BV theory with equivariant action functional $S+I^\fh$, 
and a BV quantization $\{S[L]\}$ of the non-equivariant action functional $S$.
Following Costello, it is convenient to write $S$ as $S_{\text{free}} + I$, 
where the first ``free'' term is a quadratic functional and the second ``interaction'' term is cubic and higher.
In this situation, the effective action $S[L] = S_{\text{free}} + I[L]$, 
i.e., only the interaction changes with the length scale.

\begin{dfn} \label{eqQFT} 
An {\em $\fh$-equivariant BV quantization} is a collection of effective interactions $\{I^\fh[L]\}_{L \in (0\infty)}$
satisfying
\begin{enumerate}[(a)]
\item the RG flow equation
\[
W(P_\epsilon^L, I[\epsilon]+I^\fh[\epsilon]) = I[L] + I^\fh[L]
\]
for all $0 < \epsilon < L$,
\item the equivariant scale $L$ quantum master equation, which is that
\[
Q(I[L]+I^\fh[L]) + \d_\fh I^\fh[L] + \frac{1}{2}\{I[L]+I^\fh[L],I[L]+I^\fh[L]\}_{L} + \hbar \Delta_L(I[L]+I^\fh[L])
\]
lives in $\clie^*(\fh)$ for every scale $L$, and
\item the locality axiom, with the additional condition that as $L \to 0$, we recover the equivariant classical action functional $S+ I^\fh$ modulo $\hbar$.
\end{enumerate}
\end{dfn}

In other words, we simply follow the constructions of \cite{CosRenormalization,CG} working over the base ring $\clie^*(\fh)$.
A careful reading of those texts shows that the freedom to work over interesting dg commutative algebras is built into the formalism.
%\owen{We may need to be a bit careful since a version of nilpotency is used there but ..}
%Note that our situation is particularly simple since the non-equivariant classical field theory is free and hence admits a very simple quantization,
%with $I[L] = 0$ for all $L$.

\subsection{The case of a local Lie algebra}

The above formalism works equally well, with some slight modifications, if we replace the Lie algebra $\fh$ by a {\em local Lie algebra} on the manifold where the theory lives.

\brian{do this}

\begin{lem}\label{lem: innner action}
Suppose $\{I^{\sL}[L]\}$ is an effective family satisfying the $\sL$-equivariant quantum master equation modulo $\cloc^*(\sL)$.
Then, the obstruction to lifting this action to an inner action, that is the anomaly to solving the quantum master equation, is a degree $+1$ cocycle in $\cloc^*(\sL)$. 
\end{lem}


\begin{rmk}
Equivariant quantization is essentially a version of the background field method in QFT.
One treats elements of $\sL$ as background fields and 
the interaction terms $I^\sL[L]$ encode the variation of the path integral measure with respect to these background fields.
(Solving the QME is our definition of well-posedness of the measure.)
This should not be confused with {\em gauging} a theory by $\sL$, which involves putting the elements of $\sL$ in the theory as propagating fields.
\end{rmk}

\end{document}

