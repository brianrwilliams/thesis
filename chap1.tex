\documentclass[10pt]{amsart}

\usepackage{macros,slashed}

\linespread{1.25}

\usepackage{tikz}
\usetikzlibrary{arrows,shapes}
\usetikzlibrary{trees}
\usetikzlibrary{matrix,arrows}
\usetikzlibrary{positioning}
\usetikzlibrary{calc,through}
\usetikzlibrary{decorations.pathreplacing}
\usepackage{pgffor}

\def\sAd{\sA{\rm d}}

\title{Holomorphic quantum field theory}

\def\brian{\textcolor{blue}{BW: }\textcolor{blue}}


\begin{document}
\maketitle
\tableofcontents

%\chapter{Holomorphic quantum field theory}

Our main objective in this chapter is two-fold. 
First we will define the concept of a holomorphic field theory and set up notation and terminology that we will use throughout the text. 
Our next goal is more technical, but will provide the backbone for much of the analysis throughout the remainder of this thesis.
We will show how certain holomorphic theories are surprisingly well-behaved when it comes to the problem of renormalization. 

In \cite{CostelloRenormalization} Costello has provided a mathematical formulation of the Wilsonian approach to quantum field theory.
The main takeaway is that to construct a full quantum field theory it suffices to define the theory at each energy (or length) scale and to ask that these descriptions be compatible as we vary the scale.
The infamous infinities of quantum field theory arise due to studying behavior of theories at arbitrarily high energies (or small lengths). 
In physics this is called the ultra-violet (UV) divergence. 
Classically, a theory is defined by a {\em local functional} $I^{cl}$ which is a functional on the space of fields obtained by integrating a Lagrangian density.
At each scale $L$ a theory is defined by an action functional $I[L]$, which is a function on the space of fields. 

Summarizing, there are two main steps to construct a QFT in our formalism.
\begin{itemize}
\item[{\bf Renormalization:}] For each scale $L$ and regulator $\epsilon > 0$ consider the RG flow from scale $\epsilon$ to $L$:
\be
W(P_{\epsilon < L} , I) .
\ee
In general, the limit $\epsilon \to 0$ will not be defined, but by Costello's main result there exists counterterms $I^{CT}(\epsilon)$ such that the $\epsilon \to 0$ limit of 
\ben
W(P_{\epsilon<L} , I - I^{CT}(\epsilon))
\een
is well-defined. 
Denote this limit by $I[L]$.
The family $\{I[L]\}$ defines a prequantization.
\item[{\bf Gauge consistency:}] We then ask if the family $\{I[L]\}$ defines a consistent quantization.
For each $L$ we require that $I[L]$ satisfy the scale $L$ quantum master equation....
\end{itemize}

In this section we are concerned with the first step: renormalization. 
The complication here is that even very natural field theories can have a very complicated collections of counterterms. 
For instance, the naive quantization of Chern-Simons theory on a three-manifold has counterterms even at one-loop. 
For holomorphic theories, however, we will show how the situation becomes much simpler at least at the level of one-loop.  

\begin{lem}\label{lem: hol renorm}
Let $\sE$ be a holomorphic theory on $\CC^d$ with classical interaction $I^{cl}$.  
Then, there exists a one-loop prequantization $\{I[L] \; | \; L > 0\}$ of $I^{cl}$ involving no counterterms. 
That is, we can find a propagator $P_{\epsilon < L}$ for which the $\epsilon \to 0$ limit of
\ben
W(P_{\epsilon<L} , I) \mod \hbar^2
\een
exisits.
Moreover, if $I$ is holomorphically translation invariant we can pick the family $\{I[L]\}$ to be holomorphically translation invariant as well.
\end{lem}

We will use this result repeatedly throughout this thesis. 
This result tells us that the analytic difficulties are manageable and that the key focus will be on obstructions to satisfying the quantum master equation.
In particular, a corollary of this result will give us a procedure for computing the one-loop obstruction explicitly in terms of Feynman diagrams. 
We conjecture that an extension of this result should hold to all orders in $\hbar$ which would give a constructive way of analyzing the obstruction theory order by order in $\hbar$. 
Nevertheless, we will leverage the one-loop behavior to formulate and prove index theorems in the context of holomorphic QFT.

One surprising aspect of this comes from thinking about holomorphic theories in a different way. 
Any supercharge $Q$ of a supersymmetric theory satisfying $Q^2 = 0$ allows one to construct a ``twist". 
In some cases, where Clifford multiplication with $Q$ spans all translations such a twist becomes a topological theory (in the weak sense). 
In any case, however, such a $Q$ defines a ``holomorphic twist", which results in the type of holomorphic theories we consider.
Regularization in supersymmetric theories, especially gauge theories, is notoriously difficult. 
Our result implies that after twisting the analytic difficulties become much easier to deal with. 
Consequently, phenomena such as anomalies can be cast in a more algebraic framework.
We will see such an example of this in the case of the holomorphic $\sigma$-model in the next chapter. 

Already, in \cite{LiVertex} Li has used a complex one-dimensional version of this fact to all orders in $\hbar$. 
He uses this to give an elegant interpretation of the quantum master equation for two-dimensional chiral conformal field theories using vertex algebras.
We do not make any statements in this work past one-loop
quantizations, but the higher loop behavior remains a very interesting and subtle problem that we hope to return to.

\section{The definition of a quantum field theory}

The goal of this section is to review the classical and quantum Batalin-Vilkovisky formalisms.
We will also set up the requisite conventions and notations that we will use throughout the thesis. 

\subsection{Classical field theory}

Classical field theory is a formalism for describing a physical system in terms of objects called {\em fields}. 
Mathematically, the space of fields is a (most often infinite dimensional) vector space $\sE$. 
Classical physics is described by the critical locus of a (usually real or complex valued) linear functional on the space of fields 
\be\label{actionfnl}
S : \sE \to \RR \;\; {\rm or} \;\; \CC,
\ee
called the {\em action functional}. 
The critical locus is the locus of fields that have zero variation
\be
{\rm Crit}(S) := \{\varphi \in \sE \; | \; \d S (\varphi) = 0\} .
\ee
A field $\varphi$ satisfying the equation $\d S (\varphi) = 0$ is said to be a {\em solution to the classical equations of motion}. 

Even in the finite dimensional case, if the functional $S$ is not sufficiently well-behaved the critical locus can be still be highly singular. 
The starting point of the {\em classical Batalin-Vilkovisky} formalism is to instead consider the {\em derived} critical locus.
To get a feel for this, we review the finite dimensional situation.
Let $M$ be a manifold, which is our ansatz for $\sE$ at the moment, and suppose $S : M \to \RR$ is a smooth map.
The critical locus is the intersection of the graph of $\d S$ in $T^*M$ with the zero section $0 : M \to T^*M$.
Thus, functions on the critical locus are of the form
\ben
\sO({\rm Crit}(S)) = \sO(\Gamma(\d S)) \tensor_{\sO(T^*M)} \sO(M) .
\een
The derived critical locus is a derived space whose dg ring of functions is 
\ben
\sO({\rm Crit}^{h}(S)) = \sO(\Gamma(\d S)) \tensor^{\LL}_{\sO(T^*M)} \sO(M).
\een
We have replaced the strict tensor product with the derived one.
Using the Koszul resolution of $\sO(M)$ as a $\sO(T^*M)$-module one can write this derived tensor product as a complex of polyvector fields equipped with some differential:
\ben
\sO({\rm Crit}^h(S)) \simeq \left({\rm PV}^{-*}(M), \iota_{\d S}\right) .
\een
In cohomological degree $-i$ we have ${\rm PV}^{-i} (M) = \Gamma(M, \wedge^i TM)$ and $\iota_{\d S}$ denotes contraction with the one-form $\d S$ (which raises cohomological degree with our regrading convention).
With our grading convention we have $\sO(T^*[-1] M) = {\rm PV}^{-*}(M)$. 
The space $\sO(T^*[-1]M)$ has natural shifted Poisson structure, which takes the form of the familiar Schouten-Nijenhuis bracket of polyvector fields.

The takeaway is that the derived critical locus of a functional $S : M \to \RR$ has the structure of a $(-1)$-symplectic space.
This will be the starting point for our definition of a theory in the BV formalism in the general setting.
 
In all non-trivial examples the space of fields $\sE$ is infinite dimensional and we must be careful with what functionals $S$ we allow.
The space of fields we consider will always have a natural topology, and we will choose functionals that are continuous with respect to it. 
We divert for a moment to discuss these issues of infinite dimensional linear algebra.

\subsubsection{Some functional analysis}
Homological algebra will play a paramount role in our approach to perturbative field theory.
A problem with this is that the category of topological vector spaces is not an abelian category.
It is therefore advantageous to enlarge this to the category of {\em differentiable vector spaces}.
The details of this setup are carried out in the Appendix of \cite{CG1}, but we will recall some key points.

Let ${\rm Mfld}$ be the site of smooth manifolds.
The covers defining the Grothendieck topology are given by surjective local diffeomorphisms.
There is a natural sheaf on this site given by smooth functions $C^\infty : M \mapsto C^\infty(M)$. 
By definition, a $C^\infty$-module is a module sheaf over $C^\infty$ on ${\rm Mfld}$. 

For any $p$ the assignment $\Omega^p : M \mapsto \Omega^p (M)$ defines a $C^\infty$-module.
Similarly, if $F$ is any $C^\infty$-module we have the $C^\infty$-module of $p$-forms with values on $F$ defined by the assignment 
\ben
\Omega^1(F) : M \in {\rm Mfld} \mapsto \Omega^p(M, F) = \Omega^p(M) \tensor_{C^\infty(M)} F(M) .
\een

\begin{dfn}
A {\em differentiable vector space} is a $C^\infty$-module equipped with a map of sheaves on ${\rm Mfld}$
\ben
\nabla : F \to \Omega^1(F) 
\een 
such that for each $M$, $\nabla(M)$ defines a flat connection on the $C^\infty(M)$-module $F(M)$. 
A map of differentiable vector spaces is one of $C^\infty$-modules that intertwines the flat connections. 
This defines a category that we denote ${\rm DVS}$ 
\end{dfn}

Our favorite example of differentiable vector spaces are imported directly from geometry.

\begin{eg}
Suppose $E$ is a vector bundle on a manifold $X$. 
Let $\sE(X)$ denote the space of smooth global sections.
Let $C^\infty(M, \sE(X))$ be the space of sections of the bundle $\pi_X^*E$ on $M \times X$ where $\pi_X : M \times X \to X$ is projection. 
The assignment $M \mapsto C^\infty(M , \sE(X))$ is a $C^\infty$-module with flat connection, so defines a differentiable vector space.
Similarly, the space of compactly supported sections $\sE_c(X)$ is a DVS. 
\end{eg}

Many familiar categories of topological vector spaces embed inside the category of differentiable vector spaces. 
Consider the category of locally convex topological vector spaces ${\rm LCTVS}$.
If $V$ is such a vector space, there is a notion of a smooth map $f : U \subset \RR^n \to V$.
One can show, Proposition B.3.0.6 of \cite{CG1}, that this defines a functor ${\rm dif}_t : {\rm LCTVS} \to {\rm DVS}$ sending $V$ to the $C^\infty$-module $M \mapsto C^\infty(M, V)$.
If ${\rm BVS} \subset {\rm LCTVS}$ is the subcategory with the same objects but whose morphisms are bounded linear maps, this functor restricts to embed ${\rm BVS}$ as a full subcategory ${\rm BVS} \subset {\rm DVS}$. 

There is a notion of completeness that is useful when discussing tensor products. 
A topological vector space $V \in {\rm BVS}$ is {\em complete} if every smooth map $c : \RR \to V$ has an anti-derivative \cite{KM97}.
There is a full subcategory ${\rm CVS} \subset {\rm BVS}$ of complete topological vector spaces.
The most familiar example of a complete topological vector space will be the smooth sections $\sE(X)$ of a vector bundle $E \to X$.

We let ${\rm Ch}({\rm DVS})$ denote the category of cochain complexes in differentiable vector spaces (we will refer to objects as differentiable vector spaces). 
It is enriched over the category of differential graded vector spaces in the usual way.
We say that a map of differentiable cochain complexes $f : V \to W$ is a quasi-isomorphism if and only if for each $M$ the map $f : C^\infty(M, V) \to C^\infty(M,W)$ is a quasi-isomorphism.

\begin{thm}[Appendix B \cite{CG1}]
The full subcategory ${\rm dif}_c : {\rm CVS} \subset {\rm DVS}$ is closed under limits, countable coproducts, and sequential colimits of closed embeddings. 
Furthermore, {\rm CVS} has the structure of a symmetric monoidal category with respect to the completed tensor product $\Hat{\tensor}_{\beta}$. 
\end{thm}

We will not define the tensor product $\Hat{\tensor}_\beta$here, but refer the reader the cited reference for a complete exposition.
We will recall its key properties below.
Often times we will write $\tensor$ for $\Hat{\tensor}_\beta$ where there is no potential conflict of notation. 
The fundamental property of the tensor product that we use is the following.
Suppose that $E,F$ are vector bundles on manifolds $X,Y$ respectively.
Then, $\sE(X), \sF(Y)$ lie in ${\rm CVS}$, so it makes sense to take their tensor product using $\Hat{\tensor}_\beta$. 
There is an isomorphism
\be\label{tensor1}
\sE(X) \Hat{\tensor}_\beta \sF(Y) \cong \Gamma(X \times Y, E \boxtimes F)
\ee
where $E \boxtimes F$ denotes the external product of bundles, and $\Gamma$ is smooth sections. 

If $E$ is a vector bundle on a manifold $X$, then the spaces $\sE(X), \sE_c(X)$ both lie in the subcategory ${\rm CVS} \subset {\rm DVS}$. 
The differentiable structure arises from the natural topologies on the spaces of sections. 

We will denote by $\Bar{\sE}(X)$ ($\Bar{\sE}_c(X)$) the space of (compactly supported) distributional sections.
It is useful to bear in mind the following inclusions
\ben
\begin{tikzcd}        
    \sE_c(X)  \arrow[hook]{r} \arrow[hook]{rd} & \Bar{\sE}_c(X) \arrow[hook]{r} & \Bar{\sE}(X) \\
    & \sE(X) \arrow[hook]{ru} & .
\end{tikzcd}
\een

Denote by $E^\vee$ the dual vector bundle whose fiber over $x \in X$ is the linear dual of $E_x$. 
Let $E^!$ denote the vector bundle $E^\vee \tensor {\rm Dens}_X$, where ${\rm Dens}_X$ is the bundle of densities. 
In the case $X$ is oriented, ${\rm Dens}_X$ is isomorphic to the top wedge power of $T^*X$. 
Let $\sE^!(X)$ denote the space of sections of $E^!$. 
The natural pairing 
\ben
\sE_c(X) \tensor \sE^!(X) \to \CC
\een
that pairs sections of $E$ with the evaluation pairing and integrates the resulting compactly supported top form exhibits $\Bar{\sE}_c(X)$ as the continuous dual to $\sE^!(X)$. 
Likewise, $\sE_c(X)$ is the continuous dual to $\Bar{\sE}^!(X)$. 
In this way, the topological vector spaces $\Bar{\sE}(X)$ and $\Bar{\sE}_c(X)$ obtain a natural differentiable structure.

If $V$ is any differentiable vector space then we define the space of linear functionals on $V$ to be the space of maps $V^* = {\rm Hom}_{\rm DVS}(V, \RR)$. 
Since ${\rm DVS}$ is enriched over itself this is again a differentiable vector space. 
Similarly, we can define the polynomial functions of homogeneous degree $n$ to be the space
\ben
\Sym^n(V^*) = {\rm Hom}^{multi}_{\rm DVS}(V \times \cdots \times V, \RR)_{S_n}
\een
where the hom-space denotes multi-linear maps, and we have taken $S_n$-coinvariants on the right-hand side.
The algebra of functions on $V$ is defined by
\ben
\sO(V) = \prod_{n} \Sym^n(V^*) .
\een

As an application of Equation (\ref{tensor1}) we have the following identification.

\begin{lem}\label{lem: fnls}
Let $E$ be a vector bundle on $X$. 
Then, there is an isomorphism
\ben
\sO(\sE(X)) \cong \prod_{n} \sD_c(X^n , (E^!)^{\boxtimes n})_{S_n}
\een
where $\sD_c(X^n , (E^!)^{\boxtimes n})$ is the space of compactly supported distributional sections of the vector bundle $(E^!)^{\boxtimes n}$.
Again, we take $S_n$-coinvariants on the right hand side.
\end{lem}

\subsubsection{Local functionals}

In our approach, the space of fields will always be equal to the space of smooth sections of a $\ZZ$-graded vector bundle $E\to X$ on a manifold $\sE = \Gamma(X, E)$. 
The class of functionals $S : \sE \to \RR$ defining the classical theories we consider are required to be {\em local}, or given by the integral of a Lagrangian density. 
We define this concept now.

Let $D_X$ denote the sheaf of differential operators on $X$. 
The $\infty$-jet bundle ${\rm Jet}(E)$ of a vector bundle $E$ is the vector bundle whose fiber over $x \in X$ is the space of formal germs at $x$ of sections of $E$. 
It is a standard fact that ${\rm Jet}(E)$ is equipped with a flat connection giving its space of sections $J(E) = \Gamma(X, {\rm Jet}(E))$ the structure of a $D_X$-module.

Above, we have defined the algebra of functions $\sO(\sE(X))$ on the space of sections $\sE(X)$.
Similarly, let $\sO_{red}(\sE(X)) = \sO(\sE(X)) / \RR$ be the quotient by the constant polynomial functions. 
The space $\sO_{red}(J(E))$ inherits a natural $D_X$-module structure from $J(E)$. 
We refer to $\sO_{red}(J(E))$ as the space of {\em Lagrangians} on the space of sections of the vector bundle $E$. 
Indeed, every element $F \in \sO_{red}(J(E))$ can be expanded as $F = \sum_n F_n$ where each $F_n$ is an element 
\ben
F_n \in {\rm Hom}_{C^\infty_X} (J(E)^{\tensor n}, C^\infty_X)_{S_n} \cong {\rm PolyDiff}(\sE^{\tensor n}, C^\infty(X))_{S_n}
\een
where the right-hand side is the space of polydifferential operators.
The proof of the isomorphism on the right-hand side can be found in Chapter 5 of \cite{CostelloRenormalization}.

A local functional is given by a Lagrangian densities modulo total derivatives.
The mathematical definition is the following.

\begin{dfn}
Let $E$ be a graded vector bundle on $X$.
Define the sheaf of {\em local functionals} on $X$ to be
\ben
\oloc(\sE) = {\rm Dens}_X \tensor_{D_X} \sO_{red}(J(E)),
\een
where we use the natural right $D_X$-module structure on densities.
\end{dfn}

Note that we always consider local functionals coming from Lagrangians modulo constants. 
We will not be concerned with local functions associated to constant Lagrangians. 

From the expression for functionals in Lemma \ref{lem: fnls} we see that integration defines an inclusion of sheaves
\be\label{local inclusion}
i : \oloc(\sE) \hookrightarrow \sO_{red}(\sE_c) .
\ee
Often times when we describe a local functional we will write down its value on test compactly supported sections, then check that it is given by integrating a Lagrangian density, which amounts to lifting the functional along $i$. 

\subsubsection{The definition of a classical field theory}

Before giving the definition, we need to recall what the proper notion of a shifted symplectic structure is in the geometric setting that we work in.

\begin{dfn}\label{dfn: symplectic}
Let $E$ be a graded vector bundle on $X$.
A $k$-{\em shifted symplectic structure} is an isomorphism of graded vector spaces
\ben
E \cong_{\omega} E^![k] = \left({\rm Dens}_X \tensor E^\vee\right)[k]
\een
that is graded anti-symmetric.
\end{dfn}

If $\omega^*$ is the formal adjoint of the isomorphism $\omega^* : E \cong E^![k]$, anti-symmetry amounts to the condition $\omega^* = - \omega$. 
In general, $\omega$ does {\em not} induces a Poisson structure on the space of all functionals $\sO(\sE)$. 
This is because, as we have seen above, elements of this space are given by distributional sections and hence we cannot pair elements with overlapping support.
The symplectic structure does, however, induce a Poisson bracket on {\em local} functionals. \footnote{Note that $\oloc(\sE)$ is not a shifted Poisson algebra since there is no natural commutative product.}
We will denote the bracket induced by a shifted symplectic structure by $\{-,-\}$. 

We are now ready to give the precise definition of a classical field theory.

\begin{dfn}
A {\em classical field theory} in the BV formalism on a smooth manifold $X$ is a $\ZZ$-graded vector bundle $E$ equipped with a $(-1)$-shifted symplectic structure together with a local functional $S \in \oloc(\sE)$ such that:
\begin{enumerate}
\item the functional $S$ satisfies the {\em classical master equation} 
\ben
\{S, S\} = 0;
\een
\item $S$ is at least quadratic, so we can write it (in a unique way) as 
\ben
S(\varphi) = \omega(\varphi, Q \varphi) + I(\varphi)
\een
where $Q$ is a linear differential operator such that $Q^2 = 0$, and  $I \in \oloc(\sE)$ is at least cubic;
\item the complex $(\sE, Q)$ is elliptic.
\end{enumerate}
\end{dfn}

In the physics literature, the operator $Q$ is known as the linearized BRST operator, and $\{S,-\} = Q + \{I,-\}$ is the full BRST operator.
Ellipticity of the complex $(\sE,Q)$ is a technical requirement that will be very important in our approach to the issue of renormalization in perturbative quantum field theory.
The classical master equation is equivalent to
\ben
Q I + \frac{1}{2} \{I,I\} = 0 .
\een

A {\em free theory} is a classical theory with $I = 0$ in the notation above. 
Thus, a free theory is a simply an elliptic complex equipped with a $(-1)$-shifted symplectic pairing where the differential in the elliptic complex is graded skew-self adjoint for the pairing.  

Although the space $\sO(\sE)$ does not have a well-defined shifted Possoin bracket induced from the symplectic pairing, the operator $\{S,-\} : \sO(\sE) \to \sO(\sE)[1]$ {\em is} well-defined since $S$ is local by assumption. 
By assumption, it is also square zero. 
The complex of global classical observables of the theory is defined by
\ben
\Obs^{\cl}_{\sE}(X) = (\sO(\sE(X)), \{S,-\}) .
\een
This complex is the general replacement for functions on the derived locus from the beginning of this section.
Although it does not have a $P_0$-structure, there is a subspace that does. 
This is sometimes referred to as the {\em BRST} complex in the physics literature.

\subsubsection{A description using $L_\infty$ algebras}

There is a completely equivalent way to describe a classical field theory that helps to illuminate the mathematical meaningfulness of the definition given above. 
The requisite concept we need to introduce is that of a {\em local Lie algebra} (or local $L_\infty$ algebra).

First, recall that an $L_\infty$ algebra is a modest generalization of a dg Lie algebra where the Jacobi identity is only required to hold up to homotopy.
The data of an $L_\infty$ algebra is a graded vector space $V$ with, for each $k \geq 1$, a $k$-ary bracket
\ben
\ell_k : V^{\tensor k} \to V[2-k]
\een
of cohomological degree $2-k$. 
These maps are required to satisfy a series of conditions, the first of which says $\ell_1^2 = 0$.
The next says that $\ell_2$ is a bracket satisfying the Jacobi identity up to a homotopy given by $\ell_3$.
For a detailed definition see we refer the reader to \cite{StasheffDG, GetzlerLie}.

We now give the definition of a local $L_\infty$ algebra on a manifold $X$.
This has appeared in Chapter 4 of \cite{CG2}. 

\begin{dfn} 
A {\em local $L_\infty$ algebra} on $X$ is the following data:
\begin{itemize}
\item[(i)] a $\ZZ$-graded vector bundle $L$ on $X$, whose sheaf of smooth sections we denote $\sL^{sh}$, and
\item[(ii)] for each positive integer $n$, a polydifferential operator in $n$ inputs
\ben
\ell_n : \underbrace{\sL \times \cdots \times \sL}_{\text{$n$ times}} \to \sL[2-n]
\een
\end{itemize}
such that the collection $\{\ell_n\}_{n \in \NN}$ satisfy the conditions of an $L_\infty$ algebra.
In particular, $\sL$ is a sheaf of $L_\infty$ algebras. 
\end{dfn}

Just as in the case of an ordinary graded vector bundle, we can discuss local functionals on a local Lie algebra $L$. 
In this case, the $L_\infty$ structure maps give this the structure of a sheaf of complexes. 
Indeed, the $\infty$-jet bundle $J L$ is an $L_\infty$ algebra object in $D_X$-modules and so we can define the $D_X$-module of reduced Chevalley-Eilenberg cochains $\cred^*(J L)$. 
Mimicking the definition above, we arrive at the following local version of Lie algebra cohomology that will come up again and again in this thesis.

\begin{dfn}
Let $L$ be a local Lie algebra. 
The local Chevalley-Eilenberg cochain complex is the sheaf of cochain complexes
\ben
\cloc^*(\sL) = {\rm Dens}_X \tensor_{D_X} \cred^*(L) .
\een
\end{dfn}

It turns out that the definition of a classical field theory can be repackaged in terms of certain structures on a local $L_\infty$ algebra.
The first piece of data we need to transport to the $L_\infty$ side is that of a symplectic pairing. 
The underlying data of a local $L_\infty$ algebra $L$ is a graded vector bundle. 
In Definition \ref{dfn: symplectic} we have already defined a $k$-shifted symplectic pairing. 
On the local Lie algebra sign, we ask for $k=-3$ shifted symplectic structures that are also invariant for the $L_\infty$ structure maps. 

Also, an important part of a classical field theory is ellipticity. 
We say a local $L_\infty$ algebra is {\em elliptic} if the complex $(\sL, \d = \ell_1)$ is an elliptic complex.

\begin{prop}
The following are equivalent:
\begin{enumerate}
\item a classical field theory in the BV formalism $(\sE, \omega, S)$;
\item an elliptic local Lie algebra structure on $L = E [1]$ equipped with a $(-3)$-shifted symplectic structure.
\end{enumerate}
\end{prop}

\begin{proof} (Sketch) 
The underlying graded vector bundle of the space of fields $\sE$ is $E$ and we obtain the bundle underlying the local $L_\infty$ algebra by shifting this down $L = E[1]$. 
The $(-1)$-shifted symplectic structure on $E$ transports to a $(-3)$-shifted on on $L$. 
The $L_\infty$ structure maps for $L$ come from the Taylor components of the action functional $S$. 
The exterior derivative of $S$ is a section
\ben
\d S \in \cloc^*(\sL, \sL^![-1]),
\een
where on the right-hand side we have zero differential.
The Taylor components are of the form $(\d S)_n : \sL^{\tensor n} \to \sL^![-1]$. 
Using the shifted symplectic pairing we can identify these Taylor components with maps $(\d S)_n : \sL^{\tensor n} \to \sL[2]$. 
Thus, $\d S$ can be viewed as a section of $\cloc^*(\sL, \sL[2])$. 
This is precisely the space controlling deformations of $\sL$ as a local Lie algebra.
One checks immediately that the classical master equation is equivalent to the fact that $\d S$ is a derivation, hence it determines the structure of a local Lie algebra. 
The first Taylor component $\ell_1$ is precisely the operator $Q$ before, so ellipticity of $(\sE, Q)$ is equivalent to ellipticity of $(\sL, \ell_1)$. 
\end{proof}

%The functionals we consider are required to be continuous with a certain topology on the topological vector space $\sE$. 

\subsubsection{Moduli problems and Koszul duality}

\brian{add this}

\subsection{Quantum field theory}

We now introduce the notion of a {\em quantum field theory} in the BV formalism.
We follow the effective approach defined by Costello in \cite{CostelloRenormalization}.

\subsubsection{Regularization}

We have seen that part of the data of a classical field theory is that of a $(-1)$-shifted symplectic structure on the space of fields. 
If $E$ is the graded vector bundle underlying the theory, the symplectic form determined an isomorphism of bundles $E \cong E^![-1]$. 
We can represent the inclusion $\sE_c \hookrightarrow \Bar{\sE}$ via its integral kernel $K_0 \in \Bar{\sE} \tensor \Bar{\sE}^!$. 
Using the symplectic pairing this is further identified with an element 
\ben
K_0 \in \Bar{\sE} \tensor \Bar{\sE} [-1] .
\een
That is, $K_0$ is a degree one element in $\Bar{\sE} \tensor \Bar{\sE}$.
The na\"{i}ve BV Laplacian $\Delta = \Delta_{K_0}$ is ill-defined acting on functions on $\sE$, $\sO(\sE)$. 
The point of regularization is to find a replacement for this operator.

The first step in regularization is to find a replacement of $K_0$ as a smooth, i.e. non-distributional, section in the tensor product $\sE \tensor \sE$. 
This is a reasonable thing to ask, since by ellipticity we know that the inclusion $\sE \tensor \sE \hookrightarrow \Bar{\sE} \tensor \Bar{\sE}$ is a quasi-isomorphism. 
So, we can replace $K_0$ by such a smooth section up to homotopy.
We refer to this as a {\em regularization} of the kernel.

We use a systematic way of regularization using heat kernels, which can be found in \cite{CostelloRenormalization} or Chapter 8 of \cite{CG2}. 
First, we fix the following data, that of a {\em gauge fixing} operator.
This is an operator 
\ben
Q^{GF} : \sE \to \sE[-1]
\een 
of cohomological degree $-1$.
We require that $D = [Q,Q^{GF}]$ is a generalized Laplacian acting on sections $\sE$ in the sense of \cite{GetzlerDirac}, in addition to other conditions that can be found in Definition 5.4.0.5 in \cite{CG2}. ]

The utility of introducing the gauge fixing operator is that it allows us to introduce the operator $e^{-t D}$ which has a kernel that we denote $K_t \in \Bar{\sE} \tensor \Bar{\sE}$ for any $t \geq 0$. 
This kernel satisfies the usual conditions of a heat kernel:
\begin{enumerate}
\item $K_t$ satisfies the heat equation
\ben
\frac{\partial}{\partial t} K_t + D K_t = 0
\een
\item $K_0$ is the kernel for the identity operator as above.
\end{enumerate}
Moreover, when $t > 0$ the operator $e^{-t D}$ is {\em smoothing} so that $K_t \in \sE \tensor \sE \subset \Bar{\sE} \tensor \Bar{\sE}$. 

The point of introducing this heat kernel is that it provides a regularization of $K_0$.
Indeed, for any $\epsilon, L \geq 0$ introduce the {\em propagator} 
\ben
P_{\epsilon < L} = \int_{t = \epsilon}^L (Q^{GF} \tensor 1) K_t \d t \in \Bar{\sE} \tensor \Bar{\sE} .
\een 
Then, one immediately checks that
\ben
K_L - K_\epsilon = Q P_{\epsilon < L},
\een
so that $P_{\epsilon < L}$ is a homotopy between $K_L$ and $K_\epsilon$. 
In particular, $P_{0<L}$ provides a homotopy between the identity kernel $K_0$ and $K_L$. 

\begin{dfn}
The scale $L > 0$ BV Laplacian is the order two operator
\ben
\Delta_{L} = \partial_{K_L} : \sO(\sE) \to \sO(\sE)
\een
given by contraction with the kernel $K_L \in \sE \tensor \sE$. 
\end{dfn}

We have already mentioned that the bracket $\{-,-\}$ is not defined on the whole space of functionals.
The regularized BV operator allows us to define the following {\em scale $L$ bracket}:
\ben
\{I, J\}_L := \Delta_L(IJ) - \Delta_L(I) J - (-1)^{|I|} I \Delta_L(J) .
\een
For $L > 0$ this bracket is defined on all of $\sO(\sE)$, just as $\Delta_L$ is.

\subsubsection{Effective BV quantization}

Fix a free BV theory together with a gauge fixing operator.
This is the data of an elliptic complex $(\sE, Q)$ with a $(-1)$-shifted symplectic form $\omega$.
In addition, let $Q^{GF}$ be a gauge fixing operator so that the regularized heat kernels $K_L$ and propagators $P_{\epsilon < L}$ are defined. 

We introduce the formal variable $\hbar$ and consider $\hbar$-dependent functionals $\sO(\sE)[[\hbar]]$. 
Let $\sO^+(\sE)[[\hbar]] \subset \sO(\sE)[[\hbar]]$ be the subset of functionals that are at least cubic modulo $\hbar$.
We define a map
\ben
W(P_{\epsilon < L}, -) : \sO^+(\sE)[[\hbar]] \to \sO^+(\sE)[[\hbar]],
\een
{\em renormalization group flow}. 
Formally, $W(P_{\epsilon<L},I)$ is defined by the formula
\ben
e^{W(P_{\epsilon<L},I)/\hbar} = e^{\hbar \partial_{P_{\epsilon < L}}} e^{I / \hbar} .
\een
Concretely, $W(P_{\epsilon<L},I)$ can be written as a sum over graphs $\Gamma$
\ben
W(P_{\epsilon < L} , I) = \sum_{\Gamma} \frac{\hbar^{g(\Gamma)}}{|{\rm Aut}(\Gamma)|}W_\Gamma(P_{\epsilon<L},I),
\een
where $W_\Gamma(P_{\epsilon<L},I)$ is the weight of the graph $\Gamma$ whose edges are labeled by $P_{\epsilon < L}$ and vertices labeled by $I$.
This is our mathematical definition of the Feynman weight of the graph $\Gamma$, and the precise definition can be found in Chapter 2 of \cite{CostelloRenormalization}. 

In the BV formalism, as developed in \cite{CostelloRenormalization,CG1,CG2}, one has the following definition of a quantum field theory.

\begin{dfn}
A {\em quantum field theory} in the BV formalism consists of a free BV theory $(\sE, Q, \omega)$ and an effective family of functionals
\ben
\{I[L]\}_{L \in (0,\infty)} \subset \sO^+_{P,sm}(\sE)[[\hbar]]
\een
that satisfy:
\begin{enumerate}[(a)]
\item the exact renormalization group (RG) flow equation
\ben
I[L'] = W(P_{L<L'}, I[L]);
\een
\item the scale $L$ quantum master equation (QME) at every length scale $L$:
\ben
(Q + \hbar \Delta_L) e^{I[L]/\hbar} = 0.
\een
Equivalently,
\ben
Q I[L] + \hbar \Delta_L I[L] + \frac{1}{2} \{I[L], I[L]\}_L = 0 ;
\een
\item as $L \to 0$, the functional $S[L]$ has an asymptotic expansion that is local.
\end{enumerate}
\end{dfn}

The subspace $\sO_{P,sm}^+(\sE)[[\hbar]] \subset \sO(\sE)[[\hbar]]$ is of smooth and proper functionals that are at least cubic modulo $\hbar$. 
Smooth and properness is a technical condition that we will not delve into in this work, but refer to the original reference of Costello and Gwilliam.
The first condition ensures that the scale $L$ action functional $S[L]$ determines the functional at every other scale.
The second can be interpreted as saying that we have a proper path integral measure at scale $L$ 
(i.e., the QME can be seen as a definition of the measure).
The third condition implies that the effective action is a quantization of a classical field theory,
since a defining property of a classical theory is that its action functional is local.
(A full definition is available in Section 8.2 of \cite{CG2}.)

\begin{rmk}
The length scale is often associated with a choice of Riemannian metric on the underlying manifold,
but the formalism of \cite{CostelloRenormalization} keeps track of how the space of quantum BV theories depends upon such a choice 
(and other choices that might go into issues like renormalization).
Hence, when the choices should not be essential --- such as with a topological field theory --- one can typically show rigorously that different choices give equivalent answers.
The length scale is also connected with the use of heat kernels in \cite{CostelloRenormalization},
but one can work with more general parametrices (and hence more general notions of ``scale''),
as explained in Chapter 8 of \cite{CG2}.
We use a natural length scale in this section; 
when it becomes relevant, in the context of factorization algebras, one must switch to general parametrices.
\end{rmk}

The locality condition ensures that the limit $I^{cl} = \lim_{L \to 0} I[L] \mod \hbar$ exists and is a local functional.
The QME modulo $\hbar$ implies that $I^{cl}$ satisfying the CME, so that $(\sE, Q, \omega, I^{cl})$ defines a classical theory in the BV formalism.

\subsubsection{Deformation theory for quantizations}

There is a well-established deformation theory for studying quantizations of a fixed classical field theory.
If $(\sE, Q, \omega, I)$ is a fixed classical theory, one would like to study the problem of finding quantizations which modulo $\hbar$ are equal to this classical theory. 

According to the definition of a QFT there are two main steps. 

\begin{enumerate}
\item Find an effective family $\{I[L]\}$ which, modulo $\hbar$, agrees with the classical theory $I$, and satisfies the RG flow equation. 
The main result of \cite{CostelloRenormalization} is that this step always has a solution. 
Naively, the proposed family is of the form $I[L] = W(P_{0<L}, I)$, but since $P_{0<L}$ is distributional this functional may not be well-defined. (This is the problem of UV divergence in QFT)
The key fact is that there exists a family of counterterms $I^{CT}(\epsilon) \in \sO(\sE)[[\hbar]]$ such that the limit
\ben
I[L] = \lim_{\epsilon \to 0} W(P_{\epsilon < L}, I - I^{CT}(\epsilon))
\een
does exist.
Moreover, it automatically satisfies the RG flow equation.
\item Once we have the effective family $\{I[L]\}$, the remaining condition to defining a QFT is the quantum master equation.
In general this equation is not satisfied, and there may in fact be obstructions to having a solution.
\end{enumerate}

For holomorphic theories we will study both problems above.
We will show that the analysis involved in finding counterterms for holomorphic theories is extremely well-behaved. 
In fact, the counterterms for holomorphic theories on $\CC^d$ are all zero.
Using this, we will show how solving the QME for holomorphic theories can be done a systematic way.

To study the problem of solving the quantum master equation in general, we work order by order in the formal parameter $\hbar$.
Suppose that $I[L]$ is defined modulo $\hbar^{n+2}$ and sastisfies the QME modulo $\hbar^{n+1}$.
The obstruction to satisfying the QME at scale $L$ modulo $\hbar^{n+2}$ is the functional
\ben
\Theta_{n+1} [L] = \hbar^{-n-1} (Q I[L] + \frac{1}{2}\{I[L], I[L]\}_L + \hbar \Delta_L) .
\een
The obstruction $\Theta[L]$ satisfies the classical master equation and hence the limit $\Theta_{n+1} = \lim_{L \to 0} \Theta[L]$ is a local functional and is closed for the differential $Q + \{I, -\}$. 
It is thus a closed element of degree one of the {\em deformation complex}
\ben
\Def_{\sE} = \left(\oloc(\sE), Q + \{I,-\}\right) .
\een
If $\Theta_{n+1}$ is cohomologically trivial in $H^1(\Def_{\sE})$, the space of possible lifts of $\{I[L]\}$ to a solution of the QME modulo $\hbar^{n+2}$ is a torsor for $H^0(\Def_{\sE})$. 

%\subsection{The quantum master equation}

\section{Holomorphic field theories} \label{sec: hol theory}

The goal of this section is to define the notion of a holomorphic field theory. 
This is a variant of Costello's definition of a BV theory, see the previous section, and we will take for granted that the reader is familiar with the general format.
In summary, we modify the definition of a theory by inserting the word ``holomorphic" in front of most objects (bundles, differential operators, etc..).
By applying the Dolbeault complex in appropriate locations, we will recover Costello's definition of a theory, but with a holomorphic flavor, see Table \ref{table: holtoBV}. 

There are many references in the physics literature to codify the concept of a holomorphic field theory.
See, most closely related to our approach, special cases of this in the work of Nekrasov and collaborators in \cite{NekThesis, NekChiral, NekCFT}. 
We will discuss in more detail the relationship of our analysis of holomorphic theories to this work in Chapters \ref{chap: holsig, chap: symmetries}. 

\subsection{The definition of a holomorphic theory}

We give a general definition of a classical holomorphic theory on a general complex manifold $X$ of complex dimension $d$.
We start with the definition of a {\em free} holomorphic field theory. 
After that we will go on to define what an interacting holomorphic theory is.

\subsubsection{Free holomorphic theories}

The fields of any theory are always expressed as sections of some $\ZZ$-graded vector bundle.
Here, the $\ZZ$-grading is the cohomological, or BRST, grading of the theory.
For a holomorphic theory we take this graded vector bundle to be holomorphic.  
By a {\em holomorphic} $\ZZ$-graded vector bundle we mean a $\ZZ$-graded vector bundle $
V = \oplus_i V^i$ such that each graded piece $V^i$ is a holomorphic vector bundle. 
Thus, the data we start with is the following:

\begin{itemize}
\item[(1)] a $\ZZ$-graded holomorphic vector bundle $V^* = \oplus_i V^i [-i]$, so that the finite dimensional holomorphic vector bundle $V^i$ is in cohomological degree $i$. 
\end{itemize}

A free classical theory is made up of a space of fields as above together with the data of a linearized BRST differential $Q^{BRST}$ and a symplectic pairing. 
Ordinarily, the BRST operator is a differential operator on the vector bundle defining the fields. 
For the class of theories we are considering, we want this operator to be holomorphic. 

If $E$ and $F$ are two holomorphic vector bundles on $X$, we can speak of holomorphic differential operators between $E$ and $F$. 
First, note that the Hom-bundle ${\rm Hom}(E,F)$ inherits a natural holomorphic structure. 
By definition, a holomorphic differential operator of order $m$ is a linear map
\ben
D : \Gamma^{hol}(X ; E) \to \Gamma^{hol}(X ; F)
\een
such that, with respect to a holomorphic coordinate chart $\{z_i\}$ on $X$, $D$ can be written as
\be\label{local holomorphic}
D|_{\{z_i\}} = \sum_{|I| \leq m} a_I (z) \frac{\partial^{|I|}}{\partial z_I}
\ee
where $a_I(z)$ is a local holomorphic section of ${\rm Hom}(E,F)$.
Here, the sum is over all multi-indices $I = (i_1,\ldots, i_d)$ and 
\ben
\frac{\partial^{|I|}}{\partial z_I} := \prod_{k=1}^d \frac{\partial^{i_k}}{\partial z_k^{i_k}} . 
\een 
The length is defined by $|I| := i_1 + \cdots + i_d$. 

\begin{eg}
The most basic example of a holomorphic differential operator is the $\partial$ operator for the trivial vector bundle. 
For each $1 \leq \ell \leq d = \dim_\CC(X)$, it is a holomorphic differential operator from $E = \wedge^\ell T^{1,0*}X$ to $F = \wedge^{\ell+1} T^{1,0*}X$ which on sections is
\ben
\partial : \Omega^{\ell, hol}(X) \to \Omega^{\ell+1, hol}(X) .
\een
Locally, of course, it has the form
\ben
\partial = \sum_{i = 1}^{d} (\d z_i \wedge (-)) \frac{\partial}{\partial z_i},
\een
where $\d z_i \wedge (-)$ is the vector bundle homomorphism $\wedge^\ell T^{1,0*}X \to \wedge^{\ell+1} T^{1,0*}X$ sending $\alpha \mapsto \d z_i \wedge \alpha$. 
\end{eg}

The next piece of data we fix is:
\begin{itemize}
\item[(2)] a square zero holomorphic differential operator 
\ben
Q^{hol} : \sV \to \sV[-1]
\een
of cohomological degree $+1$. 
Here $\sV$ denotes the holomorphic sections of $V$. 
\end{itemize}

Finally, to define a free theory we need the data of a symplectic pairing. 
For reasons to become clear in a moment, we must choose this pairing to have a strange cohomological degree. 
The last piece of data we fix is:
\begin{itemize}
\item[(3)] an invertible bundle map
\ben
(-,-)_V : V \tensor V \to K_X[d-1]
\een
Here, $K_X$ is the canonical bundle on $X$. 
\end{itemize}

The definition of the fields of an ordinary field theory are the {\em smooth} sections of the vector bundle $V$. 
In our situation this is a silly thing to do since we lose all of the data of the complex structure we used to define the objects above.
The more natural thing to do is to take the {\em holomorphic} sections of the vector bundle $V$. 
By construction, the operator $Q^{hol}$ and the pairing $(-,-)_V$ are defined on holomorphic sections, so on the surface this seems reasonable.
\brian{what should I say the problem is with doing things in the analytic category?}

The solution to this problem is in the existence of a resolution for the holomorphic sections of a vector bundle by smooth sections of bundles. 
Given any holomorphic vector bundle $E$ we can define its {\em Dolbeault complex} $\Omega^{0,*}(X , E)$ with its Dolbeault operator 
\ben
\dbar : \Omega^{0,p}(X, E) \to \Omega^{0,p+1}(X, E) .
\een
Here, $\Omega^{0,p}(X, E)$ denotes smooth sections of the vector bundle $\Wedge^p T^{0,1*} X \tensor E$. 

We now take a graded holomorphic vector bundle $V$ as above, equipped with the differential operator $Q^{hol}$. 
We can then define the totalization of the Dolbeault complex with the operator $Q^{hol}$:
\ben
\sE_V = \left(\Omega^{0,*}(X, E), \dbar + Q^{hol}\right) .
\een
The operator $\dbar + Q^{hol}$ will be the linearized BRST operator of our theory.
By assumption, we have $\dbar Q^{hol} = Q^{hol} \dbar^*$ so that $(\dbar + Q^{hol})^2 = 0$ and hence the fields still define a complex. 
The $(-1)$-shifted symplectic pairing is obtained by composition of the pairing $(-,-)_V$ with integration on $\Omega^{d,hol}_X$. 
The thing to observe here is that $(-,-)_V$ extends to the Dolbeault complex in a natural way: we simply combine the wedge product of forms with the pairing on $V$.
The $(-1)$-shifted pairing $\omega_V$ on $\sE$ is defined by the diagram
\ben
\xymatrix{
\sE_V \tensor \sE_V \ar[r]^-{(-,-)_V} \ar@{.>}[dr]_-{\omega_V} & \Omega^{0,*}(X , K_X) [d-1] \ar[d]^-{\int_X} \\
& \CC[-1] .
}
\een
We note that the top Dolbeault forms with values in the canonical bundle $K_X$ are precisely the top forms on the smooth manifold $X$, so integration makes sense. 

We arrive at the following definition. 

\begin{dfn/lem}\label{dfn hol free theory}
A {\em free holomorphic theory} on a complex manifold $X$ is the data $(V, Q^{hol}, (-,-)_V)$ as in (1), (2), (3) above such that $Q^{hol}$ is a square zero elliptic differential operator that is graded skew self-adjoint for the pairing $(-,-)_V$.
The triple $(\sE_V, Q_V = \dbar + Q^{hol}, \omega_V)$ defines a free BV theory in the usual sense.
\end{dfn/lem}

The usual prescription for writing down the associated action functional holds in this case.
If $\varphi \in \Omega^{0,*}(X , V)$ denotes a field the action is
\ben
S(\varphi) = \int_X \left(\varphi, (\dbar + Q^{hol}) \varphi \right)_V .
\een

The first example we explain is related to the subject of Chapter \ref{chap: holsig} and will serve as the fundamental example of a holomorphic theory. 

\begin{eg}\label{eg bg} {\em The free $\beta\gamma$ system}.
Suppose that 
\ben
V = \ul{\CC} \oplus K_X [d - 1] .
\een
Let $(-,-)_V$ be the pairing
\ben
(\ul{\CC} \oplus K_X) \tensor (\ul{\CC} \oplus K_X) \to K_X \oplus K_X \to K_X 
\een 
sending $(\lambda, \mu) \tensor (\lambda',\mu') \mapsto (\lambda \mu', \lambda'\mu) \mapsto \lambda\mu' + \lambda' \mu$.
In this example we set $Q^{hol} = 0$. 
One immediately checks that this is a holomorphic free theory as above.
The space of fields can be written as
\ben
\sE_V = \Omega^{0,*}(X) \oplus \Omega^{d,*}(X)[d - 1] .
\een 
We write $\gamma \in \Omega^{0,*}(X)$ for a field in the first component, and $\beta \in \Omega^{d,*}(X)[d - 1]$ for a field in the second component. 
The action functional reads
\ben
S(\gamma + \beta, \gamma'+\beta') = \int_{X} \beta \wedge \dbar \gamma' + \beta' \wedge \dbar \gamma .
\een 
When $d = 1$ this reduces to the ordinary chiral $\beta\gamma$ system from conformal field theory. 
The $\beta\gamma$ system is a bosonic version of the ghost $bc$ system that appears in the quantization of the bosonic string, see Chapter ?? of \cite{Polchinski1}.
We will discuss this higher dimensional version further in Section \ref{chap: holsig}.
For instance, we will see how this theory is the starting block for constructing general holomorphic $\sigma$-models. 
\end{eg}

Of course, there are many variants of the $\beta\gamma$ system that we can consider.
For instance, if $E$ is {\em any} holomorphic vector bundle on $X$ we can take 
\ben
V = E \oplus K_{\CC^d} \tensor E^\vee
\een
where $E^\vee$ is the linear dual bundle. 
The pairing is constructed as in the case above where we also use the evaluation pairing between $E$ and $E^\vee$.
In thise case, the fields are $\gamma \in \Omega^{0,*}(X, E)$ and $\beta \in \Omega^{d,*}(X, E^\vee)[d-1]$. 
The action functional is simply
\ben
S(\gamma, \beta) = \int {\rm ev}_E(\beta \wedge \dbar \gamma) .
\een
When $E$ is a tensor bundle of type $(r,s)$ this theory is a bosonic version of the $bc$ ghost system of spin $(r,s)$. 
For a general bundle $E$ we will refer to it as the $\beta\gamma$ system with coefficients in the bundle $E$. 

\begin{eg}
{\em The free chiral scalar}.
Another basic example is the free chiral scalar. 
This is a bit outside\brian{finish}
Let $X$ be a complex manifold with Hermitian metric $g$. 
Let $V = \ul{\CC}$, the trivial vector bundle. 
\brian{do this}
\end{eg}

\subsubsection{Interacting holomorphic theories}

\def\olochol{\sO_{\rm loc}^{hol}}

We now define what an interacting holomorphic theory is.
In general, an interacting field theory on a manifold $M$ is prescribed by the data of a free theory plus a local functional $I \in \oloc(\sE)$ that satisfies the classical master equation. 
Recall, the sheaf of local functionals on $\sE = \Gamma(E)$ is defined as the sheaf of Lagrangian densities
\ben
\oloc(\sE) = {\rm Dens}_M \tensor_{D_M} \sO_{red}(JE) .
\een
In the expression above $JE$ stands for the sheaf of smooth sections of the $\infty$-jet bundle ${\rm Jet}(E)$ which has the structure of a $D_X$-module.

If $V$ is a holomorphic vector bundle we can define the bundle of holomorphic $\infty$-jets ${\rm Jet}^{hol}(V)$, \cite{GriffithsGreen, Wong}. 
This is a pro-vector bundle that is holomorphic in a natural way.
The fibers of this infinite rank bundle ${\rm Jet}^{hol}(V)$ are isomorphic to 
\ben
{\rm Jet}^{hol}(V)|_w = V_w \tensor \CC[[z_1,\ldots,z_d]] 
\een
where $w \in X$ and where $\{z_i\}$ is the choice of a formal coordinate near $w$. 
We denote by $J^{hol} V$ denote the sheaf of holomorphic sections of this jet bundle.
The sheaf $J^{hol}V$ has the structure of a $D_X^{hol}$-module, that is, it is equipped with a holomorphic flat connection $\nabla^{hol}$.
This is completely analogous to the smooth case.
Locally, the holomorphic flat connection is of the form
\ben
\nabla^{hol} |_w = \sum_{i=1}^d \d w_i \left(\frac{\partial}{\partial w_i} - \frac{\partial}{\partial z_i}\right),
\een
where $\{w_i\}$ is the local coordinate on $X$ near $w$ and $z_i$ is the fiber coordinate labeling the holomorphic jet expansion.
Using holomorphic jets we can make a completely analogous definition in our setting.

Differential operators between holomorphic bundles are the same as bundle maps between the associated jet bundles. 
Suppose $V,W$ are holomorphic vector bundles with spaces of holomorphic sections given by $\sV,\sW$ respectively.
Then we can express polydifferential operators from $V$ to $W$ as
\ben
{\rm PolyDiff}^{hol}(\sV \times \cdots \times \sV, \sW) \cong {\rm Hom}({\rm Jet}^{\rm hol}(V) \tensor \ldots \tensor {\rm Jet}^{\rm hol}(V) , W) .
\een

\begin{dfn}\label{dfn hol lag}
Let $V$ be a vector bundle. 
The space of {\em holomorphic Lagrangian densities} on $V$ is
\ben
\sO_{red}^{hol}(V) = \prod_{n > 0} {\rm Hom} ({\rm Jet}^{hol}(V)^{\tensor n} , K_X)_{S_n} ,
\een
where Hom is taken in the category of holomorphic vector bundles.\footnote{The holomorphic vector bundle ${\rm Jet}^{hol}(V)$ is infinite dimensional and can be expressed as a pro-object in the category of holomorphic vector bundles. 
We require the bundle maps to be continuous with respect to the natural adic topology.}
Equivalently, a holomorphic Lagrangian density is of the form $F = \sum_n F_n \in \sO_{red}^{hol}(V)$ where each $F_n$ is a holomorphic polydifferential operator 
\ben
F_n : \sV \times \cdots \times \sV \to \Omega^{d,hol}_X .
\een
\end{dfn}

%\begin{dfn}
%Suppose $V$ is a graded holomorphic vector bundle.
%We define the sheaf of {\em holomorphic} local functionals on $V$ by
%\ben
%\olochol(V) = \Omega^{d,hol}_X \tensor_{D^{hol}_X} \sO_{red}(J^{hol}V) [d]
%\een
%\end{dfn}

Suppose that $V$ is part of the data of a free holomorphic theory $(V, Q^{hol},(-,-)_V)$.
The pairing $(-,-)_V$ endows the space of holomorphic Lagrangians with a sort of bracket.
Indeed, suppose $F, F' \in \sO_{red}^{hol}(V)$.
For simplicity suppose $F,F'$ are of homogenous symmetric degree $k,k'$ respectively.
Then, their product $F \tensor F'$ is a symmetric element in the homomorphism space
\ben
{\rm Hom}({\rm Jet}^{hol}(V)^{\tensor k +k'} , K_X \tensor K_X) .
\een
Now, the bundle map $(-,-)_V : V \tensor V \to K_X[d-1]$ is invertible, hence it determines an element $(-,-)^{-1}_V \in V \tensor K_X^*$, where $K_X^*$ is the dual bundle. 
We can then consider the composition
\ben
\xymatrix{
{\rm Hom}({\rm Jet}^{hol}(V)^{\tensor n} , K_X \tensor K_X) \ar[r]^-{(-,-)^{-1}_V} & {\rm Hom}({\rm Jet}^{hol}(V)^{\tensor k +k' -2} , K_X^* \tensor K_X \tensor K_X) \ar[r] & {\rm Hom}({\rm Jet}^{hol}(V)^{\tensor k +k' -2} , K_X).
}
\een
In the first arrow we have evaluated $(-,-)^{-1}_V$ on the first two factors and the second arrow is simply the evaluation pairing. 
We symmetrize this to obtain an element $\{F, F'\}^{hol} \in \Sym^{k+k'-2}({\rm Jet}^{hol}(V)^{\tensor n} , K_X)_{S_{k+k'-2}}$. 
In this way, we have produced a map
\ben
\{-,-\}^{hol} : \sO_{red}^{hol}(V) \times \sO_{red}^{hol}(V) \to \sO_{red}^{hol}(V) [d-1].
\een
Note that this bracket is of cohomological degree $-d+1$. 

\begin{rmk}
Note that the above still makes sense if $(-,-)_V$ is a polydifferential operator...\brian{should i say more?}
\end{rmk}

We can now state the definition of a classical holomorphic theory. 

\begin{dfn}
A {\em classical holomorphic theory} on a complex manifold $X$ is the data of a free holomorphic theory $(V, Q^{hol}, (-,-)_V)$ plus a functional
\ben
I^{hol} \in \prod_{n \geq 3} {\rm Hom} ({\rm Jet}^{hol}(V)^{\tensor n} , K_X)_{S_n} \subset \sO_{red}^{hol}(V) 
\een
of cohomological degree $d$ such that $Q^{hol} I^{hol} + \{I^{hol}, I^{hol}\}^{hol} = 0$.
\end{dfn} 

%\begin{dfn/lem}
%Let $(V, Q^{hol}, (-,-)_V, I^{hol})$ be the data of an interacting holomorphic theory. 
%Then $Q^{hol} + \{I^{hol},-\}$ equips $\olochol(V)$ with the structure of a sheaf of cochain complexes that we will denote
%\ben
%\Def^{hol}_{V} := \left(\olochol(V), Q^{hol} + \{I^{hol}, -\}^{hol}\right) .
%\een
%\end{dfn/lem}

Just as in the free case, we see that classical holomorphic theories define ordinary classical BV theories with interactions.
The underlying space of fields, as we have already seen is $\sE_V = \Omega^{0,*}(X , V)$. 
We will write $I^{hol} = \sum_k I^{hol}_k$ where $I^{hol}_k$ is symmetric degree $k$.
Now, we know that $I^{hol}_k$ is a $\Omega^{d,hol}_X$-valued functional of the form
\ben
I_k^{hol} : (\varphi_1,\ldots,\varphi_k) \mapsto D_1(\varphi_1)\cdots D_k(\varphi_k) \in \Omega^{d,hol}_X
\een
where $\varphi_i\in \sV = \Gamma^{hol}(X, V)$ and $D_i$ is a holomorphic differential operator on $\sV$.
Every holomorphic differential operator on the holomorphic vector bundle $V$ extends to a differential operator on its Dolbeault complex $\sE_V = \Omega^{0,*}(X, V)$. 
Thus, we can define the functional
\ben
I_k^{\Omega^{0,*}} : (\varphi_1,\ldots, \varphi_k) \mapsto \int D_1 (\varphi_1)\cdots D_k(\varphi_k)
\een
where, now $\varphi_i$ is a section of the Dolbeault complex $\Omega^{0,*}(X , V)$. 
The symbol $\int$ reminds us that we are working modulo total derivatives, so that the above expression defines an element of $\oloc(\sE_V)$. 
This defines a linear map $\olochol(V) \to \oloc(\sE_V)$ that we denote $I^{hol} \mapsto I^{\Omega^{0,*}}$. 
Note that since $I^{hol}$ is cohomological degree $d$, the local functional $I^{\Omega^{0,*}}$ is degree zero.

\begin{lem} Every classical holomorphic theory $(V, Q^{hol},(-,-)_V, I^{hol})$ determines the structure of a classical BV theory.
The underlying free BV theory is given in Definition/Lemma \ref{dfn hol free theory} $(\sE_V, Q, \omega_V)$ and the interaction is $I^{\Omega^{0,*}}$. 
\end{lem}
\begin{proof}
We must show that $Q^{hol}I^{hol} + \frac{1}{2} \{I^{hol},I^{hol}\}^{hol} = 0$ implies the ordinary classical master equation for $I^{\Omega^{0,*}}$:
\ben
\dbar I^{\Omega^{0,*}} + Q^{hol}I^{\Omega^{0,*}} + \frac{1}{2} \{I^{\Omega^{0,*}},I^{\Omega^{0,*}}\} = 0 .
\een
Since $I^{\Omega^{0,*}}$ is defined using holomorphic differential operators, the first term vanishes.
The fact that $Q^{hol}I^{hol} + \frac{1}{2} \{I^{hol},I^{hol}\}^{hol} = 0$ implies $Q^{hol}I^{\Omega^{0,*}} + \frac{1}{2} \{I^{\Omega^{0,*}},I^{\Omega^{0,*}}\} = 0$ follows immediately from our definitions.
\end{proof}

Table \ref{table: holtoBV} is a useful summary showing how we are producing a BV theory from a holomorphic theory.

\begin{table}
\begin{tabular}{ |c|c|c| } 
 \hline
 Holomorphic theory & BV theory \\
 \hline \hline
Holomorphic bundle $V$ & Space of fields $\sE_V = \Omega^{0,*}(X, V)$  \\ 
Holomorphic differential operator $Q^{hol}$ & Linear BRST operator $\dbar + Q^{hol}$ \\ 
Non-degenerate pairing $(-,-)_V$ & $(-1)$-symplectic structure $\omega_{V}$ \\ 
Holomorphic Lagrangian $I^{hol}$ & Local functional $I^{\Omega^{0,*}} \in \oloc(\sE_V)$ \\ 
 \hline
\end{tabular}
\caption{From holomorphic to BV}
\label{table: holtoBV}
\end{table}



\begin{eg} {\em Holomorphic $BF$-theory}
Let $\fg$ be a Lie algebra and $X$ any complex manifold.
Consider the following holomorphic vector bundle on $X$:
\ben
V = \ul{\fg}_X \oplus K_X \tensor \fg^* [d-1] .
\een
The pairing $V \tensor V \to K_X[d-1]$ is similar to the pairing for the $\beta\gamma$ system, except we use the evaluation pairing $\<-.-\>_\fg$ between $\fg$ and its dual. 
In this example, $Q^{hol} = 0$.
Write $f \in \sO^{hol}_X$ and $\beta \in K_X$ and consider
\ben
I^{hol} (f_1 \tensor X_1, f_2 \tensor X_2, \beta \tensor X^\vee) = f_1f_2 \beta \<X^\vee, [X_1,X_2]\> + \cdots
\een
where the $\cdots$ means that we symmetrize the inputs.
This defines an element $I^{hol} \in \olochol(V)^+$ and the Jacobi identity ensures $\{I^{hol}, I^{hol}\}^{hol} = 0$. 
The fields of the corresponding BV theory are
\ben
\sE_V = \Omega^{0,*}(X, \fg) \oplus \Omega^{d,*}(X, \fg^*) [d-1] .
\een
The induced local functional $I^{\Omega^{0,*}}$ on $\sE_V$ is
\ben
I^{\Omega^{0,*}} (\alpha, \beta) = \int_X \<\beta, [\alpha,\alpha]\>_\fg .
\een
The total action is $S(\alpha,\beta) = \int \<\beta, \dbar \alpha\> + \<\beta,[\alpha,\alpha]\>_\fg$.
This is formally similar to $BF$ theory (see below) and for that reason we refer to it as {\em holomorphic} BF theory.
In \cite{johansen1}, or for a more mathematical treatment see \cite{CostelloYangian}, it is shown that this theory is a twist of $N=1$ supersymmetric Yang-Mills on $\RR^4$.
\end{eg}

\begin{eg} {\em Topological $BF$-theory}
\brian{do this}

In this case $Q^{hol} = \partial$. 
\end{eg}

When we construct a BV theory from a holomorphic theory $V \rightsquigarrow \sE_V$ it is natural to expect that deformations of the theory must come from holomorphic data.
In the special case that $Q^{hol} = 0$ we have the following result which relates the deformation complex of the classical theory $\sE_V$ to a sheaf built from holomorphic differential operators.

\begin{lem}
Suppose $(V, 0, (-,-)_V, I^{hol})$ is the data of a holomorphic theory with $Q^{hol} = 0$.
Let $(\sE_V, Q = \dbar, \omega_V, I)$ be the corresponding BV theory.
Then, there is a quasi-isomorphism of sheaves
\ben
\Def_{\sE_V}  \simeq \Omega^{d,hol}_X \tensor^{\LL}_{D_X^{hol}} \sO_{red}^{hol}(V)
\een
that is compatible with the brackets and $\{-,-\}$ and $\{-,-\}^{hol}$ on both sides.
\end{lem}

\begin{rmk}
Just as in the ordinary case we can formulate the data of a classical holomorphic theory in terms of sheaves of $L_\infty$ algebras. 
We will not do that here, but hope the idea of how to do so is clear.
\end{rmk}

%We start with the definition of a {\em holomorphic local Lie algebra} on a complex manifold. 
%Enhancing this to a definition of a holomorphic classical BV theory will be immediate. 
%
%Fix a complex manifold $X$, of complex dimension $d$.
%The starting data of a local Lie algebra on $X$ is a $\ZZ$-graded vector bundle $L$ on $X$. 
%
%\begin{dfn} 
%A {\em holomorphic} local Lie algebra on a complex manifold $X$ is the data
%\begin{enumerate}
%\item A $\ZZ$-graded holomorphic vector bundle $L$ on $X$;
%\item For each $n \geq 1$ a holomorphic differential operator 
%\ben
%\ell_n : L^{\boxtimes n} \to L;
%\een
%\end{enumerate}
%such that $\{\ell_n\}$ endow the sheaf of holomorphic sections $\Gamma_{hol}(L)$ with the structure of a sheaf of $L_\infty$ algebras. 
%\end{dfn}
%
%A holomorphic local Lie algebra is {\em not} a local Lie algebra in the usual sense. 
%The problem is that, in the definition, we have utilized the space of {\em holomorphic sections}, instead of the space of all smooth sections. 
%There is a natural resolution that will allow us to turn every holomorphic local Lie algebra into an ordinary local Lie algebra. 
%Of course, given a holomorphic vector bundle $L$ we can consider its Dolbeault complex ${\rm Dol}(L) = \Omega^{0,*}(X , L)$.
%For a holomorphic local Lie algebra $L$ this is really the object we want to consider.
%
%Applying the Dolbeault complex functor to the $L_\infty$ maps $\ell_n : L^{\boxtimes n} \to L$ we obtain maps
%\ben
%{\rm Dol}(\ell_n) : \Omega^{0,*}(X , L)^{\tensor n} \to \Omega^{0,*}(X , L) . 
%\een 
%
%\begin{lem} Suppose $L$ is a holomorphic local Lie algebra. 
%Then, ${\rm Dol}(L) = \Omega^{0,*}(X , L)$ is endowed with the structure of a local Lie algebra with structure maps given by $\ell_1 = \dbar + {\rm Dol}(\ell^L_1)$, and $\ell_n = {\rm Dol}(\ell^L_n)\}$ for $n \geq 2$. 
%\end{lem}
%\begin{proof} 
%We need to check that for each $n \geq 1$ the map
%\ben 
%{\rm Dol}(\ell^L_n) : \Omega^{0,*}(X, L)^{\tensor n} = \Omega^{0,*}(X^{\times n} , L^{\boxtimes n}) \to \Omega^{0,*}(X , L)
%\een
%is compatible with the Dolbeault differential. 
%This is a local calculation, so it suffices to assume all operators $\ell_n^L$ are of the form (\ref{local holomorphic}). 
%Indeed, since each of the coefficients $a_I(z)$ are holomorphic we see that ${\rm Dol}(\ell_n^L)$ is compatible with $\dbar$. 
%\end{proof}

\subsection{Holomorphically translation invariant theories}

When working on affine space $\RR^n$ one can ask for a theory to be invariant with respect to translations. 
In this section, we consider the affine manifold $\CC^d = \RR^{2d}$ equipped with its standard complex structure and define what a {\em holomorphically translation invariant} theory is on it. 
It will be a very special case of a general holomorphic theory as defined above. 

%The definition we give of a holomorphically translation invariant %theory is slightly different than the structure of the last %section. 

Let $V$ be a holomorphic vector bundle on $\CC^n$ and suppose we fix an identification of bundles 
\ben
V \cong \CC^d \times V_0
\een
where $V_0$ is the fiber of $V$ at $0 \in \CC^d$. 
We want to consider a classical theory with space of fields given by $\Omega^{0,*}(\CC^d, V) \cong \Omega^{0,*}(\CC^d) \tensor_\CC V_0$. 
Moreover, we want this theory to be invariant with respect to the group of translations on $\CC^d$. 
Per usual, it is best to work with the corresponding Lie algebra of translations. 
Using the complex structure, we choose a presentation for the Lie algebra of all translations given by
\ben
\CC^{2d} \cong {\rm span}_\CC \left\{\frac{\partial}{\partial z_i}, \frac{\partial}{\partial \zbar_i}\right\}_{1 \leq i \leq d}.
\een

To define a theory, we need to fix a non-degenerate pairing on $V$.
Moreover, we want this to be translation invariant. 
So, suppose
\be\label{pairing 1}
(-,-)_V : V \tensor V \to K_{\CC^d} [d-1]
\ee
is a skew-symmetric bundle map that is equivariant for the Lie algebra of translations. 
The shift is so that the resulting pairing on the Dolbeault complex is of the appropriate degree.
Here, equivariance means that for sections $v,v'$ we have
\ben
(\frac{\partial}{\partial z_i} v, v')_V = L_{\partial_{z_i}} (v,v')_V
\een
where the right-hand side denotes the Lie derivative applied to $(v,v')_V \in \Omega^{d,hol}_{\CC^d}$. 
There is a similar relation for the anti-holomorphic derivatives. 
We obtain a $\CC$-valued pairing on $\Omega^{0,*}_c(\CC^d , V)$ via integration:
\ben
\int_{\CC^d} \circ (-,-)_V : \Omega^{0,*}_c (\CC^d , V) \tensor \Omega^{0,*}_c(\CC^d , V) \xto{\wedge \cdot (-,-)_V} \Omega^{d,*}(\CC^d) \xto{\int} \CC .
\een
The first arrow is the wedge product of forms combined with the pairing on $V$. 
The second arrow is only nonzero on forms of type $\Omega^{d,d}$. 
Clearly, integration is translation invariant, so that the composition is as well. 

This pairing $\Omega^{0,*}(\CC^d , V)$ together with the differential $\dbar$ are enough to define a free theory. 
However, it is convenient to consider a slightly generalized version of this situation. 
We want to allow deformations of the differential $\dbar$ on Dolbeault forms of the form
\ben
Q = \dbar + Q^{hol}
\een
where $Q^{hol}$ is a holomorphic differential operator of the form
\be\label{hol operator}
Q^{hol} = \sum_I \frac{\partial}{\partial z^I} \mu_I
\ee
where $I$ is some multi-index and $\mu_I : V \to V$ is a linear map of cohomological degree $+1$. 
Note that we have automatically written $Q^{hol}$ in a way that it is translation invariant.
Of course, for this differential to define a free theory there needs to be some compatibility with the pairing on $V$. 

We can summarize this in the following definition, which should be viewed as a slight modification of a free theory to this translation invariant holomorphic setting. 

\begin{dfn} A {\em holomorphically translation invariant free BV theory} is the data of a holomorphic vector bundle $V$ together with
\begin{enumerate}
\item an identification $V \cong \CC^d \times V_0$;
\item a translation invariant skew-symmetric pairing  $(-,-)_V$ as in (\ref{pairing 1});
\item a holomorphic differential operator $Q^{hol}$ as in (\ref{hol operator});
\end{enumerate}
such that the following conditions hold
\begin{enumerate}
\item the induced $\CC$-valued pairing $\int \circ (-,-)_V$ is non-degenerate;
\item the operator $Q^{hol}$ satisfies $(\dbar + Q^{hol})^2 = 0$ and is skew self-adjoint for the pairing:
\ben
\int (Q^{hol} v, v')_V = \pm \int (v, Q^{hol} v').
\een
\end{enumerate}
\end{dfn}

The first condition is required so that we obtain an actual $(-1)$-shifted symplectic structure on $\Omega^{0,*}(\CC^d, V)$. 
The second condition implies that the derivation $Q = \dbar + Q^{hol}$ defines a cochain complex
\ben
\sE_V = \left(\Omega^{0,*}(\CC^d, V), \dbar + Q^{hol}\right),
\een
and that $Q$ is skew self-adjoint for the symplectic structure. 
Thus, in particular, $\sE_V$ together with the pairing define a free BV theory in the ordinary sense. 
In the usual way, we obtain the action functional via
\ben
S(\varphi) = \int (\varphi, (\dbar + Q^{hol}) \varphi)_V .
\een 

Before going further, we will give a familiar example from the last section.

\begin{eg}\label{eg bg affine} {\em The free $\beta\gamma$ system on $\CC^d$}.
Consider the $\beta\gamma$ system with coefficients in any holomorphic vector bundle from Example \ref{eg bg} (and the remarks after it) specialized to the manifold $X = \CC^d$.
One immediately checks that this is a holomorphically translation invariant free theory.
%where the operators $\eta_i = \frac{\partial}{\partial (\d z_i)}$ act in the natural way.
\end{eg}


\subsubsection{Translation invariant interactions}

Let's fix a general free holomorphically translation invariant theory as above.
We now define what a holomorphically translation invariant interacting theory is.
Recall, translations span a $2d$-dimensional abelian Lie algebra $\CC^{2d} = \CC\left\{\frac{\partial}{\partial z_i}, \frac{\partial}{\partial \zbar_i}\right\}$. 
The first condition that an interaction be holomorphically translation invariant is that it be translation invariant, so invariant for this Lie algebra.

Let $\Bar{\eta}_i$ denote the operator on Dolbeault forms given by contraction with the antiholomorphic vector field $\frac{\partial}{\partial \zbar_i}$. 
Note that $\eta_i$ acts on the Dolbeault complex on $\CC^d$ with values in any vector bundle.
In particular it acts on the fields of a free holomorphically translation invariant theory as above, in addition to functionals on fields.

Of course, to be holomorphically translation invariant, it is necessary to look at local functionals that are translation invariant.
The extra bit of holomorphicity we take into account on local functionals $I \in \oloc(\sE_V)$ is the following.

\begin{dfn}
A {\em holomorphically translation invariant} local functional is a translation invariant local functional $I \in \oloc(\sE_V)^{\CC^{2d}}$ such that $\eta_i I = 0$ for all $1 \leq i \leq d$. 
\end{dfn}

There is a succinct way of expressing holomorphic translation invariance as the Lie algebra invariants of a certain super Lie algebra.
Let the abelian $d$-dimensional Lie alegebra spanned by the odd elements $\{\eta_i\}$ be denoted by $\CC^d[1]$.
We want to consider deformations that are invariant for the action by the total {\em dg} Lie algebra $\CC^{2d|d} = \CC^{2d} \oplus \CC^d[1]$.
The differential sends $\eta_i \mapsto \frac{\partial}{\partial \zbar_i}$.
The space of holomorphically translation invariant local functionals are denoted by $\oloc(\sE_V)^{\CC^{2d|d}}$.
The enveloping algebra of $\CC^{2d|d}$ is of the form
\ben
U(\CC^{2d|d}) = \CC \left[\frac{\partial}{\partial z_i},  \frac{\partial}{\partial \zbar_i}, \eta_i \right]
\een
with differential induced from that in $\CC^{2d|d}$. 
Note that this algebra is quasi-isomorphic to the algebra of constant coefficient holomorphic differential operators $\CC[\partial / \partial z_i] \xto{\simeq} U(\CC^{2d|d})$. 

Any translation invariant local functional is a sum of functionals of the form
\ben
\varphi \mapsto \int_{\CC^d} F(D_1\varphi, \ldots, D_k \varphi)
\een
where $F : V^{\tensor k} \to \CC \cdot \d^d z$ is a linear map and each $D_\alpha$ is an operator in the space 
\ben
\CC \left[\d \zbar_i, \frac{\partial}{\partial z_i}, \frac{\partial}{\partial \zbar_i}, \eta_i \right] .
\een
The condition $\eta_i I = 0$ means that none of the $D_i$'s have any $\d \zbar_j$-dependence. 
Using this description we can exhibit the space of holomorphically translation functionals as follows.
Note that if $E$ is any vector bundle on $\CC^d$ we can consider the fiber at zero of its jet bundle that we denote $J_0 E$. 

\begin{lem}\label{lem: hol trans local}
Let $V$ be a holomorphic vector bundle on $\CC^d$ and denote $\sE_V = \Omega^{0,*}(X, V)$. 
Then
\ben
\oloc(\sE_V)^{\CC^{2d|d}} \cong \CC \cdot \d^d z \tensor_{U(\CC^{2d|d})} \sO_{red} (J_0 E_V)
\een
where $E_V$ is the vector bundle on $\CC^d$ such that $\sE_V = \Gamma(E_V)$.
\end{lem}

This description of holomorphically translation invariant local functionals allows us to give a convenient description of deformations of holomorphically translation invariant theories. 
Suppose $(V,Q^{hol},(-,-)_V, I)$ be the data of an interacting holomorphically translation invariant theory on $\CC^d$.
We have already encountered the space of local functionals $\oloc(\sE_V)$ and the deformation complex of the interacting BV theory is
\ben
\Def_{\sE_V} = \left(\oloc(\sE_V), \dbar + Q^{hol} + \{I,-\}\right) .
\een
We'd like to characterize deformations that preserve holomorphically translation invariance. 

Recall that in the holomorphic case there is the holomorphic jet bundle $J^{hol}V$.
The fiber at zero of this jet bundle may be identified as $J^{hol}_0 V = V_0 [[z_1,\ldots,z_d]]$ where the $z_i$'s denote the formal jet coordinate. 

\begin{cor}
Suppose that $Q^{hol} = 0$.
Then, there is a quasi-isomorphism
\ben
\left(\Def_{\sE_V}\right)^{\CC^{2d|d}} \simeq \CC \cdot \d^d z \tensor^{\LL}_{\CC[\partial_{z_1}, \ldots, \partial_{z_d}]} \sO_{red}(V_0[[z_1,\ldots,z_d]])[d].
\een
Equipped with differential $\{I^{hol},-\}$ where $I^{hol}$ only depends on holomorphic differential operators.
Here, $\partial_{z_i} = \frac{\partial}{\partial z_i}$ and $\CC \cdot \d^d z$ denotes the trivial right $\CC[\partial_{z_i}]$-module. 
\end{cor}

The local functional $I$ defining the classical holomorphic theory endows $J^{hol}V[-1]$ the structure of a $L_\infty$ algebra in $D_{\CC^d}$-modules. 
Repackaging the statement using Lie algebraic data we can rewrite the equivalence in the lemma as
\ben
\left(\Def_{\sE_V}\right)^{\CC^{2d|d}}\simeq \CC \cdot \d^d z \tensor^{\LL}_{\CC[\partial_{z_1}, \ldots, \partial_{z_d}]} \cred^*\left(V_0[[z]][-1])\right) [d].
\een

\begin{proof}

By Lemma \ref{lem: hol trans local} we have an expression for the holomorphically translation local functionals
\ben
\left(\Def_{\sE_V}\right)^{\CC^{2d|d}} = \left(\CC \cdot \d^d z \tensor_{U(\CC^{2d|d})} \sO_{red} (J_0 E_V)[d] , \dbar + \{I,-\}\right) .
\een
Since $\sO_{red}(J_0 E_V)$ is flat as a $U(\CC^{2d|d})$-module, it follows that we can replaces the tensor product by the derived tensor product $\tensor^{\LL}$ up to quasi-isomorphism so that
\ben
\left(\Def_{\sE_V}\right)^{\CC^{2d|d}} \simeq \left(\CC \cdot \d^d z \tensor^{\LL}_{U(\CC^{2d|d})} \sO_{red} (J_0 E_V) [d] , \dbar + \{I,-\}\right) .
\een
Consider the complex $\left(\sO_{red}(J_0 E_V) , \dbar + \{I,-\}\right)$.
This complex is graded by symmetric degree, and the associated spectral sequence has first page the associated graded of $\sO_{red}(J_0 E_V)$ equipped with the $\dbar$ differential.
Moreover, at the $E_1$-page, we have the quasi-isomorphism
\ben
\left(\sO(J_0 E_V), \dbar\right)= \left(\sO_{red}(V_0 [[z_i, \zbar_i]][\d \zbar_i]), \dbar\right) \simeq \sO_{red}(V_0[[z_i]]) .
\een

Finally, we have already remarked that there is a quasi-isomorphism of algebras $U(\CC^{2d|d}) \simeq U(\CC^d)$ where the right-hand site is generated by the constant holomorphic vector fields. 
The proof of the claim follows. 

\end{proof}

%\begin{proof}
%By Lemma 6.7.1 we know that we can identify the translation invariant deformations as:
%\ben
%\left(\Def_{\sE_V}\right)^{\CC^{2d}} \simeq \CC \cdot \d^d z \d^d \zbar \tensor^{\LL}_{\CC[\partial_{z_1}, \partial_{\zbar_i}]} \sO_{red}(J_0 \sE_V).
%\een
%Here $J_0 \sE_V$ is the fiber at zero of the bundle of jets of the Dolbeault complex.
%Hence, we have an identification
%\ben
%J_0 \sE_V = V_0[[z_1,\ldots,z_d, \zbar_1,\ldots,z_d]][d \zbar_1,\ldots, \d \zbar_d] .
%\een
%Now, consider the filtration on $V$ by cohomological degree, and consider the spectral sequence computing the cohomology of the translation invariant deformations.
%Since $Q^{hol} = 0$ by assumption, the differential on the first page is simply $\dbar$.
%By the formal Poincar\'{e} lemma for the Dolbeault operator, we see that the first page is quasi-isomorphic to
%\ben
%\CC \cdot \d^d z \d^d \zbar \tensor^{\LL}_{\CC[\partial_{z_1}, \partial_{\zbar_i}]} \sO_{red}(V_0[[z_1,\ldots,z_d]]) .
%\een
%
%Now, consider the resolution of the trivial $\CC[\partial_{\zbar_i}]$-module $\CC \cdot \d^d \zbar$ given by
%\ben
%\CC[\d \zbar_i] \tensor \CC[\partial_{\zbar_i}][d]
%\een
%where the differential sends $1 \mapsto \d \zbar_i \partial_{\zbar_i}$.  
%This is a formal version of the Spencer resolution.
%Taking the $\CC^d[1] = \CC\{\Bar{\eta}_1,\ldots,\Bar{\eta}_d\}$ invariants, we see that the deformation complex becomes
%\ben
%\CC \cdot \d^d z \tensor^{\LL}_{\CC[\partial_{z_i}, \partial_{\zbar_i}, \Bar{\eta}_i]} \sO_{red}(V_0[[z_1,\ldots,z_d]]) [d].
%\een
%The lemma follows once we observe that there is a quasi-isomorphism of algebras 
%\ben
%\CC[\partial_{z_i}] \xto{\simeq} \CC[\partial_{z_i}, \partial_{\zbar_i}, \Bar{\eta}_i].
%\een
%
%\end{proof}

%\subsubsection{Interacting holomorphic theories}
%
%It is convenient to introduce the following set of degree $-1$ derivations of $\Omega^{0,*}(\CC^d)$ given by
%\ben
%\Bar{\eta}_i := \frac{\partial}{\partial (\d \zbar_i)} .
%\een
%The right-hand side is sometimes written using the interior derivative notation $\iota_{\partial / \partial \zbar_i}$. 
%By a holomorphic version of ``Cartan's magic formula" these derivations satisfy the relation
%\ben
%L_{\frac{\partial}{\partial \zbar_i}} = \dbar \Bar{\eta}_i + \Bar{\eta}_i \dbar .
%\een
%In addition, they serve to define homotopies for the following holomorphic version of Poincar\'{e}'s lemma. 
%First, consider the algebraic case. 
%Let $\AA^d$ be the complex $d$-dimensional affine space with space of smooth algebraic functions $\sO^{alg, sm}(\AA^d) = \CC[z_1,\ldots, z_d, \zbar_1,\ldots,\zbar_d]$. 
%Then, we can build the algebraic Dolbeault complex 
%\ben
%\Omega^{0,*}_{alg} (\AA^d) = \CC[z_1,\ldots,z_d, \zbar_1,\ldots,\zbar_d][\d \zbar_1,\ldots,\d \zbar_d]
%\een
%where the $\d \zbar_i$'s are in cohomological degree $1$. 
%The Dolbeault differential $\dbar$ is define in the same way. 
%Note that the operators $\Bar{\eta}_i$ also make sense on $\Omega^{0,*}_{alg}(\AA^d)$. 
%
%\begin{lem} 
%The map 
%\ben
%\CC[z_1,\ldots, z_d] \hookrightarrow \Omega^{0,*}_{alg}(\AA^d)
%\een
%is a quasi-isomorphism.
%\end{lem}
%\begin{proof}
%Note that
%\ben
%\Omega^{0,*}_{alg}(\AA^d) \cong \CC[z_1,\ldots, z_d] \tensor_\CC \CC[\zbar_1,\ldots,\zbar_d][\d \zbar_1,\ldots,\d \zbar_d] .
%\een
%The right-hand term is one-dimensional, concentrated in degree zero by the ordinary Poincar\'{e} lemma. 
%An explicit homotopy for a \brian{...}
%\end{proof}
%
%The analogous result holds with $\AA^d$ replaced by the formal $d$-disk $\hD^d$. 
%
%For the general case...
%
%\begin{lem} \brian{find good citation. should I just state the general Stein resut?}
%The map
%\ben
%\sO^{hol} (\CC^d) \hookrightarrow \Omega^{0,*}(\CC^d)
%\een
%is a quasi-isomorphism.
%\end{lem}
%
%\brian{Include equivalence with certain structured local Lie algebras. Namely holomorphically translation invariant local Lie algebras.}
%
%Local Lie algebras \footnote{Local Lie algebras will mean $L_\infty$...} provide a convenient language to cast the data of an interacting classical field theory. 
%In \brian{ref} it is shown that the data of a local Lie algebra together with a non-degenerate pairing of degree $-3$ is equivalent to the data of a classical interacting BV theory. 
%It will be convenient for us to formulate, under this equivalence, a Lie theoretic interpretation of holomorphically translation invariant interacting BV theories. 
%First, we introduce the following definition. 
%Recall, data of a local Lie algebra is a $\ZZ$-graded vector bundle $L$ together with poly-differential operators 
%\ben
%\ell_n : L \tensor \cdots \tensor L \to L 
%\een 
%for $n \geq 1$, satisfying some conditions. 
%The sheaf of smooth sections of this bundle will be denoted by $\sL$, which inherits from the operators $\{\ell_n\}$ the structure of a sheaf of $L_\infty$ algebras. 
%We will often refer to the local Lie algebra simply by its sheaf of sections. 
%
%\begin{dfn}
%A translation invariant local Lie algebra on $\RR^n$ is a local Lie algebra $\sL$,
%together with an identification $L = \RR^n \times L_0$ such that for each $n$ the structure map
%\ben
%\ell_n : \RR^n \times (L_0 \tensor \cdots \tensor L_0) \to \RR^n \times L_0
%\een
%is compatible with translations. 
%\end{dfn}


%In this thesis we will often encounter theories that are holomorphically translation invariant.
%The first step in understanding their quantizations is to describe the deformations. 
%Of course, it is completely natural to look for deformations that are also translation invariant. 
%For holomorpically translation invariant theories, we can provide a general characterization of the deformation complex that we will be necessary at various stages in this work.


%Any local Lie algebra on a manifold endows the structure of an $L_\infty$ algebra on its fibers. 
%In particular, if $L$ the graded vector bundle associated to local Lie algebra on $\CC^d$, its fiber over $0$, $L_0$, is equipped with the structure of an $L_\infty$ algebra. 

%Suppose $\sL$ is a holomorphically translation invariant local Lie algebra on $\CC^d$ of the form $\Omega^{0,*}(\CC^d, L)$ where $L$ is a graded holomorphic vector bundle.
%In this situation, we are interested in studying the local cochains of $\sL$ that are translation invariant.
%We will use the following result over and over again throughout this work.

%\begin{lem} Let $\sL_0$ denote the fiber of $\sL$ over $0 \in \CC^d$. 
%Then, if $\ell_1 = \dbar$, there is an equivalence of $L_\infty$ algebras
%\ben
%L_0 \xto{\simeq} \sL_0
%\een 
%where $L_0$ is the fiber of the holomorphic bundle $L$ over $0 \in \CC^d$. 
%\end{lem}

%\begin{prop} Suppose $\sL$ is a holomorphically translation invariant local Lie algebra on $\CC^d$ such that $\ell_1 = \dbar$.
%Then, one has
%\ben
%\cloc^*(\sL)^{\CC^d} \simeq \CC \cdot \d^d z \tensor^{\LL}_{\CC[\frac{\partial}{\partial z_i}]} \cred^*(L_0 [[z_1,\ldots,z_d]]) [d] .
%\een
%\end{prop}

%For instance, if $L = \ul{\fg}$ is the constant bundle on $\CC^d$ where $\fg$ is an ordinary Lie (or $L_\infty$) algebra one has $L_0 = \fg$ so that
%\ben
%\cloc^*(\Omega^{0,*}(\CC^d, \fg))^{\CC^d} \simeq \CC \cdot \d^d z \tensor^{\LL}_{\CC[\frac{\partial}{\partial z_i}]} \cred^*(\fg [[z_1,\ldots,z_d]]) .
% \een

%\subsubsection{Holomorphic deformations on an arbitrary complex manifold}
%
%\brian{fix and move this to above}
%There is a more general result that holds on an arbitrary complex manifold. 
%Recall, that on a complex manifold $X$ we have introduced the notion of a holomorphic local Lie algebra $L$. 
%Let $\sL = {\rm Dol}(L) = \Omega^{0,*}(X , L)$ be its associated local Lie algebra. 
%Ordinarily, the local Lie algebra cohomology of a local Lie algebra is computed in terms of the Lie algebra cohomology of the associated jet bundle. 
%With holomorphicity and some mild assumption, we can, up to quasi-isomorphism, exhibit a smaller complex computing this local cohomology. 
%
%\begin{prop} 
%Suppose $L$ is a holomorphic local Lie algebra with $\ell_1 = 0$, and let $\sL = \Omega^{0,*}(X, L)$ be its associated local Lie algebra.
%There is a quasi-isomorphisms of sheaves of cochain complexes
%\ben
%\cloc^*(\sL) \simeq \Omega^{d,hol}_X \tensor^{\LL}_{D_X^{hol}} \clie^*(J \sL^{hol}) .
%\een
%\end{prop}
%\begin{proof}
%Recall, the definition of $\cloc^*(\sL)$ is given in terms of $D$-module data by
%\ben
%\cloc^*(\sL) = \Omega^{d,d}_{X} \tensor^{\LL}_{D_X} \clie^*(J \sL)
%\een
%where $J \sL$ denotes the $\infty$-jet bundle of $\Omega^{0,*}(X , L)$. 
%Of course, $J\sL$ is a bundle equipped with a natural flat connection, and hence the structure of a $D_X$-module. 
%The Chevalley-Eilenberg complex $\clie^*(J\sL)$ inherits this $D_X$-module structure.  
%
%On the other hand, if we view $L$ as a holomorphic vector bundle, it makes sense to look at the {\em holomorphic} jet bundle $J^{hol} L$. 
%This holomorphic vector bundle is equipped with a holomorphic flat connection, and hence is a module for the sheaf of holomorphic differential operators $D^{hol}_X$. 
%
%\begin{lem} Let $(J^{hol} L)^{C^\infty}$ be the $D_X^{hol}$-module $J^{hol} L$ viewed, via the forgetful functor, as a $D_X$-module. 
%Then, there is a quasi-isomorphism of dg $D_X$-modules $(J^{hol} L)^{C^\infty} \simeq J \Omega^{0,*}(X , L)$.
%Furthermore, this quasi-isomorphism is compatible with the $L_\infty$ structures, so that we obtain a quasi-isomorphism of dg $D_X$-modules $\clie^*(J^{hol} L)^{C^\infty} \simeq \clie^*(J \sL)$. 
%\end{lem}
%\begin{proof}
%For any holomorphic vector bundle $E$, the Dolbeault complex of $E$ is a resolution of the sheaf of holomorphic sections of $E$.
%Thus, there is an equivalence of sheaves on $X$
%\ben
%\Omega^{0,*}_X(E) \simeq \Gamma^{hol}_X(E) .
%\een 
%Thus, there is an equivalence of sheaves on $X$:
%\ben
%J^{hol} L \simeq J \Omega^{0,*}(X , L) .
%\een
%We need to see that this equivalence respects the $D_X$-module structure present on both sides....
%
%\end{proof}
%
%To finish the proof, we verify the following general lemma. 
%
%\begin{lem} Suppose $V$ is a $D_X^{hol}$-module, and let $V^{C^\infty}$ denote its underlying $D_X$-module. 
%Then, there is a quasi-isomorphism of sheaves of cochain complexes
%\ben
%\Omega^{d,hol}_X \tensor^\LL_{D_X^{hol}} V [d] \simeq \Omega^{d,d}_X \tensor_{D_X}^\LL V^{C^\infty} .
%\een 
%\end{lem}
%
%\end{proof}

\section{Renormalization of holomorphic theories}

In this section we study the renormalization of holomorphically translation invariant field theories on $\CC^d$ for any $d \geq 1$. 
We start with a classical interacting holomorphic theory on $\CC^d$ and consider one-loop homotopy RG flow from some finite scale $\epsilon$ to scale $L$.
That is, we consider the sum over graphs of genus zero and one where at each vertex we place the holomorphic interaction.
To obtain a prequantization of a classical theory one must make sense of the $\epsilon \to 0$ limit of this construction. 
In general, this involves introducing a family of counterterms.
Our main result is that for a holomorphic theory no such counterterms are required, and one obtains a well-defined $\epsilon \to 0$ limit. 

We can write the fields of a holomorphic theory on $\CC^d$ as
\ben
\sE_V = \left(\Omega^{0,*}(\CC^d, V), \dbar + Q^{hol}\right)
\een
where $V$ is a graded holomorphic vector bundle and $Q^{hol}$ is a holomorphic differential operator.

Since the theory is holomorphically translation invariant we have an identification $\Omega^{0,*}(\CC^d , V) \cong \Omega^{0,*}(\CC^d) \tensor_\CC V_0$ where $V_0$ is the fiber of $V$ over $0 \in \CC^d$.
Further, we can write the $(-1)$-shifted symplectic structure defining the classical BV theory in the form
\ben
\omega_V(\alpha \tensor v, \beta \tensor w) = (v,w)_{V_0} \int \d^d z (\alpha \wedge \beta)
\een
where $(-,-)_{V_0}$ is a degree $(d-1)$-shifted pairing on the finite dimensional vector space $V_0$. 

We will assume that the holomorphic Lagrangian $I^{hol}$ is also translation invariant and so defines an interaction of the form $I = \sum_k I_k$ where $I_k$ is symmetric degree $k$ and
\ben
I_k = \int I^{hol}_k(\varphi) = \int D_{k,1}(\varphi)\cdots D_{k,k} (\varphi) \d^d z 
\een
where each $D_{i,j}$ is a translation invariant holomorphic differential operator. 

\subsection{Holomorphic gauge fixing}

To begin the process of renormalization we must fix the data of a gauge fixing operator. 
Recall, a gauge fixing operator is an operator on fields
\ben
Q^{GF} : \sE_V \to \sE_V[-1]
\een
of cohomological degree $-1$ such that $[Q, Q^{GF}]$ is a generalized Laplacian on $\sE$ where $Q$ is the linearized BRST operator. 


For holomorphic theories there is a convenient choice for a gauge fixing operator. 
To construct it we fix the standard flat metric on $\CC^d$. 
Doing this, we let $\dbar^*$ be the adjoint of the operator $\dbar$.
Using the coordinates on $(z_1,\ldots, z_d) \in \CC^d$ we can write this operator as
\ben
\dbar^* = \sum_{i=1}^d \frac{\partial}{\partial (\d \zbar_i)} \frac{\partial}{\partial z_i} .
\een
Equivalently $\frac{\partial}{\partial (\d \zbar_i)}$ is equal to contraction with the anti-holomorphic vector field $\frac{\partial}{\partial \zbar_i}$. 
The operator $\dbar^*$ extends to the complex of fields via the formula
\ben
Q^{GF} = \dbar^* \tensor {\rm id}_V : \Omega^{0,*}(X , V) \to \Omega^{0,*-1}(X, V),
\een
We claim that this is a gauge fixing operator for our holomorphic theory.
Indeed, since $Q^{hol}$ is a translation invariant holomorphic differential operator we have
\ben
[\dbar + Q^{hol}, Q^{GF}] = [\dbar,\dbar^*] \tensor \id_{V} .
\een
The operator $[\dbar,\dbar^*]$ is simply the Dolbeault Laplacian on $\CC^d$, which is certainly a generalized Laplacian.
In coordinates it is
\ben
[\dbar,\dbar^*] = -\sum_{i=1}^d \frac{\partial}{\partial \zbar_i}\frac{\partial}{\partial z_i} 
\een

By definition, the scale $L>0$ heat kernel $K_L^V \in \sE_V(\CC^d) \tensor \sE_V(\CC^d)$ satisfies
\ben
\omega_V(K_L, \varphi) = e^{-L[Q,Q^{GF}] } \varphi
\een
for any field $\varphi \in \sE_V$.
Pick a basis $\{e_i\}$ of $V_0$ and let 
\ben
{\bf C}_{V_0} = \sum_{i,j} \omega_{ij} (e_i \tensor e_j) \in V_0 \tensor V_0
\een
be the quadratic Casimir.
Here, $(\omega_{ij})$ is the inverse matrix to the pairing $(-,-)_{V_0}$. 
We see that for the holomorphic theory we can write this regularized heat kernel as
\ben
K_{L}^V (z,w) = K^{an}_L(z,w) \cdot {\bf C}_{V_0} 
\een
where the analytic part is independent of $V$ and equal to
\ben
K_L^{an} (z,w) = \frac{1}{(4 \pi L)^d} e^{-|z-w|^2/ 4L} \prod_{i=1}^d (\d \zbar_i - \d \zbar_j)  \in \Omega^{0,*} (\CC^d) \tensor \Omega^{0,*} (\CC^d) \cong \Omega^{0,*} (\CC^d \times \CC^d) .
\een
The propagator is defined by
\ben
P_{\epsilon < L}^V(z,w) = \int_{t=\epsilon}^L \d t (Q^{GF} \tensor 1) K_{L}^V(z,w) .
\een
Since ${\bf C}_{V_0}$ is independent of the coordinate on $\CC$ this propagator is of the form $P_{\epsilon < L}^V(z,w) = P_{\epsilon < L}^{an}(z,w) \cdot {\bf C}_{V_0}$ where
\begin{align*}
P_{\epsilon < L}^{an}(z,w) & = \int_{t=\epsilon}^L \d t (\dbar^* \tensor 1) K_{L}^V(z,w) \\
& = \int_{t=\epsilon}^L \d t \frac{1}{(4 \pi t)^d} \sum_{j=1}^d \left(\frac{z_j - w_j}{4 t} \right)  e^{-|z-w|^2 / 4}  \prod_{i \ne j}^d (\d \zbar_i - \d \zbar_j) .
\end{align*}

Our goal in this section is to show that one-loop RG flow produces a prequantization modulo $\hbar^2$ that requires no counterterms. 
The one-loop RG flow from $\epsilon$ to $L$ is defined by the weight expansion
\ben
W(P_{\epsilon<L}^V , I) = \sum_{\Gamma} \frac{\hbar^{g(\Gamma)}}{|{\rm Aut}(\Gamma)|} W_\Gamma (P_{\epsilon<L}^V, I) 
\een
where the sum is over graphs of genus $\leq 1$ and $W_\Gamma$ is the weight associated to the graph $\Gamma$. 

For the genus zero graphs, or trees, we do not have any analytic difficulties to worry about. 
The propagator $P_{\epsilon<L}^V$ is smooth so long as $\epsilon,L > 0$ but when $\epsilon \to 0$ it inherits a singularity along the diagonal $z = w$.
But, if $\Gamma$ is a tree the weight $W_\Gamma(P_{0<L}^V, I)$ only involves multiplication of distributions with transverse singular support, so is well-defined.
Thus we have the following.

\begin{lem} 
If $\Gamma$ is a tree then $\lim_{\epsilon \to 0} W_{\Gamma}(P_{\epsilon < L}, I)$ exists.
\end{lem}

\subsection{One-loop weights}

The only possible divergences in the $\epsilon \to 0$ limit, then, must come from graphs of genus one. 
Every graph of genus one is a wheel with some trees protruding from the external edges of the tree.
Thus, we can write the weight of a genus one graph as a product of weights associated to trees times the weight associated to a wheel.
As we just saw, the weights associated to trees are automatically convergent in the $\epsilon \to 0$ limit, thus it suffices to focus on genus one graphs that are purely wheels with external edges as in Figure \brian{fig}. 

The definition of the weight of the wheel involves placing the propagator at each internal edge and the interaction $I$ at each vertex. 
The weights are evaluated by placing compactly supported fields $\varphi \in \sE_{V,c} = \Omega^{0,*}_c(\CC^d, V)$ at each of the external edges.
We will make two simplifications:
\begin{enumerate}
\item the only $\epsilon$ dependence appears in the analytic part of the propagator $P_{\epsilon<L}^{an}$, so we can forget about the combinatorial factor ${\bf C}_{V_0}$ and assume all external edges are labeled by compactly supported Dolbeault forms in $\Omega^{0,*}_c(\CC^d)$;
\item each vertex labeled by $I$ is a sum of interactions of the form
\ben
\int_{\CC^d} D_1(\varphi) \cdots D_k(\varphi) \d^d z
\een
where $D_i$ is a translation invariant differential operator. 
Some of the differential operators will hit the compactly supported Dolbeault forms placed on the external edges of the graph.
The remaining operators will hit the internal edges labeled by the propagators.
Since a holomorphic differential operator preserves the space of compactly supported Dolbeault forms that is independent of $\epsilon$, we replace each input by an arbitrary compactly supported Dolbeault form.
\end{enumerate}

Thus, for the $\epsilon \to 0$ behavior it suffices to look at weights of wheels with arbitrary compactly supported functions as inputs where each of the internal edges are labeled by some translation invariant holomorphic differential operator 
\ben
D = \sum_{n_1,\ldots n_d} \frac{\partial^{n_1}}{\partial z_{1}^{n_1}}\cdots \frac{\partial^{n_d}}{\partial z_{d}^{n_d}}
\een
applied to the propagator $P_{\epsilon<L}^{an}$.
This motivates the following definition. 

\begin{dfn}
Let $\epsilon , L > 0$. 
In addition, fix the following data.
\begin{enumerate}[(a)]
\item An integer $k \geq 1$ that will be the number of vertices of the graph.
\item For each $\alpha = 1, \ldots, k$ a sequence of integers
\ben
\vec{n}^\alpha = (n_1^\alpha, \ldots, n_d^{\alpha}) .
\een
We denote by $(\vec{n}) = (n_{i}^j)$ the corresponding $d \times k$ matrix of integers. 
\end{enumerate}
The analytic weight associated to the pair $(k, (\vec{n}))$ is the smooth distribution
\ben
W_{\epsilon < L}^{k, (n)} : C_c^\infty((\CC^d)^k) \to \CC,
\een
that sends a smooth compactly supported function $\Phi \in C_c^\infty((\CC^d)^k) = C_c^\infty(\CC^{dk})$ to
\be\label{weight1}
W_{\epsilon < L}^{k, (n)} (\Phi) = \int_{(z^1,\ldots, z^k) \in (\CC^d)^k} \prod_{\alpha=1}^k \d^d z^\alpha \Phi(z^1,\ldots,z^\alpha) \prod_{\alpha = 1}^k \left(\frac{\partial}{\partial z^i}\right)^{\vec{n}^\alpha} P_{\epsilon < L}^{an}(z^\alpha, z^{\alpha+1}) .
\ee
In the above expression, we use the convention that $z^{k+1} = z^1$. 
\end{dfn}

We will refer to the collection of data $(k, (\vec{n}))$ in the definition as {\em wheel data}.
The motivation for this is that the weight $W_{\epsilon < L}^{k, (n)}$ is the analytic part of the full weight $W_{\Gamma}(P^V_{\epsilon<L}, I)$ where $\Gamma$ is a wheel with $k$ vertices. 

We have reduced the proof of Lemma \ref{lem: hol renorm} to showing that the $\epsilon \to 0$ limit of the analytic weight $W_{\epsilon < L}^{k, (n)}(\Phi)$ exists for any choice of wheel data $(k, (\vec{n}))$.
To do this, there are two steps. 
First, we show a vanishing result that says when $k \geq d$ the  weights vanish for purely algebraic reasons. 
The second part is the most technical aspect of the chapter where we show that for $k > d$ the weights have nice asymptotic behavior as a function of $\epsilon$.

\begin{lem} Let $(k, (\vec{n}))$ be a pair of wheel data.
If the number of vertices $k$ satisfies $k \leq d$ then
\ben
W_{\epsilon < L}^{k, (n)}  = 0
\een
as a distribution on $\CC^{dk}$ for any $\epsilon,L > 0$. 
\end{lem}
\begin{proof}
In the integral expression for the weight (\ref{weight1}) there is the following factor involving the product over the edges of the propagators:
\be\label{productprops2}
\prod_{\alpha = 1}^k \left(\frac{\partial}{\partial z^i}\right)^{\vec{n}^\alpha} P_{\epsilon < L}^{an}(z^i, z^{i+1}) .
\ee
We will show that this expression is identically zero.
To simplify the expression we first make the following change of coordinates on $\CC^{dk}$:
\begin{align}
w^i & = z^{\alpha+1} - z^\alpha \;\;\; , \;\;\; 1\leq \alpha < k \label{coords1}\\
w^k & = z^k \label{coords2} .
\end{align}
Introduce the following operators
\ben
\eta^\alpha = \sum_{i=1}^{d} \wbar_i^\alpha \frac{\partial}{\partial (\d \wbar_i^\alpha)}
\een
acting on differential forms on $\CC^{dk}$.
The operator $\eta^\alpha$ lowers the anti-holomorphic Dolbuealt type by one : $\eta : (p,q) \to (p,q-1)$.
Equivalently, $\eta^\alpha$ is contraction with the anti-holomorphic Euler vector field $\wbar_i^\alpha \partial / \partial \wbar_i^\alpha$.

Once we do this, we see that the expression (\ref{productprops2}) can be written as 
\ben
\left(\left(\sum_{\alpha=1}^{k-1} \eta^\alpha \right) \prod_{i=1}^d \left(\sum_{\alpha = 1}^{k-1} \d \wbar_{i}^\alpha\right) \right) \prod_{\alpha=1}^{k-1}\left( \eta^\alpha \prod_{i=1}^d \d \wbar_i^\alpha\right) .
\een
Note that only the variables $\wbar_i^{\alpha}$ for $i=1,\ldots,d$ and $\alpha = 1,\ldots, k-1$ appear. 
Thus we can consider it as a form on $\CC^{d(k-1)}$.
As such a form it is of Dolbeault type $(0, (d-1) + (k-1)(d-1)) = (0, (d-1)k)$. 
If $k < d$ then clearly $(d-1)k > d(k-1)$ so the form has greater degree than the dimension of the manifold and hence it vanishes. 

The case left to consider is when $k = d$.
In this case, the expression in (\ref{productprops2}) can be written as
\be\label{productprops1}
\left(\left(\sum_{\alpha=1}^{d-1} \eta^\alpha \right) \prod_{i=1}^d \left(\sum_{\alpha = 1}^{d-1} \d \wbar_{i}^\alpha\right) \right) \prod_{\alpha=1}^{d-1}\left( \eta^\alpha \prod_{i=1}^d \d \wbar_i^\alpha\right) .
\ee
Again, since only the variables $\wbar_i^{\alpha}$ for $i=1,\ldots,d$ and $\alpha = 1,\ldots, d-1$ appear, we can view this as a differential form on $\CC^{d(d-1)}$. 
Furthermore, it is a form of type $(0, d(d-1))$. 
For any vector field $X$ on $\CC^{d(d-1)}$ the interior derivative $i_X$ is a graded derivation. 
Suppose $\omega_1,\omega_2$ are two $(0,*)$ forms on $\CC^{d(d-1)}$ such that the sum of their degrees is equal to $d^2$. 
Then, $\omega_1 \iota_X \omega_2$ is a top form for any vector field on $\CC^{d(d-1)}$.
Since $\omega_1 \omega_2 = 0$ for form type reasons, we conclude that $\omega_1 \iota_X \omega_2 = \pm (i_X \omega_1) \omega_2$ with sign depending on the dimension $d$. 
Applied to the vector field $\zbar_i^1\partial / \partial \wbar_i^1$ in (\ref{productprops1}) we see that the expression can be written (up to a sign) as 
\ben
\eta^1 \left(\sum_{\alpha=1}^{d-1} \eta^\alpha \prod_{i=1}^d \left(\sum_{\alpha = 1}^{d-1} \d \wbar_{i}^\alpha\right) \right) \left(\prod_{i=1}^d \d \wbar_i^1\right) \prod_{\alpha=2}^{d-1} \left( \eta^\alpha \prod_{i=1}^d \d \wbar_i^\alpha\right) .
\een
Repeating this, for $\alpha =2,\ldots,k-1$ we can write this expression (up to a sign) as
\ben
\left(\eta_{k-1} \cdots \eta_2 \eta _1 \sum_{\alpha=1}^{k-1} \eta^\alpha \prod_{i=1}^d \left(\sum_{\alpha = 1}^{k-1} \d \wbar_{i}^\alpha\right) \right) \prod_{\alpha=1}^{k-1} \prod_{i=1}^d \d \wbar_i^\alpha 
\een
The expression inside the parentheses is zero since each term in the sum over $\alpha$ involves a term like $\eta^\beta \eta^\beta = 0$. 
This completes the proof for $k=d$. 
\end{proof}

\begin{lem}
Let $(k, (\vec{n}))$ be a pair of wheel data such that $k > d$.
Then the $\epsilon \to 0$ limit of the analytic weight
\ben
\lim_{\epsilon \to 0} W_{\epsilon < L}^{k, (n)}
\een
exists as a distribution on $\CC{dk}$. 
\end{lem}

\begin{proof}

We will bound the absolute value of the weight in Equation (\ref{weight1}) and show that it has a well-defined $\epsilon\to 0$ limit.
First, consider the change of coordinates as in Equations (\ref{coords1}),(\ref{coords2}).
The weight applied to the compactly supported function $\Phi$ can be written as
\be\label{weight2}
\int_{w^k \in \CC^d} \d^{d} w^k \int_{(w_1,\ldots,w_{k-1}) \in (\CC^d)^{k-1}} \left(\prod_{\alpha=1}^{k-1} \d^{d} w^\alpha\right) \Phi(w^1,\ldots,w^k) \left(\prod_{\alpha=1}^{k-1} \left(\frac{\partial}{\partial w^\alpha}\right)^{\vec{n}^\alpha}P^{an}_{\epsilon < L} (w^\alpha) \right) \sum_{\alpha=1}^{k-1} \left(\frac{\partial}{\partial w^\alpha}\right)^{\vec{n}^k} P^{an} \left(\sum_{\alpha=1}^{k-1} w^\alpha\right) .
\ee
For $\alpha = 1,\ldots,k-1$ the notation $P^{an}_{\epsilon < L} (w^\alpha)$ makes sense since $P^{an}_{\epsilon<L}(z^\alpha,z^{\alpha+1})$ is only a function of $w^\alpha = z^{\alpha+1}-z^\alpha$.
Similarly $P^{an}_{\epsilon<L}(z^{k+1},z^1)$ is a function of 
\ben
z^k - z^1 = \sum_{\alpha=1}^{k-1} w^\alpha . 
\een
Expanding out the propagators the weight takes the form
\ben
\begin{array}{lll}
& \displaystyle \int_{w^k \in \CC^d} \d^{2d} w^k \int_{(w_1,\ldots,w_{k-1}) \in (\CC^d)^{k-1}} \left(\prod_{\alpha=1}^{k-1} \d^{2d} w^\alpha\right) \Phi(w^1,\ldots,w^k) \int_{(t_1,\ldots,t_k) \in [\epsilon,L]^k} \prod_{\alpha=1}^k \frac{\d t_\alpha}{(4 \pi t_\alpha)^d} \\
& \displaystyle \times \sum_{i_1,\ldots,i_{k-1} =1}^d \left(\frac{\wbar^1_{i_1}}{t_1} \frac{(\wbar^1)^{n^1}}{t^{|n^1|}}\right) \cdots \left(\frac{\wbar^{k-1}_{i_{k-1}}}{t_{k-1}}\frac{(\wbar^{k-1})^{n^{k-1}}}{t^{|n^{k-1}|}}\right) \left(\sum_{\alpha=1}^{k-1} \frac{\wbar^\alpha_{i_k}}{t_k} \cdot \frac{1}{t^{|n^k|}} \left(\sum_{\alpha=1}^{k-1} \wbar^\alpha\right)^{n^k}\right) \\
& \displaystyle \times \exp\left(- \sum_{\alpha=1}^{k-1} \frac{|w^{\alpha}|^2}{t_\alpha} - \frac{1}{t_k} \left|\sum_{\alpha=1}^{k-1} w^\alpha \right|^2\right)
\end{array}
\een
The notation used above warrants some explanation. 
Recall, for each $\alpha$ the vector of integers is defined as $n^\alpha = (n^{\alpha}_1,\ldots,n^{\alpha}_d)$. 
We use the notation
\ben
(\wbar^\alpha)^{n^\alpha} = \wbar^{n^\alpha_1}_1 \cdots \wbar^{n^\alpha_d}_d .
\een
Furthermore, $|n^\alpha| = n_1^\alpha + \cdots + n_d^\alpha$. 
Each factor of the form $\frac{\wbar^\alpha_{i_\alpha}}{t_\alpha}$ comes from the application of the operator $\frac{\partial}{\partial z_i}$ in $\dbar^*$ applied to the propagator. 
The factor $\frac{(\wbar^\alpha)^{n^\alpha}}{t^{|n^\alpha|}}$ comes from applying the operator $\left(\frac{\partial}{\partial w}\right)^{n^\alpha}$ to the propagator. 
Note that $\dbar^*$ commutes with any translation invariant holomorphic differential operator, so it doesn't matter which order we do this.

To bound this integral we will recognize each of the factors
\ben
\frac{\wbar^\alpha_{i_\alpha}}{t_\alpha} \frac{(\wbar^\alpha)^{n^\alpha}}{t^{|n^\alpha|}}
\een
as coming from the application of a certain holomorphic differential operator to the exponential in the last line.
We will then integrate by parts to obtain a simple Gaussian integral which will give us the necessary bounds in the $t$-variables. 
Let us denote this Gaussian factor by
\ben
E(w,t) := \exp\left(- \sum_{\alpha=1}^{k-1} \frac{|w^{\alpha}|^2}{t_\alpha} - \frac{1}{t_k} \left|\sum_{\alpha=1}^{k-1} w^\alpha \right|^2\right)
\een

For each $\alpha,i_\alpha$ introduce the $t=(t_1,\ldots,t_k)$-dependent holomorphic differential operator
\ben
D_{\alpha, i_\alpha}(t) := \left(\frac{\partial}{\partial w^\alpha_{i_\alpha}} - \sum_{\beta = 1}^{k-1} \frac{t_\beta}{t_1+\cdots + t_k} \frac{\partial}{\partial w_{i_\alpha}^{\beta}}\right)
\prod_{j=1}^d \left(\frac{\partial}{\partial w_j^\alpha} - \sum_{\beta =1}^{k-1} \frac{t_\beta}{t_1+\cdots + t_k} \frac{\partial}{\partial w_{j}^\beta}\right)
^{n_j^\alpha} .
\een

The following lemma is an immediate calculation
\begin{lem}\label{lem: diff applied E}
One has
\ben
D_{\alpha,i_\alpha} E(w,t) = \frac{\wbar^\alpha_{i_\alpha}}{t_\alpha} \frac{(\wbar^\alpha)^{n^\alpha}}{t^{|n^\alpha|}} E(w,t) . 
\een
\end{lem}

Note that all of the $D_{\alpha,i_{\alpha}}$ operators mutually commute. 
Thus, we can integrate by parts iteratively to obtain the following expression for the weight:
\ben
\begin{array}{lll}
& \displaystyle \pm \int_{w^k \in \CC^d} \d^{2d} w^k \int_{(w_1,\ldots,w_{k-1}) \in (\CC^d)^{k-1}}\left(\prod_{\alpha=1}^{k-1} \d^{2d} w^\alpha\right) \int_{(t_1,\ldots,t_k) \in [\epsilon,L]^k} \prod_{\alpha=1}^k \frac{\d t_\alpha}{(4 \pi t_\alpha)^d}  \\ 
& \displaystyle \times\left( \sum_{i_1,\ldots, i_d} D_{1, i_1} \cdots D_{k-1,i_{k-1}} \sum_{\alpha=1}^{k-1} D_{\alpha, i_k} \Phi(w^1,\ldots,w^k) \right) \times \exp\left(- \sum_{\alpha=1}^{k-1} \frac{|w^{\alpha}|^2}{t_\alpha} - \frac{1}{t_k} \left|\sum_{\alpha=1}^{k-1} w^\alpha \right|^2\right) .
\end{array}
\een
\brian{all the differential operators $D_{\alpha, i_\alpha}$ are uniformly bounded in $t$. To make these precise I should find what the uniform bound is.}

Thus, the absolute value of the weight is bounded by 
\be\label{weight bound1}
|W_{\epsilon < L}^{k, (n)}(\Phi)| \leq C \int_{w^k \in \CC^d} \d^{2d} w^k \int_{(w^1,\ldots,w^{k-1}} \prod_{\alpha=1}^{k-1} \d^{2d} w^\alpha \Psi(w^1,\ldots,w^{k-1},w^k) \int_{(t_1,\ldots,t_k) \in [\epsilon,L]^k} \d t_1 \ldots \d t_k \frac{1}{(4\pi)^{dk}} \frac{1}{t^d_1\cdots t^d_k} \times E(w,t)
\ee
To compute the right hand side we will perform a Gaussian integration with respect to the variables $(w^1,\ldots,w^{k-1})$. 
To this end, notice that the exponential can be written as
\ben
E(w,t) = \exp\left(-\frac{1}{4} M_{\alpha\beta} (w^\alpha, w^\beta)\right)
\een
where $(M_{\alpha\beta})$ is the $(k-1)\times (k-1)$ matrix given by
\ben
\begin{pmatrix}
a_1 & b & b & \cdots & b \\
b & a_2 & b & \cdots & b \\
b & b & a_3 & \cdots & b \\
\vdots & \vdots & \vdots &  \ddots & \vdots \\
b & b & b & \cdots & a_{k-1}
\end{pmatrix} 
\een
where $a_\alpha = t_\alpha^{-1} + t_k^{-1}$ and $b = t_k^{-1}$.
The pairing $(w^{\alpha}, w^{\beta})$ is the usual Hermitian pairing on $\CC^d$, $(w^{\alpha}, w^{\beta}) = \sum_i w^{\alpha}_i \wbar^\beta_i$.
After some straightforward linear algebra we find that 
\ben
\det(M_{\alpha\beta})^{-1} = \frac{t_1\cdots t_k}{t_1+\cdots+t_k} .
\een 
We now perform a Wick expansion for the Gaussian integral in the variables $(w^1,\ldots,w^{k-1})$.
For a reference similar to the notation used here see the Appendix of \cite{EWY}.
The inequality in (\ref{weight bound1}) becomes
\begin{align}\label{weight bound2}
|W_{\epsilon < L}^{k, (n)}(\Phi)| & \leq C' \int_{w^k \in \CC^d} \d^{2d} w^k \Psi(0, \ldots, 0, w^k) \int_{(t_1,\ldots,t_k) \in [\epsilon,L]^k} \d t_1 \ldots \d t_k \frac{1}{(4\pi)^{dk}} \frac{1}{(t_1\cdots t_k)^d}\left(\frac{t_1\cdots t_k}{t_1+\cdots+t_k}\right)^d + O(\epsilon) \\ & = C' \int_{w^k \in \CC^d} \d^{2d} w^k \Psi(0, \ldots, 0, w^k) \int_{(t_1,\ldots,t_k) \in [\epsilon,L]^k} \d t_1 \ldots \d t_k \frac{1}{(4\pi)^{dk}} \frac{1}{(t_1+\cdots+t_k)^d} + O(\epsilon) .
\end{align}
The first term in the Wick expansion is written out explicitly. 
The $O(\epsilon)$ refers to higher terms in the Wick expansion, which one can show all have order $\epsilon$, so disappear in the $\epsilon \to 0$ limit.
The expression $\Psi(0, \ldots, 0, w^k)$ means that we have evaluate the function $\Psi(w^1,\ldots, w^k)$ at $w^1=\ldots=w^{k-1} =0$ leaving it as a function only of $w^k$. 
In the original coordinates this is equivalent to setting $z^1=\cdots=z^{k-1} = z^k$.

Our goal is to show that $\epsilon \to 0$ limit of the right-hand side exists. 
The only $\epsilon$ dependence on the right hand side of (\ref{weight bound2}) is in the integral over the regulation parameters $t_1,\ldots, t_k$. 
Thus, it suffices to show that the $\epsilon \to 0$ limit of 
\ben
\int_{(t_1,\ldots,t_k) \in [\epsilon,L]^k} \frac{\d t_1 \ldots \d t_k}{(t_1+\cdots+t_k)^d}
\een
exists.
By the AM/GM inequality we have $(t_1+\cdots+t_k)^d \geq (t_1 \cdots t_d)^{d/k}$. 
So, the integral is bounded by
\ben
\int_{(t_1,\ldots,t_k) \in [\epsilon,L]^k}\frac{\d t_1 \ldots \d t_k}{(t_1+\cdots+t_k)^d} \leq \int_{(t_1,\ldots,t_k) \in [\epsilon,L]^k}\frac{\d t_1 \ldots \d t_k }{(t_1 \cdots t_k)^{d/k}} = \frac{1}{(1-d/k)^k} \left(\epsilon^{1-d/k} - L^{1-d/k}\right)^k .
\een
By assumption, $d < k$, so the right hand side has a well-defined $\epsilon \to 0$ limit. 
This concludes the proof.

%Now, since $t_\alpha / \sum_\beta t_\beta < 1$ for each $\alpha$ we have the following bound for the operators $D_{\alpha, i_\alpha}$:
%\bestar
%\left|D_{\alpha,i_{\alpha}} \Phi\right| & \leq \left(\left|\frac{\partial}{\partial w^\alpha_{i_\alpha}} \Phi\right| +  \sum_{\beta = 1}^{k-1}\frac{\partial}{\partial w_{i_\beta}^{\beta}}\right)
%\prod_{j=1}^d \left(\frac{\partial}{\partial w_j^\alpha} - \frac{1}{k} \sum_{\beta =1}^{k-1} \frac{\partial}{\partial w_{j}^\beta}\right)
%^{n_j^\alpha} \right| 
\end{proof}

\subsection{A general fact about chiral anomalies}

Once any theory has been renormalized, the next step to constructing a quantization is to solve the quantum master equation. 
In general, there may be an obstruction to solving this equation.
Such obstructions in the physics literature are known as {\em anomalies}.
In general, it may be difficult to characterize such anomalies, but in the case of holomorphic theories on $\CC^d$ our result in the previous section makes this problem much easier. 
Indeed, since there are no counterterms requires, we can plug in the RG flow of the classical action functional 
and study the quantum master equation directly. 
As is usual in perturbation theory, we work order by order in $\hbar$ to construct a quantization.
In this section we will study the first step, which is to promote a classical theory to a solution of the quantum master equation modulo $\hbar^2$. 

As above, $\sE$ will be a holomorphically translation invariant theory on $\CC^d$ and $I$ will be the holomorphic interaction. 
The linearized BRST operator is of the form $Q = \dbar + Q^{hol}$ where $Q^{hol}$ is a holomorphic differential operator. 
For this section, it will be most convenient to set $Q^{hol} = 0$. 

Define $I[L] = W(P_{\epsilon<L}, I) \mod \hbar^2$ as in the last section.
Recall, from Section \ref{sec: qme} that the regularized quantum master equation at scale $L$ is 
\ben
Q I [L] + \hbar \Delta_L I[L] + \frac{1}{2} \{I[L], I[L]\}_L = 0 .
\een
This is equivalent to the equation $(Q + \hbar \Delta_L) e^{I[L]/\hbar} = 0$. 
Therefore, the obstruction to satisfying the quantum master equation modulo $\hbar^2$ at scale $L$ is
\ben
\Theta[L] = \hbar^{-1} \left(Q I [L] + \hbar \Delta_L I[L] + \frac{1}{2} \{I[L], I[L]\}_L \right),
\een
or, equivalently $\Theta[L] = e^{-I[L]/\hbar} (Q + \hbar\Delta_L)e^{I[L]/\hbar}$. 
By definition, $I[L] = \lim_{\epsilon \to 0} W(P_{\epsilon<L}, I)$ which is equivalent to $e^{I[L]/\hbar} = \lim_{\epsilon \to 0} e^{\hbar \partial_{P_{\epsilon<L}}} e^{I / \hbar} \mod \hbar^2$ as a formal series in $\hbar$. 
Thus, we can rewrite
\ben
(Q + \hbar \Delta_L) e^{I[L]/\hbar} = \lim_{\epsilon \to 0} (Q + \hbar \Delta_L)  \left(e^{\hbar \partial_{P_{\epsilon<L}}} e^{I / \hbar}\right) .
\een
Now, the operator $Q$ commutes with $e^{\hbar \partial_{P_{\epsilon < L}}}$.
Moreover, $\Delta_L e^{\hbar \partial_{P_{\epsilon < L}}} = e^{\hbar \partial_{P_{\epsilon < L}}} \Delta_\epsilon$ acting on functionals. 
Thus, 
\ben
\lim_{\epsilon \to 0} (Q + \hbar \Delta_L)  \left(e^{\hbar \partial_{P_{\epsilon<L}}} e^{I / \hbar}\right) = \lim_{\epsilon \to 0}  e^{\hbar \partial_{P_{\epsilon < L}}} (Q + \hbar \Delta_\epsilon) e^{I / \hbar} .
\een
Since $\Delta_\epsilon$ is a BV operator for the bracket $\{-,-\}_{\epsilon}$, we can rewrite the right-hand side as
\ben
\frac{1}{\hbar} \lim_{\epsilon \to 0} e^{\hbar P_{\epsilon < L}} (Q I + \hbar \Delta_\epsilon I + \frac{1}{2}\{I, I\}_\epsilon) e^{I / \hbar}.
\een
For every $\epsilon > 0$ we have $\Delta_\epsilon I = 0$.
Moreover, since $I$ is holomorphic (and since $Q^{hol} = 0$) we have $Q I = \dbar I = 0$.

We conclude that the one-loop anomaly is 
\ben
\Theta = \lim_{L \to 0} \Theta[L] = \frac{1}{2} \lim_{L \to 0} \lim_{\epsilon \to 0} e^{-I/\hbar} e^{\hbar \partial_{P_{\epsilon < L}}} \left(\{I,I\}_\epsilon e^{I/\hbar}\right) \mod \hbar^2
\een

The main result of this section is the following.

\begin{lem}\label{lem: gen chiral anomaly}
The obstruction $\Theta = \lim_{L \to 0} \Theta[L]$ to satisfying the one-loop quantum master equation is given by the expression
\ben
\Theta = \lim_{L \to 0} \lim_{\epsilon \to 0} \sum_{\Gamma \in {\rm Wheel}_{d+1}} W_\Gamma(P_{\epsilon < L}, K_\epsilon,I)
\een
where the sum is over all wheels with $(d+1)$-vertices. 
\end{lem}

\begin{proof}

Like the proof of the non-existence of counterterms for holomorphic theories, the proof of this result will be the consequence of an explicit calculations and bounds of certain Feynman diagrams. 

If we expand the quantity 
\be\label{graph exp}
\lim_{\epsilon \to 0} e^{-I/\hbar} e^{\hbar \partial_{P_{\epsilon < L}}} \left(\{I,I\}_\epsilon e^{I/\hbar}\right) \mod \hbar^2
\ee
over a sum of graphs, we see that the following two types of weights occur. 
Note that, by assumption, we are only looking at graphs of genus one which look like wheels with possible trees attach.
Graphically, the quantity $\{I,I\}_\epsilon$ is the graph of two vertices with a separating edge labeled by the heat kernel $K_\epsilon$.
Thus, all weights appearing in the expansion of (\ref{graph exp}) attach the propagator $P_{\epsilon<L}$ to all edges besides a single distinguished edge $e$, which is labeled by $K_\epsilon$. 
The two types of weights are the following:
\begin{enumerate}
\item the distinguished edge $e$ is separating;
\item the distinguished edge $e$ is {\em not} separating, and so appears as the internal edge of the wheel portion of the graph.
\end{enumerate}

By the classical master equation, we see that the $\epsilon \to 0$ limit of weights of Type (1) go to zero.
Thus, we must only consider the weights of Type (2). 

The result will follow from two steps.
\begin{enumerate}
\item If $\Gamma$ is a wheel with $k < d+1$ vertices, then $ W_\Gamma(P_{\epsilon < L}, K_\epsilon,I)
 = 0$ identically. 
\item If $\Gamma$ is a wheel with $k > d+1$ vertices, then $\lim_{\epsilon \to 0} W_\Gamma(P_{\epsilon < L}, K_\epsilon,I) =0$.
\end{enumerate} 

The proof of (1) follows a similar to 
\end{proof}
%By the compatibility between the QME and RG flow, it suffices to solve this equation at any scale $L$. 
%Thus, $(Q + \hbar \Delta_L)e^{I[L] / \hbar} = 0$ if and only if 
%\begin{lem}
%\ben
%\lim_{L \to 0} \lim_{\epsilon \to 0} e^{-I / \hbar} e^{\hbar \partial_{P_{\epsilon < L}}} \left(\{I, I\}_\epsilon e^{I / \hbar}\right) = \lim_{L \to 0} \lim_{\epsilon \to 0} \sum_{\Gamma} W_\Gamma(P_{\epsilon<L}, K_\epsilon, I) \mod \hbar^2 .
%\een
%\end{lem}


\subsection{Relation to renormalization of topological theories}

\brian{it's a fun corollary to see that a slight modification of the above actually shows that topological BF theory in any dimension has a one-loop quantization with no counterterms, not sure if I'll put that here.}

\section{Equivariant BV quantization}

Equivariant BV quantization is an enhancement of ordinary BV quantization where one takes into account the action of a group or Lie algebra. 
We will heavily rely on techniques of equivariant BV quantization throughout this thesis, notably in the construction of the holomorphic $\sigma$-model in Chapter \ref{chap hol sig} and in the proof of a local version of the Grothendieck-Riemann-Roch theorem in Chapter \ref{chap symmetries} using Feynman diagrammatic expansions.

In this section we define the notion of an equivariance in our formalism through the lens of a central result of classical field theory: Noether's theorem.
Roughly speaking, this states that symmetries of a theory are encoded by a conserved quantity. 
For instance, a symmetry by translations gives rise to conservation of energy through the the stress-energy momentum tensor. 
There is an enhancement of Noether's theorem using the language of factorization algebras proved in \cite{CG2} that we will not review here, but will recall in Chapter 3. 
For us, the manifestation of Noether's theorem will come from a description of a symmetry through a functional satisfying a certain equivariant version of the classical or quantum master equations. 

The symmetries of BV theories that we consider is a direct analog of symmetries in ordinary Hamiltonian mechanics, which we briefly recall. 
Suppose that $\fh$ is a Lie algebra on $(M,\omega)$ is an ordinary symplectic manifold. 
A symplectic action of $\fh$ on $X$ is a map of Lie algebras 
\ben
\rho : \fh \to {\rm SympVect}(M)
\een
where ${\rm SympVect}(M)$ is the Lie algebra of symplectic vector fields, i.e. those vector fields $X$ which preserve the symplectic form $L_X \omega = 0$.
On any symplectic manifold, the Poisson algebra of functions admits a Lie algebra map $\sO(M) \to {\rm SympVect}(M)$ sending a function $f$ to its Hamiltonian vector field $X_f = \{f,-\}$, where $\{-,-\}$ is the Poisson bracket.
An action $\rho$ is said to be {\em inner} if it lifts to a map of Lie algebras $\Tilde{\rho} : \fh \to \sO(M)$. 
Recall that on any symplectic manifold the kernel of $f \mapsto X_f$ is precisely the constant functions. 

All of our classical theories arise as $(-1)$-shifted symplectic formal moduli problems.
Hence, suppose we replace the symplectic manifold $M$ by a formal moduli problem $B \fg$, where $\fg$ is some dg Lie (or $L_\infty$ algebra). 
To give $B \fg$ the structure of a $(-1)$-shifted symplectic structure is equivalent to having a $(-3)$-shifted non-degenerate pairing on $\fg$. 
Functions on $B\fg$ are precisely the Chevalley-Eilenberg cochains $\sO(B\fg) = \clie^*(\fg)$. 
The $(-1)$-shifted symplectic structure equips $\clie^*(\fg) [-1]$ with the structure of a dg Lie algebra.
Since all symplectic vector fields are Hamiltonian in this case we see that 
\ben
{\rm SympVect}(B \fg) = \clie^*(\fg)[-1] / \CC = \cred^*(\fg)[-1]
\een
where we have taken the quotient by the constants, which by definition is the reduced cochains. 
We modify the notion of a symplectic action slightly to allow for more general maps of Lie algebras.
A symplectic action of $\fh$ on the $(-1)$-shifted symplectic formal moduli space $B \fg$ is a map of $L_\infty$ algebras, or a homotopy coherent map of dg Lie algebras
\ben
\rho : \fh \rightsquigarrow \cred^*(\fg)[-1] .
\een
Such a map $\rho$ is equivalent to a Maurer-Cartan element in the dg Lie algebra
\ben
I^{\fh} \in \clie^*(\fh) \tensor \cred^*(\fg)[-1].
\een
This is a cohomological degree $+1$ element $I^\fh$ such that $\d I^\fh + \frac{1}{2}\{I^\fh, I^\fh\} = 0$.
Here $\{-,-\}$ is the bracket on $\cred^*(\fg)$ and $\d$ is the sum of the Chevalley-Eilenberg differentials on $\fh$ and $\fg$. 
This is a version of the classical master equation over the base ring $\clie^*(\fh)$. 

\subsection{Classical equivariance}

We proceed to mimic the above discussion to define the notion of equivariance for a general classical BV theory. 
Again, let $\fh$ be an $L_\infty$ algebra.
A classical field theory, in the BV formalism, is given by an elliptic formal moduli problem satisfying some conditions. 
In the beginning of this chapter, we saw that this is encoded by a space of fields $\sE$, an action functional $S \in \oloc(\sE)$, and a $(-1)$-shifted symplectic structure. 

We have just seen that we can express an action of $\fh$ using a Maurer-Cartan element that is a functional of both $\fh$ and $\fg$.
The essential difference with this general situation is that we require our functionals to be {\em local} with respect to their dependence on the fields $\sE$. 
Recall that the shifted symplectic structure induced a $P_0$-bracket on local functionals $\oloc(\sE)$.
Thus, $\oloc(\sE)[-1]$ has the structure of a dg Lie algebra with differential given by $\{S,-\}$. 

\begin{dfn}\label{dfn: fh equiv}
An action of $\fh$ on a classical theory $(\sE, S, \omega)$ is a Maurer-Cartan element of the dg Lie algebra
\ben
I^\fh \in \cred^*(\fh) \tensor \oloc (\sE)[-1] .
\een
In other words, $I^{\fh}$ satisfies the {\em equivariant classical master equation}:
\ben
\d_{\fh} I^{\fh} + \{S, I^{\fh}\} + \frac{1}{2} \{I^{\fh}, I^{\fh}\} = 0 .
\een
\end{dfn}

Analogous to the manipulations above, we see that such an $I^{\fh}$ defines a sequence of maps
\ben
\fh^{\tensor m} \tensor \sE(X)^{\tensor n} \to \sE(X)
\een
combining to give $\sE(X)$ the structure of an $L_\infty$-module over $\fh$. 
The equivariant classical master equation exhibits $I^{\fh}$ as a conserved quantity encoding the symmetry by the Lie algebra $\fh$.
This is the fundamental idea of Noether's theorem.

\begin{rmk}
There is a natural map $\clie^*(\fh) \to \cred^*(\fh)$. 
An {\em inner} action of $\fh$ on $\sE$ is a lift of an action $I^{\fh}$ to an Maurer-Cartan element of the dg Lie algebra $\clie^*(\fh) \tensor \oloc(\sE)[-1]$. 
Note that there is, in general, an obstruction to lifting which lives in the cohomology $H^1_{\rm Lie}(\fh)$.  
Thus, if $\fh$ is semi-simple we see that actions always lift to inner actions.
We will be more interested in this problem in the case that $\fh$ is a {\em local} Lie algebra, where the obstruction theory is more interesting.
\end{rmk}

\subsection{Quantum equivariance}

If we start with an $\fh$-equivariant classical BV theory with fields $\sE$ with action functional $S$ --- so that $\fh$ has an $\L8$ action on the fields that preserves the pairing and the action functional $S$ --- then we can encode the action of $\fh$ as a Maurer-Cartan element $I^\fh$ in $\clie^*(\fh) \otimes \oloc (\sE)$.
We then view the sum $S + I^\fh$ as the \emph{equivariant} action functional:
the operator $\{S+ I^\fh,-\}$ is the twisted differential on $\clie^*(\fh) \otimes \oloc(\sE)$ with $I^\fh$ as the twisting cocycle,
and this operator is square-zero because $\{S + I^\fh, S+ I^\fh\}$ is a ``constant'' (i.e., lives in $\clie^*(\fh)$ and hence is annihilated by the BV bracket).

This perspective suggests the following definition of an equivariant quantum BV theory.
The starting data is two-fold:
an $\fh$-equivariant classical BV theory with equivariant action functional $S+I^\fh$, 
and a BV quantization $\{S[L]\}$ of the non-equivariant action functional $S$.
Following Costello, it is convenient to write $S$ as $S_{\text{free}} + I$, 
where the first ``free'' term is a quadratic functional and the second ``interaction'' term is cubic and higher.
In this situation, the effective action $S[L] = S_{\text{free}} + I[L]$, 
i.e., only the interaction changes with the length scale.

As in Section \ref{??} we let $\sO_{P,sm}^+(\sE)$ be the functionals that are at least cubic, have proper support, and have smooth first derivative. 

\begin{dfn} \label{eqQFT} 
An {\em $\fh$-equivariant BV quantization} is a collection of effective interactions $\{I^\fh[L]\}_{L \in (0\infty)} \subset \cred^*(\fh) \tensor \sO_{P,sm}^+(\sE)[[\hbar]]$
satisfying
\begin{enumerate}[(a)]
\item the RG flow equation
\[
W(P_\epsilon^L, I[\epsilon]+I^\fh[\epsilon]) = I[L] + I^\fh[L]
\]
for all $0 < \epsilon < L$,
\item the equivariant scale $L$ quantum master equation, which is that
\[
Q(I[L]+I^\fh[L]) + \d_\fh I^\fh[L] + \frac{1}{2}\{I[L]+I^\fh[L],I[L]+I^\fh[L]\}_{L} + \hbar \Delta_L(I[L]+I^\fh[L])
\]
lives in $\clie^*(\fh)$ for every scale $L$, and
\item the locality axiom, with the additional condition that as $L \to 0$, we recover the equivariant classical action functional $S+ I^\fh$ modulo $\hbar$.
\end{enumerate}
\end{dfn}

In other words, we simply follow the constructions of \cite{CosRenormalization} working over the base ring $\clie^*(\fh)$.
A careful reading of those texts shows that the freedom to work over interesting dg commutative algebras is built into the formalism.
%\owen{We may need to be a bit careful since a version of nilpotency is used there but ..}
%Note that our situation is particularly simple since the non-equivariant classical field theory is free and hence admits a very simple quantization,
%with $I[L] = 0$ for all $L$.

\subsection{The case of a local Lie algebra}\label{sec: local equiv}

The above formalism works equally well, with some slight modifications, if we replace the Lie algebra $\fh$ by a {\em local Lie algebra} $\sL$ on the manifold where the theory $\sE$ lives.
This is done in detail in Chapter 11 of \cite{CG2}, and we refer the reader there for more details. 

For the classical case, the first thing we must define is where the classical Noether current $I^{\fh} \leftrightarrow I^\sL$ lives.
Naively, we expect this to live in the space
\be\label{dgLie1}
\cred^*(\sL_c(X)) \tensor \oloc(\sE)[-1] .
\ee
This is not quite good enough for our purposes since we have not taken into account the {\em locality} in the Lie algebra direction.
Note that (\ref{dgLie1}) is a still a dg Lie algebra, just as above.
The inclusion (\ref{local inclusion}) determines an inclusion of vector spaces
\ben
\oloc(\sL[1] \oplus \sE) \hookrightarrow \cred^*(\sL_c(X)) \tensor \oloc(\sE)
\een
We can further quotient this subspace by $\oloc(\sL[1]) \oplus \oloc(\sE) \subset \oloc(\sL[1] \oplus \sE)$ consisting of those local functionals that depend solely on $\sL$ or $\sE$ to obtain an inclusion of vector spaces 
\ben
{\rm Act}(\sL, \sE) := \oloc(\sL[1] \oplus \sE) / \oloc(\sL[1]) \oplus \oloc(\sE)\hookrightarrow \cred^*(\sL_c(X)) \tensor \oloc(\sE) .
\een
Thus, ${\rm Act}(\sL, \sE)$ consists of functionals on $\sL [1] \oplus \sE$ that are local as both a function of $\sL[1]$ and $\sE$ and do not depend solely on $\sL[1]$ and $\sE$. 

\begin{lem}[Chapter ?? \cite{CG2}]
The differential and bracket defining the dg Lie algebra $\cred^*(\sL_c(X)) \tensor \oloc(\sE)[-1]$ in (\ref{dgLie1}) restricts to give a dg Lie algebra structure on the subspace ${\rm Act}(\sL,\sE)[-1]$. 
\end{lem} 

Using this lemma, the following definition is well-posed.

\begin{dfn}\label{dfn: sL equiv}
Let $\sL$ be a local Lie algebra and $\sE$ a classical field theory.
An $\sL$ action on $\sE$ is a Maurer-Cartan element
\ben
I^{\sL} \in {\rm Act}(\sL, \sE) [-1] .
\een
In other words, $I^{\sL}$ satisfies the equivariant classical master equation
\ben
\d_{\sL} I^{\sL} + \{S, I^{\sL}\} + \frac{1}{2} \{I^{\sL}, I^{\sL}\} = 0 .
\een
\end{dfn}

\begin{rmk}
Given any $L_\infty$ algebra $\fh$ one can define the local Lie algebra $\Omega^*_X \tensor \fh$ on $X$. 
The data of an action of $\fh$ on a theory as in Definition \ref{dfn: fh equiv} is equivalent (up to homotopy) to the data of an action of the local Lie algebra $\Omega^*_X \tensor \fh$ in the definition above.
In fact, there is an equivalence of dg Lie algebras $\cred^*(\fh) \tensor \oloc(\sE)[-1] \simeq {\rm Act}(\Omega^*_X \tensor \fh, \sE)$. 
\end{rmk}

\subsubsection{}

The quantum story for an action by a local Lie algebra is also similar to the case of an ordinary Lie algebra.
There are two spaces of functionals that appear when discussing actions of a local Lie algebra $\sL$ on a quantum field theory.
We will fix a quantum field theory as in Definition \ref{dfn: qft}.
This is the data of a free BV theory $(\sE, Q, \omega)$ together with a family of functionals $\{I[L]\}$ satisfying RG flow and the QME (plus a locality condition). 

\begin{dfn}
An $\sL$-action on the quantum field theory $(\sE, Q, \omega, \{I[L]\})$ is the data of a family of functionals
\ben
\{I^\sL[L]\} \subset \sO_{P, sm}^+(\sL[1] \oplus \sE) / \sO_{P,sm} (\sL[1]) [[\hbar]]
\een
satisfying the following properties:
\begin{enumerate}[(a)]
\item The RG equation $W(P_{L<L'}, I^\sL[L]) = I^{\sL} [L']$;
\item The equivariant quantum master equation at scale $L$:
\ben
\d_\sL I^\sL[L] + Q I^\sL[L] + \frac{1}{2} \{I^\sL[L], I^\sL[L]\}_L + \hbar \Delta_L I^{\sL}[L] = 0
\een
where $\d_\sL$ is the Chevalley-Eilenberg differential on $\sL$;
\item the locality axiom as in Definition \ref{dfn: qft};
\item under the natural quotient map
\ben
\sO_{P, sm}^+(\sL[1] \oplus \sE) / \sO_{P,sm} (\sL[1]) [[\hbar]] \to \sO^+_{P,sm}(\sE)[[\hbar]] 
\een
sends $I^{\sL}[L] \mapsto I[L]$ for each $L > 0$. 
\end{enumerate}
\end{dfn}

In the definition above we require $I^{\sL}[L]$ to be an element in $\sO_{P, sm}^+(\sL[1] \oplus \sE) / \sO_{P,sm} (\sL[1]) [[\hbar]]$, which is the space of smooth and proper functionals on $\sL[1] \oplus \sE$ that are at least cubic modulo $\hbar$ and do not depend solely on $\sL$. 
A stricter definition is that of an {\em inner} action, where we allow the functionals that depend solely on $\sL[1]$. 

\begin{dfn}
An {\em inner} action of $\sL$ on the QFT $(\sE, Q, \omega, \{I[L]\})$ is an effective family
\ben
\{I^\sL[L]\} \subset \sO_{P, sm}^+(\sL[1] \oplus \sE)
\een
satisfying conditions (a)-(c) above and under the natural map
\ben
\sO_{P, sm}^+(\sL[1] \oplus \sE) [[\hbar]] \to \sO^+_{P,sm}(\sE)[[\hbar]] 
\een
we have $I^{\sL}[L] \mapsto I[L]$ for each $L > 0$. 
\end{dfn}

Every inner action clearly defines an ordinary action on a QFT.
In practice, we will study the problem of {\em lifting} an ordinary action to an inner action. 
Just as in the obstruction theory discussed in Section \ref{sec: obstruction} there is a deformation complex controlling this lifting problem. 
Indeed, suppose $I^{\sL}[L] \in \sO_{P, sm}^+(\sL[1] \oplus \sE) / \sO_{P,sm} (\sL[1]) [[\hbar]]$ is a family satisfying the condition of having an action by $\sL$. 
We can lift this to a family of functionals
\ben
\Tilde{I}^{\sL}[L] \in \sO_{P, sm}^+(\sL[1] \oplus \sE) [[\hbar]] 
\een
that satisfy RG flow and the locality axioms, but in general they do not satisfy the equivariant quantum master equation. 
The obstruction is an element 
\ben
\Theta[L] = \d_{\sL} \Tilde{I}^{\sL}[L] + Q \Tilde{I}^{\sL}[L] + \frac{1}{2} \{\Tilde{I}^{\sL}[L], \Tilde{I}^{\sL}[L]\}_L + \hbar \Delta_L \Tilde{I}^{\sL}[L]  .
\een
Since the right-hand side is zero modulo $\sO(\sL[1])[[\hbar]]$, by assumption, we must have $\Theta[L] \in \sO(\sL[1])[[\hbar]]$. 
By homotopy RG flow it suffices to solve this equation at any scale $L$.
Moreover, by the locality axiom the limit $\lim_{L \to 0} \Theta[L]$ exists and is a local functional of $\sL[1]$. 
Thus we arrive at the following.

\begin{lem}\label{lem: innner action}
Suppose $\{I^{\sL}[L]\}$ is an effective family defining an action of $\sL$ on a QFT.
Then, the obstruction to lifting this action to an inner action, that is the anomaly to solving the equvariant quantum master equation, is the degree $+1$ cocycle in $\Theta = \lim_{L \to 0} \Theta[L] \in \cloc^*(\sL)$. 
\end{lem}


\begin{rmk}
Equivariant quantization is essentially a version of the background field method in QFT.
One treats elements of $\sL$ as background fields and 
the interaction terms $I^\sL[L]$ encode the variation of the path integral measure with respect to these background fields.
(Solving the QME is our definition of well-posedness of the measure.)
This should not be confused with {\em gauging} a theory by $\sL$, which involves putting the elements of $\sL$ in the theory as propagating fields.
\end{rmk}



\end{document}