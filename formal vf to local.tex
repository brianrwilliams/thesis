\documentclass[10pt]{amsart}

\usepackage{macros,slashed}

\linespread{1.25}

\def\brian{\textcolor{blue}{BW: }\textcolor{blue}}

\title{Local functionals from formal vector fields}

\begin{document}
\maketitle

\section{Local functionals from Gelfand-Fuks cohomology}

\subsection{Gelfand-Fuks cohomology}

In this section we recall some facts about the Lie algebra cohomology of formal vector fields $\W_d$ on the $d$-disk with values in certain non-trivial modules. 
We refer to Section \ref{sec: gk formal geometry} for the requisite notation for objects living on the formal disk.

In Section \ref{sec: formal atiyah} we have constructed the formal Atiyah class for any formal vector bundle $\cV$ on $\hD^n$. 
It is an element of the relative Gelfand-Fuks cohomology
\ben
\At^{\GF}(\cV) \in \clie^1(\W_d,\GL_d; \hOmega^1_d \otimes_{\hO_d} \End_{\hO_d}(\cV)) .
\een
From the Atiyah class we have built the formal Chern character using the usual formula 
\ben
\ch^{\GF}(\cV) = \Tr\left(\exp\left(\frac{1}{2\pi i} \At^{\GF}(\cV)\right)\right),
\een
and have studied how components of this formal Chern character give rise to $L_\infty$ extensions of $\W_d$ that appear as natural universal symmetries of quantizations of higher dimensional holomorphic $\sigma$ models with target $\hD^d$. 

In this section we arrive at the Lie algebra of formal vector fields, and its cohomology, from a different perspective. 
Instead of using formal geometry to construct universal objects on the {\em target} of a $\sigma$ model, we will see how Gelfand-Fuks classes characterize holomorphic symmetries on the higher {\em world-sheet}, or source manifold. 

The symmetry is that of holomorphic reparametrizations. 
Infinitesimally, this is described by the Lie algebra of holomorphic vector fields. 
We have already seen \brian{ref} that classical theories on a complex manifold $X$ with such a symmetry by holomorphic reparametrizations admit an action by the local Lie algebra $\sT_X = \Omega^{0,*}(X , T_X^{1,0})$. 

The Gelfand-Fuks classes we will consider in this section appear as anomalies for quantizing an action by the local Lie algebra $\sT_X$. 
In other words, these classes parametrize shifted central extensions of $\sT_X$, just as the classes $\theta \in \Sym^{d+1}(\fg^*)^\fg$ defined central extensions of the current algebra $\fg^X$. 
By our usual yoga of studying equivariant quantizations, we know such anomalies live in the local cohomology complex $\cloc^*(\sT_X)$. 


\begin{dfn/lem} Consider the following two classes of cocycles on $\W_d$.
\begin{itemize}
\item[Chern type:] For $1 \leq k \leq n$, let $\tau_k \in \clie^k(\W_d; \hOmega_d^k)$ be the cocycle
\ben
\tau_k = \sigma_k \left(\At^{\GF}(\hT_d)\right) ...finish...
\een
\item[$\GL$ type:] For $1 \leq i \leq d$ let $\xi_i \in \clie^{2i-1}(\W_d ; \hO_d)$ be the cocycle 
\ben
\xi_i : (f_{i_1} \partial_{i_1} ,\ldots, f_{i_{2i-1}} \partial_{i_{2i-1}}) \mapsto \sum ..
\een
\end{itemize}
\end{dfn/lem}

We will use the notation $\hOmega_d^\# = \oplus_{k} \hOmega^k_d [-k]$ to denote the graded $\W_d$-module with $\hOmega^k_d$ sitting in degree $k$. 
The wedge product of forms endows this $\W_d$-module with the structure of a graded commutative algebra. 

If $V$ is a graded vector space then we use the notation $\CC[V]$ to denote the free graded $\CC$-algebra on $V$.
If $V$ is spanned by vectors $\{v_i\}$ we will use the shorthand $\CC[v_i]$ for this graded algebra. 

\begin{thm}[\cite{GFnontrivial}] \label{thm nontrivial coeff} The bigraded commutative algebra $H^*(\W_d ; \hOmega_d^{\#})$ is isomorphic to the bigraded commutative algebra 
\ben
\left. \left(\CC[\xi_1,\ldots, \xi_{2d-1}, \tau_1,\ldots,\tau_d]\right) \right/ \left(c_1^{j_1}\cdots c_d^{j_d} \right),
\een
where the quotient is over all indices $\{j_1,\ldots,j_d\}$ that satisfiy $j_1 + 2j_2 + \cdots + d j_d > d$. 
Here $\xi_{2i-1}$ is in bidegree $(2i-1,0)$ and $\tau_j$ is in bidegree $(j,j)$. 
\end{thm}

In the above result we have not turned on the de Rham differential $\d_{dR} : \hOmega^k_d \to \hOmega_d^{k+1}$. 
This endows $\hOmega^*_d = (\hOmega^\#_d, \d_{dR})$ with the structure of a dg commutative algebra in $\W_d$-modules. 
The formal Poincar\'{e} lemma asserts that the inclusion of the trivial $\W_d$-module 
\ben
\CC \xto{\simeq} \hOmega^*_d
\een 
is a quasi-isomorphism. 
In turn, we obtain a quasi-isomorphism of Chevalley-Eilenberg complexes
\ben
\clie^*(\W_d) \xto{\simeq} \clie^*(\W_d ; \hOmega_d^*) . 
\een 
We may think of the cochain complex $\clie^*(\W_d ; \hOmega_d^*)$ as the total complex of the double complex with vertical differential given by the $\W_d$ Chevalley-Eilenberg differential for the graded module $\hOmega_d^\#$, and horizontal differential equal to the de Rham differential. 

To any double complex there is a spectral sequence abutting to the cohomology of the total complex. 
The $E_1$ page of this spectral sequence is given by the cohomology of the vertical differential. 
Moreover, if the double complex is a bigraded algebra so are each of the pages. 
In this case, the $E_1$ page is precisely the bigraded algebra of Theorem \ref{thm nontrivial coeff} and we have a spectral sequence
\be\label{ss1}
E^{p,q}_2 = \left(H^q(\W_d ; \hOmega^p_d), \d_{dR}\right) \implies H^*(\W_d ; \hOmega^*_d) \cong H^*(\W_d) .
\ee

\begin{eg}
For the case $d = 1$ the spectral sequence collapses at the $E_2$ page. 
The only nontrivial cohomology is $\CC$ in bidegree $(0,0)$ and $\xi_1 \cdot \tau_1$ in bidgree $(1,2)$. 
The $1$-cocycle valued in formal power series $\xi_1$ is given by $\xi_1(f_i \partial_i) = \partial_i f_i \in \hO_n$. 
The $1$-cocycle valued in formal $1$-forms $\tau_1$ is given by $\tau_1(g_j \partial_j) = \d_{dR}(\partial_j g_j)$. 
To obtain the generator of $H^3({\rm W}_1)$ we perform the following zig-zag:
\ben
\xymatrix{
\clie^3({\rm W}_1) \ar[r] & \clie^3({\rm W}_1 ; \hO_1) & \\
& \clie^2({\rm W}_1 ; \hO_1) \ar[u]^-{\d_{CE}} \ar[r]^-{\d_{dR}} & \clie^2({\rm W}_1; \hOmega^1_1) . 
} 
\een
The de Rham differential kills $\xi_1 \cdot \tau_1$, so there exists an $\alpha \in \clie^2({\rm W}_1 ; \hO_1)$ such that $\d_{dR} \alpha = - \xi_1 \cdot \tau_1$. 
Now, the class $\d^{\hO}_{CE} \alpha \in \clie^3({\rm W}_1 ; \hO_n)$ satisfies
\begin{align*}
\d_{dR} (\d^{\hO}_{CE} \alpha) & = - \d_{CE} (\xi_1 \tau_1) = 0 \\
\d_{CE} \d_{CE}^{\hO} \alpha & = 0 .
\end{align*}
Here, $\d_{CE}^{\hO}$ denote the Chevalley-Eilenberg differential for $\clie^*(\W_1 ; \hO_1)$ and $\d_{CE}$ is the restriction of this Chevalley-Eilenberg differential to $\clie^*(\W_1)$. 
The first line says that $\d_{CE}\alpha$ lifts to $\clie^3(\W_1)$, and the second line says that it is a cocycle for the absolute cohomology.  
Finally, note that $(\d_{CE}^{\hO} + \d_{dR} ) \alpha = \d_{CE}^{\hO} \alpha - \xi_1 \tau_1$. 
Thus, in the total complex $\d_{CE}^{\hO} \alpha$ is homotopic to $\xi_1 \tau_1$, and so $[\d_{CE}^{\hO} \alpha]$ is the generator of $H^3(\W_1)$. 
\end{eg}

For general $d \geq 1$, one can apply this spectral sequence to understand the cohomology $H^*(\W_d)$. 
To describe it, we introduce the following topological space. 
Let ${\rm Gr}(d,n)$ be the complex Grassmannian of $d$-planes in $\CC^n$. 
Denote by ${\rm Gr}(d,\infty)$ the colimit of the natural sequence 
\ben
{\rm Gr}(d,d) \to {\rm Gr}(d, d+1) \to \cdots . 
\een 
It is a standard fact that ${\rm Gr}(d, \infty)$ is a model for the classifying space $B\U(d)$ of principal $\U(d)$-bundles. 
Let $E\U(d) \to B\U(d)$ be the universal principal $\U(d)$-bundle. 
Using the colimit description above, we have a natural skeletal filtration of $B\U(d)$ by 
\ben
{\rm sk}_{k} B\U(d) = {\rm Gr}(d, k) .
\een 
Let $X_d$ denote the restriction of $E\U(d)$ over the $2d$-skeleton:
\ben
\xymatrix{
X_d \ar[r] \ar[d] & E \U(d) \ar[d] \\
{\rm sk}_{2d} B \U(d) \ar[r] & B\U(d) .
}
\een

\begin{rmk}
Though not the way the Gelfand and Fuks originally proved the result, one can use the computation of the cohomology of $\W_d$ with coefficients in $\hOmega^k_d$ together with the spectral sequence (\ref{ss1}) to prove this description of $H^*(\W_d)$. 
Indeed, the spectral sequence (\ref{ss1}) is isomorphic, up to regradings, to the Serre spectral sequence for the principal $\U(d)$-bundle $X_d \to {\rm sk}_{2d} B \U(d)$. 
In other words, the formal de Rham differential on $\hOmega^*_d$ is exactly the $E_2$ differential for the Serre spectral sequence. 
\end{rmk}

\begin{thm}[\cite{GF1} Theorem 2.2.4] 
There is an isomorphism of graded vector spaces
\ben
H^*(\W_d) \cong H^*_{dR} (X_d) .
\een
Moreover, the commutative algebra structure on $H^*(\W_d)$ is trivial. 
\end{thm}

As a simple example, note that when $d = 1$ we have ${\rm sk}_2 B \U(1) = \PP^1 \subset \PP^\infty = B \U(1)$. 
Moreover, the restriction of the universal bundle is Hopf fibration $U(1) \to S^3 \to \PP^1$. 
In particular, one has $X_1 = S^3$. 

\subsection{Local cocycles on holomorphic vector fields} 

We now turn to a description of local central extensions of the local Lie algebra of holomorphic vector fields $\sT_X = \Omega^{0,*}(X ; T_X^{1,0})$ for any complex $d$-fold $X$. 
Recall, such a central extension is determined by a cocycle in complex of local functionals $\cloc^*(\sT_X)$. 
Our main result is to identify such local cocycles with Gelfand-Fuks cocycles we have just discussed. 

Our first goal is to construct, from a Gelfand-Fuks class in $\clie^*(\W_d)$, a local functional on $\sT_X$. 
We have seen that the cochain complex $\clie^*(\W_d ; \hOmega_d^*)$, equipped with the total differential $\d_{CE} + \d_{dR}$, computes the absolute Gelfand-Fuks cohomology $H^*(\W_d)$. 
We will use this property to represent elements of $H^*(\W_d)$ by local cocycles on $\sT_X$. 

Using the natural framing on the formal disk, we can decompose a class $\alpha \in \clie^k(\W_d ; \hOmega^*_d)$ as 
\ben
\alpha = f^I \d t_I
\een
where the sum is over the multi-index $I = (i_1,\ldots, i_k)$ where $1 \leq i_j \leq d$, and for each $I$, $f^I$ is a $k$ multi-linear symmetric functional on $\W_d$ valued in $\hO_d$
\ben
f^I : \Sym^k(\W_d [1]) \to \hO_d .
\een 
We extend $f^I$ to a functional on the Dolbeault complex $\Omega^{0,*}(\CC^d ; T^{1,0} \CC^d)$ as follows. 
Using the framing on $\CC^d$, every element of the Dolbeualt complex can be written as
\ben
X^{J} (z,\zbar) \d \zbar_J
\een
where $J = (j_1,\ldots, j_l)$ is a multi-index and $X^J$ is an ordinary holomorphic vector field on $\CC^d$.
We extend $f^I$ to a Dolbeualt valued functional $\Omega^{0,*}(\CC^d ; T^{1,0}\CC^d)$ via the formula
\ben
\begin{array}{cccc}
f^I_{\Omega^{0,*}} : & \Sym^k\left(\Omega^{0,*}(\CC^d ; T^{1,0})\right) & \to & \Omega^{0,*}(\CC^d) \\ 
&\left(X_1^{J(1)} (z,\zbar) \d \zbar_{J(1)}, \ldots, X^{J(k)}_k (z,\zbar) \d \zbar_{J(k)} \right) & \mapsto & f^I(X_1^{J(1)} , \ldots, X^{J(k)}_k) \d \zbar_{J(1)} \wedge \cdots \d \zbar_{J(k)} 
\end{array}
\een

The local functional corresponding to the original class $\alpha = f^I \d t_I \in \clie^*(\Vect ; \hOmega^*_d)$ is defined by the $k$-multi-linear functional
\ben
(\xi_1, \ldots, \xi_k) \mapsto \int_{\CC^d} f^I_{\Omega^{0,*}}(\xi_1,\ldots, \xi_k) \d z_I .
\een
Denote this functional by $J^{GF}(\alpha)$. 
Note that it is only nonzero when the multi-index $I$ is a permutation of $(1,\ldots, d)$. 
Since it is given by the integral of a some multi-differential operators against a density it is manifestly a local functional. 

\begin{prop}
Let $\cloc^*(\sT_{\CC^d})$ be the local functionals of $\sT_{\CC^d}$ on $\CC^d$. The map
\ben
J^{GF} : \clie^*(\W_d ; \hOmega^*_n)[2d] \to \cloc^*(\sT_{\CC^d}) 
\een
sending $\alpha \mapsto J^{GF}(\alpha)$ is a map of cochain complexes. 
Moreover, it is a quasi-isomorphism. 
\end{prop}

\begin{thm}
Let $X$ be a complex $d$-fold. 
Then, the map
\ben
J^{GF} : \ul{\clie^*(\W_d ; \hOmega^*_n)}[2d] \to \cloc^*(\sT_X)
\een
is a quasi-isomorphism of sheaves.
In particular, there is an isomorphism of graded vector spaces
\be\label{central charges}
H^{*+2d}(\Vect) \cong H^*(X, \cloc^*(\sT_X)) ,
\ee
where the right-hand side denotes the hypercohomology. 
\end{thm}

\begin{eg}
Again, take the case $d=1$. 
We can describe the local cocycle corresponding to the generator $H^3(\W_1) \cong H^1(\sT_{\CC})$ explicitly. 
Recall, the generator of $H^3(\W_1)$ came from the element $\xi_1 \tau_1 \in \clie^2(\W_1 ; \hOmega^1_1)$ on the $E_2$ page of the spectral sequence (\ref{ss1}). 
Using the formulas for $\xi_1,\tau_1$ above, we see that the local functional $J^{GF}(\xi_1 \tau_1)$ is given by
\ben
\left(f(z,\zbar) \frac{\partial}{\partial z}, g(z,\zbar)\d \zbar \frac{\partial}{\partial z} \right) \mapsto \int_\CC \left(\frac{\partial}{\partial z} f \right) \partial \left(\frac{\partial}{\partial z} g\right) \d \zbar .
\een 
For instance, the linear functional $\tau_1 : g(t) \frac{\partial}{\partial t} \mapsto \d_{dR} (\partial_t g(t))$ is mapped to the functional on the Dolbeualt complex of holomorphic vector fields given by $g(z,\zbar) \frac{\partial}{\partial z} \mapsto \partial (\partial_z g(z,\zbar))$. 

If we integrate by parts, we can put $J^{GF}(\xi_1 \tau_1)$ in the form $\int f \partial^3_z g \d z \d \zbar$. 
If one restricts this local functional to the annulus and performs the radial integration, we recover the usual formula for the generator of $H^2({\rm Vect}(S^1))$ \brian{citation} defining the central extension of the Virasoro Lie algebra. 
In fact, in \cite{BWVir} \brian{finish}
\end{eg}


\end{document}