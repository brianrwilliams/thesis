\documentclass[10pt]{amsart}

\usepackage{macros,slashed}

\linespread{1.25}

\usepackage{tikz}
\usetikzlibrary{arrows,shapes}
\usetikzlibrary{trees}
\usetikzlibrary{matrix,arrows}
\usetikzlibrary{positioning}
\usetikzlibrary{calc,through}
\usetikzlibrary{decorations.pathreplacing}
\usepackage{pgffor}

\title{Holomorphic descent}

\def\brian{\textcolor{blue}{BW: }\textcolor{blue}}

\begin{document}
\maketitle

\subsection{The colored operad of holomorphic disks}

%\subsection{Reduction along spheres}
%$\pi : \CC^{d} \setminus \{0\} = S^{2d-1} \times \RR_{>0} \to \RR_{>0}$
%\begin{prop}
%Suppose $\sF$ is a holomorphically translation invariant factorization algebra and \brian{assumptions}. 
%Then, the sub factorization algebra 
%\ben
%?? \subset \pi_* \sF
%\een
%is a locally constant factorization algebra on $\RR_{>0}$. 
%\end{prop}
%
%This proposition tells us that to every holomorphically translation $\sF$ invariant factorization algebra satisfying those mild conditions above there is an associated associative algebra that we will denote by $\sA_\sF$. 

\subsection{Holomorphic descent}

\subsubsection{Topological descent}

\brian{review}
Before jumping in to the construction of operators in holomorphic theories using a descent procedure, we'd like to a review a more familiar topological situation. 
This concept was introduced by Witten in his introduction of cohomological field theories \cite{WittenCohomological}. 
Expositions of this construction in the context of topological conformal field theory can be found in \cite{WittenZwiebach, DijkgraafVV}

Suppose we have a translation invariant theory on $\RR^d$ for which all infinitesimal translations are exact for the BRST differential.
If $Q^{BRST}$ is the BRST differential this means that for $i=1,\ldots,d$ there exists operators $G_i$ on the space of fields such that
\be\label{G operator}
[Q^{BRST},G_i] = \frac{\partial}{\partial x_i} .
\ee
Note that since $\partial / \partial x_i$ has BRST degree zero, the operators $G_i$ decrease the BRST degree by one. 
Here, one thinks of the collection $\{G_i\}$ as providing a homotopy trivialization of the action by infinitesimal translations on the theory. 
In particular, this means that $\partial/\partial x_i$ acts trivially on the $Q^{BRST}$-cohomology.

In turn, $G_i$ also acts on the local operators of the theory. 
Using translation invariance, we can view a local operator $\cO$ as a function on space-time $\RR^d$. 
Suppose $\cO$ has pure BRST degree $k$.
Using the operator $G_i$ we can consider the function valued operator $G_i \cO$ which is of BRST degree $k-1$.
Using the frame on $\RR^d$ we can then define the {\em 1-form} valued operator
\ben
\cO^{(1)} = \sum_i (G_i \cO) \d x_i .
\een
By construction, the following relation is satisfied
\ben
\d_{dR} \cO = \sum_i \frac{\partial}{\partial x_i} \cO \d x_i = [Q^{BRST}, \cO^{(1)}] .
\een
This is the first so-called {\em topological descent equation}. 
In general, we can iterate the above construction to define
\ben
\cO^{(l)} = \sum_{i_1,\ldots i_l} G_{i_1}\cdots G_{i_l} \cO \d x_{i_1} \cdots d x_{i_l} .
\een
This is an $l$-form valued operator of BRST degree $k-l$. 

The operator $\cO^{(l)}$ allows us to define a new class of operators that depend on choosing an $l$-cycle inside of $\RR^d$. 
Indeed, suppose $Z \subset \RR^d$ is a closed $l$-dimensional submanifold.
Define the operator
\ben
\cO_Z = \int_{Z} \cO^{(l)} .
\een
The topological descent equations imply that if $\cO$ is BRST invariant $Q^{BRST} \cO = 0$, then $Q^{BRST} \cO_Z = 0$ as well.

Interesting examples of cohomological field theories arise as topological twists of supersymmetric theories.
Another class of examples come from topological vertex algebras \cite{Huang, LianZuckerman}.
In Section \brian{ref} we will discuss a class of such theories by considering a higher dimensional version of holomorphic gravity. 

We know that the local operators of a quantum field theory have the structure of a factorization algebra.
In the world of factorization algebras, there is also a notion of being topological: being (homotopically) locally constant.
This means that for every embedding of open balls $B \hookrightarrow B'$, the induced factorization structure map $\sF(B) \to \sF(B')$ is a quasi-isomorphism.

It would be natural to expect that the observables of a topological field theory, in which the infinitesimal translations are BRST exact, should give rise to such a factorization algebra. 
This is not exactly the case.
The relations (\ref{G operator}) guarantee a slightly weaker condition on the factorization algebra of observables. 
Indeed, the resulting action of the operators $G_i$ on the factorization algebra provide us with a sort of ``flat connection" on the factorization algebra. 
The difference between this structure and the locally constant condition is analogous to the discrepancy between $D$-modules and local systems. 
It is current work of Elliott and Safranov \cite{} to show how topological twists of supersymmetric theories give rise to such locally constant factorization algebras. 

We discuss a more direct way in which we can extract a shadow of a locally constant factorization algebra from a topological field theory using descent. 
There is an algebraic object associated to any locally constant factorization algebra.
Indeed, a famous theorem of Lurie \cite{LurieAlg} states an equivalence of categories 
\ben
\{{\rm Locally\;constant\;factorization\;algebras\;on\;} \RR^d\} \; \simeq \; \{E_d-{\rm \; algebras}\} .
\een
The cohomology of an $E_d$-algebra has the structure of a $P_d$-algebra.
In the category of cochain complexes, we have the following concrete definition of a $P_d$-algebra. 

\begin{dfn}
Let $d \geq 0$.
A $P_d$ algebra in cochain complex is a commutative dg algebra $(A, \d)$ together with the data of a bracket of degree $1-d$
\ben
\{-,-\} : A \tensor A \to A [d-1]
\een
such that:
\begin{enumerate}
\item the bracket is graded anti-symmetric:
\ben
\{a,b\} = -(-1)^{|a| + d-1} (-1)^{|b| + d-1};
\een
\item the bracket satisfies graded Jacobi:
\ben
\{a,\{b,c\}\} = \{\{a,b\},c\} + (-1)^{|a| + d-1} (-1)^{|b|+d-1} \{a,\{b,c\}\} ;
\een
\item the bracket is a graded bi-derivation for the commutative product:
\ben
\{a,b \cdot c\} = \{a,b\} \cdot c + (-1)^{|b|(|a| + d-1)} b \cdot \{a,c\} .
\een
\end{enumerate}
for all $a,b,c \in A$. 
\end{dfn}

\brian{finish}

\begin{eg} Topological $BF$ theory.
\brian{give references}
For any dg Lie algebra $(\fg,\d_{\fg}, [-,-])$ one can define the following $d$-dimensional topological field theory.
The fields of $BF$ theory with values in $\fg$ consist of 
\ben
(A,B) \in \Omega^*(\RR^d ; \fg[1]) \oplus \Omega^{*}(\RR^{d} ; \fg)[d-2]
\een 
with action functional
\ben
S(A,B) = \int \<B, d A + [A,A]\>_\fg,
\een 
where $\<-,-\>_\fg$ denotes a chosen invariant non-degenerate pairing on $\fg$. 
The name comes from the fact that $S = \int B F(A)$ where $F(A) = \d A + [A,A]$ is the curvature. 
The differential is a sum $\d = \d_{dR} + \d_{\fg}$. 
Note that the theory is translation invariant and has a natural action by the infinitesimal translations $\{\frac{\partial}{\partial x_i}\}$ via Lie derivative.

The class of local operators we consider are defined as
\begin{align*}
\cO_{A,a}(x) : & A \in \Omega^{0}(\RR^d; \fg)[1] \mapsto \<a,A(x)\>_\fg \\
\cO_{B,a}(x) : & B \in \Omega^{0}(\RR^d, \fg)[d-2] \mapsto \<a, B(x)\> .
\end{align*}
where $x \in \RR^d$ is a fixed point and $a \in \fg$ is a fixed element.
Using translation invariance, we view $\cO_{A,a}, \cO_{B,a}$ as function valued operators on $\RR^{d}$. 
The total space of local operators can be identified with functions on the shifted tangent bundle to the formal moduli space $B\fg$, $\sO(T[d-1] B \fg)$. 
The operator $\cO_{A,a}$ corresponds to the linear coordinate on the base of $B\fg$ and $\cO_{B,a}$ corresponds to a linear coordinate on the fiber.

We consider the differential operator
\ben
G_i = \frac{\d}{\d(\d x_i)}
\een
that also acts on the space of fields. 
This operator is equal to the contraction with the vector field $\frac{\partial}{\partial x_i}$. 
Since $G_i$ commutes with the differential and bracket on the Lie algebra, the Cartan formula implies
\ben
[Q^{BRST}, G_i] = \left[\d_{dR}, \frac{\d}{\d(\d x_i)}\right] = \frac{\partial}{\partial x_i} .
\een

Following the descent procedure above, we go on to define the form valued local operators
\ben
\cO_{A,a}^{(l)} = \sum_{i_1,\ldots, i_l} G_i \cO_{A,a} \d x_i
\een
and similarly for $\cO_{B,a}$. 
Then, for any $l$-cycle $Z \subset \RR^d$ we obtain operators $\int_Z \cO_{A,a}^{(l)}, \int_Z \cO_{B,a}^{(l)}$. 
For example, one can check that the latter operator is of the form
\ben
\int_Z \cO_{B,a}^{(l)} : B \in \Omega^{l}(\RR^d)[d-2-l] \mapsto \int_Z B ,
\een
which is of degree $- d + 2 + l$. 


To obtain the $P_n$-bracket via descent we consider the $(d-1)$-sphere $Z = S^{d-1}$, which we assume is centered at the origin.
Then, the bracket between the linear operators $\sO_{A,a}, \cO_{B,a'}$ is computed by the operator product expansion of $\cO_{A,a}$ and the descended operator $\int_{S^{d-1}} \cO_{B,a'}^{(d-1)}$:
\ben
\{\cO_{A,a}, \cO_{B,a'}\}_{P_d} = \cO_{A,a}(0) \star \int_{S^{d-1}} \cO_{B,a'}^{(d-1)} .
\een
A simple OPE calculation \brian{finish this}
\end{eg}


\subsubsection{General theory}

We will now summarize the steps in defining the higher dimensional OPE for holomorphically translation invariant quantum field theories. 
We note that this is a schematic, and as is usual we will need to regularize at various stages to obtained a well-defined construction. 

\begin{enumerate}
\item Suppose $\cO \in \sObs_0$ is a local operator supported at $0 \in \CC^d$. 
Let $z \in \CC^d$ be another point, and consider the translated operator 
\ben
\cO(z) := \tau_z \cO .
\een 
By the property of holomorphic translation invariance, this assignment defines a $\sO^{hol}(\CC^d)$-valued local operator. 

\item We perform ``holomorphic descent" to the function valued operator $\sO^{hol}(\CC^d)$ to obtain Dolbeualt valued operator 
\ben
\cO^{(0,*)}(z) \in \Omega^{0,*}(\CC^d) \tensor \sObs_0 .
\een 
Explicitly, 
\ben
\cO^{(0,k)} (z) = \sum_{I} (\Bar{\eta}_I \cdot \cO(z)) \d \zbar_I
\een
where $I = (i_1,\ldots,i_k)$, $1 \leq i_k \leq d$, is a multi-index of length $k$ and $\eta_I = \eta_{i_1\cdots i_k}$, $\d \zbar_I = \d \zbar_{i_1} \cdots \d \zbar_{i_k}$. 

\item For any $f(z) \d^d z \in \Omega^{d,hol}(\CC^d)$, and $w \in \CC^d$, define the sphere supported operator
\ben
\cO_{f}(w, r) := \int_{z \in S^{2d-1}_{w,r}} f(z) \d^d z \cO^{(0,d-1)}(z)
\een 
where $S^3_{w,r}$ is the sphere of radius $r$ centered at $w$. 

\item If $\cO'$ is another local operator supported at zero, we define the $f$-bracket by
\ben
\{\cO, \cO'\}_f := \cO_f(0, r) \star \cO' \in \sObs_0
\een
where $\star$ denotes the factorization product of a small disk with a small neighborhood of $S^{2d-1}_{0,r}$. 

\end{enumerate}

\subsubsection{}

The observables of the $\beta\gamma$ system comes naturally equipped with null-homotopies of the operators $\frac{\partial}{\partial \zbar_i}$. 

So far, in Section \brian{ref} we have described the space of local operators on the $d$-disk of the $\beta\gamma$ system with values in a vector space $V$. 
For disks centered at $z \in \CC^d$ there are two main classes of operators $O_\gamma (\vec{n}, z ; v^*)$ and $O_{\beta}(\vec{m}, z ; v)$ where $\vec{n} = (n_1,\ldots,n_d) \in (\ZZ_{\geq 0})^d$, $(m_1,\ldots,m_d) \in (\ZZ_{\geq 1})^d$, $v \in V$, and $v^* \in V$. 

\end{document}