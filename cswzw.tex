\documentclass[10pt]{amsart}

\usepackage{macros,slashed}

\linespread{1.25}

\usepackage{tikz}
\usetikzlibrary{arrows,shapes}
\usetikzlibrary{trees}
\usetikzlibrary{matrix,arrows}
\usetikzlibrary{positioning}
\usetikzlibrary{calc,through}
\usetikzlibrary{decorations.pathreplacing}
\usepackage{pgffor}

\title{Chern-Simons/WZW correspondence}

\def\brian{\textcolor{blue}{BW: }\textcolor{blue}}

\begin{document}
\maketitle

As a warmup to ...

\begin{prop}
The algebra of operators on the boundary of perturbative Chern-Simons theory at level $k$ is the Kac-Moody algebra at level $k - h^\vee$ where $h^\vee$ is the dual Coxeter number of $\fg$.
\end{prop}

If $M$ is any $3$-manifold, perturbative Chern-Simons theory at level $k$ in the BV formalism has fields
\ben
A \in \Omega^*(M) \tensor \fg [1]
\een
with action functional $S(A) = \int_M CS(A)$ where $CS(A)$ is the usual Chern-Simons $3$-form. 

We start with Chern-Simons theory at level $k$ placed on a $3$-manifold of the form
\ben
\Sigma \times \RR
\een
where $\Sigma$ is a Riemann surface. 
Soon we will replace $\RR$ by $\RR_{\geq 0}$ and discuss the boundary condition that leads to the Kac-Moody vertex algebra.
In choosing a Riemann surface we have fixed a complex structure on $\Sigma$. 
We can write our fields in a way that reflects this complex structure. 
Indeed, the fields together with the linearized BRST differential take the form
\ben
\xymatrix{
\Omega^{0,*}(\Sigma) \tensor \Omega^*(\RR) \tensor \fg[1] \ar[r]^-{\partial} & \Omega^{1,*}(\Sigma) \tensor \Omega^*(\RR) \tensor \fg .}
\een
Here, the $\dbar$ differential on $\Sigma$ and the de Rham differential $\d_{dR}$ on $\RR$ are implicit in the notation $\Omega^{i,*}(\Sigma) \tensor \Omega^*(\RR)$ for $i=0,1$. 
We will write fields with respect to this decomposition as $A_+ \in \Omega^{0,*}(\Sigma) \tensor \Omega^*(\RR) \tensor \fg[1]$ and $A_- \in \Omega^{1,*}(\Sigma) \tensor \Omega^*(\RR) \tensor \fg$. 
In this notation, the Chern-Simons action functional reads
\ben
S(A_+ + A_-) = \frac{ik}{4 \pi} \int \left(\<A_- , F(A_+)\>_\fg + \frac{1}{2} \<A_+ , \d A_-\>_\fg \right)
\een
where $F(A_+) = \d A_+ + \frac{1}{2} [A_+, A_+]$, and where we have fixed an invariant non-degenerate pairing $\<-,-\>_\fg$ on $\fg$. 
If one makes the change of coordinates $A = A_+$, $B = \frac{ik}{4\pi} A_-$, one can rewrite the action functional as
\ben
S(A,B) = \int \<B , F(A)\>_\fg + \frac{ik}{8 \pi} \int \<A, \partial A\>_\fg
\een
where we have used the fact that if $A \in \Omega^{0,*}(\Sigma) \tensor \Omega^*(\RR)$ then $A \d A = A \partial A$. 
Note that in the limit $k \to 0$ we recover precisely $3$-dimensional $BF$ theory, which we discussed in Section \brian{ref}. 
This level zero limit is also referred to as Chern-Simons at the critical level \brian{ref} because of the one-loop effect that we will soon witness. 

We have formulated topological $BF$ theory is a cotangent BV theory.
It is constructive to think about Chern-Simons theory on $\Sigma \times \RR$ as a {\em twisted} cotangent theory.
Indeed, if $\sL$ denotes the local Lie algebra $\Omega^{0,*}(\Sigma) \tensor \fg$, note that when $k \to 0$ the theory coincides with the formal moduli space $T^*[-1] B \sL$. 
For finite $k$, we can realize the theory as a linear deformation by the map $\partial : \sL \to \sL^! [-1]$.
We will denote this twisted (and shifted) cotangent bundle describing the theory by $T_\partial^*[-1] \sL$. 

\subsection{The chiral boundary condition}

We now consider putting Chern-Simons theory on the manifold with boundary $\Sigma \times \RR_{\geq 0}$.
In order to do this, we must fix a boundary condition at $\Sigma \times \{0\}$. 
The boundary condition we choose is defined by
\ben
\left\{A|_\Sigma = 0 \right\} \subset {\rm Fields}|_{\Sigma}.
\een

This boundary condition defines a Lagrangian subspace of the space of fields supported on the boundary. 
Indeed, the boundary fields are given by twisted cotangent bundle $T^*_{\partial} B (\sL|_\Sigma)$ where
\ben
\sL|_\Sigma = \Omega^{0,*}(\Sigma) \tensor \fg
\een 
Again, $T_\partial$ means that we are deforming the cotangent bundle by the linear operator $\partial : \sL \to \sL^!$ mapping the base to the cotangent fiber:
\ben
T^*_{\partial} B (\sL|_\Sigma) = \Omega^{0,*}(\Sigma) \tensor \fg[1] \xto{\partial} \Omega^{1,*}(\Sigma) \tensor \fg .
\een
The $(-1)$-shifted symplectic structure defined on the bulk fields restricts to an ordinary ($0$-shifted) symplectic structure on the fields supported on the boundary.
Concretely, the boundary condition we choose is given by the Lagrangian subspace 
\ben
\Omega^{1,*}(\Sigma) \tensor \fg  \subset T^*_{\partial} B (\sL|_\Sigma) , 
\een
which is precisely the cotangent fiber of the twisted cotangent bundle.
Note that $\Omega^{1,0}(\Sigma) \tensor \fg$ sits in degree zero inside of the boundary fields.

The observables on the boundary are functions on the Lagrangian subspace determining the boundary condition. 
So, for us, the boundary observables are functions of only the $B$-fields in the coordinates we used above to define Chern-Simons.
\brian{describe these $\cO_{B, n}(z)$}

The bulk observables of any perturbative quantum field theory have the structure of a factorization algebra.
Classically, the factorization algebra of observables has a $P_0$ structure, induced from the $(-1)$-symplectic pairing on fields. 
It is the main result of \cite{ButsonYoo} that for sufficiently well-behaved boundary conditions of field theories that are topological in the direction normal to the boundary that the classical boundary observables form a factorization algebra with a $P_0$ structure compatible with the $P_0$ structure on the bulk observables. 
In general, and is the case here, this $P_0$ structure is not induced from a symplectic structure.
It could happen that the bracket defining the $P_0$ structure be degenerate.
For this reason, such boundary conditions are said to define {\em degenerate} theories on the boundary. 

\begin{prop}
The classical limit of the Kac-Moody factorization algebra on $\Sigma$ is quasi-isomorphic to the factorization algebra of classical boundary operators of Chern-Simons theory at level $k$ with the chiral boundary condition.
The $P_0$ structure on boundary operators agrees with the $P_0$ structure on the classical limit of Kac-Moody given by
\ben
\Pi = [-,-]_\fg + \# k \partial_\Sigma .
\een
\end{prop}

\subsection{The boundary OPE}

In this section we compute the boundary OPE of Chern-Simons theory on the $3$-manifold $\CC \times \RR_{\geq 0}$ with chiral boundary condition introduced in the previous section. 

\brian{derive propagator}
We conclude that the propagator, with both UV and IR cutoffs, is of the form
\be\label{3dprop}
P_{\epsilon<L} (z,t;w,s) = \left(\d t \frac{\partial}{\partial z} - \d \zbar \frac{\partial}{\partial t}\right) \int_{T = \epsilon}^L \frac{\d T}{(4 \pi T)^{3/2}} e^{-r^2 / 4 T}, 
\ee
where $r^2 = (t-s)^2 + |z-w|^2$. 
In the limit $\epsilon \to 0$, $L\to \infty$ the propagator has a more palatable form.
Indeed, making the substitution $u = r^2 / 4$ we can write the above as
\ben
P_{\epsilon}^L (z,t;w,s) = \left(\d t \frac{\partial}{\partial z} - \d \zbar \frac{\partial}{\partial t}\right) \left(\frac{1}{r}\right) \int_{\epsilon}^L u^{1/2} e^{-u} \d u
\een
\brian{work out factors of $2,\pi$.}
In the desired limit we can evaluate the Gaussian to obtain
\ben
P (z,t;w,s) = \lim_{\epsilon \to 0}\lim_{T \to \infty} P_{\epsilon}^L (z,t;w,s) = C \left(\d t \frac{\partial}{\partial z} - \d \zbar \frac{\partial}{\partial t}\right) \left(\frac{1}{r}\right) .
\een


\begin{prop}
The boundary OPE between the operators $\cO_{B,n}(z)$ agrees with the OPE of the Kac-Moody vertex algebra at level $k - h^\vee$. 
\end{prop}

\begin{proof}

For non-zero $k$, we treat the term $(k / 8\pi) \int A \partial A$ as an interaction term of the theory. 
In general, this would be an ill-defined thing to do, since bivalent vertices can span unbounded Feynman diagrams, but due to the nature of the symplectic pairing in the bulk theory the Feynman expansion is well behaved. 
Alternatively, we could introduce a formal deformation parameter attached to this vertex, and expand formally in this parameter. 
This added complication is unnecessary here. 

\ben
\cO_{B,-1}(M_1, z_1) \cdot \cO_{B,-1}(M_2, z_2) = F_{{\rm I}} (z_1,z_2) + F_{{\rm II}}(z_1,z_2) + F_{{\rm III}} (z_1,z_2)
\een

The first diagram is linear in the field $B$ and contributes the following quantity
\ben
F_{\rm I}(z_1,z_2)(B) = \int_{(z,t) \in \CC \times \RR_{\geq 0}} \<B, [A(M_1, z_1) , A(M_2, z_2)] \>_\fg,
\een
where the one-forms $A(M_i,z_i)$ are defined by pairing the bulk-boundary propagator $P(z_i, 0 ; z,t)$ with the linear operator $\cO_{B, -1}(M_i, z_i)$. 
Using the form of the bulk-boundary propagator \brian{above} we see 
\ben
A(M_i,z_i) = \left(\d \zbar \frac{\partial}{\partial t} - \d t \frac{\partial}{\partial z}\right) \left(t^2 + |z-z_i|^2\right)^{-1/2} \tensor M_i 
\een
as an element in $\Omega^{0,*}(\CC \times \RR_{\geq 0}) \tensor \fg$.
The bracket is given by
\ben
[A(M_1, z_1) , A(M_2, z_2)] = \left(\d \zbar \frac{\partial}{\partial t} - \d t \frac{\partial}{\partial z}\right) \left(t^2 + |z-z_1|^2\right)^{-1/2} \left(\d \zbar \frac{\partial}{\partial t} - \d t \frac{\partial}{\partial z}\right) \left(t^2 + |z-z_2|^2\right)^{-1/2} \tensor [M_1,M_2] .
\een

On shell, the fields on the boundary are required to be holomorphic. 
Thus, we may assume that $B$ is a holomorphic one-form $B = f(z) \d z \tensor M$, where $M \in \fg$. 
Thus, the contribution from diagram I reduces to
\ben
\<M, [M_1, M_2]\>_{\fg} \int_{(z,t) \in \CC \times \RR_{\geq 0}} f(z) \left(\d \zbar \frac{\partial}{\partial t} - \d t \frac{\partial}{\partial z}\right) \left(t^2 + |z-z_1|^2\right)^{-1/2} \left(\d \zbar \frac{\partial}{\partial t} - \d t \frac{\partial}{\partial z}\right) \left(t^2 + |z-z_2|^2\right)^{-1/2} \d z .
\een
In Section 17 of \cite{Omega} Costello has computed this 3-dimensional integral. 
He shows that it is equal to an expression of the form
\ben
C \frac{\<M, [M_1, M_2]\>_{\fg}}{z_1-z_2} + g(z_1-z_2)
\een
where $g(z_1-z_2)$ is holomorphic in the variable $z_1-z_2$ and $C$ is some constant involving the value of the holomorphic function $f$ at $z = 0$.  
Of course, $g$ does not contribute to the OPE. 
We conclude that 
\ben
F_{\rm I}(z_1,z_2) \sim c \frac{[M_1,M_2]}{z_1-z_2} 
\een 
where the $\sim$ denotes the non-singular contribution. 

The next term is $F_{\rm II}(z_1,z_2)$ which corresponds to the diagram II in the figure above. 
The contribution is computed similarly to diagram I. 
The Lie algebra factor is given by $k \<M_1,M_2\>_\fg$. 
The analytic part is of the form
\ben
\int_{(z,t) \in \CC \times \RR_{\geq 0}} \left(\d \zbar \frac{\partial}{\partial t} - \d t \frac{\partial}{\partial z}\right) \left(t^2 + |z-z_1|^2\right)^{-1/2} \frac{\partial}{\partial z} \left(\d \zbar \frac{\partial}{\partial t} - \d t \frac{\partial}{\partial z}\right) \left(t^2 + |z-z_2|^2\right)^{-1/2} \d z . 
\een
Note that when applied to the function $(t^2 + |z-z_2|^2)^{-1/2}$ the operator $\frac{\partial}{\partial z}$ agrees with $\frac{\partial}{\partial z_2}$. 
Thus we can pull it out of the integral, and we obtain by the same calculation above obtain
\ben
F_{\rm II}(z_1,z_2) \sim c k \frac{\partial}{\partial z_2} \frac{\<M_1,M_2\>_\fg}{z_1-z_2} = c  k \frac{\<M_1,M_2\>_\fg}{(z_1-z_2)^2} .
\een

Finally, we turn to diagram III. 
This will contribute to the well-established shift of the critical level of Chern-Simons at the one-loop level. 
See for instance \cite{WittenCS}.
This diagram involves a loop, so there could potentially be UV divergences. 
We find that once we regularize, using our convenient gauge fixing operator, that the UV divergence does not contribute to the OPE and we get a finite correction to level. 
We will label the boundary to bulk edges by the full propagator $P(z_i,0;z,t)$ as above. 
We label the internal edges of the loop, which occurs in the bulk, by the regularized propagator $P_{\epsilon<L} (z,t;w,s)$ from (\ref{3dprop}).
Note that we can split the function on $C^\infty((\RR_{\geq 0} \times \CC) \times (\RR_{\geq 0} \times \CC)$ as
\ben
\frac{1}{(4 \pi T)^{3/2}} e^{-((t-s)^2 + |z-w|^2) / 4 T} = \left(\frac{1}{(4\pi T)^{1/2}} e^{-(t-s)^2/ 4T} \right)\left(\frac{1}{4 \pi T} e^{-|z-w|^2 / 4T}\right) . 
\een
The right-hand side is simply the product of two heat kernels $K^{1d}_{T}(t,s) K_T^{2d}(z,w)$, where $K^{1d}_T$ is the heat kernel for the one-dimensional Laplacian, and $K^{2d}_T$ is the heat kernel for the two-dimensional Laplacian on $\CC$. 
It follows that we can write the regularized propagator as
\begin{align*}
P_{\epsilon < L}(z,t;w,s) & = \left(\d \zbar \frac{\partial}{\partial t} - \d t \frac{\partial}{\partial z}\right) \int_{T = \epsilon}^L \d T \left(\frac{1}{(4\pi T)^{1/2}} e^{-(t-s)^2/ 4T} \right)\left(\frac{1}{4 \pi T} e^{-|z-w|^2 / 4T}\right) \\ & = \d \zbar \int_{T = \epsilon}^L \d T \left(\frac{\partial}{\partial t} \frac{1}{(4\pi T)^{1/2}} e^{-(t-s)^2/ 4T} \right) \left(\frac{1}{4 \pi T} e^{-|z-w|^2 / 4T}\right) \\ & - \d t \int_{T = \epsilon}^L \d T \left(\frac{\partial}{\partial z}\frac{1}{4 \pi T} e^{-|z-w|^2 / 4T}\right)\left(\frac{1}{(4\pi T)^{1/2}} e^{-(t-s)^2/ 4T} \right) \\ 
& = \d \zbar \int_{T=\epsilon} \d T \frac{\partial}{\partial t} K_T^{1d} (t,s) K_T^{2d}(z,w) - \d t \int_{T = \epsilon}^L \d T \frac{\partial}{\partial z} K_T^{2d}(z,w) K_T^{1d}(s,t) .
\end{align*}

We now move on to compute $F_{\rm III}(z_1,z_2)$. 
The algebraic weight of the \brian{finish}
For simplicity, set $\alpha(z_i; z,t) = \left(\d \zbar \frac{\partial}{\partial t} - \d t \frac{\partial}{\partial z}\right) \left(t^2 + |z-z_1|^2\right)^{-1/2}$.
Here, $z_i$ is fixed on the boundary and $(z,t) \in \CC \times \RR_{\geq 0}$. 
The analytic part of the regularized weight is of the form
\ben
\int_{(z, t ; w, s) \in (\CC \times \RR_{\geq 0})^2} \alpha(z_1, z) \alpha(z_2, w) P_{\epsilon < L}(z,t;w,s) P_{\epsilon < L}(w,s;z,t) .
\een
Using the splitting of the propagator above, we see that we can write this integral as a sum of two integrals.
The first is of the form
\ben
\int_{(z, t ; w, s) \in (\CC \times \RR_{\geq 0})^2}  \alpha(z_1; z,t)\alpha(z_2;w,s) \d z \d w \d \zbar \d s \int_{T_1,T_2= \epsilon}^L \d T_1 \d T_2 \left(K_{T_1}^{1d}(s,t) \frac{\partial}{\partial t} K_{T_2}^{1d}(s,t)  \right) \left(K_{T_2}^{2d}(z,w) \frac{\partial}{\partial z} K_{T_1}^{2d}(z,w)\right) .
\een
The second term simply flips $(z,t) \leftrightarrow (w,s)$.

We now perform a familiar trick.
We will evaluate, using the Wick expansion, the Gaussian integral for the diagonal coordinate $z - w$ in the $\CC \hookrightarrow \CC \times \CC$ direction. 
Doing this, the integral reduces to
\be\label{CS oneloop1}
C \int_{z \in \CC , (t,s) \in (\RR_{\geq 0})^2} \alpha(z_1; z, t) \frac{\partial}{\partial z} \alpha(z_2 ; z, s) \d z \d s \int_{T_1,T_2 = \epsilon}^L \d T_1 \d T_2 \left(K_{T_1}^{1d}(t,s) \frac{\partial}{\partial t} K_{T_2}^{1d}(t,s)  \right) \frac{T_2}{(T_1 + T_2)^2}  . 
\ee
Now, consider the quantity 
\ben
K_{T_1}^{1d}(t,s) \frac{\partial}{\partial t} K_{T_2}(t,s) \d s = \frac{1}{4 \pi} \frac{1}{T_1^{1/2} T_2^{3/2}} (t-s) e^{-(t-s)^2 \left(\frac{1}{T_1} + \frac{1}{T_2}\right) /4} \d s .
\een
Making the substitution $u = (t-s) \left(\frac{1}{T_1} + \frac{1}{T_2}\right)$ this expression becomes 
\ben
\frac{1}{T_1^{1/2}T_2^{3/2}} \frac{T_1T_2}{T_1+T_2} u e^{-u^2/4} \d u .
\een
Evaluating the $u$-integral we see that (\ref{CS oneloop}) reduces to 
\ben
\int_{(z,t) \in \CC \times \RR_{\geq 0}} \alpha(z_1; z,t) \frac{\partial}{\partial z} \alpha(z_2 ; z,t) f(t) \d z \d t \int_{T_1,T_2 = \epsilon}^L \d T_1 \d T_2 \frac{T_1^{1/2}T_2^{1/2}}{(T_1 + T_2)^3} .
\een
One checks immediately that the $\epsilon \to 0$ limit of the $T_1,T_2$ integral exists and is independent of $L$. 
Moreover, ... 
\end{proof}ggg

eiorjeijmelkfmdkfmdkfmmfmfmf


\end{document}