\documentclass[10pt]{amsart}

\usepackage{macros,slashed}

\linespread{1.25}

\usepackage{tikz}
\usetikzlibrary{arrows,shapes}
\usetikzlibrary{trees}
\usetikzlibrary{matrix,arrows}
\usetikzlibrary{positioning}
\usetikzlibrary{calc,through}
\usetikzlibrary{decorations.pathreplacing}
\usepackage{pgffor}

\title{Random analysis}

\def\brian{\textcolor{blue}{BW: }\textcolor{blue}}



\begin{document}
\maketitle

\section{The higher OPE and descent}

In this section we provide a partial analysis of the higher operator product expansion present in holomorphically translation invariant quantum field theories. 

\subsection{The observables on the $d$-disk}

In this section we give a description of the observables of the $\beta\gamma$ system supported on a $d$-disk inside $\CC^d$. 
For now, we will only consider the free $\beta\gamma$ system with target a complex vector space $V$. 
Thus the observables are..\brian{finish}

We would like to study a class of local operators...

\subsubsection{Primaries of a CFT}

In ordinary chiral conformal field theory, there is a collection of operators that, in some sense, generate all other operators. 
These are called ``primary operators" (or primary fields), and are defined by those operators that are killed by the positive part of the Virasoro algebra \cite{polchinski}, that is, the ``lowering operators". 
To obtain all of the operators one considers the descendants of the primary operators which are obtained by applying the negative part of the Virasoro algebra, or the ``raising operators", to the primaries. 
For example, in the $d=1$ $\beta\gamma$ system, there are two primary operators:
\begin{align*}
\cO_{\gamma,0} (w) & : \gamma \mapsto \gamma(w) = \int_{z \in C_w} \frac{\gamma(z)}{z-w} \d z  \\
\cO_{\beta,-1} (w) & : \beta \d z \mapsto \beta(w) = \int_{z \in C_w} \frac{\beta(z)}{z-w} \d z ,
\end{align*}
where $C_w$ is any closed contour surrounding $w$. 
(The indices $0,-1$ are to indicate the conformal weight.)
Consider the operators placed at $w=0$.
We notice that each of these operators are annihilated by the positive half of the Virasoro $L_n = z^{n+1} \partial_z$, $n \geq 0$.
The descendants are obtained by iteratively applying the raising operator $L_{-1} = \partial_z$, which in this case is just the infinitesimal translations. 
Indeed, for each $n \geq 0$ we obtain
\begin{align*}
\cO_{\gamma,-n}(w) = \frac{1}{n!} \partial^n \cO_{\gamma,0}(w) & : \gamma \mapsto \partial_z^n \gamma(z = w) \\
\cO_{\beta,-n-1}(w) = \frac{1}{n!} \partial^n \cO_{\beta,1}(w) & : \beta \d z \mapsto \partial_z^n \beta(z=w) .
\end{align*}

There is an $S^1$ action on $\CC$ given by rotations, and this extends to an $S^1$ action on the $\beta\gamma$ system.
In terms of the Virasoro algebra, the infinitesimal action of $S^1$ is given by the Euler vector field $L_0 = z \partial_z$. 
There is an induced grading on the factorization algebra of the one-dimensional free $\beta\gamma$ system by the eigenvalues of this $S^1$ action.
Applied to the disk, or local, observables this is precisely the $\ZZ_{\geq 0}$ conformal weight grading of the chiral CFT.
For instance, the operators $\cO_{\gamma,-n}(w), \cO_{\beta, -n}$ lie in the weight $n$ subspace of the factorization algebra applied to $D(w,r)$ (for any $r >0$). 
We will see a similar grading in the higher dimensional holomorphic case.

\subsubsection{}
The $\beta\gamma$ system on $\CC^d$ has a symmetry by the group unitary group $U(d)$. 
Indeed, the fields of the $\beta\gamma$ system are built from sections of certain natural holomorphic vector bundles on $\CC^d$. 
The group $U(d)$ acts by automorphisms on every holomorphic vector bundle, hence it acts on sections via the pull-back. 

There is another symmetry that we wish to contemplate. 
Introduce an action of $U(1)$ on the fields of the theory such that $V$ has weight $q_f \in \ZZ$ and $V^*$ has weight $-q_f$.
The value of the fields $\gamma$ lie in the vector space $V$, so these fields are of weight $q_f$. 
Conversely, the fields $\beta$ lie in $V^*$, so have weight $-q_f$.
Since the pairing defining the free theory is only non-zero between a single $\gamma$ and single $\beta$ field, the theory is invariant under this symmetry.
In the physics literature, this is a so-called ``flavor symmetry" of the theory, and so to distinguish it from the other symmetry we will denote this group by $U(1)_f$. 
This symmetry will be especially relevant when we compute the character of the $\beta \gamma$ system.

\begin{lem}\label{lem U(d) equivariance}  The symmetry by $U(d) \times U(1)_f$ on the classical $\beta\gamma$ system with values in the complex vector space $V$ extends to a symmetry of the factorization algebra of quantum observables $\Obs^q$.
\end{lem}
\begin{proof}
The differential on the factorization algebra is of the form $\dbar + \hbar \Delta$. 
The operator $\dbar$ is manifestly equivariant for the action of $U(d)$.
Since $U(1)_f$ does not act on spacetime, $\dbar$ trivially commutes with its action. Further, the action of $U(d)$ is through linear automorphisms, and since the BV Laplacian $\Delta$ is a second order differential operator, it certainly commutes with the action of $U(d)$. 
Likewise, since $U(1)_f$ is compatible with the $(-1)$-symplectic pairing, it automatically is compatible with $\Delta$. 
\end{proof}

We will use the action of $U(d)$ to organize the class of operators we are interested in. 
The eigenvectors of $U(d)$ are labeled by the eigenvectors of a maximal torus, which we will take to be given by the subgroup
\ben
T^d = \{{\rm diag}(q_1,\ldots,q_d) \; | \; |q_i| = 1\} \subset U(d) .
\een 
Here, $q_i \in S^1 \subset \CC^\times$ are complex numbers of unit modulus. 
We say that an element $v$ of the factorization algebra has weight $(n_1,\ldots,n_k)$ if $(q_1,\ldots,q_d) \cdot v = q_1^{n_1}\cdots q_d^{n_d} v$. 
We will use the shorthand $\vec{n} = (n_1,\ldots,n_d)$. 
\begin{dfn}
\begin{enumerate}
\item Let $w \in \CC^d$ and $r > 0$. 
For any vector of non-negative integers $\vec{n} = (n_1,\ldots, n_d)$ denote by
\ben
\Obs^\q_V(r)^{(\vec{n})} \subset \Obs^\q_{V}(D(w,r))
\een 
the subcomplex of weight $\vec{n}$ elements. 
\item 
Let
\ben
\Obs_V^\q(r) := \bigoplus_{\vec{n}} \Obs^\q_V(r)^{(\vec{n})} 
\een
where the direct sum is over all vectors of non-negative integers.
\end{enumerate}
\brian{should I make separate dfn for classical and quantum}
\end{dfn}

\begin{rmk}
Note that we have excluded $w \in \CC^d$ from the notation above. 
This is because the $\beta\gamma$ system is translation invariant. \brian{say more?}
\end{rmk}

We now introduce the following operators that will be of most relevance for our study of the operator product expansion.

\begin{dfn} Let $w \in \CC^d$ and $r > 0$.
Define the following linear observables supported on $D(w,r)$.
\begin{enumerate}
\item For $n_i \in \ZZ_{\geq 0}, i = 1,\ldots d$, and $v^* \in V^*$ define
\ben
\cO_{\gamma, -\vec{n}} (w;v^*) : \gamma \in \Omega^{0,*}(D(w,r)) \mapsto \left\<v^*,\left(\left.\frac{\partial^{n_1}}{\partial z_1^{n_1}} \cdots \frac{\partial^{n_d}}{\partial z_d^{n_d}} \gamma(z,\zbar) \right|_{z=w}\right)\right\> .
\een
Here, the brackets denote the evaluation pairing between $V^*$ and $V$. 
\item For $m_i \in \ZZ_{\geq 1}, i=1,\ldots d$, and $v \in V$ define
\ben
\cO_{\beta, -\vec{m}} (w;v) : \beta \d^d z \in \Omega^{d,*}(D(w,r)) \mapsto \left\<v ,\left(\left.\frac{\partial^{m_1-1}}{\partial z_1^{m_1-1}} \cdots \frac{\partial^{m_d-1}}{\partial z_d^{m_d-1}} \beta(z,\zbar) \right|_{z=w}\right)\right\> .
\een
\end{enumerate}
\end{dfn}

Our convention is that the evaluation of a Dolbeualt form is zero $\d \zbar_i |_{z=w} = 0$.
Thus, the above observables are only nonzero when $\gamma \in \Omega^{0,0}(D(w,r))$ and $\beta \d^d z \in \Omega^{d,0}(D(w,r))$.
In particular, this implies that these operators are of the following homogenous cohomological degree:
\begin{align*}
\deg(\cO_{\gamma, -\vec{n}} (w;v^*))  & = 0 \\
\deg(\cO_{\beta, -\vec{m}} (w ; v)) & = d-1 .
\end{align*}

The minus sign in $\cO_{\gamma, -\vec{n}}(w)$ is purely conventional, and meant to match up with the physics and vertex algebra literature \brian{ref}.
One reason for using this convention is to note that for any $d$-disk $D(0,r)$ there is an embedding of topological vector spaces
\ben
z_1^{-1} \cdots z_d^{-1} \CC[z_1^{-1}, \cdots, z_d^{-1}] \to \left(\Omega^{0,*}(D(w,r))\right)^\vee 
\een
that sends a Laurent polynomial $f(z)$ functional
\ben
\gamma \in \Omega^{0,*}(D(w,r)) \mapsto \oint_{z \in S^{2d-1}} f(z-w) \gamma(z,\zbar) \d^d z \wedge \omega_{BM}(z-w,\zbar-\wbar) ,
\een
where $\omega_{BM}$ is the Bochner-Martinelli form of type $(0,d-1)$, and $S^{2d-1}$ is the sphere of radius $r$ around $w$.

\begin{rmk}
This is the analog of the fact that when $d=1$ there is an embedding
\ben
z^{-1} \CC [z^{-1}] \to \left(\sO^{hol}(D)\right)^\vee
\een
sending $f(z)$ to the residue functional $g \mapsto \Res_{z} (f(z) g(z) \d z)$. 
\end{rmk}

Similarly, there is an embedding $z_1^{-1} \cdots z_d^{-1} \CC[z_1^{-1}, \ldots,z_d^{-1}] \to \left(\Omega^{d,*}(D(w,r)\right)^\vee$ sending $f(z)$ to the functional
\ben
\beta \in \Omega^{d,*}(D(w,r)) \mapsto \oint_{S^{2d-1}} f(z-w) \beta(z,\zbar) \wedge \omega_{BM}(z-w,\zbar-\wbar).
\een

These embeddings determine a linear map 
\ben
i_D : z_1^{-1} \cdots z_d^{-1} \CC[z_1^{-1}, \ldots,z_d^{-1}] \tensor (V^* \oplus V[-d+1]) \to \left(\Omega^{0,*}(D(w,r)) \tensor V \oplus \Omega^{d,*}(D(w,r)) \tensor V^*[d-1]\right)^\vee .
\een
The right-hand side is simply the linear observables sitting inside of $\Obs^{\cl}_V(D(w,r))$.
It follows from the higher dimensional residue formula that, for $n_i \geq 0$, the image of 
\ben
z_1^{-n_1} \cdots z_d^{-n_d} \tensor v^* \in \CC[z_1^{-1},\ldots, z_d^{-1}] \tensor V^*
\een
under this map is precisely $\cO_{\gamma, -\vec{n}}(w;v^*)$, where $\vec{n} = (n_1,\ldots,n_d)$. 
Similarly, for $m_i \geq 1$, the image of
\ben
z_{1}^{-m_1+1} \cdots z_d^{-m_d+1} \tensor v \in \CC[z_1^{-1},\ldots, z_d^{-1}] \tensor V [-d+1]
\een
under this map is $\cO_{\beta,\vec{m}}(w;v)$. 
This motivates the following general definition. 

\begin{dfn}\label{dfn disk2}
For any $f(z) \in \CC[z_1^{-1},\ldots,z_1^{-1}]$, denote by $\cO_{\gamma, f}(w; v^*)$ the image of $f\tensor v^*$ under the linear map $i_D$
\ben
\cO_{\gamma,f}(w;v^*) := i_D(f \tensor v^*)
\een
In particular, note $\cO_{\gamma, z_1^{-n_1}\cdots z_d^{-n_d}}(w;v^*) = \cO_{\gamma , -\vec{n}}(w;v^*)$.
Similarly, if $g \in z_1^{-1} \cdots z_d^{-1} \CC[z_1^{-1},\ldots,z_d^{-1}]$, let
\ben
\cO_{\beta,g}(w;v) := i_D(z_1\cdots z_d g(z) \tensor v) .
\een
\end{dfn}

\begin{lem}
Let $r < s$.
Then, the factorization structure map for including disks $D(0,r) \subset D(0,s)$ induces a diagram
\ben
\xymatrix{
\Obs_V^\q(D(0,r)) \ar[r] & \Obs_V^\q(D(0,s)) \\
\Obs_V^\q(r) \ar[u] \ar[r]^-{\simeq} & \Obs_V^\q(s) \ar[u] .
}
\een
Further, the bottom horizontal map is a quasi-isomorphism.
\end{lem}

\begin{proof}
The two vertical maps are the inclusions of the $U(d)$-eigenspaces of the observables supported on disks of radius $r$ and $s$ respectively. 
It follows from Lemma \ref{lem U(d) equivariance} that the factorization algebra is $U(d)$-equivariant, so in particular the factorization algebra structure map for the inclusion of disks $D(0,r) \hookrightarrow D(0,s)$ is a map of $U(d)$-representations. 
Hence, the map restricts to each of the eigenspaces, yielding the diagram. 

In \cite{fact1} it is shown in Corollary 5.3.6.4 that for the one-dimensional $\beta\gamma$ system, the lower map above is a quasi-isomorphism. 
In fact, a similar argument applies to the $\beta\gamma$ system on $\CC^d$. 
Indeed, consider the collection
\ben
\{\cO_{\gamma, -\vec{n}_1} (0 ; v_1^*) \cdot \cO_{\gamma, -\vec{n}_k} (0 ; v_k^*) \cdot \cO_{\beta, -\vec{m}_1}(0;v_1) \cdots \cO_{\beta, - \vec{m}_l}(0;v_l)\}. 
\een
The collection runs over non-negative integers $k,l$ and sequences $\vec{n}_i = (n_{i,1},\ldots,n_{i,d})$, $n_{i,j} \geq 0$ and $\vec{m}_i = (m_{i,1},\ldots,m_{i,d})$, $m_{i,1} \geq 1$. 
It also runs over vectors $v_i, v_j^*$ in $V$ and $V^*$, respectively. 
Now, it follows from Lemma 5.3.6.2 of \cite{fact1} that the above collection form a basis for the cohomology
\ben
H^*\Obs^\cl_V(r)^{(\vec{N})} \subset H^*\Obs^\cl(D(0,r))
\een
for any $r$, where $\vec{N} = (N_1,\ldots,N_d)$
\ben
N_j = \left(n_{1,j} + \cdots + n_{k,j}\right) + \left(m_{1,j} + \cdots + m_{l,j}\right) .
\een
The result for the quantum observables follows from the spectral sequence \brian{finish}
\end{proof}

We will denote $\cV_V = \Obs_V^\q(r)$, which is well-defined up to quasi-isomorphism by the preceding proposition. 
This is the ``state space" of the higher dimensional holomorphic theory. 

\subsection{The sphere observables}

We now provide a description of the value of the factorization algebra of observables of the $\beta\gamma$ system applied to another important class of open sets in $\CC^d$: neighborhoods of the $(2d-1)$-sphere $S^{2d-1} \subset \CC^d$. 

We first describe the precise open neighborhoods of the $(2d-1)$-sphere that we will consider.
Denote the closed $d$-disk centered at $w$ of radius $r$ by
\ben
\Bar{D}(w,r) = \left\{(z_1,z_2) \in \CC^2 \; | \; |z-w| \leq r^2\right\} . 
\een
As above, the open disk is denoted $D(w,r)$. 
Let $\epsilon,r > 0$ be such that $0 < \epsilon < r$, and consider the open submanifold
\ben
N_{r, \epsilon}(w) := D(w,r + \epsilon) \setminus \Bar{D}(w, r-\epsilon) \subset \CC^d \setminus \{w\} .
\een 
For any $\epsilon > 0$, the open set $N_{r,\epsilon}$ is a neighborhood of the closed submanifold given by the sphere of radius $r$ centered at $w$, $S^{2d-1}_r(w) \subset \CC^d \setminus \{w\}$. 
Note that when $d=1$, $N_{r,\epsilon}$ is simply an annulus centered at $w$. 

Like in the case of a disk, it is convenient to get our hands on a class of simple observables supported on $N_{r,\epsilon}(w)$. 
First, we describe linear functionals on the Dolbeualt complex of $N_{r,\epsilon}(w)$. 
This will lead naturally to a description of linear observables supported on this neighborhood. 

\brian{describe the complex $A_d$}

\begin{lem}
For any neighborhood $N_{r,\epsilon}(w)$ as above, the residue along the $(2d-1)$-sphere centered at $w$ of radius $r$, $S^{2d-1}_r(w)$, determines an embedding of topological dg vector spaces
\ben
i_{S^{2d-1}} : A_{d} [d-1] \to \left(\Omega^{0,*}(N_{r,\epsilon}(w)\right)^\vee
\een
sending $\alpha \in A_d$ to the functional
\ben
i_{S^{2d-1}}(\alpha) : \omega \in \Omega^{0,*}(N_{r,\epsilon}(w)) \mapsto \oint_{S^{2d-1}_r(w)} \alpha \wedge \d^d z \wedge \omega .
\een
\end{lem}
\begin{proof}
This is a consequence of Stokes theorem. 
Suppose $\alpha = \dbar \alpha '$. 
Then, for any $\omega \in \Omega^{0,*}(N_{r,\epsilon}(w))$ we have
\ben
\oint_{S^{2d-1}} (\dbar \alpha') \wedge \d^d z \wedge \omega = \oint_{S^{2d-1}} \alpha' \wedge \d^d z \wedge \dbar \omega .
\een
The right-hand side is simply $(\dbar i_N)(\omega) = i_N(\dbar \omega)$. 
\end{proof}

Similarly, there is an embedding $A_d [d-1] \to \left(\Omega^{d,*}(N_{r,\epsilon}(w)\right)^\vee$
sending $\alpha \in A_{d} [d-1]$ to the functional
\ben
\eta \in \Omega^{d,*}(N_{r,\epsilon}(w)) \mapsto \int_{S^{2d-1}_r(w)} \alpha \wedge \eta .
\een

These two embeddings allow us to provide a succinct description of the class of linear operators on $N_{r,\epsilon}(w)$ we will be mostly interested in. 
Indeed, the linear embeddings above, determine a cochain map (that we proceed to denote by the same symbol):
\ben
i_{S^{2d-1}} : A_d \tensor \left(V^*[d-1] \oplus V\right) \to \left(\Omega^{0,*}(N_{r,\epsilon}(w)) \tensor V \oplus \Omega^{d,*}(N_{r,\epsilon}(w)) \tensor V^*[d-1] \right)^\vee \subset \Obs_V^{\cl}\left(N_{r,\epsilon}(w)\right).
\een

\begin{dfn}
Let $\alpha \in A_{d}$ and $v^* \in V^*$.
Define the linear observable
\ben
\cO_{\gamma, \alpha}(w ; v^*) := i_{S^{2d-1}}(\alpha \tensor v^*) \in \Obs^{\cl}(N_{r,\epsilon}(w)) .
\een 
Likewise, for $v \in V$, define
\ben
\cO_{\beta, z_{1}^{-1} \cdots z_d^{-1} \alpha} (w;v) := i_{S^{2d-1}}(\alpha \tensor v) .
\een 
\end{dfn}

\begin{dfn}
Define the dg vector space of {\em classical sphere observables} to be
\ben
\sA^{\cl}_V :=  \Sym \left(A_d \tensor \left(V^*[d-1] \oplus V\right)\right)
\een
equipped with the differential coming from $A_d$. 
\end{dfn}

Note that $A_d$ has the structure of a commutative dg algebra, but we are not using the multiplication here.
The same construction above, applied now to symmetric products of linear operators, determines a cochain map $i_{S^{2d-1}} : \sA^{\cl}_{V} \to \Obs^{\cl}(N_{r,\epsilon}(w))$.

Let $\sA_{V} = \sA_V^{\cl}[\hbar]$.
Then, since $\Delta|_{\sA_V} = 0$, we see that $i_{S^{2d-1}}$ extends to a cochain map
\ben
i_{S^{2d-1}} : \sA_{V} \to \Obs^\q_V(N_{r,\epsilon}(w)) .
\een
We will refer to $\sA_V$ as the {\em quantum sphere observables}, or when there is no confusion, the sphere observables. 

The cohomology of $A_d$ is concentrated in degrees $0$ and $d-1$. 
Explicitly, one can represent the zeroeth cohomology as
\ben
H^0(A_d) = \CC[z_1,\ldots,z_d] .
\een
Now, let $\omega_{BM}(z,\zbar)$ be the Bochner-Martinelli kernel of type $(0,d-1)$ from above. 
We can express the $(d-1)$st cohomology of $A_d$ as
\ben
H^{d-1}(A_d) = \CC[\partial_{z_1}, \cdots, \partial_{z_d}] \cdot \omega_{BM} 
\een 
That is, every element of $H^{d-1}(A_d)$ can be written as a holomorphic polynomial differential operator acting on $\omega_{BM}$. 
Further, it is convenient to make the $U(d)$-equivariant identification 
\be\label{U(d) identification}
 \CC[\partial_{z_1}, \cdots, \partial_{z_d}] \omega_{BM} \cong z_1^{-1} \cdots z_d^{-1} \CC[z_1^{-1}, \ldots, z_d^{-1}],
 \ee
which makes sense since $\omega_{BM}$ has $T^d \subset U(d)$-weight $(-1,\ldots,-1)$. 

\subsubsection{The associative algebra associated to the sphere}

So far we have not discussed the structure that the factorization product puts on the dg vector space $\sA_V$. 
To recover this structure, we will only be concerned with open sets that are neighborhoods of spheres, as in the previous section. 
The configurations of open sets we consider are given by nesting the neighborhoods of the form $N_{r,\epsilon}(w)$, where $w$ is a fixed center.

For simplicity, we assume that our spheres and neighborhoods are all centered at $w=0$.
For $\epsilon < r$ we have defined the open neighborhood $N_{r,\epsilon}=N_{r,\epsilon}(0)$ of the sphere $S^{2d-1}_r$ centered at zero.
Pick positive numbers $0 < \epsilon_i < r_i$ such that $r_1 < r < r_2$, $\epsilon_1 < r - r_1$, and $\epsilon_2 < r_2 - r$.
Finally, suppose $r > \epsilon > \max\{r - r_1 + \epsilon_1, r_2 - r + \epsilon_2\}$. 
We consider the factorization product structure map for $\Obs^{\q}_{V}$ corresponding to the following embedding of open sets
\be\label{fact product 1}
N_{r_1, \epsilon_1} \sqcup N_{r_2, \epsilon_2} \hookrightarrow N_{r, \epsilon}  ,
\ee
shown schematically in Figure \brian{figure}. 
The factorization structure map for this embedding of disjoint open sets is of the form 
\be\label{fact product 2}
\Obs^{\q}_{V}(N_{r_1, \epsilon_1}) \tensor \Obs^{\q}_{V}(N_{r_2, \epsilon_2}) \to \Obs^{\q}_{V}(N_{r,\epsilon}) .
\ee
%We will see that the specific choices of $r, r_i$ and $\epsilon, \epsilon_i$ are not important.

\begin{lem} \label{lem sphere alg} The factorization structure map in (\ref{fact product 2}) restricts to the subspace of sphere observables. 
That is, there is a commutative diagram
\ben
\xymatrix{
\Obs^{\q}_{V}(N_{r_1, \epsilon_1}) \tensor \Obs^{\q}_{V}(N_{r_2, \epsilon_2}) \ar[r] & \Obs^{\q}_{V} \\
\sA_V \tensor \sA_V \ar[u] \ar[r]^-{\mu_2} & \sA_V \ar[u]
}
\een
where the top line is the map in (\ref{fact product 2}). 
The same holds for an arbitrary number of nested neighborhoods of the form $N_{r,\epsilon}$. 
That is, for any $k \geq 0$ the factorization product restricts to a linear map 
\ben
\mu_k : \sA_V^{\tensor k} \to \sA_V .
\een
\end{lem}

Each of the neighborhoods $N_{r,\epsilon}$ are contained in the open submanifold $\CC^{d} \setminus \{0\}$.
Note that there is a homeomorphism $\CC^{d} \setminus \{0\} \cong S^{2d-1} \times \RR_{>0}$.
Further, we have the radial projection map
\ben
\pi : \CC^{d} \setminus \{0\} = S^{2d-1} \times \RR_{>0} \to \RR_{>0} 
\een
that sends $z = (z_1,\ldots,z_d) \mapsto |z| = \sqrt{|z_1|^2+\cdots+|z_d|^2}$. 

A fundamental feature of factorization algebras is that they push forward along smooth maps. 
We can thus push forward the factorization algebra $\Obs^\q_V$ on $\CC^{d}\setminus \{0\}$ along $\pi$ to obtain a factorization algebra on $\RR_{>0}$. 
To an open interval of the form $(r-\epsilon, r+\epsilon)\subset \RR_{>0}$ the factorization algebra assigns precisely the observables supported on $N_{r,\epsilon}$. 

Lemma \ref{lem sphere alg} implies that the inclusion $\sA_V \hookrightarrow \Obs^\q(N_{r,\epsilon})$ induces a map of factorization algebras on $\RR_{>0}$:
\ben
\sF_{\sA_V} \to \pi_*(\Obs^\q_V) 
\een
where $\sF_{\sA_V}$ assigns, to every interval, the dg vector space $\sA_V$. 
In particular $\sF_{\sA_V}$ is locally constant, and hence determines the structure of an associative (\brian{really} $A_\infty$) dg algebra on $\sA_V$. 

We would now like to identify this algebra structure. 
To do this, note that we can view $\sA_V$ as the symmetric algebra on the following dg vector space
\ben
A_d \tensor (V^*[d-1] \tensor V) \oplus \CC \cdot \hbar .
\een
This dg vector space has the structure of a dg Lie algebra, with bracket given by
\ben
[\alpha \tensor v^*, \alpha \tensor v] = \hbar \<v^*, v\> \oint_{S^{2d-1}} \d^d z \; \alpha \wedge \alpha'  .
\een
All other brackets are determined by graded anti-symmetry.
Moreover, the parameter $\hbar$ is central.
Denote this dg Lie algebra by $\sH_V$. 

\begin{prop}
There is a quasi-isomorphism of associative dg algebras
\ben
\sA_V \simeq U\left(\sH_V\right)
\een
where $U(-)$ is the universal enveloping algebra. 
The associative dg algebra structure on the left-hand side is determined by the factorization product as in Lemma \ref{lem sphere alg} and the discussion above.
\end{prop}

\begin{rmk}
If $(\fg, \d, [-,-])$ is a dg Lie algebra its universal enveloping algebra is defined explicitly by 
\ben
U(\fg) = {\rm Tens}(\fg) / (x \tensor y - (-1)^{|x||y|} y \tensor x - [x,y]) .
\een
It is immediate to check that the differential $\d$ descends to one on $U(\fg)$, giving $U(\fg)$ the structure of an associative dg algebra.
\end{rmk}

\begin{proof}
We have written down a map of factorization algebras $\sF_{\sA_V} \to \pi_*(\Obs_V^\q)$, where $\sF_{\sA_V}$ is the locally constant factorization algebra that assigns the cochain complex $\sA_V$ to every interval, and whose factorization product is induced from $\pi_*(\Obs_V^\q)$. 

Let $\ul{U(\sH_V)}$ be the locally constant factorization algebra on $\RR_{>0}$ based on the associative algebra $U(\sH_V)$. 
We will write down an explicit quasi-isomorphism of locally constant factorization algebras
\ben
\Phi : \ul{U(\sH_V)} \to \sF_{\sA_V},
\een
implying the result. 

By Poincar\'{e}-Birkoff-Witt, the dg vector spaces $U(\sH_V)$ and $\sA_V$ are isomorphic. 
Therefore, if $I \subset \RR_{>0}$ is an interval, we define $\Phi(I)$ to be the identity map. 
We also need to take into account a higher operation. 

If $I_1,I_2$ are two intervals, consider their disjoint union $I_1 \sqcup I_2 \subset \RR_{>0}$. 
We define
\ben
\Phi(I_1 \sqcup I_2) : U(\sH_V) \tensor U({\sH_V}) \to \sF_{\sA_V} (I_1 \sqcup I_2) 

We must explicitly compute the factorization product for certain observables in $\pi_*(\Obs_V^\q)$.
\ben
[\cO_{\gamma, \alpha_1}(0;v^*), \cO_{\beta, \alpha_2}(0;v)] = \cO_{\gamma, \alpha_1}(0;v^*) \star \cO_{\beta, \alpha_2}(0;v) - \cO_{\beta,\alpha_2}(0;v) \star \cO_{\gamma, \alpha_1}(0;v^*) .
\een
We will compute the factorization product using the explicit form of the propagator of the $\beta\gamma$ system computed in Section \ref{}. 
The full propagator is an element
\ben 
P (z,w) = \lim_{L\to \infty} \lim_{\epsilon \to 0} P_{\epsilon < L}(z,w) \in \Bar{\sE}_V(\CC^d) \Hat{\tensor} \Bar{\sE}_V(\CC^d)
\een
where the $\Bar{\sE}_V(\CC^d)$ denotes the space of distributional sections on $\CC^d$.
Explicitly, we found 
\ben
P(z,w) = C_d \;\omega_{BM}(z,w) 
\een
where $\omega_{BM}(z,w)$ is the Bochner-Martinelli kernel.

Contraction with $P$ determines a degree zero, order two differential operator
\ben
\partial_{P} : \Obs^{\cl}_V (U) \to \Obs^{\cl}_{V}(U)
\een
for any open set $U \subset \CC^d$. 
Recall that the classical observables on $U$ are simply given by a symmetric algebra on the continuous dual of $\sE_V(U)$. 
Since $\Bar{\sE}^\vee = \sE_c^!$, we can view the propagator as an symmetric smooth linear map
\ben
P^\vee : \sE_{V,c}^!(\CC^d) \Hat{\tensor} \sE_{V,c}^!(\CC^d) \to \CC .
\een
The contraction operator $\partial_P$ is determined by declaring it vanishes on $\Sym^{\leq 1}$, and on $\Sym^2$ is given by the linear map $P^\vee$. 

To compute the factorization product we use the isomorphism
\ben
\begin{array}{cccc}
W_0^\infty : & \Obs^{\cl}_V(U) [\hbar]  & \to & \Obs^\q_V(U) \\
& \cO & \mapsto & e^{\hbar \partial_P} \cO 
\end{array}
\een
that makes sense for any open set $U$.
This is an isomorphism of cochain complexes, with inverse given by $(W_0^\infty)^{-1} = e^{-\hbar \partial_P}$. 
By \ref{??} it determines the following formula for the factorization product. 
If $\cO,\cO'$ are observables supported on disjoint opens $U,U'$, and $V$ is and open set containing $U,U'$, then the factorization structure map is given by
\ben
\cO \star \cO' = e^{-\hbar \partial_P} \left(\left(e^{\hbar \partial_P}\cO\right) \cdot \left(e^{\hbar \partial_P} \cO'\right)\right) \in \Obs^\q(V) .
\een 
Here, the $\cdot$ refers to the symmetric product on classical observables.

The calculation of the factorization product relies on the higher dimensional residue formula involving the Bochner-Martinelli form. 
If $f$ is any any function in $C^\infty(U)$, where $U$ is a domain in $\CC^d$, then the residue formula states that for any $z \in D$ 
\ben
f(z,\zbar) = \int_{w \in \partial U} \d^d w \; f(w) \; \omega_{BM}(z,w) - \int_{w \in D} \d^d w \; (\dbar f)(w) \wedge \omega_{BM}(z,w) .
\een 

\end{proof}







\subsubsection{The disk as a module}

We point out the potential confusion with notation for disk observables in Definition \ref{dfn disk2}. 
This is remedied by the following lemma. 

\begin{lem} 
Consider the factorization algebra structure map for the inclusion $N_{r, \epsilon}(w) \hookrightarrow D(w, R)$ where $R > r + \epsilon$:
\ben
\mu : \Obs^\q_V(N_{r,\epsilon}(w)) \to \Obs^\q_V(D(w,R)) .
\een
Then, in cohomology $H^*(\mu)|_{H^* \sA_V}$ is only nonzero on elements in
\ben
\Sym \left(H^{d-1}(A_d) \tensor (V^*[d-1] \oplus V)\right) .
\een
On the linear elements inside of this symmetric algebra, the factorization map map satisfies
\ben
H^* \mu : \cO_{\gamma , \partial_z^{\vec{n}} \omega_{BM}} \mapsto \cO_{\gamma, -\vec{n}}
\een
\end{lem}

\begin{prop}
The factorization product above gives the cohomology $H^*\sV_V$ the structure of a graded module for the associative graded algebra $H^*\sA_V$.
Moreover, there is an isomorphism of $H^*\sA_V$ modules
\ben
H^*\sV \cong H^*\sA_V \tensor_{\sA_{V,+}} \CC .
\een 
\end{prop}

The tensor product $H^*\sA_V \tensor_{\sA_{V,+}} \CC$ is equal to the induction of the trivial module along the subalgebra $\sA_{V,+} \subset H^*\sA_V$. 
In particular, it implies that as a graded vector space
\ben
H^* \sV_V \cong \sA_{V,-} [-d+1],
\een
which is immediate from our identification (\ref{U(d) identification})

\subsection{The colored operad of holomorphic disks}

%\subsection{Reduction along spheres}
%$\pi : \CC^{d} \setminus \{0\} = S^{2d-1} \times \RR_{>0} \to \RR_{>0}$
%\begin{prop}
%Suppose $\sF$ is a holomorphically translation invariant factorization algebra and \brian{assumptions}. 
%Then, the sub factorization algebra 
%\ben
%?? \subset \pi_* \sF
%\een
%is a locally constant factorization algebra on $\RR_{>0}$. 
%\end{prop}
%
%This proposition tells us that to every holomorphically translation $\sF$ invariant factorization algebra satisfying those mild conditions above there is an associated associative algebra that we will denote by $\sA_\sF$. 

\subsection{Holomorphic descent}

\subsubsection{Topological descent}

\brian{review}

\subsubsection{General theory}

We will now summarize the steps in defining the higher dimensional OPE for holomorphically translation invariant quantum field theories. 
We note that this is a schematic, and as is usual we will need to regularize at various stages to obtained a well-defined construction. 

\begin{enumerate}
\item Suppose $\cO \in \sObs_0$ is a local operator supported at $0 \in \CC^d$. 
Let $z \in \CC^d$ be another point, and consider the translated operator 
\ben
\cO(z) := \tau_z \cO .
\een 
By the property of holomorphic translation invariance, this assignment defines a $\sO^{hol}(\CC^d)$-valued local operator. 

\item We perform ``holomorphic descent" to the function valued operator $\sO^{hol}(\CC^d)$ to obtain Dolbeualt valued operator 
\ben
\cO^{(0,*)}(z) \in \Omega^{0,*}(\CC^d) \tensor \sObs_0 .
\een 
Explicitly, 
\ben
\cO^{(0,k)} (z) = \sum_{I} (\Bar{\eta}_I \cdot \cO(z)) \d \zbar_I
\een
where $I = (i_1,\ldots,i_k)$, $1 \leq i_k \leq d$, is a multi-index of length $k$ and $\eta_I = \eta_{i_1\cdots i_k}$, $\d \zbar_I = \d \zbar_{i_1} \cdots \d \zbar_{i_k}$. 

\item For any $f(z) \d^d z \in \Omega^{d,hol}(\CC^d)$, and $w \in \CC^d$, define the sphere supported operator
\ben
\cO_{f}(w, r) := \int_{z \in S^{2d-1}_{w,r}} f(z) \d^d z \cO^{(0,d-1)}(z)
\een 
where $S^3_{w,r}$ is the sphere of radius $r$ centered at $w$. 

\item If $\cO'$ is another local operator supported at zero, we define the $f$-bracket by
\ben
\{\cO, \cO'\}_f := \cO_f(0, r) \star \cO' \in \sObs_0
\een
where $\star$ denotes the factorization product of a small disk with a small neighborhood of $S^{2d-1}_{0,r}$. 

\end{enumerate}

\subsubsection{}

The observables of the $\beta\gamma$ system comes naturally equipped with null-homotopies of the operators $\frac{\partial}{\partial \zbar_i}$. 

So far, in Section \brian{ref} we have described the space of local operators on the $d$-disk of the $\beta\gamma$ system with values in a vector space $V$. 
For disks centered at $z \in \CC^d$ there are two main classes of operators $O_\gamma (\vec{n}, z ; v^*)$ and $O_{\beta}(\vec{m}, z ; v)$ where $\vec{n} = (n_1,\ldots,n_d) \in (\ZZ_{\geq 0})^d$, $(m_1,\ldots,m_d) \in (\ZZ_{\geq 1})^d$, $v \in V$, and $v^* \in V$. 


\end{document}
