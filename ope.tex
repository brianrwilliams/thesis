\documentclass[10pt]{amsart}

\usepackage{macros,slashed}

\linespread{1.25}

\usepackage{tikz}
\usetikzlibrary{arrows,shapes}
\usetikzlibrary{trees}
\usetikzlibrary{matrix,arrows}
\usetikzlibrary{positioning}
\usetikzlibrary{calc,through}
\usetikzlibrary{decorations.pathreplacing}
\usepackage{pgffor}

\title{Local observables}

\def\brian{\textcolor{blue}{BW: }\textcolor{blue}}

\begin{document}
\maketitle

\section{The higher OPE and descent}

In this section we provide a partial analysis of the higher operator product expansion present in holomorphically translation invariant quantum field theories. 

In ordinary chiral conformal field theory, there is a collection of operators that, in some sense, generate all other operators. 
These are called ``primary operators" (or primary fields), and are defined by those operators that are killed by the positive part of the Virasoro algebra \cite{polchinski}, that is, the ``lowering operators". 
To obtain all of the operators one considers the descendants of the primary operators which are obtained by applying the negative part of the Virasoro algebra, or the ``raising operators", to the primaries. 
For example, in the $d=1$ $\beta\gamma$ system, there are two primary operators:
\begin{align*}
\cO_{\gamma,0} (w) & : \gamma \mapsto \gamma(w) = \int_{z \in C_w} \frac{\gamma(z)}{z-w} \d z  \\
\cO_{\beta,-1} (w) & : \beta \d z \mapsto \beta(w) = \int_{z \in C_w} \frac{\beta(z)}{z-w} \d z ,
\end{align*}
where $C_w$ is any closed contour surrounding $w$. 
(The indices $0,-1$ are to indicate the conformal weight.)
Consider the operators placed at $w=0$.
We notice that each of these operators are annihilated by the positive half of the Virasoro $L_n = z^{n+1} \partial_z$, $n \geq 0$.
The descendants are obtained by iteratively applying the raising operator $L_{-1} = \partial_z$, which in this case is just the infinitesimal translations. 
Indeed, for each $n \geq 0$ we obtain
\begin{align*}
\cO_{\gamma,-n}(w) = \frac{1}{n!} \partial^n \cO_{\gamma,0}(w) & : \gamma \mapsto \partial_z^n \gamma(z = w) \\
\cO_{\beta,-n-1}(w) = \frac{1}{n!} \partial^n \cO_{\beta,1}(w) & : \beta \d z \mapsto \partial_z^n \beta(z=w) .
\end{align*}

There is an $S^1$ action on $\CC$ given by rotations, and this extends to an $S^1$ action on the $\beta\gamma$ system.
In terms of the Virasoro algebra, the infinitesimal action of $S^1$ is given by the Euler vector field $L_0 = z \partial_z$. 
There is an induced grading on the factorization algebra of the one-dimensional free $\beta\gamma$ system by the eigenvalues of this $S^1$ action.
Applied to the disk, or local, observables this is precisely the $\ZZ_{\geq 0}$ conformal weight grading of the chiral CFT.
For instance, the operators $\cO_{\gamma,-n}(w), \cO_{\beta, -n}$ lie in the weight $n$ subspace of the factorization algebra applied to $D(w,r)$ (for any $r >0$). 
We will see a similar grading in the higher dimensional holomorphic case.

\subsection{The observables on the $d$-disk}

In this section we give a description of the observables of the $\beta\gamma$ system supported on a $d$-disk inside $\CC^d$. 
For now, we will only consider the free $\beta\gamma$ system with target a complex vector space $V$. 
Thus the observables are..\brian{finish}

\noindent{\bf Notation}: Throughout this section $\Obs^{\q}_V$ will denote the factorization algebra of smoothed quantum observables.


\subsubsection{The cohomology of the observables}

\begin{lem}
For any $d$-dimensional disk in $\CC^d$ there is an isomorphism
\ben
H^* \left(\Obs^{\q}_V(D(w,r))\right) \cong \Sym\left( \left(\sO^{hol}(D(w,r)\right)^\vee \tensor V^* \oplus \left(\Omega^{d,hol}(D(w,r))\right)^\vee\tensor V[-d+1] \right) [\hbar]
\een
where the $(-)^\vee$ is the topological dual.
\end{lem}

\subsubsection{An explicit characterization}

The $\beta\gamma$ system on $\CC^d$ has a symmetry by the unitary group $U(d)$. 
Indeed, the fields of the $\beta\gamma$ system are built from sections of certain natural holomorphic vector bundles on $\CC^d$. 
The group $U(d)$ acts by automorphisms on every holomorphic vector bundle, hence it acts on sections via the pull-back. 

There is another symmetry that will be relevant later on when we exhibit a calculation of the character for the local operators.
Introduce an action of $U(1)$ on the fields of the theory such that $V$ has weight $q_f \in \ZZ$ and $V^*$ has weight $-q_f$.
The value of the fields $\gamma$ lie in the vector space $V$, so these fields are of weight $q_f$. 
Conversely, the fields $\beta$ lie in $V^*$, so have weight $-q_f$.
Since the pairing defining the free theory is only non-zero between a single $\gamma$ and single $\beta$ field, the theory is invariant under this symmetry.
In the physics literature, this is a so-called ``flavor symmetry" of the theory, and so to distinguish it from the other symmetry we will denote this group by $U(1)_f$. 
This symmetry will be especially relevant when we compute the character of the $\beta \gamma$ system.

\begin{lem}\label{lem U(d) equivariance}  The symmetry by $U(d) \times U(1)_f$ on the classical $\beta\gamma$ system with values in the complex vector space $V$ extends to a symmetry of the factorization algebra of smoothed quantum observables $\Obs^{\q}_V$.
\end{lem}

\begin{proof}
The differential on the factorization algebra is of the form $\dbar + \hbar \Delta$. 
The operator $\dbar$ is manifestly equivariant for the action of $U(d)$.
Since $U(1)_f$ does not act on spacetime, $\dbar$ trivially commutes with its action. Further, the action of $U(d)$ is through linear automorphisms, and since the BV Laplacian $\Delta$ is a second order differential operator, it certainly commutes with the action of $U(d)$. 
Likewise, since $U(1)_f$ is compatible with the $(-1)$-symplectic pairing, it automatically is compatible with $\Delta$. 
\end{proof}

We will use the action of $U(d)$ to organize the class of operators we are interested in. 
The eigenvectors of $U(d)$ are labeled by the eigenvectors of a maximal torus, which we will take to be given by the subgroup
\ben
T^d = \{{\rm diag}(q_1,\ldots,q_d) \; | \; |q_i| = 1\} \subset U(d) .
\een 
Here, $q_i \in S^1 \subset \CC^\times$ are complex numbers of unit modulus. 
We say that an element $v$ of the factorization algebra has weight $(n_1,\ldots,n_k)$ if $(q_1,\ldots,q_d) \cdot v = q_1^{n_1}\cdots q_d^{n_d} v$. 
We will use the shorthand $\vec{n} = (n_1,\ldots,n_d)$. 
\begin{dfn}
\begin{enumerate}
\item Let $w \in \CC^d$ and $r > 0$. 
For any vector of non-negative integers $\vec{n} = (n_1,\ldots, n_d)$ denote by
\ben
\Obs^\q_V(r)^{(\vec{n})} \subset \Obs^\q_{V}(D(w,r))
\een 
the subcomplex of weight $\vec{n}$ elements. 
\item 
Let
\ben
\Obs_V^\q(r) := \bigoplus_{\vec{n}} \Obs^\q_V(r)^{(\vec{n})} 
\een
where the direct sum is over all vectors of non-negative integers.
\end{enumerate}
By setting $\hbar = 0$ this also induces weight spaces for the classical observables.
\end{dfn}

\begin{rmk}
Note that we have excluded $w \in \CC^d$ from the notation above. 
This is because the $\beta\gamma$ system, as we have already pointed out, is a translation invariant factorization algebra (in fact, it's holomorphically translation invariant). 
In particular if $z,w$ are any points then translation by $z$ induces an isomorphism
\ben
\tau_z : \Obs^\q_V(D(w,r)) \cong \Obs^\q_V(D(w-z, r)) .
\een
Translation clearly preserves the action by $U(d)$, so this isomorphism restricts to the weight spaces defined above.
\end{rmk}

We now introduce the following operators that will be of most relevance for our study of the operator product expansion.

\begin{dfn} Let $w \in \CC^d$ and $r > 0$.
Define the following linear observables supported on $D(w,r)$.
\begin{enumerate}
\item For $n_i \in \ZZ_{\geq 0}, i = 1,\ldots d$, and $v^* \in V^*$ define
\ben
\cO_{\gamma, -\vec{n}} (w;v^*) : \gamma \in \Omega^{0,*}(D(w,r)) \mapsto \left\<v^*,\left(\left.\frac{\partial^{n_1}}{\partial z_1^{n_1}} \cdots \frac{\partial^{n_d}}{\partial z_d^{n_d}} \gamma(z,\zbar) \right|_{z=w}\right)\right\>_V .
\een
Here, the brackets denote the evaluation pairing between $V^*$ and $V$. 
\item For $m_i \in \ZZ_{\geq 1}, i=1,\ldots d$, and $v \in V$ define
\ben
\cO_{\beta, -\vec{m}} (w;v) : \beta \d^d z \in \Omega^{d,*}(D(w,r)) \mapsto \left\<v ,\left(\left.\frac{\partial^{m_1-1}}{\partial z_1^{m_1-1}} \cdots \frac{\partial^{m_d-1}}{\partial z_d^{m_d-1}} \beta(z,\zbar) \right|_{z=w}\right)\right\>_V .
\een
\end{enumerate}
The braces $\<-,-\>_V$ denotes the evaluation pairing for the vector space $V$ and its dual.
\end{dfn}

Our convention is that the evaluation of a Dolbeualt form is zero $\d \zbar_i |_{z=w} = 0$.
Thus, the above observables are only nonzero when $\gamma \in \Omega^{0,0}(D(w,r))$ and $\beta \d^d z \in \Omega^{d,0}(D(w,r))$.
In particular, this implies that these operators are of the following homogenous cohomological degree:
\begin{align*}
\deg(\cO_{\gamma, -\vec{n}} (w;v^*))  & = 0 \\
\deg(\cO_{\beta, -\vec{m}} (w ; v)) & = d-1 .
\end{align*}

\begin{rmk}
The minus sign in $\cO_{\gamma, -\vec{n}}(w)$ is purely conventional, and meant to match up with the physics and vertex algebra literature \brian{ref}.
One reason for using this convention is motivated by the state-operator correspondence by realizing the above operators as coming from residues over higher dimensional spheres.
Note that for any $d$-disk $D(0,r)$ there is an embedding of topological vector spaces
\ben
z_1^{-1} \cdots z_d^{-1} \CC[z_1^{-1}, \cdots, z_d^{-1}] \to \left(\Omega^{0,*}(D(w,r))\right)^\vee
\een
that sends a Laurent polynomial $f(z)$ functional
\ben
\gamma \in \Omega^{0,*}(D(w,r)) \mapsto \oint_{z \in S^{2d-1}} f(z-w) \gamma(z,\zbar) \wedge \left(\d^d z \wedge \omega^{BM}(z-w,\zbar-\wbar)\right) ,
\een
where $\omega_{BM}$ is the Bochner-Martinelli form of type $(0,d-1)$, and $S^{2d-1}$ is the sphere of radius $r$ around $w$.
This is the higher dimensional versoin of the embedding 
\ben
z^{-1} \CC [z^{-1}] \to \left(\Omega^{1,hol}(D)\right)^\vee
\een
sending $f(z) \in z^{-1} \CC[z^{-1}] $ to the residue functional $g \mapsto \Res_{z} (f(z) g(z) \d z)$. 
We will elaborate more on these types of sphere operators in the next section.
\end{rmk}

%Similarly, there is an embedding $z_1^{-1} \cdots z_d^{-1} \CC[z_1^{-1}, \ldots,z_d^{-1}] \to \left(\Omega^{d,*}(D(w,r)\right)^\vee$ sending $f(z)$ to the functional
%\ben
%\beta \in \Omega^{d,*}(D(w,r)) \mapsto \oint_{S^{2d-1}} f(z-w) \beta(z,\zbar) \wedge \omega_{BM}(z-w,\zbar-\wbar).
%\een

%These embeddings determine a linear map 
%\ben
%i_D : z_1^{-1} \cdots z_d^{-1} \CC[z_1^{-1}, \ldots,z_d^{-1}] \tensor (V^* \oplus V[-d+1]) \to \left(\Omega^{0,*}(D(w,r)) \tensor V \oplus \Omega^{d,*}(D(w,r)) \tensor V^*[d-1]\right)^\vee .
%\een
%The right-hand side is simply the linear observables sitting inside of $\Obs^{\cl}_V(D(w,r))$.
%It follows from the higher dimensional residue formula that, for $n_i \geq 0$, the image of 
%\ben
%z_1^{-n_1} \cdots z_d^{-n_d} \tensor v^* \in \CC[z_1^{-1},\ldots, z_d^{-1}] \tensor V^*
%\een
%under this map is precisely $\cO_{\gamma, -\vec{n}}(w;v^*)$, where $\vec{n} = (n_1,\ldots,n_d)$. 
%Similarly, for $m_i \geq 1$, the image of
%\ben
%z_{1}^{-m_1+1} \cdots z_d^{-m_d+1} \tensor v \in \CC[z_1^{-1},\ldots, z_d^{-1}] \tensor V [-d+1]
%\een
%under this map is $\cO_{\beta,\vec{m}}(w;v)$. 
%This motivates the following general definition. 

%\begin{dfn}\label{dfn disk2}
%For any $f(z) \in \CC[z_1^{-1},\ldots,z_1^{-1}]$, denote by $\cO_{\gamma, f}(w; v^*)$ the image of $f\tensor v^*$ under the linear map $i_D$
%\ben
%\cO_{\gamma,f}(w;v^*) := i_D(f \tensor v^*)
%\een
%In particular, note $\cO_{\gamma, z_1^{-n_1}\cdots z_d^{-n_d}}(w;v^*) = \cO_{\gamma , -\vec{n}}(w;v^*)$.
%Similarly, if $g \in z_1^{-1} \cdots z_d^{-1} \CC[z_1^{-1},\ldots,z_d^{-1}]$, let
%\ben
%\cO_{\beta,g}(w;v) := i_D(z_1\cdots z_d g(z) \tensor v) .
%\een
%\end{dfn}

\begin{lem}
Let $r < s$.
Then, the factorization structure map for including disks $D(0,r) \subset D(0,s)$ induces a diagram
\ben
\xymatrix{
\Obs_V^\q(D(0,r)) \ar[r] & \Obs_V^\q(D(0,s)) \\
\Obs_V^\q(r) \ar[u] \ar[r]^-{\simeq} & \Obs_V^\q(s) \ar[u] .
}
\een
Further, the bottom horizontal map is a quasi-isomorphism.
\end{lem}

\begin{proof}
The two vertical maps are the inclusions of the $U(d)$-eigenspaces of the observables supported on disks of radius $r$ and $s$ respectively. 
It follows from Lemma \ref{lem U(d) equivariance} that the factorization algebra is $U(d)$-equivariant, so in particular the factorization algebra structure map for the inclusion of disks $D(0,r) \hookrightarrow D(0,s)$ is a map of $U(d)$-representations. 
Hence, the map restricts to each of the eigenspaces, yielding the diagram. 

In \cite{fact1} it is shown in Corollary 5.3.6.4 that for the one-dimensional $\beta\gamma$ system, the lower map above is a quasi-isomorphism. 
A completely similar argument applies to the $\beta\gamma$ system on $\CC^d$. 
Indeed, consider the collection
\ben
\{\cO_{\gamma, -\vec{n}_1} (0 ; v_1^*) \cdot \cO_{\gamma, -\vec{n}_k} (0 ; v_k^*) \cdot \cO_{\beta, -\vec{m}_1}(0;v_1) \cdots \cO_{\beta, - \vec{m}_l}(0;v_l)\}. 
\een
The collection runs over non-negative integers $k,l$ and sequences $\vec{n}_i = (n_{i,1},\ldots,n_{i,d})$, $n_{i,j} \geq 0$ and $\vec{m}_i = (m_{i,1},\ldots,m_{i,d})$, $m_{i,1} \geq 1$. 
It also runs over vectors $v_i, v_j^*$ in $V$ and $V^*$, respectively. 
Now, it follows from Lemma 5.3.6.2 of \cite{fact1} that the above collection form a basis for the cohomology
\ben
H^*\Obs^\cl_V(r)^{(\vec{N})} \subset H^*\Obs^\cl(D(0,r))
\een
for any $r$, where $\vec{N} = (N_1,\ldots,N_d)$
\ben
N_j = \left(n_{1,j} + \cdots + n_{k,j}\right) + \left(m_{1,j} + \cdots + m_{l,j}\right) .
\een
The result for the quantum observables follows from the spectral sequence \brian{finish}
\end{proof}

We will denote $\cV_V = \Obs_V^\q(r)$, which is well-defined up to quasi-isomorphism by the preceding proposition. 
This is the ``state space" of the higher dimensional holomorphic theory. 
\brian{elude to OPE}

\subsection{The sphere observables}

We now provide a description of the value of the factorization algebra of observables of the $\beta\gamma$ system applied to another important class of open sets in $\CC^d$: neighborhoods of the $(2d-1)$-sphere $S^{2d-1} \subset \CC^d$. 
We then study the algebraic structure that the factorization product endows the collection of sphere operators with.

Heuristically speaking, the operators we will consider are supported on $(2d-1)$ sphere.
Since the factorization algebra only takes values on open sets, we need to fix small neighborhoods of the spheres in order to define the observables precisely.
Let us explain the exact open neighborhoods of the $(2d-1)$-sphere that we will consider.
Denote the closed $d$-disk centered at $w$ of radius $r$ by
\ben
\Bar{D}(w,r) = \left\{(z_1,z_2) \in \CC^2 \; | \; |z-w| \leq r^2\right\} . 
\een
As above, the open disk is denoted $D(w,r)$. 
Let $\epsilon,r > 0$ be such that $0 < \epsilon < r$, and consider the open submanifold
\ben
N_{r, \epsilon}(w) := D(w,r + \epsilon) \setminus \Bar{D}(w, r-\epsilon) \subset \CC^d \setminus \{w\} .
\een 
For any $\epsilon > 0$, the open set $N_{r,\epsilon}$ is a neighborhood of the closed submanifold given by the sphere of radius $r$ centered at $w$, $S^{2d-1}_r(w) \subset \CC^d \setminus \{w\}$. 
Note that when $d=1$, $N_{r,\epsilon}$ is simply an annulus centered at $w$. 

Like in the case of a disk, it is convenient to get our hands on a class of simple observables supported on $N_{r,\epsilon}(w)$. 
We have the following general fact about linear functionals on the Dolbeualt complex of $N_{r,\epsilon}(w)$. 
This lemma will allow us to describe linear observables supported on these neighborhoods. 

\begin{lem}
For any neighborhood $N_{r,\epsilon}(w)$ as above, the residue along the $(2d-1)$-sphere centered at $w$ of radius $r$, $S^{2d-1}_r(w)$, determines an embedding of topological dg vector spaces
\ben
i_{S^{2d-1}} : A_{d} [d-1] \to \left(\Omega^{0,*}(N_{r,\epsilon}(w)\right)^\vee
\een
sending $\alpha \in A_d$ to the functional
\ben
i_{S^{2d-1}}(\alpha) : \omega \in \Omega^{0,*}(N_{r,\epsilon}(w)) \mapsto \oint_{S^{2d-1}_r(w)} \alpha \wedge \d^d z \wedge \omega .
\een
\end{lem}
\begin{proof}
This is a consequence of Stokes' theorem. 
Suppose $\alpha = \dbar \alpha '$. 
Then, for any $\omega \in \Omega^{0,*}(N_{r,\epsilon}(w))$ we have
\ben
\oint_{S^{2d-1}} (\dbar \alpha') \wedge \d^d z \wedge \omega = \oint_{S^{2d-1}} \alpha' \wedge \d^d z \wedge \dbar \omega .
\een
The right-hand side is simply $(\dbar i_N)(\omega) = i_N(\dbar \omega)$. 
\end{proof}

Similarly, there is an embedding $A_d [d-1] \to \left(\Omega^{d,*}(N_{r,\epsilon}(w)\right)^\vee$
sending $\alpha \in A_{d} [d-1]$ to the functional
\ben
\eta \in \Omega^{d,*}(N_{r,\epsilon}(w)) \mapsto \int_{S^{2d-1}_r(w)} \alpha \wedge \eta .
\een

These two embeddings allow us to provide a succinct description of the class of linear operators on $N_{r,\epsilon}(w)$ we are interested in. 
Indeed they determine a cochain map (that we proceed to denote by the same symbol):
\ben
i_{S^{2d-1}} : A_d \tensor \left(V^*[d-1] \oplus V\right) \to \left(\Omega^{0,*}(N_{r,\epsilon}(w)) \tensor V \oplus \Omega^{d,*}(N_{r,\epsilon}(w)) \tensor V^*[d-1] \right)^\vee \subset \Obs_V^{\cl}\left(N_{r,\epsilon}(w)\right).
\een

\begin{dfn}
Let $\alpha \in A_{d}$ and $v^* \in V^*$.
Define the linear observable
\ben
\cO_{\gamma, \alpha}(w ; v^*) := i_{S^{2d-1}}(\alpha \tensor v^*) \in \Obs^{\cl}(N_{r,\epsilon}(w)) .
\een 
Likewise, for $v \in V$, define
\ben
\cO_{\beta, z_{1}^{-1} \cdots z_d^{-1} \alpha} (w;v) := i_{S^{2d-1}}(\alpha \tensor v) .
\een 
\end{dfn}

\begin{dfn}
Define the dg vector space of {\em classical sphere observables} to be
\ben
\sA^{\cl}_V :=  \Sym \left(A_d \tensor \left(V^*[d-1] \oplus V\right)\right)
\een
equipped with the differential coming from $A_d$. 
\end{dfn}

Note that $A_d$ has the structure of a commutative dg algebra, but we are not using the multiplication here.
The same construction above, applied now to symmetric products of linear operators, determines a cochain map $i_{S^{2d-1}} : \sA^{\cl}_{V} \to \Obs^{\cl}(N_{r,\epsilon}(w))$.

Let $\sA_{V} = \sA_V^{\cl}[\hbar]$.
Then, since $\Delta|_{\sA_V} = 0$, we see that $i_{S^{2d-1}}$ extends to a cochain map
\ben
i_{S^{2d-1}} : \sA_{V} \to \Obs^\q_V(N_{r,\epsilon}(w)) .
\een
We will refer to $\sA_V$ as the {\em quantum sphere observables}, or when there is no confusion, the sphere observables. 

\subsubsection{Nesting spherical shells}

We now would like to discuss what happens when we study the factorization product on the observables supported on spheres. 
This will endow the cochain complex $\sA_V$ with the structure of an associative (really $A_\infty$) algebra. 
To recover this structure, we will only be concerned with open sets that are neighborhoods of spheres, as in the previous section. 
The factorization product is defined for any disjoint configurations of open sets. 
The configurations of open sets we consider are given by nesting the neighborhoods of the form $N_{r,\epsilon}(w)$, where $w$ is a fixed center.

For simplicity, we assume that our spheres and neighborhoods are all centered at $w=0$.
For $x\epsilon < r$ we have defined the open neighborhood $N_{r,\epsilon}=N_{r,\epsilon}(0)$ of the sphere $S^{2d-1}_r$ centered at zero.
Pick positive numbers $0 < \epsilon_i < r_i$ such that $r_1 < r < r_2$, $\epsilon_1 < r - r_1$, and $\epsilon_2 < r_2 - r$.
Finally, suppose $r > \epsilon > \max\{r - r_1 + \epsilon_1, r_2 - r + \epsilon_2\}$. 
We consider the factorization product structure map for $\Obs^{\q}_{V}$ corresponding to the following embedding of open sets
\be\label{fact product 1}
N_{r_1, \epsilon_1} \sqcup N_{r_2, \epsilon_2} \hookrightarrow N_{r, \epsilon}  ,
\ee
shown schematically in Figure \brian{figure}. 
The factorization structure map for this embedding of disjoint open sets is of the form 
\be\label{fact product 2}
\Obs^{\q}_{V}(N_{r_1, \epsilon_1}) \tensor \Obs^{\q}_{V}(N_{r_2, \epsilon_2}) \to \Obs^{\q}_{V}(N_{r,\epsilon}) .
\ee
%We will see that the specific choices of $r, r_i$ and $\epsilon, \epsilon_i$ are not important.

\begin{lem} \label{lem sphere alg} The factorization structure map in (\ref{fact product 2}) restricts to the subspace of sphere observables. 
That is, there is a commutative diagram
\ben
\xymatrix{
\Obs^{\q}_{V}(N_{r_1, \epsilon_1}) \tensor \Obs^{\q}_{V}(N_{r_2, \epsilon_2}) \ar[r] & \Obs^{\q}_{V} \\
\sA_V \tensor \sA_V \ar[u] \ar[r]^-{\mu_2} & \sA_V \ar[u]
}
\een
where the top line is the map in (\ref{fact product 2}). 
The same holds for an arbitrary number of nested neighborhoods of the form $N_{r,\epsilon}$. 
That is, for any $k \geq 0$ the factorization product restricts to a linear map 
\ben
\mu_k : \sA_V^{\tensor k} \to \sA_V .
\een
\end{lem}

Each of the neighborhoods $N_{r,\epsilon}$ are contained in the open submanifold $\CC^{d} \setminus \{0\}$.
Note that there is a homeomorphism $\CC^{d} \setminus \{0\} \cong S^{2d-1} \times \RR_{>0}$.
Further, we have the radial projection map
\ben
\pi : \CC^{d} \setminus \{0\} = S^{2d-1} \times \RR_{>0} \to \RR_{>0} 
\een
that sends $z = (z_1,\ldots,z_d) \mapsto |z| = \sqrt{|z_1|^2+\cdots+|z_d|^2}$. 

A fundamental feature of factorization algebras is that they push forward along smooth maps. 
We can thus push forward the factorization algebra $\Obs^\q_V$ on $\CC^{d}\setminus \{0\}$ along $\pi$ to obtain a factorization algebra on $\RR_{>0}$. 
To an open interval of the form $(r-\epsilon, r+\epsilon)\subset \RR_{>0}$ the factorization algebra assigns precisely the observables supported on $N_{r,\epsilon}$. 

Lemma \ref{lem sphere alg} implies that there is a factorization algebra $\sF_{\sA_V}$ associated to $\sA_V$ and that the inclusion $\sA_V \hookrightarrow \Obs^\q(N_{r,\epsilon})$ induces a map of factorization algebras on $\RR_{>0}$:
\ben
\sF_{\sA_V} \to \pi_*(\Obs^\q_V) 
\een
The factorization algebra $\sF_{\sA_V}$ assigns to every interval the dg vector space $\sA_V$. 
In particular $\sF_{\sA_V}$ is locally constant, and hence determines the structure of an $A_\infty$ algebra on $\sA_V$. 
We would now like to identify this algebra structure. 

We will proceed in two ways. 
First, we will use the Moyal formula of Section \ref{??} as well as the explicit form of the propagator from Section \ref{?} to deduce the operator product expansion between cohomology classes of operators corresponding to $\sA_V$. 
This will tell us what the algebra structure is on the cohomology $H^*(\sA_V)$. 
Second, we will use the smoothed description of the observables as a factorization enveloping algebra to nail down the precise algebra structure at the cochain level. 

Note that we can view $\sA_V$ as the symmetric algebra on the following cochain complex
\ben
A_d \tensor (V^*[d-1] \tensor V) \oplus \CC \cdot \hbar .
\een
This complex has the structure of a dg Lie algebra, with bracket given by
\be\label{HV bracket}
[\alpha \tensor v^*, \alpha \tensor v] = \hbar \<v^*, v\> \oint_{S^{2d-1}} \alpha \wedge \alpha'  \d^d z .
\ee
All other brackets are determined by graded anti-symmetry and declaring the parameter $\hbar$ is central.
Denote this dg Lie algebra by $\sH_V$. 

Our main result is that the dg algebra structure on $\sA_V$ endowed by the factorization product is equivalent to the universal enveloping algebra $U(\sH_V)$ of the dg Lie algebra $\sH_V$.

\begin{rmk}
If $(\fg, \d, [-,-])$ is a dg Lie algebra its universal enveloping algebra is defined explicitly by 
\ben
U(\fg) = {\rm Tens}(\fg) / (x \tensor y - (-1)^{|x||y|} y \tensor x - [x,y]) .
\een
It is immediate to check that the differential $\d$ descends to one on $U(\fg)$, giving $U(\fg)$ the structure of an associative dg algebra.
\end{rmk}

\subsubsection{Using the Moyal formula}

As eluded to before, we now identify the algebra structure on the cohomology of $\sA_V$
induced by the map of factorization algebras $\sF_{\sA_V} \to \pi_*(\Obs_V^\q)$, where $\sF_{\sA_V}$ is the locally constant factorization algebra that assigns the cochain complex $\sA_V$ to every interval.

Let $\ul{U(\sH_V)}$ be the locally constant factorization algebra on $\RR_{>0}$ based on the associative algebra $U(\sH_V)$. 
We will write down an explicit isomorphism of locally constant factorization algebras
\ben
\Phi : \ul{U(H^*\sH_V)} \to H^*\sF_{\sA_V},
\een
implying the result. 

By Poincar\'{e}-Birkoff-Witt, the dg vector spaces $U(\sH_V)$ and $\sA_V$ are isomorphic. 
Therefore, if $I \subset \RR_{>0}$ is an interval, we define $\Phi(I)$ to be the identity map. 
Thus, it suffices to show that the associative algebra structure on the spherical observables agrees with that of $U(\sH_V)$ in cohomology.

%We also need to take into account a higher operation. 

%If $I_1,I_2$ are two intervals, consider their disjoint union $I_1 \sqcup I_2 \subset \RR_{>0}$. 
%We define
%\ben
%\Phi(I_1 \sqcup I_2) : U(\sH_V) \tensor U({\sH_V}) \to \sF_{\sA_V} (I_1 \sqcup I_2) 
%\een

We turn to an explicit calculation of factorization product for observables in $\pi_*(\Obs_V^\q)$.
If $\sO, \sO' \in U(\sH_V)$ then we can compute the commutator $[\sO,\sO']$ in the factorization algebra as follows.
For $i = 1,2,3$ let $\epsilon_i, r_i > 0$ be such that 
\ben
\epsilon \leq \epsilon_1 < r_1 \leq \epsilon_2 < r_2 \leq \epsilon_3 < r_3 \leq r
\een 
and consider the configurations
\ben
i_{12} : N_{r_1, \epsilon_1} \sqcup N_{r_2, \epsilon_2} \hookrightarrow N_{r, \epsilon}
\een
and
\ben
i_{23} :  N_{r_2, \epsilon_2} \sqcup N_{r_3, \epsilon_3} \hookrightarrow N_{r, \epsilon}
\een
in $\CC^\d \setminus \{0\}$. 
If $I_i = (r_i - \epsilon_i, r_i + \epsilon$ and $I = (r- \epsilon, r+\epsilon)$, these correspond to the configurations $i_{12} : I_1 \sqcup I_2 \hookrightarrow I$ and $i_{23} : I_2 \sqcup I_3 \hookrightarrow I$ in $\RR_{>0}$, respectively. 
The induced factorization structure maps are
\be\label{starprods}
\begin{array}{ccc}
\star_{12} & : & \Obs_V^\q(N_{r_1, \epsilon_1}) \tensor \Obs_V^\q( N_{r_2, \epsilon_2}) \to \Obs_V^\q(N_{r, \epsilon}) \\
\star_{23} & : & \Obs_V^\q(N_{r_2, \epsilon_2}) \tensor \Obs_V^\q( N_{r_3, \epsilon_3}) \to \Obs_V^\q(N_{r, \epsilon}) .
\end{array}
\ee
The commutator $[\sO, \sO']$ is computed via the formula
\be\label{commutator}
\sO \star_{12} \sO' - \sO' \star_{23} \sO .
\ee
In the notation $\sO \star_{12} \sO'$ we view $\sO$ as having support in $N_{r_1,\epsilon_1}$ and $\sO'$ as having support in $N_{r_2,\epsilon_2}$.

We compute this commutator at the level of cohomology.
The cohomology of $A_d$ is concentrated in degrees $0$ and $d-1$. 
Explicitly, one can represent the zeroeth cohomology as
\ben
H^0(A_d) = \CC[z_1,\ldots,z_d] .
\een
Now, let $\omega_{BM}(z,\zbar)$ be the Bochner-Martinelli kernel of type $(0,d-1)$ from above. 
We can express the $(d-1)$st cohomology of $A_d$ as
\ben
H^{d-1}(A_d) = \CC[\partial_{z_1}, \cdots, \partial_{z_d}] \cdot \omega_{BM} 
\een 
That is, every element of $H^{d-1}(A_d)$ can be written as a holomorphic polynomial differential operator acting on $\omega_{BM}$. 
Further, it is convenient to make the $U(d)$-equivariant identification 
\be\label{U(d) identification}
 \CC[\partial_{z_1}, \cdots, \partial_{z_d}] \omega_{BM} \cong z_1^{-1} \cdots z_d^{-1} \CC[z_1^{-1}, \ldots, z_d^{-1}],
 \ee
which makes sense since $\omega_{BM}$ has $T^d \subset U(d)$-weight $(-1,\ldots,-1)$. 

Recall that $\sH_V = A_d \tensor (V^*[d-1] \oplus V)$.
It follows from above that the cohomology of $\sH_V$ is concentrated in degrees $-(d-1), 0, d-1$. 
The non-trivial Lie algebra structure on $\sH_V$ comes from the ordinary symplectic pairing on this space, as we've already discussed. 

Suppose $v,v^*$ are in $V,V^*$, respectively and $\alpha,\alpha' \in A_d$.
The corresponding classical observables $\cO_{\gamma,\alpha}(0;v^*)$ and $\cO_{\beta, z_1^{-1}\cdots z_d^{-1} \alpha'}(0;v)$ have cohomological degrees
\begin{align*}
{\rm deg}\left(\cO_{\gamma,\alpha}(0;v^*)\right) = |\alpha| - d + 1 \\
\left|\cO_{\beta, z_1^{-1}\cdots z_d^{-1} \alpha'}(0;v)\right| = |\alpha'|,
\end{align*}
where $|\alpha|$ denotes the differential form degree.
In cohomology the only nontrivial form degrees of $\alpha,\alpha'$ that survive are $0,d-1$. 
Suppose that $|\alpha| = 0$.
Then, the only way we could obtain a nontrivial commutator between the operators above is if $|\alpha'| = d-1$. 

We will compute the factorization product in (\ref{commutator}) using our explicit formula for the propagator of the $\beta\gamma$ system computed in Section \ref{?}.
We diverge a moment to recall how this construction works.
The main idea is that the propagator allows us to promote a classical observable to a quantum observable.
Recall, the full propagator is an element
\ben 
P (z,w) = \lim_{L\to \infty} \lim_{\epsilon \to 0} P_{\epsilon < L}(z,w) \in \Bar{\sE}_V(\CC^d) \Hat{\tensor} \Bar{\sE}_V(\CC^d)
\een
where the $\Bar{\sE}_V(\CC^d)$ denotes the space of distributional sections on $\CC^d$.
Explicitly, we showed that 
\ben
P(z,w) = C_d \;\omega_{BM}(z,w) 
\een
where $\omega_{BM}(z,w)$ is the Bochner-Martinelli kernel.

Contraction with $P$ determines a degree zero, order two differential operator
\ben
\partial_{P} : \Obs^{\cl}_V (U) \to \Obs^{\cl}_{V}(U)
\een
for any open set $U \subset \CC^d$. 
Recall that the classical observables on $U$ are simply given by a symmetric algebra on the continuous dual of $\sE_V(U)$. 
Since $\Bar{\sE}^\vee = \sE_c^!$, we can view the propagator as an symmetric smooth linear map
\ben
P^\vee : \sE_{V,c}^!(\CC^d) \Hat{\tensor} \sE_{V,c}^!(\CC^d) \to \CC .
\een
The contraction operator $\partial_P$ is determined by declaring it vanishes on $\Sym^{\leq 1}$, and on $\Sym^2$ is given by the linear map $P^\vee$. 

To compute the factorization product we use the isomorphism
\ben
\begin{array}{cccc}
W_0^\infty : & \Obs^{\cl}_V(U) [\hbar]  & \to & \Obs^\q_V(U) \\
& \cO & \mapsto & e^{\hbar \partial_P} \cO 
\end{array}
\een
that makes sense for any open set $U$.
This is an isomorphism of cochain complexes, with inverse given by $(W_0^\infty)^{-1} = e^{-\hbar \partial_P}$. 
By \ref{??} it determines the following formula for the factorization product. 
If $\cO,\cO'$ are observables supported on disjoint opens $U,U'$, and $V$ is and open set containing $U,U'$, then the factorization structure map is given by
\ben
\cO \star \cO' = e^{-\hbar \partial_P} \left(\left(e^{\hbar \partial_P}\cO\right) \cdot \left(e^{\hbar \partial_P} \cO'\right)\right) \in \Obs^\q(V) .
\een 
Here, the $\cdot$ refers to the symmetric product on classical observables.

The calculation of the factorization product relies on the higher dimensional residue formula involving the Bochner-Martinelli form. 
If $f$ is any any function in $C^\infty(U)$, where $U$ is a domain in $\CC^d$, then the residue formula states that for any $z \in D$ 
\ben
f(z,\zbar) = \int_{w \in \partial U} \d^d w \; f(w) \; \omega_{BM}(z,w) - \int_{w \in D} \d^d w \; (\dbar f)(w) \wedge \omega_{BM}(z,w) .
\een 
In particular, if $f(z,\zbar)$ is holomorphic the second term drops out and we get the familiar expression for the higher dimensional residue.

We can now perform the main calculation. 
Recall, we have fixed observables $\cO_{\gamma, \alpha}(0;v^*)$ and $\cO_{\beta, z_1^{-1}\cdots z_d^{-1} \alpha'} (0;v)$.
In the notation of Equation (\ref{starprods}), we have
\bestar
\cO_{\gamma, \alpha}(0;v^*) \star_{12} \cO_{\beta, z_1^{-1}\cdots z_d^{-1} \alpha'} (0;v) & = &  \cO_{\gamma, \alpha}(0;v^*) \cdot \cO_{\beta, z_1^{-1}\cdots z_d^{-1} \alpha'} (0;v) \\ & & + \hbar \<v, v^*\>\oint_{|z^1| = r_1} \oint_{|z^2| = r_2} \alpha(z^1) \d^d z^1 \alpha'(z^2) P(z^1,z^2) \\ & = & \cO_{\gamma, \alpha}(0;v^*) \cdot \cO_{\beta, z_1^{-1}\cdots z_d^{-1} \alpha'} (0;v) \\ & & + \hbar \<v, v^*\> \oint_{|z^1| = r_1} \oint_{|z^2| = r_2}  \alpha(z^1) \alpha'(z^2) \d^d z^1 \omega_{BM}(z^1,z^2) \\ & = & \cO_{\gamma, \alpha}(0;v^*) \cdot \cO_{\beta, z_1^{-1}\cdots z_d^{-1} \alpha'} (0;v)  +  \hbar \<v, v^*\> \oint_{|z| = r_1} \alpha(z) \alpha'(z) \d^d z \\ & & +  \hbar \<v, v^*\> \oint_{|z^1| = r_1} \int_{z^2 \in D(0, r_2)} \; \alpha(z^1) (\dbar \alpha')(z^2) \omega_{BM}(z^1,z^2) . 
\eestar 
In the first line we have used the Moyal formula.
In the second line we have used the explicit form of the propagator. 
In the third line we have used the higher residue formula. 
Finally, since we are only interested in the cohomology class of the product, we can assume that $\alpha,\alpha'$ are both holomorphic. 
In particular, the third term in the last line vanishes. 
The calculation for the $\star_{23}$ product is similar. 
We conclude that in cohomology the commutator between the quantum observables $\cO_{\gamma, \alpha}(0;v^*)$ and $\cO_{\beta, z_1^{-1}\cdots z_d^{-1} \alpha'} (0;v)$ is precisely
\ben
\# \hbar \<v, v^*\> \oint_{|z| = r_1} \alpha(z) \alpha'(z) \d^d z .
\een
This agrees with the commutator $(\ref{HV bracket})$ in $\sH_V$. 
The extension to commutators between non-linear observables is completely analogous. 
Thus, we conclude that as associative graded algebras one as 
\ben
U(H^* \sH_V) \cong H^* \sA_V .
\een

\subsubsection{Using smoothed observables}

We now provide a refined description of the algebra of sphere operators, yet this approach may seem more indirect. 
It relies on the existence of the subfactorization algebra of $\Obs^\q_{V}$ given by the {\em smoothed observables}. 
\brian{have we already encountered this?}

The linear smoothed observables, equipped with the linearized BRST differential, on any $U \subset \CC^d$ form the subcomplex 
\ben
\Omega^{d,*}_c(U) \tensor V^*[d] \oplus \Omega^{0,*} (U) \tensor V [1] \subset \Obs^{\cl}_V(U) .
\een
Using the $P_0$ bracket restricted to the linear observables, we can form the central extension of dg Lie algebras
\ben
0 \to \CC [-1] \cdot \hbar \to \sH'_V(U) \to \Omega^{d,*}_c(U) \tensor V^*[d] \oplus \Omega^{0,*} (U) \to 0 .
\een
This is similar to the construction of the ordinary Heisenberg algebra (such as $\sH_V$ above).
For classical linear observables the Lie bracket is defined by $[\cO, \cO'] = \hbar \{\cO, \cO'\}$, where $\{-,-\}$ is the $P_0$ bracket. 
Since the $P_0$ bracket is degree $+1$ to make this a dg Lie algebra we must put $\hbar$ in degree $+1$ as well.
Note that this construction works well as we vary the open set $U$. 
Namely, $U \mapsto \sH'_V(U)$ is a cosheaf of Lie algebras on $\CC^d$. 
An elementary observation identifies the smoothed quantum observables with the factorization enveloping algebra of $\Tilde{\sH}_V$:
\ben
\Obs^{\q}_{V} \cong \UU(\sH'_V) .
\een
Indeed, the right hand side assigns to each open $U$ the cochain complex $\clieu_*(\Tilde{\sH}_V(U)) = \left(\Sym(\sH_V'(U)), \dbar + \d_{CE}\right)$. 
One checks directly that $\d_{CE}$ is precisely the BV Laplacian $\hbar \Delta$. 

We now introduce the following algebra, which is a deformation of the dg associative algebra $U(\sH_V)$ from the previous section.
Let $\Tilde{\sH}_V$ denote the following extension of dg Lie algebras
\ben
0 \to \CC \cdot \hbar \to \Tilde{\sH}_V \to A_d \tensor (V^*[d-1] \tensor V) \to 0
\een
determined by the $2$-cocycle
\ben
(\alpha \tensor v^* , \alpha' \tensor v) \mapsto \hbar \<v,v^*\> \oint_{S^{2d-1}} \alpha \alpha' \d^d z + \hbar \<v, v^*\> \oint_{z \in S^{2d-1}} \alpha(z) \d^d z \int_{w \in D} \dbar \alpha' (w) \omega_{BM}(z,w)  + (\alpha \leftrightarrow \alpha') .
\een 
%To check that this is a cocycle, note that by the residue formula one has 
%\ben
%\oint_{z \in S^{2d-1}} \dbar \alpha(z) \d^d z \int_{w \in D} \dbar \alpha'(w) \omega_{BM}(z,w) = ...
%\een
The first term above defined the bracket for $\sH_V$, so we are deforming the bracket by the second term. 
\brian{call $\Tilde{\sA}_V$ the algebra with commutator that includes the error term in the residue}.

%\ben
%\xymatrix{
%\Obs^{\q, sm}(N_{r,\epsilon}) \ar@{^{(}->}[r] & \Obs^\q_V(N_{r,\epsilon}) \\
%\sA_V \ar@{.>}[u] \ar[ur]_-{i_{S^{2d-1}}} .
%}
%\een

\begin{prop}
There is a locally constant factorization algebra $\sF_V$ on $\RR_{>0}$ with the following properties:
\begin{enumerate}
\item $\sF_V$ admits a map of factorization algebras
\ben
\sF_V \to \rho_* (\Obs^\q_V)
\een
that is dense at the level of cohomology.
\item As a locally constant one-dimensional factorization algebra $\sF_V$ is equivalent to the dg algebra $\Tilde{\sA}_V$. 
\end{enumerate}
\end{prop}

\begin{proof}
We will write down the factorization algebra $\sF_V$ and then prove the above two properties we claim it satisfies. 
Consider the local Lie algebra on $\RR_{>0}$ whose compactly supported sections are $\Omega^*_{\RR_{>0},c} \tensor \Tilde{\sH}_V$.
The Lie bracket is encoded by the Lie bracket on $\Tilde{\sH}_V$ combined with the wedge product of forms on $\RR_{>0}$. 
Now, we define $\sF_V$ as the factorization envelope of this local Lie algebra 
\ben
\sF_V = \UU\left(\Omega^*_{\RR_{>0},c} \tensor \Tilde{\sH}_V\right) .
\een

We have just expressed $\Obs_V^\q$ as a factorization enveloping algebra as well.
Since the pushforward commutes with the functor $\UU(-)$, to construct the map in (1) it suffices to provide a map of factorization Lie algebras
\ben
\Omega^*_{\RR_{>0},c} \tensor \Tilde{\sH}_V \to \rho_* \sH_V' .
\een

\end{proof}

\begin{rmk}
At the level of cohomology $H^*(\Tilde{\sA}_V) \cong H^*(\sA_V)$ as graded algebras....\brian{comment}
\end{rmk}



\subsubsection{The disk as a module}

We point out the potential confusion with notation for disk observables in Definition \ref{dfn disk2}. 
This is remedied by the following lemma. 

\begin{lem} 
Consider the factorization algebra structure map for the inclusion $N_{r, \epsilon}(w) \hookrightarrow D(w, R)$ where $R > r + \epsilon$:
\ben
\mu : \Obs^\q_V(N_{r,\epsilon}(w)) \to \Obs^\q_V(D(w,R)) .
\een
Then, in cohomology $H^*(\mu)|_{H^* \sA_V}$ is only nonzero on elements in
\ben
\Sym \left(H^{d-1}(A_d) \tensor (V^*[d-1] \oplus V)\right) .
\een
On the linear elements inside of this symmetric algebra, the factorization map map satisfies
\ben
H^* \mu : \cO_{\gamma , \partial_z^{\vec{n}} \omega_{BM}} \mapsto \cO_{\gamma, -\vec{n}}
\een
\end{lem}

\begin{prop}
The factorization product above gives the cohomology $H^*\sV_V$ the structure of a graded module for the associative graded algebra $H^*\sA_V$.
Moreover, there is an isomorphism of $H^*\sA_V$ modules
\ben
H^*\sV \cong H^*\sA_V \tensor_{\sA_{V,+}} \CC .
\een 
\end{prop}

The tensor product $H^*\sA_V \tensor_{\sA_{V,+}} \CC$ is equal to the induction of the trivial module along the subalgebra $\sA_{V,+} \subset H^*\sA_V$. 
In particular, it implies that as a graded vector space
\ben
H^* \sV_V \cong \sA_{V,-} [-d+1],
\een
which is immediate from our identification (\ref{U(d) identification})

\subsection{The colored operad of holomorphic disks}

%\subsection{Reduction along spheres}
%$\pi : \CC^{d} \setminus \{0\} = S^{2d-1} \times \RR_{>0} \to \RR_{>0}$
%\begin{prop}
%Suppose $\sF$ is a holomorphically translation invariant factorization algebra and \brian{assumptions}. 
%Then, the sub factorization algebra 
%\ben
%?? \subset \pi_* \sF
%\een
%is a locally constant factorization algebra on $\RR_{>0}$. 
%\end{prop}
%
%This proposition tells us that to every holomorphically translation $\sF$ invariant factorization algebra satisfying those mild conditions above there is an associated associative algebra that we will denote by $\sA_\sF$. 

\subsection{Holomorphic descent}

\subsubsection{Topological descent}

\brian{review}
Before jumping in to the construction of operators in holomorphic theories using a descent procedure, we'd like to a review a more familiar topological situation. 
This concept was introduced by Witten in his introduction of cohomological field theories \cite{WittenCohomological}. 
Expositions of this construction in the context of topological conformal field theory can be found in \cite{WittenZwiebach, DijkgraafVV}

Suppose we have a translation invariant theory on $\RR^d$ for which all infinitesimal translations are exact for the BRST differential.
If $Q^{BRST}$ is the BRST differential this means that for $i=1,\ldots,d$ there exists operators $G_i$ on the space of fields such that
\be\label{G operator}
[Q^{BRST},G_i] = \frac{\partial}{\partial x_i} .
\ee
Note that since $\partial / \partial x_i$ has BRST degree zero, the operators $G_i$ decrease the BRST degree by one. 
Here, one thinks of the collection $\{G_i\}$ as providing a homotopy trivialization of the action by infinitesimal translations on the theory. 
In particular, this means that $\partial/\partial x_i$ acts trivially on the $Q^{BRST}$-cohomology.

In turn, $G_i$ also acts on the local operators of the theory. 
Using translation invariance, we can view a local operator $\cO$ as a function on space-time $\RR^d$. 
Suppose $\cO$ has pure BRST degree $k$.
Using the operator $G_i$ we can consider the function valued operator $G_i \cO$ which is of BRST degree $k-1$.
Using the frame on $\RR^d$ we can then define the {\em 1-form} valued operator
\ben
\cO^{(1)} = \sum_i (G_i \cO) \d x_i .
\een
By construction, the following relation is satisfied
\ben
\d_{dR} \cO = \sum_i \frac{\partial}{\partial x_i} \cO \d x_i = [Q^{BRST}, \cO^{(1)}] .
\een
This is the first so-called {\em topological descent equation}. 
In general, we can iterate the above construction to define
\ben
\cO^{(l)} = \sum_{i_1,\ldots i_l} G_{i_1}\cdots G_{i_l} \cO \d x_{i_1} \cdots d x_{i_l} .
\een
This is an $l$-form valued operator of BRST degree $k-l$. 

The operator $\cO^{(l)}$ allows us to define a new class of operators that depend on choosing an $l$-cycle inside of $\RR^d$. 
Indeed, suppose $Z \subset \RR^d$ is a closed $l$-dimensional submanifold.
Define the operator
\ben
\cO_Z = \int_{Z} \cO^{(l)} .
\een
The topological descent equations imply that if $\cO$ is BRST invariant $Q^{BRST} \cO = 0$, then $Q^{BRST} \cO_Z = 0$ as well.

Interesting examples of cohomological field theories arise as topological twists of supersymmetric theories.
Another class of examples come from topological vertex algebras \cite{Huang, LianZuckerman}.
In Section \brian{ref} we will discuss a class of such theories by considering a higher dimensional version of holomorphic gravity. 

We know that the local operators of a quantum field theory have the structure of a factorization algebra.
In the world of factorization algebras, there is also a notion of being topological: being (homotopically) locally constant.
This means that for every embedding of open balls $B \hookrightarrow B'$, the induced factorization structure map $\sF(B) \to \sF(B')$ is a quasi-isomorphism.

It would be natural to expect that the observables of a topological field theory, in which the infinitesimal translations are BRST exact, should give rise to such a factorization algebra. 
This is not exactly the case.
The relations (\ref{G operator}) guarantee a slightly weaker condition on the factorization algebra of observables. 
Indeed, the resulting action of the operators $G_i$ on the factorization algebra provide us with a sort of ``flat connection" on the factorization algebra. 
The difference between this structure and the locally constant condition is analogous to the discrepancy between $D$-modules and local systems. 
It is current work of Elliott and Safranov \cite{} to show how topological twists of supersymmetric theories give rise to such locally constant factorization algebras. 

We discuss a more direct way in which we can extract a shadow of a locally constant factorization algebra from a topological field theory using descent. 
There is an algebraic object associated to any locally constant factorization algebra.
Indeed, a famous theorem of Lurie \cite{LurieAlg} states an equivalence of categories 
\ben
\{{\rm Locally\;constant\;factorization\;algebras\;on\;} \RR^d\} \; \simeq \; \{E_d-{\rm \; algebras}\} .
\een
The cohomology of an $E_d$-algebra has the structure of a $P_d$-algebra.
In the category of cochain complexes, we have the following concrete definition of a $P_d$-algebra. 

\begin{dfn}
Let $d \geq 0$.
A $P_d$ algebra in cochain complex is a commutative dg algebra $(A, \d)$ together with the data of a bracket of degree $1-d$
\ben
\{-,-\} : A \tensor A \to A [d-1]
\een
such that:
\begin{enumerate}
\item the bracket is graded anti-symmetric:
\ben
\{a,b\} = -(-1)^{|a| + d-1} (-1)^{|b| + d-1};
\een
\item the bracket satisfies graded Jacobi:
\ben
\{a,\{b,c\}\} = \{\{a,b\},c\} + (-1)^{|a| + d-1} (-1)^{|b|+d-1} \{a,\{b,c\}\} ;
\een
\item the bracket is a graded bi-derivation for the commutative product:
\ben
\{a,b \cdot c\} = \{a,b\} \cdot c + (-1)^{|b|(|a| + d-1)} b \cdot \{a,c\} .
\een
\end{enumerate}
for all $a,b,c \in A$. 
\end{dfn}

\brian{finish}

\begin{eg} Topological $BF$ theory.
\brian{give references}
For any dg Lie algebra $(\fg,\d_{\fg}, [-,-])$ one can define the following $d$-dimensional topological field theory.
The fields of $BF$ theory with values in $\fg$ consist of 
\ben
(A,B) \in \Omega^*(\RR^d ; \fg[1]) \oplus \Omega^{*}(\RR^{d} ; \fg)[d-2]
\een 
with action functional
\ben
S(A,B) = \int \<B, d A + [A,A]\>_\fg,
\een 
where $\<-,-\>_\fg$ denotes a chosen invariant non-degenerate pairing on $\fg$. 
The name comes from the fact that $S = \int B F(A)$ where $F(A) = \d A + [A,A]$ is the curvature. 
The differential is a sum $\d = \d_{dR} + \d_{\fg}$. 
Note that the theory is translation invariant and has a natural action by the infinitesimal translations $\{\frac{\partial}{\partial x_i}\}$ via Lie derivative.

The class of local operators we consider are defined as
\begin{align*}
\cO_{A,a}(x) : & A \in \Omega^{0}(\RR^d; \fg)[1] \mapsto \<a,A(x)\>_\fg \\
\cO_{B,a}(x) : & B \in \Omega^{0}(\RR^d, \fg)[d-2] \mapsto \<a, B(x)\> .
\end{align*}
where $x \in \RR^d$ is a fixed point and $a \in \fg$ is a fixed element.
Using translation invariance, we view $\cO_{A,a}, \cO_{B,a}$ as function valued operators on $\RR^{d}$. 
The total space of local operators can be identified with functions on the shifted tangent bundle to the formal moduli space $B\fg$, $\sO(T[d-1] B \fg)$. 
The operator $\cO_{A,a}$ corresponds to the linear coordinate on the base of $B\fg$ and $\cO_{B,a}$ corresponds to a linear coordinate on the fiber.

We consider the differential operator
\ben
G_i = \frac{\d}{\d(\d x_i)}
\een
that also acts on the space of fields. 
This operator is equal to the contraction with the vector field $\frac{\partial}{\partial x_i}$. 
Since $G_i$ commutes with the differential and bracket on the Lie algebra, the Cartan formula implies
\ben
[Q^{BRST}, G_i] = \left[\d_{dR}, \frac{\d}{\d(\d x_i)}\right] = \frac{\partial}{\partial x_i} .
\een

Following the descent procedure above, we go on to define the form valued local operators
\ben
\cO_{A,a}^{(l)} = \sum_{i_1,\ldots, i_l} G_i \cO_{A,a} \d x_i
\een
and similarly for $\cO_{B,a}$. 
Then, for any $l$-cycle $Z \subset \RR^d$ we obtain operators $\int_Z \cO_{A,a}^{(l)}, \int_Z \cO_{B,a}^{(l)}$. 
For example, one can check that the latter operator is of the form
\ben
\int_Z \cO_{B,a}^{(l)} : B \in \Omega^{l}(\RR^d)[d-2-l] \mapsto \int_Z B ,
\een
which is of degree $- d + 2 + l$. 


To obtain the $P_n$-bracket via descent we consider the $(d-1)$-sphere $Z = S^{d-1}$, which we assume is centered at the origin.
Then, the bracket between the linear operators $\sO_{A,a}, \cO_{B,a'}$ is computed by the operator product expansion of $\cO_{A,a}$ and the descended operator $\int_{S^{d-1}} \cO_{B,a'}^{(d-1)}$:
\ben
\{\cO_{A,a}, \cO_{B,a'}\}_{P_d} = \cO_{A,a}(0) \star \int_{S^{d-1}} \cO_{B,a'}^{(d-1)} .
\een
A simple OPE calculation \brian{finish this}
\end{eg}


\subsubsection{General theory}

We will now summarize the steps in defining the higher dimensional OPE for holomorphically translation invariant quantum field theories. 
We note that this is a schematic, and as is usual we will need to regularize at various stages to obtained a well-defined construction. 

\begin{enumerate}
\item Suppose $\cO \in \sObs_0$ is a local operator supported at $0 \in \CC^d$. 
Let $z \in \CC^d$ be another point, and consider the translated operator 
\ben
\cO(z) := \tau_z \cO .
\een 
By the property of holomorphic translation invariance, this assignment defines a $\sO^{hol}(\CC^d)$-valued local operator. 

\item We perform ``holomorphic descent" to the function valued operator $\sO^{hol}(\CC^d)$ to obtain Dolbeualt valued operator 
\ben
\cO^{(0,*)}(z) \in \Omega^{0,*}(\CC^d) \tensor \sObs_0 .
\een 
Explicitly, 
\ben
\cO^{(0,k)} (z) = \sum_{I} (\Bar{\eta}_I \cdot \cO(z)) \d \zbar_I
\een
where $I = (i_1,\ldots,i_k)$, $1 \leq i_k \leq d$, is a multi-index of length $k$ and $\eta_I = \eta_{i_1\cdots i_k}$, $\d \zbar_I = \d \zbar_{i_1} \cdots \d \zbar_{i_k}$. 

\item For any $f(z) \d^d z \in \Omega^{d,hol}(\CC^d)$, and $w \in \CC^d$, define the sphere supported operator
\ben
\cO_{f}(w, r) := \int_{z \in S^{2d-1}_{w,r}} f(z) \d^d z \cO^{(0,d-1)}(z)
\een 
where $S^3_{w,r}$ is the sphere of radius $r$ centered at $w$. 

\item If $\cO'$ is another local operator supported at zero, we define the $f$-bracket by
\ben
\{\cO, \cO'\}_f := \cO_f(0, r) \star \cO' \in \sObs_0
\een
where $\star$ denotes the factorization product of a small disk with a small neighborhood of $S^{2d-1}_{0,r}$. 

\end{enumerate}

\subsubsection{}

The observables of the $\beta\gamma$ system comes naturally equipped with null-homotopies of the operators $\frac{\partial}{\partial \zbar_i}$. 

So far, in Section \brian{ref} we have described the space of local operators on the $d$-disk of the $\beta\gamma$ system with values in a vector space $V$. 
For disks centered at $z \in \CC^d$ there are two main classes of operators $O_\gamma (\vec{n}, z ; v^*)$ and $O_{\beta}(\vec{m}, z ; v)$ where $\vec{n} = (n_1,\ldots,n_d) \in (\ZZ_{\geq 0})^d$, $(m_1,\ldots,m_d) \in (\ZZ_{\geq 1})^d$, $v \in V$, and $v^* \in V$. 




\end{document}
