\documentclass[10pt]{amsart}

\usepackage{macros,slashed}

\linespread{1.25}

\title{Deformations of higher CDOs}

\def\brian{\textcolor{blue}{BW: }\textcolor{blue}}

\begin{document}
\maketitle

\section{Deformations for an $L_\infty$ space}

We have already pointed out that our theory is a cotangent theory. 
In particular, there is an action of the abelian group $\CC^\times_{cot}$ which assigns the base direction a weight of zero and the fiber a weight of $+1$. 
Thus, if $(\gamma, \beta) \in \Omega^{0,*}(Y, \fg) \oplus \Omega^{d,*}(Y, \fg^\vee)[d-1]$, then an element $\lambda \in \CC^\times_{cot}$ acts by
\ben
\lambda \cdot (\gamma, \beta) = (\gamma, \lambda \beta) .
\een
Our first reduction is to restrict ourselves to studying deformations that are compatible with this $\CC^\times_{cot}$ action.

Note that the symplectic pairing of the theory, as well as the classical action functional, is of $\CC^{\times}_{cot}$-weight $(-1)$.
Our convention is that the parameter $\hbar$ {\em has $\CC^\times_{cot}$-weight $(-1)$} as well. 
There are two compelling reasons for making this definition. 
The first deals with studying correlation functions for the theory. 
If we require the observables of the theory to be equivariant for this rescaling of the cotangent fibers, this means that the factorization product must have $\CC^\times_{cot}$ weight zero.
In the case that the theory is free, we have seen that the factorization product between two operators of the theory $\cO, \cO'$ is computed by a Moyal type formula
\ben
\cO \star \cO' = e^{-\hbar \partial_P} \left(e^{\hbar \partial_P} \cO \cdot e^{\hbar \partial_P} \cO' \right) .
\een
Since the symplectic pairing is $\CC^\times_{cot}$-weight $(-1)$ we observe that the propagator is also $\CC^\times_{cot}$-weight $(+1)$.
\footnote{This actually requires that we also take the gauge fixing operator to be of $\CC^\times_{cot}$-weight zero, which is the natural thing to do for cotangent theories.}
For the product to have weight zero we are then forced to take $\hbar$ to have opposite weight to $P$.

The other, related reason, we choose this weight for $\hbar$ is that we would like to require our BV complex to be equivariant for rescaling the fibers as well. 
The classical BRST differential is of the form $\{S,-\} = Q + \{I,-\}$.
We have already said that the classical action is of weight $(-1)$.
Since the symplectic pairing is also degree $(-1)$, this means that the $P_0$ bracket is degree $+1$. 
Thus, the classical BRST complex is manifestly equivariant.
The quantum BV differential involves deforming this classical differential by $\hbar \Delta$. 
For the same reason as the Poisson bracket, the BV Laplacian has weight $(+1)$. 
Thus, we see that in order to have an equivariant differential we are again forced to take $\hbar$ to have weight $-1$. 

In the case of an interacting theory, we have the following restriction on the quantum interactions of the theory as well. 
We can expand an effective interaction as
\ben
I[L] = \sum_{g \geq 0} \hbar^g I^{(g)}[L] .
\een
In order for $I[L]$ to have $\CC^\times_{cot}$ weight $(-1)$ we see that $I^{(g)}[L]$ must have weight $g-1$. 
We are only studying a one-loop quantization of the holomorphic theory, so the effective action has the form $I[L] = I^{(0)} + \hbar I^{(1)}[L]$ and hence $I^{(1)}[L]$ has weight zero. 

Thus, all one-loop quantities compatible with the $\CC^\times_{cot}$ action also have weight zero, including the one-loop anomaly. 
For this reason, we will be most concerned with the piece of the deformation complex that is $\CC^{\times}_{cot}$-weight zero. 
This amounts to looking just at local functionals of the $\gamma$-field.

\begin{dfn} 
The {\em deformation complex for cotangent quantizations} of the holomorphic $\sigma$-model of maps $Y \to B \fg$ is the cochain complex 
\ben
\Def^{\rm cot}_{Y \to B\fg} = \cloc^*(\Omega^{0,*}_Y \tensor \fg) .
\een
This is simply the local cochains of the local Lie algebra $\Omega^{0,*}_Y \tensor \fg$ on $Y$. 
\end{dfn}

We will be most interested in seeing how both the anomaly and the resulting quantum correction induced by the anomaly are realized inside the complex $\Def_{Y \to Bg}^{\rm cot}$. 
Before doing this, we'd like to restrict ourselves to looking at quantizations preserving further symmetries. 

\brian{do this}

\begin{thm}
Consider the deformation complex for cotangent quantizations of the holomorphic $\sigma$-model of maps $\CC^d \to B \fg$. 
There is a quasi-isomorphism of the $\CC^d \ltimes U(d)$ invariant subcomplex with the complex of closed $(d+1)$-forms on $B\fg$:
\ben
J : \Omega^{d+1}_{cl}(B \fg) \xto{\simeq} \left(\Def_{\CC^d \to B \fg}\right)^{\CC^d \ltimes U(d)} .
\een
\end{thm}

To compute the translation invariant deformation complex we will invoke Proposition \brian{hol trans invt def} from Section \brian{ref}.
Note that the deformation complex is simply the (reduced) local cochains on the local Lie algebra $\Omega^{0,*}_{\CC^d} \tensor \fg$. 
Thus, in the notation of Section \brian{same ref} the bundle $L$ is simply the trivial bundle $\fg$.
Thus, we see that the translation invariant deformation complex is quasi-isomorphic to the following cochain complex
\ben
\left(\Def^{\rm cot}_{Y \to B\fg}\right)^{\CC^d} \; \simeq \; \CC \cdot \d^d z \tensor^{\LL}_{\CC\left[\frac{\partial}{\partial z_i}\right]} \cred^*(\fg[[z_1,\ldots,z_d]])  .
\een
We'd like to recast the right-hand side in a more algebraic way. 

Note that the the algebra $\CC\left[\frac{\partial}{\partial z_i}\right]$ is the enveloping algebra of the abelian Lie algebra $\CC^d = \CC\left\{\frac{\partial}{\partial z_i}\right\}$. 
Thus, the complex we are computing is of the form
\ben
\CC \tensor^{\LL}_{U(\CC^d)} \cred^*(\fg[[z_1,\ldots,z_d]]) .
\een
This is precisely the Chevalley-Eilenberg cochain complex computing Lie algebra homology of $\CC^d$ with values in the module $\cred^*(\fg[[z_1,\ldots,z_d]])$:
\ben
\left(\Def^{\rm cot}_{Y \to B\fg}\right)^{\CC^d} \; \simeq  \; \clieu_*\left(\CC^d ; \cred^*(\fg[[z_1,\ldots,z_d]]) \d^d z\right) .
\een
Note that $\CC^d$ acts on $\d^d z$ trivially.
We want to keep $\d^d z$ in the notation since below we are interested in computing the $\GL_d$-invariants.

To compute the cohomology of this complex, we will first describe the differential explicitly. 
There are two components to the differential.
The first is the ``internal" differential coming from the Lie algebra cohomology of $\fg [[z_1,\ldots,z_d]]$, we will write this as $\d_{\fg}$. 
The second comes from the $\CC^d$-module structure on $\clie^*(\fg[[z_1,\ldots,z_n]])$ and is the differential computing the Lie algebra homology, which we denote $\d_{\CC^d}$. 
We will employ a spectral sequence whose first term turns on the $\d_{\fg}$ differential.
The next term turns on the differential $\d_{\CC^d}$.

As a graded vector space, the cochain complex we are trying to compute has the form
\ben
\Sym(\CC^d [1]) \tensor \cred^*\left(\fg[[z_1,\ldots,z_d]])\right) \d^d z .
\een
The spectral sequence is induced by the increasing filtration of $\Sym(\CC^d [1])$ by symmetric powers
\ben
F^k = \Sym^{\leq k}(\CC^d[1]) \tensor \cred^*\left(\fg[[z_1,\ldots,z_d]])\right) \d^d z .
\een
As above, we write the generators of $\CC^d$ by $\frac{\partial}{\partial z_i}$. 
Also, note that the reduced Chevalley-Eilenberg complex has the form
\ben
\cred^*(\fg[[z_1,\ldots,z_n]]) = \left(\Sym^{\geq 1} \left(\fg^\vee [z_1^\vee,\ldots,z_d^\vee][-1] \right), \d_{\fg}\right),
\een
where $z_i^\vee$ is the dual variable to $z_i$. 

Recall, we are only interested in the $U(d)$-invariant subcomplex of this deformation complex. 
Sitting inside of $U(d)$ we have $S^1 \subset U(d)$ as multiples of the identity. 
This induces an overall weight grading to the complex.
The group $U(d)$ acts in the standard way on $\CC^d$.
Thus, $z_i$ has weight $(+1)$ and both $z_i^\vee$ and $\frac{\partial}{\partial z_i}$ have $S^1$-weight $(-1)$. 
Moreover, the volume element $\d^d z$ has $S^1$ weight $d$.
It follows that in order to have total $S^1$-weight that the total number of $\frac{\partial}{\partial z_i}$ and $z_i^\vee$ must add up to $d$.
Thus, as a graded vector space the invariant subcomplex has the following decomposition
\ben
\bigoplus_k \Sym^k(\CC^d[1]) \tensor \left(\bigoplus_{i \leq d-k} \Sym^{i} \left(\fg^\vee [z_1^\vee,\ldots,z_d^\vee][-1] \right) \right) \d^d z .
\een
It follows from Schur-Weyl that the space of $U(d)$ invariants of the $d$th tensor power of the fundamental representation $\CC^d$ is one-dimensional, spanned by the top exterior power. 
Thus, when we pass to the $U(d)$-invariants, only the unique totally antisymmetric tensor involving $\frac{\partial}{\partial z_i}$ and $z_i^\vee$ survives. 
Thus, for each $k$, we have
\be\label{U(d) invariants}
\left(\Sym^k(\CC^d[1]) \tensor \left(\bigoplus_{i \leq d-k} \Sym^{i} \left(\fg^\vee [z_1^\vee,\ldots,z_d^\vee][-1] \right) \right) \d^d z\right) \cong \wedge^{k}\left(\frac{\partial}{\partial z_i}\right) \wedge \wedge^{d-k}\left(z_i^\vee\right) \clie^*\left(\fg , \Sym^{d-k}(\fg^\vee)\right) \d^d z .
\ee
Here, $\wedge^{k}\left(\frac{\partial}{\partial z_i}\right) \wedge \wedge^{d-k}\left(z_i^\vee\right)$ is just a copy of the determinant $U(d)$-representation, but we'd like to keep track of the appearances of the partial derivatives and $z_i^\vee$. 
Note that for degree reasons, we must have $k \leq d$. 
When $k = 0$ this complex is the (shifted) space of functions modulo constants on the formal moduli space $B\fg$, $\sO_{red}(B\fg)[d]$. 
When $k \geq 1$ this the (shifted) space of $k$-forms on the formal moduli space $B\fg$, which we write as $\Omega^{k}(B \fg)[d+k]$.
Thus, we see that before turning on the differential on the next page, our complex looks like
\be\label{bg def complex1}
\xymatrix{
\ul{-2d} & \cdots & \ul{-d-1} & \ul{-d} \\
\sO_{red}(B \fg) & \cdots & \Omega^{d-1} (B \fg) & \Omega^{d}(B \fg) .
}
\ee
We've omitted the extra factors for simplicity. 

We now turn on the differential $\d_{\CC^d}$ coming from the Lie algebra homology of $\CC^d = \CC\left\{\frac{\partial}{\partial z_i}\right\}$ with values in the above module. 
Since this Lie algebra is abelian the differential is completely determined by how the operators $\frac{\partial}{\partial z_i}$ act.
We can understand this action explicitly as follows.
Note that $\frac{\partial}{\partial z_i} z_j = \delta_{ij}$, thus we may as well think of $z_i^\vee$ as the element $\frac{\partial}{\partial z_i}$. 
Consider the subspace corresponding to $k=d$ in Equation (\ref{U(d) invariants}):
\ben
\frac{\partial}{\partial z_1} \cdots \frac{\partial}{\partial z_d} \cred^*(\fg) \d^d z .
\een 
Then, if $x \in \fg^\vee [-1] \subset \cred^*(\fg)$ we observe that
\ben
\d_{\CC^d} \left(\frac{\partial}{\partial z_1} \cdots \frac{\partial}{\partial z_d} \tensor f \tensor \d^d z \right) = \det (\partial_i, z_j^\vee) \tensor 1 \tensor x \tensor \d^d z \in  \wedge^{d-1}\left(\frac{\partial}{\partial z_i}\right) \wedge \wedge^1 \left(z_i^\vee\right) \clie^*\left(\fg , \fg^\vee \right) \d^d z.
\een
This follows from the fact that the action of $\frac{\partial}{\partial z_i}$ on $x = x \tensor 1 \in \fg^\vee \tensor \CC[z_i^\vee]$ is given by
\ben
\frac{\partial}{\partial z_i} \cdot (x \tensor 1) = 1 \tensor x \tensor z_i^\vee \in \clie^*(\fg , \fg^\vee) z_i^\vee .
\een
By the Leibniz rule we can extend this to get the formula for general elements $f \in \cred^*(\fg)$. 
We find that getting rid of all the factors of $z_i$ we recover precisely the de Rham differential 
\ben
\xymatrix{ 
\cred^*(\fg) [2d] \ar@{=}[d] \ar[r]^-{\d_{\CC^d}} & \clie^*(\fg , \fg^\vee) [2d-1] \ar@{=}[d] \\
\sO_{red}(B\fg) \ar[r]^-{\partial} & \Omega^1(B \fg) .
}
\een


%where the direct sum runs over multi-indices $I = (i_1,\ldots,i_k)$. 
%As usual, the we have used the multi-index notation for differential operators and $z_I^\vee = z_{i_1} \cdots z_{i_k
%
%\brian{...}
%An element of $(\lambda_1,\ldots,\lambda_d)$ acts on $z = (z_1,\ldots, z_d)$ via $(\lambda_1z_1,\ldots,\lambda_d z_d)$.
%Further, 
%\ben
%(\lamda_1,\ldots, \lambda_d) \cdot \frac{\partial}{\partial z_i} = \lambda_{i}^{-1} \frac{\partial}{\partial z_i} .
%\een
%
%\ben
%f \left(\frac{\partial}{\partial z_1}, \cdots, \frac{\partial}{\partial z_j}) 



\end{document}