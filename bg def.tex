\documentclass[10pt]{amsart}

\usepackage{macros,slashed}

\linespread{1.25}

\title{Deformations of higher CDOs}

\def\brian{\textcolor{blue}{BW: }\textcolor{blue}}

\begin{document}
\maketitle

\section{Deformations for an $L_\infty$ space}

We have already pointed out that our theory is a cotangent theory. 
In particular, there is an action of the abelian group $\CC^\times_{cot}$ which assigns the base direction a weight of zero and the fiber a weight of $+1$. 
Thus, if $(\gamma, \beta) \in \Omega^{0,*}(Y, \fg) \oplus \Omega^{d,*}(Y, \fg^\vee)[d-1]$, then an element $\lambda \in \CC^\times_{cot}$ acts by
\ben
\lambda \cdot (\gamma, \beta) = (\gamma, \lambda \beta) .
\een
Our first reduction is to restrict ourselves to studying deformations that are compatible with this $\CC^\times_{cot}$ action.

Note that the symplectic pairing of the theory, as well as the classical action functional, is of $\CC^{\times}_{cot}$-weight $(-1)$.
Our convention is that the parameter $\hbar$ {\em has $\CC^\times_{cot}$-weight $(-1)$} as well. 
There are two compelling reasons for making this definition. 
The first deals with studying correlation functions for the theory. 
If we require the observables of the theory to be equivariant for this rescaling of the cotangent fibers, this means that the factorization product must have $\CC^\times_{cot}$ weight zero.
In the case that the theory is free, we have seen that the factorization product between two operators of the theory $\cO, \cO'$ is computed by a Moyal type formula
\ben
\cO \star \cO' = e^{-\hbar \partial_P} \left(e^{\hbar \partial_P} \cO \cdot e^{\hbar \partial_P} \cO' \right) .
\een
Since the symplectic pairing is $\CC^\times_{cot}$-weight $(-1)$ we observe that the propagator is also $\CC^\times_{cot}$-weight $(+1)$.
\footnote{This actually requires that we also take the gauge fixing operator to be of $\CC^\times_{cot}$-weight zero, which is the natural thing to do for cotangent theories.}
For the product to have weight zero we are then forced to take $\hbar$ to have opposite weight to $P$.

The other, related reason, we choose this weight for $\hbar$ is that we would like to require our BV complex to be equivariant for rescaling the fibers as well. 
The classical BRST differential is of the form $\{S,-\} = Q + \{I,-\}$.
We have already said that the classical action is of weight $(-1)$.
Since the symplectic pairing is also degree $(-1)$, this means that the $P_0$ bracket is degree $+1$. 
Thus, the classical BRST complex is manifestly equivariant.
The quantum BV differential involves deforming this classical differential by $\hbar \Delta$. 
For the same reason as the Poisson bracket, the BV Laplacian has weight $(+1)$. 
Thus, we see that in order to have an equivariant differential we are again forced to take $\hbar$ to have weight $-1$. 

In the case of an interacting theory, we have the following restriction on the quantum interactions of the theory as well. 
We can expand an effective interaction as
\ben
I[L] = \sum_{g \geq 0} \hbar^g I^{(g)}[L] .
\een
In order for $I[L]$ to have $\CC^\times_{cot}$ weight $(-1)$ we see that $I^{(g)}[L]$ must have weight $g-1$. 
We are only studying a one-loop quantization of the holomorphic theory, so the effective action has the form $I[L] = I^{(0)} + \hbar I^{(1)}[L]$ and hence $I^{(1)}[L]$ has weight zero. 

Thus, all one-loop quantities compatible with the $\CC^\times_{cot}$ action also have weight zero, including the one-loop anomaly. 
For this reason, we will be most concerned with the piece of the deformation complex that is $\CC^{\times}_{cot}$-weight zero. 
This amounts to looking just at local functionals of the $\gamma$-field.

\begin{dfn} 
The {\em deformation complex for cotangent quantizations} of the holomorphic $\sigma$-model of maps $Y \to B \fg$ is the cochain complex 
\ben
\Def^{\rm cot}_{Y \to B\fg} = \cloc^*(\Omega^{0,*}_Y \tensor \fg) .
\een
This is simply the local cochains of the local Lie algebra $\Omega^{0,*}_Y \tensor \fg$ on $Y$. 
\end{dfn}

We will be most interested in seeing how both the anomaly and the resulting quantum correction induced by the anomaly are realized inside the complex $\Def_{Y \to Bg}^{\rm cot}$. 
Before doing this, we'd like to restrict ourselves to looking at quantizations preserving further symmetries. 

\brian{do this}

To compute the translation invariant deformation complex we will invoke Proposition \brian{hol trans invt def} from Section \brian{ref}.
Note that the deformation complex is simply the (reduced) local cochains on the local Lie algebra $\Omega^{0,*}_{\CC^d} \tensor \fg$. 
Thus, in the notation of Section \brian{same ref} the bundle $L$ is simply the trivial bundle $\fg$.
Thus, we see that the translation invariant deformation complex is quasi-isomorphic to the following cochain complex
\ben
\left(\Def^{\rm cot}_{Y \to B\fg}\right)^{\CC^d} \; \simeq \; \CC \cdot \d^d z \tensor^{\LL}_{\CC\left[\frac{\partial}{\partial z_i}\right]} \cred^*(\fg[[z_1,\ldots,z_d]])  .
\een
We'd like to recast the right-hand side in a more algebraic way. 

Note that the the algebra $\CC\left[\frac{\partial}{\partial z_i}\right]$ is the enveloping algebra of the abelian Lie algebra $\CC^d = \CC\left\{\frac{\partial}{\partial z_i}\right\}$. 
Thus, the complex we are computing is of the form
\ben
\CC \tensor^{\LL}_{U(\CC^d)} \cred^*(\fg[[z_1,\ldots,z_d]]) .
\een
This is precisely the Chevalley-Eilenberg cochain complex computing Lie algebra homology of $\CC^d$ with values in the module $\cred^*(\fg[[z_1,\ldots,z_d]])$:
\ben
\left(\Def^{\rm cot}_{Y \to B\fg}\right)^{\CC^d} \; \simeq  \; \clieu_*\left(\CC^d ; \cred^*(\fg[[z_1,\ldots,z_d]])\right) .
\een

\end{document}