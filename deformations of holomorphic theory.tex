\documentclass[10pt]{amsart}

\usepackage{macros,slashed}

\linespread{1.25}

\title{Deformations of a holomorphic theory}

\def\brian{\textcolor{blue}{BW: }\textcolor{blue}}

\def\hD{\Hat{D}}

\begin{document}

\subsection{Free holomorphic theories}

We restrict ourselves to considering theories on the manifold $\CC^d = \RR^{2n}$ equipped with its standard complex structure. 
Fix a holomorphic vector bundle $V$ on $\CC^d$ and an identification of bundles 
\ben
V \cong \CC^d \times V_0
\een
where $V_0$ is the fiber of $V$ at $0 \in \CC^d$. 
We want to consider a classical theory with space of fields given by $\Omega^{0,*}(\CC^d, V)$. 
Moreover, we want this theory to be invariant with respect to the group of translations on $\CC^d$. 
Per usual, it is easiest to work with the corresponding Lie algebra of translations. 
Moreover, using the complex structure, we choose a presentation for this Lie algebra as
\ben
\CC^d \cong {\rm span} \left\{\frac{\partial}{\partial z_i}, \frac{\partial}{\partial \zbar_i}\right\}_{1 \leq i \leq d}.
\een

The first thing we need to do is fix a $(-1)$-shifted symplectic pairing.
To do this, we suppose that a we have a translation invariant pairing on $V$ valued in the canonical bundle $K_{\CC^d}$.
That is, suppose 
\be\label{pairing 1}
\< \;\;,\;\; \>_V : V \tensor V \to K_{\CC^d} [d-1]
\ee
is a skew-symmetric bundle map that is equivariant for the Lie algebra of translations. 
The shift is so that the resulting pairing on the Dolbeualt complex is of the appropriate degree.
Here, equivariance means that for sections $v,v'$ we have
\ben
\< \frac{\partial}{\partial z_i} v, v'\>_V = L_{\partial_{z_i}} \<v,v'\>_V
\een
where the right-hand side denotes the Lie derivative applied to $\<v,v'\>_V \in K_{\CC^d}$. 
There is a similar relation for the anti-holomorphic derivatives. 
We obtain a $\CC$-valued pairing on $\Omega^{0,*}_c(\CC^d , V)$ via integration:
\ben
\int_{\CC^d} \circ \<\;\;,\;\;\>_V : \Omega^{0,*}_c (\CC^d , V) \tensor \Omega^{0,*}_c(\CC^d , V) \xto{\wedge \cdot \<\;, \;\>_V} \Omega^{d,*}(\CC^d) \xto{\int} \CC .
\een
The first arrow is the wedge product of forms combined with the pairing on $V$. 
The second arrow is only nonzero on forms of type $\Omega^{d,d}$. 
Clearly, integration is translation invariant, so that the composition is as well. 

This pairing $\Omega^{0,*}(\CC^d , V)$ together with the differential $\dbar$ are enough to define a free theory. 
However, it is convenient to consider a slightly generalized version of this situation. 
We want to allow deformations of the differential $\dbar$ on Dolbeault forms of the form
\ben
Q = \dbar + Q^{hol}
\een
where $Q^{hol}$ is a holomorphic differential operator of the form
\be\label{hol operator}
Q^{hol} = \sum_I \frac{\partial}{\partial z^I} \mu_I
\ee
where $I$ is some multi-index and $\mu_I : V \to V$ is a linear map of cohomological degree $+1$. 
Note that we have automatically written $Q^{hol}$ in a way that it is translation invariant.
Of course, for this differential to define a free theory there needs to be some compatibility with the pairing on $V$. 
We can summarize this in the following definition, which should be viewed as a slight modification of a free theory to this translation invariant holomorphic setting. 

\begin{dfn} A {\em holomorphically translation invariant free BV theory} is the data of a holomorphic vector bundle $V$ together with
\begin{enumerate}
\item an identification $V \cong \CC^d \times V_0$;
\item a translation invariant skew-symmetric pairing  $\<-,-\>_V$ as in (\ref{pairing 1});
\item a holomorphic differential operator $Q^{hol}$ as in (\ref{hol operator});
\end{enumerate}
such that the following conditions hold
\begin{enumerate}
\item the induced $\CC$-valued pairing $\int \circ \<-,-\>_V$ is non-degenerate;
\item the operator $Q^{hol}$ satisfies $(\dbar + Q^{hol})^2 = 0$ and
\ben
\int \<Q^{hol} v, v'\>_V = \pm \int \<v, Q^{hol} v'\> .
\een
\end{enumerate}
\end{dfn}

The first condition is required so that we obtain an actual $(-1)$-shifted symplectic structure on $\Omega^{0,*}(\CC^d, V)$. 
The second condition implies that the derivation $Q = \dbar + Q^{hol}$ defines a cochain complex
\ben
\sE_V = \left(\Omega^{0,*}(\CC^d, V), \dbar + Q^{hol}\right),
\een
and that $Q$ is skew self-adjoint for the symplectic structure. 
Thus, in particular, $\sE_V$ together with the pairing define a free BV theory in the ordinary sense. 
In the usual way, we obtain the action functional via
\ben
S(\varphi) = \int \<\varphi, (\dbar + Q^{hol}) \varphi\>_V .
\een 

Before going further, we will list some familiar examples.

\begin{eg}\label{eg bg affine} {\em The free $\beta\gamma$ system on $\CC^d$}.
Suppose that 
\ben
V = \ul{\CC} \oplus K_{\CC^d} [d - 1] .
\een
Let $\<\;,\;\>_V$ be the pairing
\ben
(\ul{\CC} \oplus K_{\CC^d}) \tensor (\ul{\CC} \oplus K_{\CC^d}) \to K_{\CC^{d}} \oplus K_{\CC^d} \to K_{\CC^d} 
\een 
sending $(\lambda, \mu) \tensor (\lambda',\mu') \mapsto (\lambda \mu', \lambda'\mu) \mapsto \lambda\mu' + \lambda' \mu$.
Finally, let $Q^{hol} = 0$. 
One immediately checks that this is a holomorphically translation invariant free theory as above.
The space of fields can be written as
\ben
\Omega^{0,*}(\CC^d) \oplus \Omega^{d,*}(\CC^d)[d - 1] .
\een 
We write $\gamma$ for a field in the first component, and $\beta$ for a field in the second component. 
The action functional reads
\ben
S(\gamma + \beta, \gamma'+\beta') = \int_{\CC^d} \beta \wedge \dbar \gamma' + \beta' \wedge \dbar \gamma .
\een 
When $d = 1$ this reduces to the ordinary chiral $\beta\gamma$ system from conformal field theory \brian{ref}. 
We will discuss this higher dimensional version further in Section \brian{}.
For instance, we will see how this theory is the starting block for constructing general holomorphic $\sigma$-models. 
\end{eg}

Of course, there are many variants of the $\beta\gamma$ system that we can consider.
For instance, if $E$ is {\em any} holomorphic vector bundle we can take 
\ben
V = E \oplus K_{\CC^d} \tensor E^\vee
\een
where $E^\vee$ is the linear dual bundle. 
The pairing is constructed as in the case above where we also use the evaluation pairing between $E$ and $E^\vee$, ${\rm ev}_E : E \tensor E^\vee \to \CC$.
In thise case, the fields are $\gamma \in \Omega^{0,*}(\CC^d, E)$ and $\beta \in \Omega^{d,*}(\CC^d, E^\vee)[d-1]$. 
The action functional is simply
\ben
S(\gamma, \beta) = \int_{\CC^d} {\rm ev}_E(\beta \wedge \dbar \gamma) .
\een

\begin{eg} {\em Topological ...}
Consider the above example with $Q^{hol} = \partial$..\brian{finish}. 
\end{eg}

\subsection{Interacting holomorphic theories}

It is convenient to introduce the following set of degree $-1$ derivations of $\Omega^{0,*}(\CC^d)$ given by
\ben
\Bar{\eta}_i := \frac{\partial}{\partial (\d \zbar_i)} .
\een
The right-hand side is sometimes written using the interior derivative notation $\iota_{\partial / \partial \zbar_i}$. 
By a holomorphic version of ``Cartan's magic formula" these derivations satisfy the relation
\ben
L_{\frac{\partial}{\partial \zbar_i}} = \dbar \Bar{\eta}_i + \Bar{\eta}_i \dbar .
\een
In addition, they serve to define homotopies for the following holomorphic version of Poincar\'{e}'s lemma. 
First, consider the algebraic case. 
Let $\AA^d$ be the complex $d$-dimensional affine space with space of smooth algebraic functions $\sO^{alg, sm}(\AA^d) = \CC[z_1,\ldots, z_d, \zbar_1,\ldots,\zbar_d]$. 
Then, we can build the algebraic Dolbeualt complex 
\ben
\Omega^{0,*}_{alg} (\AA^d) = \CC[z_1,\ldots,z_d, \zbar_1,\ldots,\zbar_d][\d \zbar_1,\ldots,\d \zbar_d]
\een
where the $\d \zbar_i$'s are in cohomological degree $1$. 
The Dolbeault differential $\dbar$ is define in the same way. 
Note that the operators $\Bar{\eta}_i$ also make sense on $\Omega^{0,*}_{alg}(\AA^d)$. 

\begin{lem} 
The map 
\ben
\CC[z_1,\ldots, z_d] \hookrightarrow \Omega^{0,*}_{alg}(\AA^d)
\een
is a quasi-isomorphism.
\end{lem}
\begin{proof}
Note that
\ben
\Omega^{0,*}_{alg}(\AA^d) \cong \CC[z_1,\ldots, z_d] \tensor_\CC \CC[\zbar_1,\ldots,\zbar_d][\d \zbar_1,\ldots,\d \zbar_d] .
\een
The right-hand term is one-dimensional, concentrated in degree zero by the ordinary Poincar\'{e} lemma. 
An explicit homotopy for a \brian{...}
\end{proof}

The analogous result holds with $\AA^d$ replaced by the formal $d$-disk $\hD^d$. 

For the general case...

\begin{lem} \brian{find good citation. should I just state the general Stein resut?}
The map
\ben
\sO^{hol} (\CC^d) \hookrightarrow \Omega^{0,*}(\CC^d)
\een
is a quasi-isomorphism.
\end{lem}

\brian{Include equivalence with certain structured local Lie algebras. Namely holomorphically translation invariant local Lie algebras.}

Local Lie \footnote{Local Lie algebras will mean $L_\infty$...} are a convenient language to phrase the data of an interacting classical field theory. 
In \brian{ref} it is shown that the data of a local Lie algebra together with a non-degenerate pairing of degree $-3$ is equivalent to the data of a classical interacting BV theory. 
It will be convenient for us to formulate, under this equivalence, a Lie theoretic interpretation of holomorphically translation invariant interacting BV theories. 
First, we introduce the following definition. 
Recall, data of a local Lie algebra is a $\ZZ$-graded vector bundle $L$ together with poly-differential operators 
\ben
\ell_n : L \tensor \cdots \tensor L \to L 
\een 
for $n \geq 1$, satisfying some conditions. 
The sheaf of smooth sections of this bundle will be denoted by $\sL$, which inherits from the operators $\{\ell_n\}$ the structure of a sheaf of $L_\infty$ algebras. 
We will often refer to the local Lie algebra simply by its sheaf of sections. 

\begin{dfn}
A translation invariant local Lie algebra on $\RR^n$ is a local Lie algebra $\sL$,
together with an identification $L = \RR^n \times L_0$ such that for each $n$ the structure map
\ben
\ell_n : \RR^n \times (L_0 \tensor \cdots \tensor L_0) \to \RR^n \times L_0
\een
is compatible with translations. 
\end{dfn}

\subsection{Holomorphic deformations}

Any local Lie algebra on a manifold endows the structure of an $L_\infty$ algebra on its fibers. 
Suppose $\sL$ is a holomorphically translation invariant local Lie algebra on $\CC^d$ of the form $\Omega^{0,*}(\CC^d, L)$ where $L$ is a graded holomorphic vector bundle....

%\begin{lem} Let $\sL_0$ denote the fiber of $\sL$ over $0 \in \CC^d$. 
%Then, if $\ell_1 = \dbar$, there is an equivalence of $L_\infty$ algebras
%\ben
%L_0 \xto{\simeq} \sL_0
%\een 
%where $L_0$ is the fiber of the holomorphic bundle $L$ over $0 \in \CC^d$. 
%\end{lem}

\begin{prop} Suppose $\sL$ is a holomorphically translation invariant local Lie algebra on $\CC^d$ such that $\ell_1 = \dbar$.
Then, one has
\ben
\cloc^*(\sL)^{\CC^d} \simeq K_{\CC^d} \tensor^{\LL}_{\CC[\frac{\partial}{\partial z_i}]} \cred^*(L_0 [[z_1,\ldots,z_d]]).
\een
\end{prop}

For instance, if $L = \ul{\fg}$ is the constant bundle on $\CC^d$ where $\fg$ is an ordinary Lie (or $L_\infty$) algebra one has $L_0 = \fg$ so that
\ben
\cloc^*(\Omega^{0,*}(\CC^d, \fg))^{\CC^d} \simeq \CC \cdot \d^d z \tensor^{\LL}_{\CC[\frac{\partial}{\partial z_i}]} \cred^*(\fg [[z_1,\ldots,z_d]]) .
\een
\end{document}
