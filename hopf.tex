\documentclass[10pt]{amsart}

\usepackage{macros,slashed}

\linespread{1.25}

\title{The theory on Hopf surfaces}

\def\brian{\textcolor{blue}{BW: }\textcolor{blue}}

\def\SU{{\rm SU}}
\def\Spin{{\rm Spin}}
\def\dvol{{\rm dvol}}
\def\dslash{\slashed{\partial}}
\begin{document}
\maketitle

The local theory of the holomorphic $\sigma$-model where the source is $\CC^d$ has been studied in the last section.
We now turn to global aspects of the theory which amounts to putting the theory on more exotic source manifolds. 
In this section we consider the holomorphic $\sigma$-model where the source is a particular compact complex $d$-fold. 
In fact, we will allow for holomorphic families of compact complex manifolds as the source and study how the partition function varies in this family. 

\subsection{The case of an elliptic curve}

As a warmup we recall a familiar situation 

\subsection{Hopf manifolds}

We focus on a family of complex manifolds defined by Hopf in \cite{Hopf} defined in every complex dimension $d$. 
For $1 \leq i \leq d$ let $q_i \in D(0,1)^{\times}$ be a nonzero complex number of modulus $|q_i| <1$. 
The $d$-dimensional {\em Hopf manifold of type} ${\bf q} = (q_1,\ldots,q_d)$ is the following quotient of punctured affine space $\CC^d \setminus \{0\}$ by the discrete group $\ZZ^d$:
\ben
H_{\bf q} = \left. \left(\CC^d \setminus \{0\}\right) \right/ \left( (z_1,\ldots,z_d) \sim (q_1^{2\pi i \ZZ} z_1, \ldots,q_d^{2 \pi i \ZZ} z_d) \right) .
\een
Note that in the case $d = 1$ we recover the usual description of an elliptic curve $H_{\bf q} = E_q = \CC^\times / q^{2 \pi i \ZZ}$. 

For any $d$ and tuple $(q_1,\ldots, q_d)$ as above, we see that as a smooth manifold there is a diffeomorphism $H_{\bf q} \cong S^{2d-1} \times S^1$. 
Indeed, the radial projection map $\CC^d \setminus \{0\} \to \RR_{>0}$ defines a smooth $S^{2d-1}$-fibration over $\RR_{>0}$. 
Passing to the quotient, we obtain an $S^{2d - 1}$-fibration 
\ben
H_{\bf q} \to \left. \RR_{>0} \right/ \left(r \sim \lambda^{\ZZ} \cdot r \right) \cong S^1 .
\een
Here, $\lambda = (|q_1|^2 + \cdots + |q_d|^2)^{1/2} > 0$. 
Since there are no non-trivial $S^{2d-1}$ fibrations over $S^1$ we obtain $H_{\bf q} = S^{2d-1} \times S^1$. 

There is an equivalent description of $H_{\bf q}$ as a quotient of affine space that we will take advantage of. 

\section{A higher chiral algebraic index}

\begin{prop}\label{prop hopf cohomology}
Any Hopf manifold is strictly formal. 
In particular, if $X$ is a Hopf manifold, then there is a bigraded quasi-isomorphism
\ben
\left(\Omega^{*,*}(X) , \dbar \right) \simeq \left(H^{*,*}_{\dbar}(X), 0\right) .
\een
Moreover, let $\delta, \epsilon$ be formal parameters of bidegree $(0,1)$ and $(d-1,d)$. 
Then, there is a bigraded isomorphism
\ben
H^{*,*}_{\dbar} (X) = H^*(X , \Omega^*_{hol}) \cong \CC[\delta,\epsilon] / (\epsilon^2)
\een
where $\delta$ has bi-degree $(0,1)$ and $\epsilon$ has bi-degree $(d,d-1)$. 
\end{prop}
\begin{proof}
This proof is essentially in Appendix \cite{HirzebruchTopMethods}.
We use the Borel spectral sequence for a holomorphic fibration
\ben
\xymatrix{
F \ar[r] & E \ar[d] \\
& B 
}
\een
which exists so long as we assume that $F$ is K\:{a}hler \cite{Hirzebruch}. 
The spectral sequence has pages $(E_r, \d_r), r \geq 0$, each of which is trigraded $E_{r}^{p,q,s}$, where $(p,q)$ are the the Dolbeault types and $s$ is the internal degree of the spectral sequence. 
The $E_1$ page is given by
\ben
E_{1}^{p,q,s} = \oplus_{i} \Omega^{i, s-i}(B) \tensor H^{p-i, q-s+i}(F)
\een
with differential $\dbar_B$ the Dolbeualt differential on the base $B$. 
Note that since $F$ is K\:{a}hler, it is strictly formal and their exists a Dolbeault model of the form $(\CC[V], \d)$. 
By general nonsense, we can hence obtain a model for the total space $E$ by deforming the differential on $(\Omega^{*,*}(B) \tensor \CC[V], \dbar + \d)$ using the above spectral sequence.

Any Hopf manifold $X$ of complex dimension $d$ sits in a holomorphic fibration
\ben
\xymatrix{
T \ar[r] & X \ar[d] \\
& \CC P^{d-1}
}
\een
where $T = S^1\times S^1$ is a torus. 
Of course, $T$ is K\:{a}hler with Dolbeualt model given by
\ben
\left(\CC[\delta, x], 0\right)
\een
where $\delta$ has bidegree $(0,1)$ and $x$ has bidegree $(1,0)$. 
An explicit Dolbeualt model for $\CC P^{d-1}$ is given by
\ben
(\CC[y,\epsilon] , \dbar_B) \;\;\;, \;\; |y| = (1,1) \;\;, \;\; |\epsilon| = (d-1,d) \;\; , \; \; \dbar_B(\epsilon) = y^d .
\een

Thus, the $E_1$ page is of the form
\ben
\left(\CC[x,y,\delta,\epsilon] , \dbar_B\right) .
\een
The only possible differential on the $E_2$ page must have the form $\d_2(x) = c y$, where $c$ is some constant. 
By rescaling generators we can assume that $c = 0$ or $1$. 
The following fact will determine the value of this constant.

\begin{lem} \brian{either cite Hirzebruch, Kodaira, or find easy proof} For any Hopf manifold $X$ one has 
\ben
H^{1,0}_{\dbar}(X) = H^0(X , \Omega^1_{hol}) = 0 .
\een
\end{lem}

This lemma implies that $x$ must support a differential and with our normalization above $\d_2 x = y$. 
It follows that a model for $X$ is given by
\ben
\left(\CC[x,y,\delta,\epsilon] , \dbar_B + \eta\right) 
\een
where $\dbar_B(\epsilon) = y^d$ and $\eta(x) = y$. 
The statements of the proposition follow immediately.
\end{proof}

This result will allow us to write down an explicit model for the factorization homology of the higher dimensional $\beta\gamma$ system along an arbitrary Hopf surface. 
Since the method is very similar to the case of an elliptic curve, we will be brief in the details. 

Let $X$ be an arbitrary Hopf surface. 
We will first work in the free situation where the target is a vector space $V$. 
The fields of the theory on $X$ are given by
\ben
\Omega^{0,*}(X , V) \oplus \Omega^{d,*}(X, V^*)[d-1] .
\een
By Proposition \ref{prop hopf cohomology} we know that there is a quasi-isomorphism
\begin{align*}
\Omega^{0,*}(X , V) \oplus \Omega^{d,*}(X, V^*)[d-1] & \simeq H^{0,*}_{\dbar} (X, V) \tensor H^{d,*}_{\dbar}(X , V^*)[d-1] \\ & \cong \CC[\delta] \tensor V \oplus \epsilon \CC[\delta] \tensor V^* \\ & = \CC[\delta] \tensor (V \oplus \epsilon V^*) .
\end{align*}
where $\delta$ has degree $+1$ and $\epsilon$ has degree $d-1$. 
(Note that only the $(0,*)$-grading contributes to the cohomological degree in our language.)

As an immediate consequence, we see that the classical observables along $X$ satisfy
\ben
\Obs^{\cl}_V(X) \simeq \Omega^{-*}(T^*V) 
\een
where we view $\Omega^{-*}(T^*V) = \sO(\CC[\delta] \tensor (V \oplus \epsilon V[d-1])$.
Since $\epsilon$ has degree $d-1$, the graded vector space $\epsilon V[d-1]$ sits in degree zero, so we have omitted the parameter $\epsilon$ from the notation. 

\begin{prop}
Let $X$ be any Hopf manifold.
For any scale $L$, there is a quasi-isomorphism
\ben
\Obs^{\q}_V(X) [L] \simeq \left(\Omega^{-*}(T^*V) [\hbar] , \hbar L_\pi\right)
\een
where $L_\pi$ is the Lie derivative with respect to the Poisson tensor of the symplectic vector space $T^*V$.
\end{prop}


\end{document}