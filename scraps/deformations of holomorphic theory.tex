\documentclass[10pt]{amsart}

\usepackage{macros,slashed}

\linespread{1.25}

\title{Deformations of a holomorphic theory}

\def\brian{\textcolor{blue}{BW: }\textcolor{blue}}

\def\hD{\Hat{D}}

\begin{document}

The goal of this section is to define the notion of a holomorphic field theory. 

\subsection{The definition of a holomorphic theory}

We give a general definition of a classical holomorphic theory on a general complex manifold $X$ of complex dimension $d$.
We start with the definition of a {\em free} holomorphic field theory. 
After that we will go on to define what an interacting holomorphic theory is.

The fields of any theory are always expressed as sections of some $\ZZ$-graded vector bundle.
Here, the $\ZZ$-grading is the cohomological, or BRST, grading of the theory.
For a holomorphic theory we take this graded vector bundle to be holomorphic.  
By a {\em holomorphic} $\ZZ$-graded vector bundle we mean a $\ZZ$-graded vector bundle $
V = \oplus_i V^i$ such that each graded piece $V^i$ is a holomorphic vector bundle. 
Thus, the data we start with is the following:

\begin{itemize}
\item[(1)] a $\ZZ$-graded holomorphic vector bundle $V^* = \oplus_i V^i [-i]$, so that the finite dimensional holomorphic vector bundle $V^i$ is in cohomological degree $i$. 
\end{itemize}

A free classical theory is made up of a space of fields as above together with the data of a linearized BRST differential $Q^{BRST}$ and a symplectic pairing. 
Ordinarily, the BRST operator is a differential operator on the vector bundle defining the fields. 
For the class of theories we are considering, we want this operator to be holomorphic. 

If $E$ and $F$ are two holomorphic vector bundles on $X$, we can speak of holomorphic differential operators between $E$ and $F$. 
First, note that the Hom-bundle ${\rm Hom}(E,F)$ inherits a natural holomorphic structure. 
By definition, a holomorphic differential operator of order $m$ is a linear map
\ben
D : \Gamma^{hol}(X ; E) \to \Gamma^{hol}(X ; F)
\een
such that, with respect to a holomorphic coordinate chart $\{z_i\}$ on $X$, $D$ can be written as
\be\label{local holomorphic}
D|_{\{z_i\}} = \sum_{|I| \leq m} a_I (z) \frac{\partial^{|I|}}{\partial z_I}
\ee
where $a_I(z)$ is a local holomorphic section of ${\rm Hom}(E,F)$.
Here, the sum is over all multi-indices $I = (i_1,\ldots, i_d)$ and 
\ben
\frac{\partial^{|I|}}{\partial z_I} := \frac{\partial^{i_k}}{\partial z_k^{i_k}} . 
\een 
The length is defined by $|I| := i_1 + \cdots + i_d$. 

The most basic example of a holomorphic differential operator is the $\partial$ operator which, for each $1 \leq \ell \leq d$, is a holomorphic differential operator from $E = \wedge^\ell T^{1,0*}X$ to $F = \wedge^{\ell+1} T^{1,0*}X$. 
Locally, of course, it has the form
\ben
\partial = \sum_{i = 1}^{d} (\d z_i \wedge (-)) \frac{\partial}{\partial z_i},
\een
where $\d z_i \wedge (-)$ is the vector bundle homomorphism $\wedge^\ell T^{1,0*}X \to \wedge^{\ell+1} T^{1,0*}X$ sending $\alpha \mapsto \d z_i \wedge \alpha$. 

The next piece of data we fix is:
\begin{itemize}
\item[(2)] a holomorphic differential operator 
\ben
Q^{hol} : V \to V[-1]
\een
of cohomological degree $+1$. 
\end{itemize}

Finally, to define a free theory we need the data of a symplectic pairing. 
For reasons to become clear in a moment, we must choose this pairing to have a strange cohomological degree. 
The last piece of data we fix is:
\begin{itemize}
\item[(3)] a pairing 
\ben
(-,-)_V : V \tensor V \to K_{X}[d-1]
\een
such that for each $z \in X$ the induced pairing on the fibers $(-,-)_V|_{z}$ is non-degenerate.
Here $K_X$ is the holomorphic canonical bundle on $X$. 
\end{itemize}

The definition of the fields of an ordinary field theory are the {\em smooth} sections of the vector bundle $V$. 
In our situation this is a silly thing to do since we lose all of the data of the complex structure we used to define the objects above. 
The more natural thing to do is take the {\em holomorphic} sections of the vector bundle $V$. 
By construction, the operator $Q^{hol}$ and the pairing $(-,-)_V$ are defined on holomorphic sections, so on the surface this seems reasonable. 
\brian{what should I say the problem is with doing things in the analytic category?}

The solution to this problem is in the existence of a resolution for the holomorphic sections of a vector bundle by smooth sections of bundles. 
Given any holomorphic vector bundle $E$ we can define it's {\em Dolbeualt complex} $\Omega^{0,*}(X , E)$ with it's Dolbeualt operator 
\ben
\dbar : \Omega^{0,p}(X, E) \to \Omega^{0,p+1}(X, E) .
\een
Here, $\Omega^{0,p}(X, E)$ denotes sections of the vector bundle
\ben
\Wedge^p T^{0,1*} X \tensor E
\een
and $\dbar$ is defined in the usual way \brian{recall this?}.

Using this construction, we take the fields of our free theory to be the complex
\ben
\sE = \left(\Omega^{0,*}(X, E), \dbar + Q^{hol}\right) .
\een
The operator $\dbar + Q^{hol}$ is the total linearized BRST operator.
By assumption we have $\dbar Q^{hol} = Q^{hol} \dbar^*$ so that $(\dbar + Q^{hol})^2 = 0$ and so the fields still define a complex. 
The $(-1)$-shifted symplectic pairing is obtained by composition of the pairing $(-,-)_V$ with integration on $K_X$. 
The thing to observe here is that $(-,-)_V$ extends to the Dolbeualt complex in a natural way: we simply combine the wedge product of forms with the pairing on $V$.
The $(-1)$-shifted pairing $\omega$ on $\sE$ is defined by the diagram
\ben
\xymatrix{
\sE \tensor \sE \ar[r]^-{(-,-)_V} \ar@{.>}[dr]_-{\omega} & \Omega^{0,*}(X , K_X) [d-1] \ar[d]^-{\int_X} \\
& \CC[-1] .
}
\een

We arrive at the following definition.

\begin{dfn}
A {\em free holomorphic theory} on a complex manifold $X$ defined by the data $(V, Q^{hol}, (-,-)_V)$ as in (1), (2), (3) has fields $\Omega^{0,*}(X , V)$, linearized BRST differential $\dbar + Q^{hol}$, and $(-1)$-shifted symplectic structure $\omega$ as above.
\end{dfn}

It is immediate to check that this defines a classical free theory in the ordinary sense as in Definition \ref{?} of \cite{CG}. 
The usual prescription for writing down the associated action functional holds in this case.
If $\varphi \in \Omega^{0,*}(X , V)$ denotes a field the action is
\ben
S(\varphi) = \int_X \left(\varphi, (\dbar + Q^{hol}) \varphi)\right)_V .
\een

The first example we explain is related to the subject of Chapter \ref{chap holsig} and will serve as the fundamental example of a holomorphic theory. 

\begin{eg}\label{eg bg affine} {\em The free $\beta\gamma$ system}.
Suppose that 
\ben
V = \ul{\CC} \oplus K_{X} [d - 1] .
\een
Let $(-,-)_V$ be the pairing
\ben
(\ul{\CC} \oplus K_{X}) \tensor (\ul{\CC} \oplus K_{X}) \to K_{X} \oplus K_{X} \to K_{X} 
\een 
sending $(\lambda, \mu) \tensor (\lambda',\mu') \mapsto (\lambda \mu', \lambda'\mu) \mapsto \lambda\mu' + \lambda' \mu$.
In this example we set $Q^{hol} = 0$. 
One immediately checks that this is a holomorphic free theory as above.
The space of fields can be written as
\ben
\sE = \Omega^{0,*}(X) \oplus \Omega^{d,*}(X)[d - 1] .
\een 
We write $\gamma \in \Omega^{0,*}(X)$ for a field in the first component, and $\beta \in \Omega^{d,*}(X)[d - 1]$ for a field in the second component. 
The action functional reads
\ben
S(\gamma + \beta, \gamma'+\beta') = \int_{X} \beta \wedge \dbar \gamma' + \beta' \wedge \dbar \gamma .
\een 
When $d = 1$ this reduces to the ordinary chiral $\beta\gamma$ system from conformal field theory \brian{ref}. 
We will discuss this higher dimensional version further in Section \brian{}.
For instance, we will see how this theory is the starting block for constructing general holomorphic $\sigma$-models. 
\end{eg}



\subsection{Holomorphically translation invariant theories on $\CC^d$}

We first consider the affine manifold $\CC^d = \RR^{2n}$ equipped with its standard complex structure. 
Later on we will give a definition of a holomorphic theory on a general complex manifold.
Fix a holomorphic vector bundle $V$ on $\CC^d$.

and an identification of bundles 
\ben
V \cong \CC^d \times V_0
\een
where $V_0$ is the fiber of $V$ at $0 \in \CC^d$. 
We want to consider a classical theory with space of fields given by $\Omega^{0,*}(\CC^d, V)$. 
Moreover, we want this theory to be invariant with respect to the group of translations on $\CC^d$. 
Per usual, it is best to work with the corresponding Lie algebra of translations. 
Using the complex structure, we choose a presentation for this Lie algebra as
\ben
\CC^{2d} \cong {\rm span}_\CC \left\{\frac{\partial}{\partial z_i}, \frac{\partial}{\partial \zbar_i}\right\}_{1 \leq i \leq d}.
\een

The first thing we need to do is fix a $(-1)$-shifted symplectic pairing.
To do this, we suppose that a we have a translation invariant pairing on $V$ valued in the canonical bundle $K_{\CC^d}$.
That is, suppose 
\be\label{pairing 1}
\< \;\;,\;\; \>_V : V \tensor V \to K_{\CC^d} [d-1]
\ee
is a skew-symmetric bundle map that is equivariant for the Lie algebra of translations. 
The shift is so that the resulting pairing on the Dolbeualt complex is of the appropriate degree.
Here, equivariance means that for sections $v,v'$ we have
\ben
\< \frac{\partial}{\partial z_i} v, v'\>_V = L_{\partial_{z_i}} \<v,v'\>_V
\een
where the right-hand side denotes the Lie derivative applied to $\<v,v'\>_V \in K_{\CC^d}$. 
There is a similar relation for the anti-holomorphic derivatives. 
We obtain a $\CC$-valued pairing on $\Omega^{0,*}_c(\CC^d , V)$ via integration:
\ben
\int_{\CC^d} \circ \<\;\;,\;\;\>_V : \Omega^{0,*}_c (\CC^d , V) \tensor \Omega^{0,*}_c(\CC^d , V) \xto{\wedge \cdot \<\;, \;\>_V} \Omega^{d,*}(\CC^d) \xto{\int} \CC .
\een
The first arrow is the wedge product of forms combined with the pairing on $V$. 
The second arrow is only nonzero on forms of type $\Omega^{d,d}$. 
Clearly, integration is translation invariant, so that the composition is as well. 

This pairing $\Omega^{0,*}(\CC^d , V)$ together with the differential $\dbar$ are enough to define a free theory. 
However, it is convenient to consider a slightly generalized version of this situation. 
We want to allow deformations of the differential $\dbar$ on Dolbeault forms of the form
\ben
Q = \dbar + Q^{hol}
\een
where $Q^{hol}$ is a holomorphic differential operator of the form
\be\label{hol operator}
Q^{hol} = \sum_I \frac{\partial}{\partial z^I} \mu_I
\ee
where $I$ is some multi-index and $\mu_I : V \to V$ is a linear map of cohomological degree $+1$. 
Note that we have automatically written $Q^{hol}$ in a way that it is translation invariant.
Of course, for this differential to define a free theory there needs to be some compatibility with the pairing on $V$. 
We can summarize this in the following definition, which should be viewed as a slight modification of a free theory to this translation invariant holomorphic setting. 

\begin{dfn} A {\em holomorphically translation invariant free BV theory} is the data of a holomorphic vector bundle $V$ together with
\begin{enumerate}
\item an identification $V \cong \CC^d \times V_0$;
\item a translation invariant skew-symmetric pairing  $\<-,-\>_V$ as in (\ref{pairing 1});
\item a holomorphic differential operator $Q^{hol}$ as in (\ref{hol operator});
\end{enumerate}
such that the following conditions hold
\begin{enumerate}
\item the induced $\CC$-valued pairing $\int \circ \<-,-\>_V$ is non-degenerate;
\item the operator $Q^{hol}$ satisfies $(\dbar + Q^{hol})^2 = 0$ and
\ben
\int \<Q^{hol} v, v'\>_V = \pm \int \<v, Q^{hol} v'\> .
\een
\end{enumerate}
\end{dfn}

The first condition is required so that we obtain an actual $(-1)$-shifted symplectic structure on $\Omega^{0,*}(\CC^d, V)$. 
The second condition implies that the derivation $Q = \dbar + Q^{hol}$ defines a cochain complex
\ben
\sE_V = \left(\Omega^{0,*}(\CC^d, V), \dbar + Q^{hol}\right),
\een
and that $Q$ is skew self-adjoint for the symplectic structure. 
Thus, in particular, $\sE_V$ together with the pairing define a free BV theory in the ordinary sense. 
In the usual way, we obtain the action functional via
\ben
S(\varphi) = \int \<\varphi, (\dbar + Q^{hol}) \varphi\>_V .
\een 

Before going further, we will list some familiar examples.

\begin{eg}\label{eg bg affine} {\em The free $\beta\gamma$ system on $\CC^d$}.
Suppose that 
\ben
V = \ul{\CC} \oplus K_{\CC^d} [d - 1] .
\een
Let $\<\;,\;\>_V$ be the pairing
\ben
(\ul{\CC} \oplus K_{\CC^d}) \tensor (\ul{\CC} \oplus K_{\CC^d}) \to K_{\CC^{d}} \oplus K_{\CC^d} \to K_{\CC^d} 
\een 
sending $(\lambda, \mu) \tensor (\lambda',\mu') \mapsto (\lambda \mu', \lambda'\mu) \mapsto \lambda\mu' + \lambda' \mu$.
Finally, let $Q^{hol} = 0$. 
One immediately checks that this is a holomorphically translation invariant free theory as above.
The space of fields can be written as
\ben
\Omega^{0,*}(\CC^d) \oplus \Omega^{d,*}(\CC^d)[d - 1] .
\een 
We write $\gamma$ for a field in the first component, and $\beta$ for a field in the second component. 
The action functional reads
\ben
S(\gamma + \beta, \gamma'+\beta') = \int_{\CC^d} \beta \wedge \dbar \gamma' + \beta' \wedge \dbar \gamma .
\een 
When $d = 1$ this reduces to the ordinary chiral $\beta\gamma$ system from conformal field theory \brian{ref}. 
We will discuss this higher dimensional version further in Section \brian{}.
For instance, we will see how this theory is the starting block for constructing general holomorphic $\sigma$-models. 
\end{eg}

Of course, there are many variants of the $\beta\gamma$ system that we can consider.
For instance, if $E$ is {\em any} holomorphic vector bundle we can take 
\ben
V = E \oplus K_{\CC^d} \tensor E^\vee
\een
where $E^\vee$ is the linear dual bundle. 
The pairing is constructed as in the case above where we also use the evaluation pairing between $E$ and $E^\vee$, ${\rm ev}_E : E \tensor E^\vee \to \CC$.
In thise case, the fields are $\gamma \in \Omega^{0,*}(\CC^d, E)$ and $\beta \in \Omega^{d,*}(\CC^d, E^\vee)[d-1]$. 
The action functional is simply
\ben
S(\gamma, \beta) = \int_{\CC^d} {\rm ev}_E(\beta \wedge \dbar \gamma) .
\een

\begin{eg} {\em Topological ...}
Consider the above example with $Q^{hol} = \partial$..\brian{finish}. 
\end{eg}

\subsubsection{Interacting holomorphic theories}

It is convenient to introduce the following set of degree $-1$ derivations of $\Omega^{0,*}(\CC^d)$ given by
\ben
\Bar{\eta}_i := \frac{\partial}{\partial (\d \zbar_i)} .
\een
The right-hand side is sometimes written using the interior derivative notation $\iota_{\partial / \partial \zbar_i}$. 
By a holomorphic version of ``Cartan's magic formula" these derivations satisfy the relation
\ben
L_{\frac{\partial}{\partial \zbar_i}} = \dbar \Bar{\eta}_i + \Bar{\eta}_i \dbar .
\een
In addition, they serve to define homotopies for the following holomorphic version of Poincar\'{e}'s lemma. 
First, consider the algebraic case. 
Let $\AA^d$ be the complex $d$-dimensional affine space with space of smooth algebraic functions $\sO^{alg, sm}(\AA^d) = \CC[z_1,\ldots, z_d, \zbar_1,\ldots,\zbar_d]$. 
Then, we can build the algebraic Dolbeualt complex 
\ben
\Omega^{0,*}_{alg} (\AA^d) = \CC[z_1,\ldots,z_d, \zbar_1,\ldots,\zbar_d][\d \zbar_1,\ldots,\d \zbar_d]
\een
where the $\d \zbar_i$'s are in cohomological degree $1$. 
The Dolbeault differential $\dbar$ is define in the same way. 
Note that the operators $\Bar{\eta}_i$ also make sense on $\Omega^{0,*}_{alg}(\AA^d)$. 

\begin{lem} 
The map 
\ben
\CC[z_1,\ldots, z_d] \hookrightarrow \Omega^{0,*}_{alg}(\AA^d)
\een
is a quasi-isomorphism.
\end{lem}
\begin{proof}
Note that
\ben
\Omega^{0,*}_{alg}(\AA^d) \cong \CC[z_1,\ldots, z_d] \tensor_\CC \CC[\zbar_1,\ldots,\zbar_d][\d \zbar_1,\ldots,\d \zbar_d] .
\een
The right-hand term is one-dimensional, concentrated in degree zero by the ordinary Poincar\'{e} lemma. 
An explicit homotopy for a \brian{...}
\end{proof}

The analogous result holds with $\AA^d$ replaced by the formal $d$-disk $\hD^d$. 

For the general case...

\begin{lem} \brian{find good citation. should I just state the general Stein resut?}
The map
\ben
\sO^{hol} (\CC^d) \hookrightarrow \Omega^{0,*}(\CC^d)
\een
is a quasi-isomorphism.
\end{lem}

\brian{Include equivalence with certain structured local Lie algebras. Namely holomorphically translation invariant local Lie algebras.}

Local Lie algebras \footnote{Local Lie algebras will mean $L_\infty$...} provide a convenient language to cast the data of an interacting classical field theory. 
In \brian{ref} it is shown that the data of a local Lie algebra together with a non-degenerate pairing of degree $-3$ is equivalent to the data of a classical interacting BV theory. 
It will be convenient for us to formulate, under this equivalence, a Lie theoretic interpretation of holomorphically translation invariant interacting BV theories. 
First, we introduce the following definition. 
Recall, data of a local Lie algebra is a $\ZZ$-graded vector bundle $L$ together with poly-differential operators 
\ben
\ell_n : L \tensor \cdots \tensor L \to L 
\een 
for $n \geq 1$, satisfying some conditions. 
The sheaf of smooth sections of this bundle will be denoted by $\sL$, which inherits from the operators $\{\ell_n\}$ the structure of a sheaf of $L_\infty$ algebras. 
We will often refer to the local Lie algebra simply by its sheaf of sections. 

\begin{dfn}
A translation invariant local Lie algebra on $\RR^n$ is a local Lie algebra $\sL$,
together with an identification $L = \RR^n \times L_0$ such that for each $n$ the structure map
\ben
\ell_n : \RR^n \times (L_0 \tensor \cdots \tensor L_0) \to \RR^n \times L_0
\een
is compatible with translations. 
\end{dfn}

\subsection{Holomorphic theories on general complex manifolds}

There is a weaker notion of a holomorphic theory that makes sense on a general complex manifold.
It is most convenient to formulate this in terms of $L_\infty$ algebras. 
We start with the definition of a {\em holomorphic local Lie algebra} on a complex manifold. 
Enhancing this to a definition of a holomorphic classical BV theory will be immediate. 

Fix a complex manifold $X$, of complex dimension $d$.
The starting data of a local Lie algebra on $X$ is a $\ZZ$-graded vector bundle $L$ on $X$. 


If $E$ and $F$ are two holomorphic vector bundles on $X$, we can speak of holomorphic differential operators between $E$ and $F$. 
First, note that the Hom-bundle ${\rm Hom}(E,F)$ inherits a natural holomorphic structure. 
By definition, a holomorphic differential operator of order $m$ is a linear map
\ben
D : \Gamma^{hol}(X ; E) \to \Gamma^{hol}(X ; F)
\een
such that, with respect to a holomorphic coordinate chart $\{z_i\}$ on $X$, $D$ can be written as
\be\label{local holomorphic}
D|_{\{z_i\}} = \sum_{|I| \leq m} a_I (z) \frac{\partial^{|I|}}{\partial z_I}
\ee
where $a_I(z)$ is a local holomorphic section of ${\rm Hom}(E,F)$.
Here, the sum is over all multi-indices $I = (i_1,\ldots, i_d)$ and 
\ben
\frac{\partial^{|I|}}{\partial z_I} := \frac{\partial^{i_k}}{\partial z_k^{i_k}} . 
\een 
The length is defined by $|I| := i_1 + \cdots + i_d$. 

The most basic example of a holomorphic differential operator is the $\partial$ operator which, for each $1 \leq \ell \leq d$, is a holomorphic differential operator from $E = \wedge^\ell T^{1,0*}X$ to $F = \wedge^{\ell+1} T^{1,0*}X$. 
Locally, of course, it has the form
\ben
\partial = \sum_{i = 1}^{d} (\d z_i \wedge (-)) \frac{\partial}{\partial z_i},
\een
where $\d z_i \wedge (-)$ is the vector bundle homomorphism $\wedge^\ell T^{1,0*}X \to \wedge^{\ell+1} T^{1,0*}X$ sending $\alpha \mapsto \d z_i \wedge \alpha$. 

\begin{dfn} 
A {\em holomorphic} local Lie algebra on a complex manifold $X$ is the data
\begin{enumerate}
\item A $\ZZ$-graded holomorphic vector bundle $L$ on $X$;
\item For each $n \geq 1$ a holomorphic differential operator 
\ben
\ell_n : L^{\boxtimes n} \to L;
\een
\end{enumerate}
such that $\{\ell_n\}$ endow the sheaf of holomorphic sections $\Gamma_{hol}(L)$ with the structure of a sheaf of $L_\infty$ algebras. 
\end{dfn}

A holomorphic local Lie algebra is {\em not} a local Lie algebra in the usual sense. 
The problem is that, in the definition, we have utilized the space of {\em holomorphic sections}, instead of the space of all smooth sections. 
There is a natural resolution that will allow us to turn every holomorphic local Lie algebra into an ordinary local Lie algebra. 
Of course, given a holomorphic vector bundle $L$ we can consider its Dolbeualt complex ${\rm Dol}(L) = \Omega^{0,*}(X , L)$.
For a holomorphic local Lie algebra $L$ this is really the object we want to consider.

Applying the Dolbeault complex functor to the $L_\infty$ maps $\ell_n : L^{\boxtimes n} \to L$ we obtain maps
\ben
{\rm Dol}(\ell_n) : \Omega^{0,*}(X , L)^{\tensor n} \to \Omega^{0,*}(X , L) . 
\een 

\begin{lem} Suppose $L$ is a holomorphic local Lie algebra. 
Then, ${\rm Dol}(L) = \Omega^{0,*}(X , L)$ is endowed with the structure of a local Lie algebra with structure maps given by $\ell_1 = \dbar + {\rm Dol}(\ell^L_1)$, and $\ell_n = {\rm Dol}(\ell^L_n)\}$ for $n \geq 2$. 
\end{lem}
\begin{proof}
We need to check that for each $n \geq 1$ the map
\ben 
{\rm Dol}(\ell^L_n) : \Omega^{0,*}(X, L)^{\tensor n} = \Omega^{0,*}(X^{\times n} , L^{\boxtimes n}) \to \Omega^{0,*}(X , L)
\een
is compatible with the Dolbeualt differential. 
This is a local calculation, so it suffices to assume all operators $\ell_n^L$ are of the form (\ref{local holomorphic}). 
Indeed, since each of the coefficients $a_I(z)$ are holomorphic we see that ${\rm Dol}(\ell_n^L)$ is compatible with $\dbar$. 
\end{proof}

\subsection{Holomorphic deformations}

\subsubsection{Translation invariant deformations on $\CC^d$}

Any local Lie algebra on a manifold endows the structure of an $L_\infty$ algebra on its fibers. 
In particular, if $L$ the graded vector bundle associated to local Lie algebra on $\CC^d$, its fiber over $0$, $L_0$, is equipped with the structure of an $L_\infty$ algebra. 

Suppose $\sL$ is a holomorphically translation invariant local Lie algebra on $\CC^d$ of the form $\Omega^{0,*}(\CC^d, L)$ where $L$ is a graded holomorphic vector bundle.
In this situation, we are interested in studying the local cochains of $\sL$ that are translation invariant.
We will use the following result over and over again throughout this work.

%\begin{lem} Let $\sL_0$ denote the fiber of $\sL$ over $0 \in \CC^d$. 
%Then, if $\ell_1 = \dbar$, there is an equivalence of $L_\infty$ algebras
%\ben
%L_0 \xto{\simeq} \sL_0
%\een 
%where $L_0$ is the fiber of the holomorphic bundle $L$ over $0 \in \CC^d$. 
%\end{lem}

\begin{prop} Suppose $\sL$ is a holomorphically translation invariant local Lie algebra on $\CC^d$ such that $\ell_1 = \dbar$.
Then, one has
\ben
\cloc^*(\sL)^{\CC^d} \simeq \CC \cdot \d^d z \tensor^{\LL}_{\CC[\frac{\partial}{\partial z_i}]} \cred^*(L_0 [[z_1,\ldots,z_d]]) [d] .
\een
\end{prop}

For instance, if $L = \ul{\fg}$ is the constant bundle on $\CC^d$ where $\fg$ is an ordinary Lie (or $L_\infty$) algebra one has $L_0 = \fg$ so that
\ben
\cloc^*(\Omega^{0,*}(\CC^d, \fg))^{\CC^d} \simeq \CC \cdot \d^d z \tensor^{\LL}_{\CC[\frac{\partial}{\partial z_i}]} \cred^*(\fg [[z_1,\ldots,z_d]]) .
\een

\subsubsection{Holomorphic deformations on an arbitrary complex manifold}

There is a more general result that holds on an arbitrary complex manifold. 
Recall, that on a complex manifold $X$ we have introduced the notion of a holomorphic local Lie algebra $L$. 
Let $\sL = {\rm Dol}(L) = \Omega^{0,*}(X , L)$ be its associated local Lie algebra. 
Ordinarily, the local Lie algebra cohomology of a local Lie algebra is computed in terms of the Lie algebra cohomology of the associated jet bundle. 
With holomorphicity and some mild assumption, we can, up to quasi-isomorphism, exhibit a smaller complex computing this local cohomology. 

\begin{prop} 
Suppose $L$ is a holomorphic local Lie algebra with $\ell_1 = 0$, and let $\sL = \Omega^{0,*}(X, L)$ be its associated local Lie algebra.
There is a quasi-isomorphisms of sheaves of cochain complexes
\ben
\cloc^*(\sL) \simeq \Omega^{d,hol}_X \tensor^{\LL}_{D_X^{hol}} \clie^*(J \sL^{hol}) .
\een
\end{prop}
\begin{proof}
Recall, the definition of $\cloc^*(\sL)$ is given in terms of $D$-module data by
\ben
\cloc^*(\sL) = \Omega^{d,d}_{X} \tensor^{\LL}_{D_X} \clie^*(J \sL)
\een
where $J \sL$ denotes the $\infty$-jet bundle of $\Omega^{0,*}(X , L)$. 
Of course, $J\sL$ is a bundle equipped with a natural flat connection, and hence the structure of a $D_X$-module. 
The Chevalley-Eilenberg complex $\clie^*(J\sL)$ inherits this $D_X$-module structure.  

On the other hand, if we view $L$ as a holomorphic vector bundle, it makes sense to look at the {\em holomorphic} jet bundle $J^{hol} L$. 
This holomorphic vector bundle is equipped with a holomorphic flat connection, and hence is a module for the sheaf of holomorphic differential operators $D^{hol}_X$. 

\begin{lem} Let $(J^{hol} L)^{C^\infty}$ be the $D_X^{hol}$-module $J^{hol} L$ viewed, via the forgetful functor, as a $D_X$-module. 
Then, there is a quasi-isomorphism of dg $D_X$-modules $(J^{hol} L)^{C^\infty} \simeq J \Omega^{0,*}(X , L)$.
Furthermore, this quasi-isomorphism is compatible with the $L_\infty$ structures, so that we obtain a quasi-isomorphism of dg $D_X$-modules $\clie^*(J^{hol} L)^{C^\infty} \simeq \clie^*(J \sL)$. 
\end{lem}
\begin{proof}
For any holomorphic vector bundle $E$, the Dolbeualt complex of $E$ is a resolution of the sheaf of holomorphic sections of $E$.
Thus, there is an equivalence of sheaves on $X$
\ben
\Omega^{0,*}_X(E) \simeq \Gamma^{hol}_X(E) .
\een 
Thus, there is an equivalence of sheaves on $X$:
\ben
J^{hol} L \simeq J \Omega^{0,*}(X , L) .
\een
We need to see that this equivalence respects the $D_X$-module structure present on both sides....

\end{proof}

To finish the proof, we verify the following general lemma. 

\begin{lem} Suppose $V$ is a $D_X^{hol}$-module, and let $V^{C^\infty}$ denote its underlying $D_X$-module. 
Then, there is a quasi-isomorphism of sheaves of cochain complexes
\ben
\Omega^{d,hol}_X \tensor^\LL_{D_X^{hol}} V [d] \simeq \Omega^{d,d}_X \tensor_{D_X}^\LL V^{C^\infty} .
\een 
\end{lem}

\end{proof}

\end{document}
