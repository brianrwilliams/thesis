\documentclass[10pt]{amsart}

\usepackage{macros,slashed}

\linespread{1.25}

\title{Local operators for the $\beta\gamma$}

\def\Obs{{\rm Obs}}
\def\ev{{\rm ev}}
\def\Wedge{\bigwedge}

\def\brian{\textcolor{blue}{BW: }\textcolor{blue}}

\begin{document}

\subsection{The character}

In this section we compute the character of the action of $U(d) \times U(1)_f$ on the local observables of the free $\beta\gamma$ system with values in $V$. 
By definition, the character is conjugation invariant, so it is completely determined by its value on the subgroup $T^d \times U(1)_f \subset U(d) \times U(1)_f$. 
Choose the following basis for the maximal torus of $U(d)$: 
\ben
T^d = \{{\rm diag}(q_1,\ldots,q_d) \; | \; |q_i| = 1\} \subset U(d).
\een 
We label the coordinate on $U(1)_f$ by $u$. 
\brian{something about filtrations. I.e., why does the "formal character" make sense?} 
We conclude that the character is valued in the power series ring $\CC[[q_i^{\pm}, u^{\pm q_f}]]$. 

We now turn to the case that the complex dimension $d = 2$, with an aim to compare to the formula for the character of the $\cN = 1$ supersymmetric chiral multiplet on $\RR^4$. 

The local operators of the theory are equal to the observables on a complex $2$-disk $D^2 \subset \CC^2$. 
By translation invariance it suffices to consider a disk centered at the origin $0 \in \CC^2$. 
When $d=2$ we use Proposition \ref{} to read off the cohomology of the disk observables $H^*\Obs^q (D^2)$:
\ben
\Sym\left((\sO^{hol}(D^2) \tensor V)^\vee \right) \tensor \Sym \left( (\Omega^{2,hol}(D^2) \tensor V^*)^\vee [-1] \right) .
\een 

\begin{prop} The $U(2) \times U(1)_f$ character of the local operators of the $\beta\gamma$ system on $\CC^2$ is equal to
\ben
\prod_{n_1, n_2 \geq 0} \frac{1 - u^{q_f} q_1^{n_1 - 1} q_2^{n_2 - 1}}{1 - u^{-q_f} q_1^{n_1} q_2^{n_2}} \in \CC[[q_1^{\pm},q_2^{\pm}, u^{\pm q_f}]]
\een
\end{prop}
\begin{proof}
We will write down a basis for a dense subspace of the observables on a $2$-disk.
For integers $n_1,n_2 \geq 0$ and elements $v \in V$, $v^* \in V^*$ consider the following linear observables on the $2$-disk:
\ben
\begin{array}{ccclll} 
O_{\gamma} (n_1,n_2 ; v^*) & : & \gamma \tensor w & \in \sO^{hol}(D^2) \tensor V & \mapsto & \ev(v^*, w) \frac{\partial^{n_1}}{\partial z_1^{n_1}} \frac{\partial^{n_2}}{\partial z_2^{n_2}} \gamma (0) \\
O_{\beta} (n_1+1,n_2+1; v) & : & \beta \d z_1 \d z_2 \tensor w^* & \in \Omega^{2,hol} (D^2) \tensor V^* & \mapsto & \ev(w^*, v) \frac{\partial^{n_1}}{\partial z_1^{n_1}} \frac{\partial^{n_2}}{\partial z_2^{n_2}} \beta (0) .
\end{array} 
\een

Since the field $\gamma \tensor w \in \sO^{hol} \tensor V$ has $U(2)$ weight zero, we see that the 

For fixed $n_1,n_2 \geq 0$, let $V^*_{n_1,n_2}$ denote the linear span of operators $O_{\gamma}(n_1, n_2; v^*)$. 
As a vector space $V_{n_1,n_2}^* \cong V^*$, but we want to remember the weights under $U(2)$. 
Likewise, for $n_1 , n_2 > 0$, let $V_{n_1,n_2} \cong V$ be the linear span of the operators $O_{\beta}(n_1, n_2 ; v)$. 

There is an injective map of graded vector spaces
\ben
\Sym \left( \left(\bigoplus_{n_1,n_2 \geq 0} V_{n_1,n_2}^*\right) \oplus \left(\bigoplus_{n_1,n_2 > 0}  V_{n_1,n_2}[-1] \right)\right)  \to \Sym\left( \left(\sO^{hol}(D^2) \tensor V\right)^\vee \oplus \left( \Omega^{2,hol}(D^2) \tensor V^*\right)^\vee [-1] \right),
\een
where the right-hand side is the cohomology of the observables on $D^2$. 
Moreover, this map is {\em dense}. \brian{explain}

Thus, to compute the character of the local operators it suffices to compute it on the vector space
\ben
\Sym \left( \left(\bigoplus_{n_1,n_2 \geq 0} V_{n_1,n_2}^*\right) \oplus \left(\bigoplus_{n_1,n_2 > 0} \oplus V_{n_1,n_2}[-1] \right)\right) \cong \Sym \left(\bigoplus_{n_1,n_2 \geq 0} V_{n_1,n_2}^*\right) \tensor \Wedge \left(\bigoplus_{n_1,n_2 > 0} V_{n_1,n_2} \right) .
\een
We have used the convention that as (ungraded) vector spaces the symmetric algebra of a vector space in odd degree is the exterior algebra. 
For instance, $\Sym(W[-1]) = \Wedge(W)$ as ungraded vector spaces. 
We can further simplify the right-hand side as
\ben
\bigotimes_{n_1, n_2 \geq 0} \left(\Sym(V^*_{n_1,n_2})\right) \bigotimes \bigotimes_{n_1,n_2 > 0} \left(\Wedge (V_{n_1,n_2})\right) .
\een 
The character of the symmetric algebra $\Sym(V^*_{n_1,n_2})$ is equal to $(1-u^{-q_f}q_1^{n_1}q_2^{n_2})^{-1}$ and the character of $\Wedge(V_{n_1,n_2})$ is equal to $(1- u^{q_f} q_1^{n_1}q_2^{n_2})$. 
The formula for character in the statement of the proposition follows from the fact that the character of a tensor product is the product of the characters. 
\end{proof}

We have seen in Proposition \brian{ref} that when the complex dimension $d = 2$, the free $\beta\gamma$ system is equivalent to the holomorphic twist of the free $\cN=1$ chiral multiplet in four dimensions. 
In \cite{Closset1} Equation 5.58 the index for the $\cN=1$ chiral multiplet is computed, and our answer is easily seen to agree with theirs. 
We conclude that in this instance that under the holomorphic twist the superconformal index was sent to the character of the local observables of the holomorphic theory. 
We will see \brian{ref} that this is a general fact about superconformal indices.

\brian{Do general case. Relate to elliptic gamma functions. Relate to Witten index, which is the parition function on $S^3 \times S^1$.}



\end{document}
