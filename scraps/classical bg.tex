\documentclass[10pt]{amsart}

\usepackage{macros,slashed}

\linespread{1.25}

\title{The classical theory}

\def\brian{\textcolor{blue}{BW: }\textcolor{blue}}

\begin{document}
\maketitle

Let $Y$ be a complex $d$-fold and $X$ a complex manifold of any dimension. 
In this section we construct the classical BV theory describing the cotangent theory of the formal neighborhood of some fixed holomorphic map inside ${\rm Map}^{hol}(Y,X)$. 
As we have already justified, this is a higher dimensional generalization of the curved $\beta\gamma$ system. 

Just like in the ordinary $\beta\gamma$ system, there exists a modification of this theory given the data of holomorphic vector bundle $V$ on $Y$. 
It will be convenient for us to include this data in our general construction of the classical theory.
At the level of the Dolbeualt complex, this modification is easy to understand. 
The holomorphic vector bundle $V$ defines an operator 
\ben
\dbar : \Gamma(Y, V) \to \Omega^{0,1}(Y, V),
\een
which extends to all Dolbeault forms and defines an elliptic complex $\Omega^{0,*}(Y,V)$.
There is also a geometric description of this modification at the level of the holomorphic $\sigma$-model. 
Indeed, the base fields of the original cotangent theory consist of smooth maps $\gamma : Y \to X$ which, on-shell, are required to be holomorphic....\brian{finish}

We will provide a construction of the classical holomorphic $\sigma$ model based on formal geometry...

\subsection{The free theory}
As above, let $Y$ be a complex $d$-fold. 
Furthermore, we fix a complex vector space $W$, that will be the ``target" of our free theory. 
Denote by ${\rm ev}_W : W \tensor W^* \to \CC$ the evaluation pairing between $W$ and its linear dual.
%The definition of a free theory on a manifold $Y$ in the BV formalism consists of three pieces: 1) a finite rank $\ZZ$-graded vector bundle on $Y$, 2) a density valued pairing for this vector bundle that is fiberwise nondegenerate, and 3) an operator $Q$ of 
\begin{dfn}
The free $\beta\gamma$ system on $Y$ with values in $W$ has complex of fields
\ben
\Omega^{0,*}(Y , W) \oplus \Omega^{d,*}(Y, W^*)[d-1]
\een 
equipped with the Dolbeualt operator 
\ben
\dbar = \dbar_{\Omega^{0,*}} \tensor {\rm id}_{W} + \dbar_{\Omega^{d,*}} \tensor {\rm id}_{W^*}.
\een 
We will denote these fields by $\gamma \in \Omega^{0,*}(Y, W)$ and $\beta \in \Omega^{d,*}(Y, W^*)[d-1]$. 
The $(-1)$-shifted symplectic pairing is defined by ``wedge and integrate"
\ben
\<\gamma + \beta, \gamma'+\beta'\> = \int_Y \ev_{W} (\gamma \wedge \beta') + \ev_W(\beta \wedge \gamma') .
\een 
The action functional is
\ben
S_{free}(\gamma, \beta) = \<\beta, \dbar \gamma\> = \int_Y {\rm ev}_W(\beta \wedge \dbar \gamma) .
\een 
\end{dfn}

When $Y= \CC^d$ this the theory we introduced in Section \brian{ref} Example \ref{eg bg affine}, where we saw that it defines a holomorphically translation invariant theory. 



\end{document}
