\documentclass[10pt]{amsart}

\usepackage{macros,slashed}

\linespread{1.25}

\title{The theory on Hopf surfaces}

\def\brian{\textcolor{blue}{BW: }\textcolor{blue}}

\def\SU{{\rm SU}}
\def\Spin{{\rm Spin}}
\def\dvol{{\rm dvol}}
\def\dslash{\slashed{\partial}}
\begin{document}
\maketitle

The local theory of the holomorphic $\sigma$-model where the source is $\CC^d$ has been studied in the last section.
We now turn to global aspects of the theory which amounts to putting the theory on more exotic source manifolds. 
In this section we consider the holomorphic $\sigma$-model where the source is a particular compact complex $d$-fold. 
In fact, we will allow for holomorphic families of compact complex manifolds as the source and study how the partition function varies in this family. 

\subsection{The case of an elliptic curve}

As a warmup we recall a familiar situation 

\subsection{Hopf manifolds}

We focus on a family of complex manifolds defined by Hopf in \cite{Hopf} defined in every complex dimension $d$. 

\begin{dfn}
Fix an integer $d \geq 1$.
Let $f : \CC^d \to \CC^d$ be a polynomial map such that $f(0) = 0$ such that its Jacobian at zero $Jac(f)(0)$ is invertible with eigenvalues $\{\lambda_i\}$ all satisfying $|\lambda_i|<1$. 
Define the {\em Hopf manifold associated to $f$} to be the $d$-dimensional complex manifold
\ben
Y_f := \left. \left(\CC^d \setminus \{0\}\right) \right/ (x \sim f(x)) .
\een
\end{dfn}

Note that $Y_{f}$ is compact for any $f$. 
In the case $d=1$ all Hopf surfaces are equivalent to elliptic curves.
\brian{give some history}

\begin{lem} 
For any $f$ there is a diffeomorphism $Y_f \cong S^{2d-1} \times S^1$.
\end{lem}

This implies that when $d > 1$, the cohomology $H^{2}_{dR} (Y_f) = 0$ for any $f$. 
In particular, $Y_f$ is not K\"{a}hler when $d > 1$. 

For $1 \leq i \leq d$ let $q_i \in D(0,1)^{\times}$ be a nonzero complex number of modulus $|q_i| <1$. 
The $d$-dimensional {\em Hopf manifold of type} ${\bf q} = (q_1,\ldots,q_d)$ is the following quotient of punctured affine space $\CC^d \setminus \{0\}$ by the discrete group $\ZZ^d$:
\ben
H_{\bf q} = \left. \left(\CC^d \setminus \{0\}\right) \right/ \left( (z_1,\ldots,z_d) \sim (q_1^{2\pi i \ZZ} z_1, \ldots,q_d^{2 \pi i \ZZ} z_d) \right) .
\een
Note that in the case $d = 1$ we recover the usual description of an elliptic curve $H_{\bf q} = E_q = \CC^\times / q^{2 \pi i \ZZ}$. 

For any $d$ and tuple $(q_1,\ldots, q_d)$ as above, we see that as a smooth manifold there is a diffeomorphism $H_{\bf q} \cong S^{2d-1} \times S^1$. 
Indeed, the radial projection map $\CC^d \setminus \{0\} \to \RR_{>0}$ defines a smooth $S^{2d-1}$-fibration over $\RR_{>0}$. 
Passing to the quotient, we obtain an $S^{2d - 1}$-fibration 
\ben
H_{\bf q} \to \left. \RR_{>0} \right/ \left(r \sim \lambda^{\ZZ} \cdot r \right) \cong S^1 .
\een
Here, $\lambda = (|q_1|^2 + \cdots + |q_d|^2)^{1/2} > 0$. 
Since there are no non-trivial $S^{2d-1}$ fibrations over $S^1$ we obtain $H_{\bf q} = S^{2d-1} \times S^1$. 

There is an equivalent description of $H_{\bf q}$ as a quotient of affine space that we will take advantage of. 

\section{A higher chiral algebraic index}

Let $V^{*,*} = \oplus_{p,q} V^{p,q}$ be a bigraded vector space.
If $v \in V^{p,q}$ we say that $v$ is of homogeneous bidegree and write $|v| = (p,q)$. 
The total degree $n=p+q$ endows $V$ with the structure of an ordinary $\ZZ$-graded vector space. 
We will denote by $\CC[V] = \Sym(V)$ the graded symmetric algebra with respect to this $\ZZ$-grading. 
Note that since $V$ is bigraded, there is an induced bigrading on the symmetric algebra 
\ben
\CC[V] = \oplus_{p,q} \CC[V]^{p,q}
\een
where $\CC[V]^{p,q}$ consists of polynomials $v_1\ldots v_l$ such that $|v_1| + \cdots + |v_l| = (p,q)$. 

\begin{dfn}
Let $X$ be a complex manifold.
A {\em Dolbeualt model} for $X$ is a bigraded vector space $V^{*,*} = \oplus_{p,q} V^{p,q}$ together with:
\begin{itemize}
\item an operator 
\ben
\dbar_V : V^{p,q} \to \CC[V]^{p,q+1}
\een
such that $\dbar_V^2 = 0$;
\item a map of cochain complexes
\ben
\varphi : (\CC[V], \dbar_V) \to \left(\Omega^{0,*}(X), \dbar\right)
\een
\end{itemize}
\end{dfn}

\begin{dfn}
We say that a complex manifold is {\em Dolbeualt formal} if there exists a Dolbeault model $(\CC[V],\dbar)$ and quasi-isomorphisms
\ben
\xymatrix{
(\Omega^{0,*}(X), \dbar) & \ar[l]_-{\varphi}^-{\simeq} (\CC[V],\dbar_V) \ar[r]^-{\psi}_{\simeq} & (H^{0,*}_{\dbar} (X), 0) .
}
\een
\end{dfn}

Note that this definition is slightly different than ordinary formality of $X$ as a smooth manifold, which is defined using the de Rham complex of $X$.
There is a definition that ties these two notions together. 

\begin{dfn}
A complex manifold $X$ is said to be {\em strictly formal} if there exists...
\end{dfn}

A famous result of Deligne-Griffiths-Morgan-Sullivan states that compact K\"{a}hler manifolds are formal. 
Without much more work, one can show that their result actually implies that any compact K\"{a}hler manifold is also strictly formal (and hence Dolbeualt formal).

The manifolds we are most interested in throughout this section are Hopf manifolds.
As we've already pointed out these complex manifolds are {\em not} K\"{a}hler.
Nevertheless, we have the following result. 

\begin{prop}\label{prop hopf cohomology}
Any Hopf manifold $X$ is strictly formal, so that $\Omega^{*,*}(X)$ is quasi-isomorphic (in a way that preserves the bigrading) to its cohomology $H^{*,*}_{\dbar}(X)$.
Moreover, let $\delta, \epsilon$ be formal parameters of bidegree $(0,1)$ and $(d-1,d)$. 
Then, there is a bigraded isomorphism
\ben
H^{*,*}_{\dbar} (X) = H^*(X , \Omega^*_{hol}) \cong \CC[\delta,\epsilon] / (\epsilon^2)
\een
where $\delta$ has bi-degree $(0,1)$ and $\epsilon$ has bi-degree $(d,d-1)$. 
\end{prop}
\begin{proof}
This proof is essentially in Appendix \cite{HirzebruchTopMethods}.
We use the Borel spectral sequence for a holomorphic fibration
\be\label{hopf fibration}
\xymatrix{
F \ar[r] & E \ar[d] \\
& B 
}
\ee
which exists so long as we assume that $F$ is K\"{a}hler \cite{Hirzebruch}. 
The spectral sequence has pages $(E_r, \d_r), r \geq 0$, each of which is trigraded $E_{r}^{p,q,s}$, where $(p,q)$ are the the Dolbeault types and $s$ is the internal degree of the spectral sequence. 
The $E_1$ page is given by
\ben
E_{1}^{p,q,s} = \oplus_{i} \Omega^{i, s-i}(B) \tensor H^{p-i, q-s+i}(F)
\een
with differential $\dbar_B$ the Dolbeualt differential on the base $B$. 
Note that since $F$ is K\:{a}hler, it is strictly formal and their exists a Dolbeault model of the form $(\CC[V], \d)$. 
By general nonsense, we can hence obtain a model for the total space $E$ by deforming the differential on $(\Omega^{*,*}(B) \tensor \CC[V], \dbar + \d)$ using the above spectral sequence.

Any Hopf manifold $X$ of complex dimension $d$ sits in a holomorphic fibration
\ben
\xymatrix{
T \ar[r] & X \ar[d] \\
& \CC P^{d-1}
}
\een
where $T = S^1\times S^1$ is a torus. 
Of course, $T$ is K\:{a}hler with Dolbeualt model given by
\ben
\left(\CC[\delta, x], 0\right)
\een
where $\delta$ has bidegree $(0,1)$ and $x$ has bidegree $(1,0)$. 
An explicit Dolbeualt model for $\CC P^{d-1}$ is given by
\ben
(\CC[y,\epsilon] , \dbar_B) \;\;\;, \;\; |y| = (1,1) \;\;, \;\; |\epsilon| = (d-1,d) \;\; , \; \; \dbar_B(\epsilon) = y^d .
\een

Thus, the $E_1$ page is of the form
\ben
\left(\CC[x,y,\delta,\epsilon] , \dbar_B\right) .
\een
The only possible differential on the $E_2$ page must have the form $\d_2(x) = c y$, where $c$ is some constant. 
By rescaling generators we can assume that $c = 0$ or $1$. 
The following fact will determine the value of this constant.

\begin{lem} \brian{either cite Hirzebruch, Kodaira, or find easy proof} For any Hopf manifold $X$ one has 
\ben
H^{1,0}_{\dbar}(X) = H^0(X , \Omega^1_{hol}) = 0 .
\een
\end{lem}

This lemma implies that $x$ must support a differential and with our normalization above $\d_2 x = y$. 
It follows that a model for $X$ is given by
\ben
\left(\CC[x,y,\delta,\epsilon] , \dbar_B + \eta\right) 
\een
where $\dbar_B(\epsilon) = y^d$ and $\eta(x) = y$. 
The statements of the proposition follow immediately.
\end{proof}

This result will allow us to write down an explicit model for the factorization homology of the higher dimensional $\beta\gamma$ system along an arbitrary Hopf surface. 
Since the method is very similar to the case of an elliptic curve, we will be brief in the details. 

Let $X$ be an arbitrary Hopf surface. 
We will first work in the free situation where the target is a vector space $V$. 
The fields of the theory on $X$ are given by
\ben
\Omega^{0,*}(X , V) \oplus \Omega^{d,*}(X, V^*)[d-1] .
\een
By Proposition \ref{prop hopf cohomology} we know that there is a quasi-isomorphism
\begin{align*}
\Omega^{0,*}(X , V) \oplus \Omega^{d,*}(X, V^*)[d-1] & \simeq H^{0,*}_{\dbar} (X, V) \tensor H^{d,*}_{\dbar}(X , V^*)[d-1] \\ & \cong \CC[\delta] \tensor V \oplus \epsilon \CC[\delta] \tensor V^* [d-1] \\ & = \CC[\delta] \tensor (V \oplus \epsilon V^*[d-1]) .
\end{align*}
where $\delta$ has degree $+1$ and $\epsilon$ has degree $d-1$. 
(Note that only the $(0,*)$-grading contributes to the cohomological degree in our language.)

As an immediate consequence, we see that the classical observables along $X$ satisfy
\ben
\Obs^{\cl}_V(X) \simeq \Omega^{-*}(T^*V) 
\een
where we view $\Omega^{-*}(T^*V) = \sO(\CC[\delta] \tensor (V \oplus \epsilon V[d-1])$.
Since $\epsilon$ has degree $d-1$, the graded vector space $\epsilon V[d-1]$ sits in degree zero, so we have omitted the parameter $\epsilon$ from the notation. 

\begin{prop}
Let $X$ be any Hopf manifold.
For any scale $L$, there is a quasi-isomorphism
\ben
\Obs^{\q}_V(X) [L] \simeq \left(\Omega^{-*}(T^*V) [\hbar] , \hbar L_\pi\right)
\een
where $L_\pi$ is the Lie derivative with respect to the Poisson tensor of the symplectic vector space $T^*V$.
\end{prop}
\begin{proof}
As we've already seen, the scale $L$ global observables are defined by
\ben
\Obs^\q_V(X) [L] = \left(\Obs^{\cl}_V(X) [\hbar], \dbar + \hbar \Delta_L\right)
\een
Recall, the operator $\Delta_L$ is given by contraction with the regularized heat kernel $K_L$. 
By the formality result, which we used in the calculation above, we know there is a quasi-isomorphism
\be\label{hopf quasi}
\xymatrix{
\CC[\delta] \tensor V \oplus \epsilon \CC[\delta] \tensor V^*[d-1] \ar@{^{(}->}[r]^-{\simeq} & \Omega^{0,*}(X , V) \oplus \Omega^{d,*}(X, V^*)[d-1]  .
}
\ee
We can understand the elements $\delta,\epsilon$ concretely in the Dolbeualt complex of $X$ as follows.
Let $z=(z_1,\ldots,z_d)$ be the coordinate on $\CC^d$.
Note that
\ben
\dbar \log \left(|z|^2\right) = \sum_{i} \d \zbar_i \frac{z_i}{|z|^2}
\een
is a well-defined $(0,1)$ form on $\CC^d \setminus \{0\}$ and descends along quotient $\CC^d \setminus \{0\} \to X$ to define a $(0,1)$ form on $X$.
The above quasi-isomorphism (\ref{hopf quasi}) sends $\delta \mapsto \dbar \log (|z|^2)$. 
Next, recall from (\ref{hopf fibration}), that every Hopf manifold sits in a holomorphic fibration $T \to X \xto{\pi} \CC P^{d-1}$. 
Let $\omega$ be the K\"{a}hler form on $\CC P^{d-1}$. 
From the existence of Dolbeualt models for compact complex manifolds, we know there exists a $(d,d-1)$-form $\alpha \in \Omega^{d,d-1}(\CC P^{d-1})$ such that $\dbar \alpha = \omega^{d}$.
The map (\ref{hopf quasi}) sends $\epsilon \mapsto \pi^* \alpha$.

The left-hand side of (\ref{hopf quasi}) has natural $(-1)$-shifted symplectic structure given by fermionic integration against $\delta$ combined with the evaluation pairing between $V$ and its dual. 
Moreover, this quasi-isomorphism is compatible with the $(-1)$-symplectic structure on fields. 
We have just seen that classically this quasi-isomorphism induces $\Obs^{\cl}_V(X) \simeq \Omega^{-*}(T^*V)$, so we need to check that the scale $L$ BV Laplacian agrees with $L_\pi$ in the smaller complex $\Omega^{-*}(T^*V)$. 
Since $\Delta_L$ is chain homotopy equivalent to $\Delta_{L'}$ for any $L,L'$ it suffices to compute this BV Laplacian at any scale. 
Just as in the $1$-dimensional case we see that on $\Omega^{-*}(T^*V)$ the BV Laplacian is well defined as $L\to \infty$.
In this limit, we recover exactly $L_\pi$ as desired.

%The zig-zag of quasi-isomorphisms 
%\ben
%\xymatrix{
%& \left(\CC[x,y,\delta,\epsilon] , \dbar_B + \eta\right) \ar[ld] \ar[rd] & \\
%(\Omega^{*,*}(X), \dbar) & & (H^{*,*}_{\dbar}(X), 0)
%}
%\een
%induces a zig-zag of quasi-isomorphisms

\end{proof}

\end{document}