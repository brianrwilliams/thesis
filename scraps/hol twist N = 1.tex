\documentclass[10pt]{amsart}

\usepackage{macros,slashed}

\linespread{1.25}

\title{The holomorphic twist of $\cN=1$ in four dimensions}

\def\brian{\textcolor{blue}{BW: }\textcolor{blue}}

\def\SU{{\rm SU}}
\def\Spin{{\rm Spin}}
\def\dvol{{\rm dvol}}
\def\dslash{\slashed{\partial}}
\begin{document}
\maketitle

In this section we describe how the $\cN = 1$ chiral supermultiplet in four dimensions is related to the two dimensional $\beta\gamma$ system. 

\subsection{}

\subsubsection{}

We work on the four manifold $\RR^4$ equipped with the flat metric. We will write $\Omega^i$ for the space of smooth $i$-forms on $\RR^4$. The space of sections of the spinor bundles are denoted by $\sS_{\pm} = C^\infty(\RR^4 ; S_{\pm})$ where $S_{\pm}$ are the defining representations of the two copies of $\SU(2)$ in $\Spin(4)$. Write $\dvol = \d x_1 \cdots \d x_4$ and $\star$ for the Hodge star operator. 

We fix an even dimensional real vector space $V$ equipped with a complex structure, denoted by $J$, and a Hermitian inner product, denoted by $h$. 

\subsubsection{}

\begin{dfn}
The $N=1$ chiral supermultiplet on $\RR^4$ with values in the complex vector space $V$ equipped with a Hermitian pairing $h$ has space of fields
\bestar
\Phi_+ & = & (\varphi_+ , \psi_+, F_+) \in \Omega^0 \tensor V \oplus \cS_+ \tensor V \oplus \Omega^2 \tensor V \\
\Phi_- & = & (\varphi_-,\psi_-,F_-) \in \Omega^0 \tensor V \oplus \cS_- \tensor V \oplus \Omega^2 \tensor V 
\eestar
and action functional given by
\ben
S^{\rm SO}(\varphi,\psi,F) = \int h(\varphi \tensor \Delta \varphi)\dvol + \int h(\<\psi_+ , \dslash \psi_-\>) \dvol +\int F^2
\een
where $\<-,-\>$ denotes the standard symplectic pairing $S_+ \tensor S_+ \to \CC$. 
\end{dfn}

We have written the fields as a pair of superfields, a chiral one $\Phi_+$ and an anti-chiral one $\Phi_-$. 

It will be useful for our purposes to write the chiral supermultiplet in the BV-BRST formalism. There are no gauge symmetries, so all that this amounts to is the the introduction of the anti-fields to the fields we have already written.
\bestar
\Phi_+^\vee & = & (\varphi_+^\vee, \psi_+^\vee, F_+^\vee) \in \Omega^4 \tensor V \oplus \cS_-' \tensor V \oplus \Omega^2 \tensor V \\
\Phi_-^\vee & = & (\varphi_-^\vee, \psi_-^\vee, F_-^\vee) \in \Omega^4 \tensor V \oplus \cS_-' \tensor V \oplus \Omega^2 \tensor V .
\eestar
We note that the anti-field to a positive spinor $\psi_+ \in \cS_{\pm}$ is a spinor $\psi_\pm^\vee \in \cS_+$ of the same chirality. The prime simply indicates the same underlying spinor bundle except we are viewing it as an anti-field. Using this notation, we can define the classical theory in the BV formalism in a succinct way. 

\begin{dfn} The $N=1$ chiral supermultiplet on $\RR^4$ in the BV formalism is the $\ZZ \times \ZZ/2$ graded theory with complex of fields given by
\end{dfn}

The $R$-charge of an anti-field is opposite to that of the corresponding field. Thus, $\psi_{\pm}^\vee$ has $R$-charge $\mp 1$ and $F^\vee_{\pm}$ has $R$-charge $\mp 2$. 

We will perform a twist of the free chiral supermultiplet by a fixed constant spinor $Q \in \cS_-$. This element acts on the fields $\Phi = (\varphi_+, \psi_+, F)$ as above:
\ben
Q \cdot (\varphi_+, \psi_+, F) = (0, Q \cdot (\d \varphi_+), \<Q, \dslash \psi_+\>) .
\een
The action of $Q$ on the anti-fields reads is determined by compatibility with the $(-1)$-shifted symplectic pairing. Explicitly it is
\ben
Q \cdot (\varphi_+^\vee, \psi_+^\vee, F) = (\<Q, \dslash \psi_+^\vee\>, Q \cdot (\star F_+^\vee), 0) .
\een

We have arrived at the following. 

\begin{prop} The twist of the $N=1$ free chiral supermultiplet on $\RR^4$ with values in the hermitian vector space $V$ by an element $Q \in \cS_-$ is equivalent to the free $\beta\gamma$ system on $\CC^2$ with values in $V$:
\ben
(\gamma,\beta) \in \Omega^{0,*}(\CC^2 ; V) \oplus \Omega^{1,*}(\CC^2 ; V^\vee)[1] .
\een
The action functional is $S(\gamma,\beta) = \int \<\beta, \dbar \gamma\>$ where $\<-,-\>$ is the evaluation pairing on $V$. 
\end{prop}

\subsubsection{}
We will introduce the following ``first-order" reformulation of the chiral supermultiplet that will be convenient for our description of it as a BV theory. Introduce additional scalar fields of the form
\ben
B \in \Omega^3 \tensor V
\een
and define the action 
\bestar
S^{\rm FO}(\varphi, B , \psi , F) & = &  \int h(B \wedge \d \varphi) - \frac{1}{2} \int h(B \wedge \star B)  \\ & + & \int h(\<\psi_+ , \dslash \psi_-\>) \dvol 
\eestar

\begin{lem} The theories $S^{\rm SO}$ and $S^{\rm FO}$ are classically equivalent. 
\end{lem}
\begin{proof} We show that the spaces of solutions to the classical equations of motion are equivalent. The equations of motion for the chiral supermultiplet read
\bestar
\Delta \varphi & = & 0 \\
\dslash \psi & = & 0 \\
F & = & 0 .
\eestar 
Next, consider $S^{\rm FO}$. The pieces of the action functional involving $\psi, F$ are identical. We use the variation of the scalar field $\varphi \mapsto \varphi + \delta \varphi$ to obtain the equation of motion $\d B = 0$. The variation $B \mapsto B + \delta B$ yields the equation $\d \varphi - h^\vee(\star B)$. This equation is equivalent to $\star \d \varphi = h^\vee (B)$. Applying $\d$ to this equation and using the equation $\d B = 0$ we obtain $\d \star \d \varphi = \Delta \varphi = 0$, as desired. 

An explicit equivalence at the level of fields can be written as follows. \brian{Want $\varphi \mapsto \varphi, B \mapsto \star h^\vee(\d \varphi) + h^\vee(F)$.}
\end{proof}

\brian{physics description of susy action}

\subsubsection{}
We describe the theory $S^{\rm FO}$ as a classical theory in the BV formalism. The space of fields
\[
\xymatrix{
& \ul{0} & \ul{1} \\
{\rm Fermion \; degree}\;\; \ul{0} & \Omega^0 \tensor V \ar[r]^-{\d_+} & (\Omega^1 \tensor V)_+ \\
{\rm Fermion \; degree} \; \; \ul{0} & (\Omega^{3} \tensor V)_- \ar[r]^-{\d} & \Omega^4 \tensor V  \\
{\rm Fermion \; degree} \; \; \ul{1} & (\sS_+ \oplus \sS_-) \tensor V \ar[r]^-{\dslash} & (\sS'_-\oplus \sS'_+) \tensor V \\
}
\]


\end{document}
