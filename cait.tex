\documentclass[10pt]{amsart}

\usepackage{macros,slashed}

\linespread{1.25}

\usepackage{tikz}
\usetikzlibrary{arrows,shapes}
\usetikzlibrary{trees}
\usetikzlibrary{matrix,arrows}
\usetikzlibrary{positioning}
\usetikzlibrary{calc,through}
\usetikzlibrary{decorations.pathreplacing}
\usepackage{pgffor}

\title{A chiral algebraic index theorem}

\def\brian{\textcolor{blue}{BW: }\textcolor{blue}}


\begin{document}
\maketitle

\section{A chiral algebraic index theorem}

In this section we would like to leverage our analysis of the holomorphic $\sigma$-model to formulate an index theorem for operators on the higher sphere spaces, ${\rm Map}(S^{2d-1}, X)$, of the target manifold. 
We are inspired, in part, by Witten's original study of elliptic genera appearing as the index of Dirac operators on the loop space $LX = {\rm Map}(S^1, X)$ of $X$\cite{WittenDirac,WittenElliptic}.
First, we will focus on this $d = 1$ case by giving a mathematical formulation of Witten's result that has the flavor of an ``algebraic index theorem" in the sense of Fedosov, Nest-Tsygan \cite{Fedosov, NestTsygan}. 

The method we employ to formulate the chiral index theorem is reminiscent of Nest-Tsygan's \cite{NestTsygan} approach based on formal geometry to prove the original algebraic index theorem, which we recall now. 
This approach is further clarified in \cite{FFS} for the case of a general deformation quantization. 
There are two main steps.
The first is to prove a ``local" algebraic index theorem which characterizes the unique trace present on the Weyl algebra in any number of variables.
Precisely, the statement is that the Hochschild homology of the $n$-dimensional Weyl algebra is one-dimensional and concentrated in degree $-2n$. 

The next step uses a version of Gelfand-Kazhdan formal geometry to ``descend" this result over any manifold. 
The crucial fact that makes this procedure possible is that the cocycle representing the degree $-2n$ Hochschild homology of the Weyl algebra is equivariant for formal vector fields. 
The cocycle then yields a differential form on the manifold, and in the case the manifold is closed, one can integrate it.
The global index theorem is obtained by explicitly evaluating the integral of this descended cocycle, which is where the $\Hat{A}$-class appears.

\brian{make this better} no surprise that our approach to constructing $\sigma$-models in the BV formalism using formal geometry is applicable to index theorems involving 
Our technique is to construct a local version of the index theorem on the formal $n$-disk, and then use the technique of descent to deduce the index theorem on a general manifold (satisfying the usual anomaly conditions for the chiral case). 

The above process should appear familiar in our construction of the quantization of $\sigma$-models. 
We first study the BV quantization with target the formal disk, and then analyze how the quantization is equivariant for formal vector fields by placing them in the background.
Before embarking on the chiral index theorem, we recall how the ordinary algebraic index theorem is related to the study of one-dimensional quantum field theory, that is, quantum mechanics.

\subsection{The algebraic index theorem and QM}

Recently, in \cite{GLL} it is shown how to deduce the algebraic index theorem using the Batalin-Vilkovisky approach to perturbative quantum field theory. 
Our summary here highlights the application of formal geometry, \brian{finish}

Topological quantum mechanics is a one-dimensional $\sigma$-model which we will write in the BV formalism.
We start with the case in which the target is an $n$-disk $D^n$. 
The complex of fields is
\begin{align*}
(f,g) \in \Omega^*(\RR, \RR^n \oplus (\RR^n)^*) .
\end{align*}
The dual $(-)^*$ is to remind ourselves that $f$ and $g$ are conjugate to each other.
The BRST differential is simply the de Rham operator. 
The action functional reads
\ben
S_{1d}(f,g) = \int_{\RR} g \d f,
\een
where we are implicitly using the evaluation pairing on $\RR^n$. 
The equations of motion on $\RR$ reduce to looking just at constant maps. 
Being a free theory, it has a unique quantization. 
The factorization algebra of quantum observables $\Obs_{1d}$ is a locally constant factorization algebra on $\RR$, which is equivalent as an associative algebra to the Weyl algebra $\Weyl^\hbar_n$ on generators $\{t_1,\ldots,t_n, \partial_{t_1}, \ldots, \partial_{t_n}\}$. 
The commutation relation is $[\partial_{t_i},t_j] = \hbar \delta_{ij}$. 

As a factorization algebra on $\RR$, $\Obs_{1d}$ is translation invariant. 
Thus, we can place this factorization algebra on the circle $S^1$, viewing $S^1 = \RR / \ZZ$. 
Equivalently, this is the quantization of the $\sigma$-model of maps $S^1 \to D^n$. 
Now, the factorization homology, or global sections, of any topological factorization algebra on the circle $S^1$ is equivalent to the Hochschild homology of the underlying algebra. 
In fact, in \cite{GLL}, an explicit cochain homotopy equivalence is written down
\ben
\Phi : {\rm Hoch}(\Weyl_n^\hbar) \xto{\simeq} \int_{S^1} \Obs_{1d}.
\een
We will explain this equivalence further in Section \brian{ref}.

Using the standard flat metric on $S^1$ we can consider the finite dimensional subspace of harmonic forms on the circle $\HH(S^1) \subset \Omega^*(S^1)$. 
It follows that there is quasi-isomorphism of cochain complexes 
\ben
\int_{S^1} \Obs_{1d} \xto{\simeq} \left(\sO(\HH(S^1) \tensor (\RR^n \oplus (\RR^n)^*)) [[\hbar]], \hbar \Delta_\infty\right).
\een
Here $\Delta_\infty$ \brian{how to say this best}.
We will see how we can interpret this map as taking homotopy RG flow to ``scale $\infty$". 

In \cite{GradyGwilliamCS1} this theory is called one-dimensional abelian Chern-Simons theory. 
It makes sense when we replace the target $D^n$ with any $L_\infty$ space. 
In particular, if $X$ is a smooth manifold, we can substitute the $L_\infty$ space $\fg_X$ encoding the smooth structure to obtain a BV formulation of the perturbative $\sigma$-model of maps $\RR \to X$ that are infinitesimally close to the constant maps
\ben
T^*[-1] \Hat{\rm Map}_0(\RR, X) .
\een
Here ${\rm Map}_0$ denotes the formal neighborhood of constant maps inside the space of all smooth maps $\RR \to X$. 

Alternatively, which is the approach we will take in the chiral case, we can replace the target disk $D^n$ by the formal disk $\hD^n$ and formulate this theory using Gelfand-Kazhdan descent. 
This means that we need to study how the free theory is equivariant for the Lie algebra of formal vector fields $\Vect$ on the $n$-disk. 
The following theorem will appear in joint work with Grady and Gwilliam.

\begin{thm}[\cite{GradyGwilliamW}]
There is a $\Vect$-equivariant quantization of topological quantum mechanics on $\RR$ with target the formal disk $\hD^n$. 
In turn, the factorization algebra $\Obs_{1d}$ is $\Vect$-equivariant and hence determines a sheaf $\Obs_{X}$ of locally constant factorization algebras on any smooth $n$-manifold $X$. 
This sheaf is equivalent to the sheaf of differential operators $D_X$ as associative algebras.
\end{thm}

\brian{finish}

\subsection{Stating the chiral algebraic index theorem}

We now proceed to formulate the chiral algebraic index theorem. 
Our starting point is the holomorphic $\sigma$-model of maps from a Riemann surface to a complex manifold $X$ that we have just considered earlier in this thesis. 
We will propose a higher dimensional version of this story in the end of this section. 

So far, in our analysis of the holomorphic $\sigma$-model we have been considered with the local theory so we have taken the source to be of the form $\CC^d$. 
For the $d=1$ chiral algebraic index theorem we are most interested in the case that the source of this $\sigma$-model is an elliptic curve 
\ben
E_\tau = \CC / (\ZZ + \tau \ZZ)
\een
where $\tau$ lies in the upper-half plane $\HH$. 
It will sometimes be convenient to use the ``$q$-presentation" where $q = e^{2 \pi i \tau}$ and write the elliptic curve as a quotient of $\CC^\times$ by the integers $E_q = \CC^\times / q^\ZZ$.
Of course, the exponential induces an isomorphism $E_\tau \cong E_q$. 

As a smooth manifold, the elliptic curve is just a torus $S^1 \times S^1$. 
We may therefore write the fields of the $\sigma$-model on $E_\tau$ with target $X$ as $T^*[-1]{\rm Map}(E_\tau, X) = T^*[-1] {\rm Map}(S^1, L X)$ where $LX$ is the free loop space of the manifold $X$.
Thus, via compactification along one of the circles, we obtain a one-dimensional $\sigma$-model akin to the one we have just discussed in the previous section. 
Heuristically, one can think of this as quantum mechanics on the free loop space. 
We will state the precise relationship between the two-dimensional chiral theory and this one-dimensional quantum mechanics in Section \ref{sec: }.

Of course, the space $LX$ is not a smooth manifold in the ordinary sense.
Consequently, there are many analytic difficulties when doing quantum mechanics. 
In \cite{WittenDirac}, Witten provides a physical description of the path integral for quantum mechanics on the free loop space.
One of the main outcomes of his work is the partition function of this system, from which he extracts the so-called {\em Witten genus} of the manifold $X$.
We will not work with the space of topological loops $LX$, but rather work with an algebraic replacement that we will call the algebraic loop space $LX^{alg}$.
There is an associated sheaf of associative algebras on $X$ that deserves to be called differential operators on $LX^{alg}$, denoted $D^{\hbar}_{LX^{alg}}$ that we will define in Section \brian{ref}. 
The fundamental object in the chiral algebraic index theorem will be a $q$-twisted version of a trace on this sheaf of algebras. 

In \cite{CostelloWG1,CostelloWG2} it is shown how the partition function of the holomorphic $\sigma$-model along an elliptic curve $E_\tau$ recovers the Witten genus of the target manifold. 
We will see how to arrive at this result through the lens of formal geometry, which will be the key step in deducing the global chiral algebraic index theorem. 
The one main advantage of working directly with the elliptic curve, as opposed to doing quantum mechanics on the loop space, is that the modular properties of the Witten genus are manifest. 

The advantage of the perspective of quantum mechanics is that we have at our disposal many algebraic tools that will allow us to view the calculation of the Witten genus on an elliptic curve as a ``chiral index theorem". 

\begin{thm}
(Chiral algebraic index theorem)
Let $q \in D(0,1)^\times$, $X$ be a compact complex manifold, and $\alpha$ a trivialization of $\ch_2(TX) \in H^2(X , \Omega^2_{cl})$, so that the factorization algebra $\Obs^\q_{X,\alpha}$ of quantum observables of the two-dimensional holomorphic $\sigma$-model from $E_q$ to $X$ is defined (Theorem \ref{thm curved quantization}). 
Then:
\begin{enumerate}
\item there is quasi-isomorphism of sheaves of cochain complexes on $X$
\ben
\Phi^q : {\rm Hoch}(D_{LX^{alg}}^{\hbar} ; q) \xto{\simeq} \int_{E_q} \Obs^\q_{X,\alpha} ;
\een 
\item there is a $q$-twisted trace map
\ben
{\rm Tr}^q_X : {\rm Hoch}(D_{LX^{alg}}^{\hbar} ; q) \to \CC[[\hbar,\hbar^{-1}];
\een
\brian{some uniqueness?}
\item if $$1 \in H^* \int_{E_q} \Obs^q_{X,\alpha} \cong HH^*(D_{LX^{alg}}^{\hbar} ; q)$$ denotes the unit observable, then in cohomology the trace map satisfies
\ben
{\rm Tr}^q_X(1) = \int_X {\rm Wit}(X , q) .
\een
\end{enumerate}
\end{thm}

Item (1) will be a formal consequence of the quantization of the two-dimensional holomorphic $\sigma$-model and will be proved in \brian{ref}.
Item (2) is a $q$-twisted version of the trace present in ordinary deformation quantization, which we will be a consequence of Proposition \brian{ref}. 
Finally, item (3) will follow from an explicit Feynman diagram calculation which is akin to Costello's original calculation of the Witten genus as the partition function of the holomorphic $\sigma$-model. 

\subsection{The local chiral algebraic index theorem}

In this section we prove a local version of the chiral algebraic index theorem in the case the target manifold is the complex $n$-disk. 
For now, we take the source of our theory to be the punctured complex line $\CC^\times$.
Eventually, we will descend this to the elliptic curve $E_q$. 
The two-dimensional chiral theory describing this holomorphic $\sigma$-model is the usual $\beta\gamma$ system on $\CC^\times$ with values in the $n$-dimensional affine space $\CC^n$. 
In this section it will be convenient to replace $\CC^n$ with an arbitrary finite dimensional complex vector space $V$. 

The fields of the theory are of the form
\ben
(\gamma, \beta) \in \Omega^{0,*}(\CC^\times) \tensor V \oplus \Omega^{1,*}(\CC^\times) \tensor V^*
\een
with the usual action $\int \beta \dbar \gamma$. 
As in Section \brian{ref} it is convenient to use a smoothed version of the factorization algebra of observables that we denote by $\Obs_V^{\q, sm}$. 
To an open set $U \subset \CC^\times$ the smoothed observables assign the cochain complex
\ben
{\Obs}^{\q,sm}_V(U) = (\Sym(\Omega^{1,*}_c(U) \otimes V [1] \oplus \Omega^{0,*}_c(U) \otimes V[1])[[\hbar]], \dbar + \hbar \Delta). 
\een
Note that the naive BV Laplacian is well-defined on the smoothed observables. 

Consider the vector field $L_0 = z \frac{\partial}{\partial z}$. 
This is simply the Euler vector field, which represents infinitesimal scaling automorphism on $\CC^\times$.
There is a natural action of $L_0$ on the fields of the $\beta\gamma$ system by Lie derivative.
This further induces an action on the factorization algebra of quantum observables.

\subsubsection{Quantum mechanics on the loop space}

We consider a reduction of the two dimensional theory on $\CC^\times$ to a one-dimensional theory.
It is convenient to view $\CC^\times$ as a cylinder
\ben
\CC^\times \cong \RR_{>0} \times S^1 .
\een
We can view the radial direction $\RR_{>0}$ as a ``time" parameter and the $S^1$ as the ``loop" parameter. 
We consider the reduction of our theory ``along the circle $S^1$". 
At the level of factorization algebras, this is equivalent to pushing forward along the radial projection map $\pi : \CC^\times \to \RR_{>0}$ which we can view pictorially as
\ben
\begin{tikzpicture}
\draw (0,0) ellipse (1.25 and 0.5);
\draw (-1.25,0) -- (-1.25,-3.5);
\draw (-1.25,-3.5) arc (180:360:1.25 and 0.5);
\draw [dashed] (-1.25,-3.5) arc (180:360:1.25 and -0.5);
\draw (1.25,-3.5) -- (1.25,0);  
%\fill [gray,opacity=0.5] (-1.25,0) -- (-1.25,-3.5) arc (180:360:1.25 and 0.5) -- (1.25,0) arc (0:180:1.25 and -0.5);

\draw[->] (2, -1.75) -- (5,-1.75) ;

\node (pi) at (3.5, -1.5) {$\pi$};
\draw (6, -3.5) -- (6, 0);
\draw (5.9,-3.45) arc(180:360:0.1cm); 

\end{tikzpicture}
\een

Inside of the pushforward $\Obs^{\q, sm}_{V}$ along $\pi$ is a locally constant factorization algebra on $\RR_{>0}$ that we will see can be viewed as the observables of topological quantum mechanics with target the algebraic loop space of $V$. 
To see this, we will express our two-dimensional theory in terms of the natural coordinates on the cylinder. 

We will use the coordinates on $\CC^\times$ coming from the flat coordinates on $\CC$ via the exponential map
\ben
\CC \to \CC^\times \;\; , \;\; w \mapsto e^{2 \pi i w} .
\een
We will use the coordinate $z = e^{2\pi i w} \in \CC^\times$. 
Thus the constant holomorphic and anti-holomorphic one-forms read
\begin{align*}
\d w & = \frac{1}{2 \pi i} \frac{\d z}{z} \\
\d \Bar{w} & = - \frac{1}{2 \pi i} \frac{\d \zbar}{\zbar} .
\end{align*}
Using this coordinate we fix trivializations
\begin{align*}
\Omega^{0,*}(\CC^\times) & \cong C^\infty(\CC^\times) \left[\frac{\d \zbar}{2 \pi i z}\right] \\
\Omega^{1,*}(\CC^\times) & \cong \frac{\d z}{2 \pi i} C^\infty(\CC^\times) \left[\frac{\d \zbar}{2 \pi i z}\right] .
\end{align*}

The zero degree piece of the $\gamma$ field is a map from $\CC^\times$ to a vector space $V$.
In terms of the coordinates $(r, e^{i \theta}) \in \RR_{>0} \times S^1$ we can expand $\gamma$ as:
\ben
\gamma (z,\zbar) = \sum_{n \in \ZZ} f_n(r) e^{in\theta}
\een
where $f_n : \RR_{>0} \to V$ is defined by $f_n(r) = \frac{1}{2\pi}\int_{S^1}\d\theta\, \gamma e^{-in\theta}$. 
There is a similar mode expansion of $\beta \frac{\d z}{z} \in \Omega^{1,0}(\CC^\times) \tensor V^*$. 
\ben
\beta(z,\zbar) \frac{\d z}{2 \pi i z} = \sum_{n \in \ZZ} g_n(r) e^{in\theta} \frac{1}{2\pi i} \left(\frac{\d r}{r} + i \,\d\theta\right),
\een
where we used the fact $\d z/z = \d r/r + i \, \d\theta$. 

With respect to the mode expansion, the action functional becomes
\begin{align}
\label{action for loop mechanics in q coordinates}
S(\gamma,\beta) & = \int_{\CC^\times} \left(\beta \frac{\d z}{2 \pi i z}\right) \dbar \gamma \\ 
& = \frac{1}{2} \sum_{n \in \ZZ} \int_{r \in \RR_{> 0}} \d r \, g_{-n}(r) \left( \partial_r  - \frac{n}{r}\right)f_n(r),
\end{align}
where we have performed an integration over the loop parameter $\theta$. 

We can rewrite this action functional as $S = \sum_{n \in \ZZ} S^{(n)}$ where $S^{(n)}(g_{-n}, f_{n}) = \int_{\RR_{>0}} g_{-n} \nabla^{(n)} f_n$,
using the connection $\nabla^{(n)} = \d - \frac{n}{r} \d r$ on the trivial bundle on $\RR_{>0}$. 
We think of the collection $\{S^{(n)}\}_{n \in \ZZ}$ defining a family of quantum mechanics theories, one for each mode number $n \in \ZZ$. 

The algebraic loop space of the vector space $V$ is defined by $LV = V [u,u^{-1}]$ where $u$ is a formal loop parameter. 
What we have written down is a version of quantum mechanics on $\RR_{>0}$ with target this algebraic loop space $LV$.
The piece of the action functional $S^{(n)}$ comes restricting the target to $z^n V \subset L V$. 
Note that $S$ does not define a one-dimensional BV theory in the usual sense, since the fields are not the sections of a finite dimensional vector bundle. 
Nevertheless, we can still make sense of the observables of this theory. 

For each $n$ the action functional $S^{(n)}$ describes a free theory.
This theory is described by the $(-1)$-symplectic elliptic complex
\ben
\left(\Omega^*(\RR_{>0}) \tensor (u^n V \oplus u^{-n} V^*), \nabla^{(n)}\right),
\een
where $\nabla^{(n)}$ is as above. 
Of course, as a vector space $u^n V = V$, but the notation reminds us of the mode number. 
Correspondingly, there is a factorization algebra of smoothed quantum observables $\Obs^{\q,sm}_{z^n V}$, which assigns to an interval $I \subset \RR_{>0}$ the cochain complex
\be\label{loop obs}
\left(\Sym\left(\Omega_c^*(I) \tensor (u^n V \oplus u^{-n}V^*)^*\right), \nabla^{(n)} + \hbar \Delta\right) .
\ee

\begin{dfn}
Define the factorization algebra (valued in pro-cochain complexes) of {\em smoothed quantum observables for topological quantum mechanics on $LV$} to be the colimit
\be\label{colimit1}
\Obs_{LV}^{\q,sm} := \underset{N \to \infty}{\rm colim} \bigotimes_{|n| \leq N} \Obs^{\q,sm}_{z^nV} .
\ee
\end{dfn}

Heuristically, we think of the factorization algebra $\Obs_{LV}^{\q, sm}$ as being associated to the BV theory of quantum mechanics on the loop space. 
To make this precise, we'd have to modify our definition of a BV theory to include fields that are sections of infinite dimensional vector bundles, but we omit that discussion here. 
For now, we will work with the above well-posed definition.

We can associate to any vector space $V$ the following associative algebra. 
Consider first the one-dimensional central extension of the abelian Lie algebra $V \oplus V^*$
\ben
0 \to \CC \cdot \hbar \to \sH(V) \to V^* \oplus V \to 0
\een
with bracket given by $[v^*, v] = \hbar \ev(v,v^*)$.
This is the Heisenberg algebra on $V$. 
The Weyl algebra on $V$ is, by definition, the universal enveloping algebra of $\sH(V)$, which we denote by $D^\hbar_V$. 
Of course, $D^\hbar_V$ is simply the associative algebra of polynomial differential operators on $V$. 

\begin{dfn} Let $D^{\hbar}_{z^n V}$ denote the Weyl algebra on the symplectic vector space $z^n \, V \oplus z^{-n} \, V^*$. 
Define the {\em Weyl algebra of $LV$} by the colimit:
\be\label{colimit2}
D^{\hbar}_{LV} := \underset{N \to \infty}{\rm colim} \bigotimes_{|n| \leq N} D^{\hbar}_{z^n V} .
\ee 
\end{dfn}

The algebra $D^\hbar_{LV}$ has a natural weight filtration induced by the operator $L_0 = z \partial_z$ on $LV$.  
For any $n$, $D^{\hbar}_{z^n V}$ is isomorphic as an algebra to the ordinary Weyl algebra on $V$.
The notation is to keep track of the $L_0$ weight $n$. 

We can describe this Weyl algebra in the following equivalent way.
Consider the following bilinear map
\ben
\omega : L V \times LV^* \to \CC \;\; , \;\; (v \tensor f , v^* \tensor g) \mapsto {\rm ev}(v,v^*) \Res_{z=0} \left(f \d g\right) . 
\een 
Here, ${\rm ev}(v,v^*)$ denotes the evaluation pairing between $V$ and $V^*$. 
Note that $\omega(v \tensor z^n, v^* \tensor z^m)$ is only nonzero when $n+m = 0$. 
This equips $LV \oplus LV^{*}$ with the structure of a symplectic vector space. 

For each $N \geq 0$ denote by $LV^{(N)}$ be the finite sum
\ben
LV^{(N)} = \bigoplus_{n \leq N} z^n \cdot V \subset LV .
\een 
Then, $\omega$ restricts to a symplectic form on the finite dimensional vector space $LV^{(N)} \oplus LV^{*(N)}$. 
Denote by $\Weyl(LV^{(N)})$ the corresponding Weyl algebra on this finite dimensional vector space. 
There is a $L_0$-equivariant equivalence of algebras
\ben
D^{\hbar}_{LV} \cong \underset{N \to \infty}{\rm colim} \; \Weyl(LV^{(N)}) .
\een

\begin{lem} The following hold about the associative algebra $D_{LV}^\hbar$ and the factorization algebras $\Obs^{\q,sm}_{LV}$, $\Obs^{\q,sm}_V$.
\begin{enumerate}
\item The factorization algebra $\Obs^{\q,sm}_{LV}$ is locally constant and is equivalent, as an associative algebra, to $D^{\hbar}_{LV}$. Moreover, this equivalence is $L_0$-equivariant.
%\item there is a quasi-isomorphism of $L_0$-equivariant factorization algebras on $\RR_{>0}$ 
%\ben
%\ul{D^{\hbar}_{LV}} \xto{\simeq} \Obs^{\q,sm}_{LV} .
%\een 
\item
There is a map of factorization algebras on $\RR_{>0}$
\be\label{dense1}
\iota : \Obs^{\q,sm}_{LV} \to \pi_*\Obs_V^{\q,sm},
\ee
which is dense inclusion when applied to each interval. 
\end{enumerate}
\end{lem}

\begin{proof}
Consider $\Obs_{z^0V}^{\q, sm}$, the factorization algebra associated to topological quantum mechanics from the first part of this section.
Equivalently, this is the zero mode part of the theory on the loop space described by the action $S^{(0)}$.
It is clear that this factorization algebra is locally constant. 
Moreover, in Section 4.4 of \cite{fact1} it is shown that the corresponding associative algebra is equivalent to the Weyl algebra.
%Pick a bump function $f$ for the interval $(-1,1) \subset \RR$. 
%This means that $f$ is compactly supported on $(-1,1)$, is zero outside $[-1,1]$, and satisfies $\int f \d x = 0$ 
%WeFor each interval $I$ there is a map of cochain complexes
%\ben
%D_{V}^\hbar \to \Obs^{\q, sm}_{1d}(I)
%\een
To see (1) we observe that since $\d_{dR} (n \log r) = \frac{n}{r} \d r$, for each $n$ the connection $\nabla^{(n)} = \d_{dR} - \frac{n}{r} \d r$ is gauge equivalent to the de Rham differential $\d_{dR}$. 
This implies that the classical observables of the theory $S^{(n)}$ are quasi-isomorphic to the classical observables of $S^{(0)}$. 
Since the BV Laplacian $\Delta$ commutes with this gauge transformation, we see that this implies that the quantum observables are also quasi-isomorphic. 
Thus, for each $n$, $\Obs^{\q,sm}_{z^n V} \simeq \Obs^{\q,sm}_{z^0V}$.  
At each $N$ the factorization algebra 
\ben
\bigotimes_{|n| \leq N} \Obs^{\q,sm}_{z^nV}
\een
is thus equivalent to the associative algebra $\otimes_{|n|\leq N} D_{z^n V}^{\hbar}$.
Moreover, this equivalence preserves the $L_0$-equivariance. 
Both the observables $\Obs^{\q,sm}_{LV}$ and the Weyl algebra $D_{LV}^\hbar$ we are considering are defined as colimits (\ref{colimit1}), (\ref{colimit2}), so the result follows. 

The claim (2) is a general fact about pushing forward holomorphic theories along the radial projection. 
First, consider the cosheaf of compactly supported Dolbeualt forms on $\CC^\times$, $\Omega^{0,*}_{c}$.
We can pushforward this cosheaf along $\pi : \CC^\times \to \RR_{>0}$ whose value on an interval $I \subset \RR_{>0}$ is $(\pi_* \Omega_{c}^{0,*})(I) = \Omega_c^{0,*}(\pi^{-1}(I))$. 
Let $\Omega^*_c$ be the cosheaf of compactly supported forms on $\RR_{>0}$. 
Pull-back induces a map of cosheaves $\pi^* : \Omega^*_c \to \pi_* \Omega^{0,*}_c$. 
Moreover, for every $n$ we can define another map of cosheaves
\ben
\pi^*_n : \Omega^*_c \to \pi_* \Omega^{0,*}_c
\een
that sends a function $f(r)$ to $z^n f(|z|)$ and a one-form $f(r) \d r$ to the one-form $(z^n f(|z|)) \d (|z|)$. 
We can then consider the direct sum $\oplus_{n \in \ZZ} \pi^*_n : \oplus_n \Omega^*_c \to \pi_* \Omega^{0,*}_c$. 
Now, every open set in $\CC$ is Stein, so that the cohomology of $H^*(\Omega^{0,*}_c(\pi^{-1}(I)))$ is concentrated in degree one and given by
\ben
H^1(\Omega^{0,*}_c(\pi^{-1}(I))) \cong \left(\Omega^1_{hol}(\pi^{-1}(I)) \right)^\vee .
\een
In cohomology, $\oplus_n \pi_n^*$ exhibits a dense map $\CC[z,z^{-1}] \to \left(\Omega^1_{hol}(\pi^{-1}(I)) \right)^\vee$. 
The same argument goes through to show that the smoothed observables of quantum mechanics on the loop space, which is built from the compactly supported forms on $\RR_{>0}$, maps densely in cohomology to the push forward of the observables of the $\beta\gamma$ system as desired. 
For a more detailed argument that is similar to this, see Section \brian{ref Kac moody section}. 
\end{proof}

\subsubsection{Global sections}

We now turn to the statement of the local chiral algebraic index theorem using factorization homology. 
We will use the results above about the factorization algebras on $\CC^\times$ and $\RR_{>0}$ to deduce statements about the factorization homologies along $E_q$ and $S^1$, respectively.

First, we recall the general fact about discretely equivariant factorization algebras. 
Let $\sF$ be any factorization algebra on $X$, and that $G$ is a discrete group acting on $X$. 
We say that $\sF$ is $G$-equivariant if, for every $g \in G$, there is a quasi-isomorphism of factorization algebras 
\ben
\rho_{g} : g^*(\sF) \xto{\simeq} \sF
\een
on $X$ such that for any $g,h$ $\rho_{gh} = \rho_h \circ h^*\rho_g$.
We note that the pull-back of factorization algebras is defined since the action by $g : X \to X$ is a homeomorphism.
In fact, as long as $f : X \to Y$ is an open immersion, the pullback $f^*\sG$ of a factorization algebra $\sG$ on $Y$ is defined \cite{CG} \brian{place?}.
We denote by ${\rm Fact}_X^{G}$ the category of $G$-equivariant factorization algebras. 
Note that this definition is sufficiently simpler than the case where $G$ is a continuous, or Lie, group since we do not need to impose any smoothness conditions.

\begin{prop}[\cite{CG} Proposition \brian{??}] \label{prop discrete equivariance}Suppose $G$ acts properly discontinuously on $X$ and let $p : X \to X / G$ be the resulting open immersion.
Then, the induced functor $p^* : {\rm Fact}_{X/G} \to {\rm Fact}_{X}^{G}$ is an equivalence of categories.
\end{prop}

\begin{rmk}
This equivalence of categories factors through an equivalence of locally constant factorization algebras.
\end{rmk}

In our situation, for each $q \in D(0,1)^\times$, we have an action of the discrete group $\ZZ$ on two different spaces: $\CC^\times$ and $\RR_{>0}$.
On $\CC^\times$ the action is by $n \in \ZZ : z \mapsto q^n z$.
Equivalently, this is induced by the exponentiatation of the Euler vector field $L_0 = z \frac{\partial}{\partial z}$. 
We have already seen that the factorization algebra $\Obs^{\q, sm}_{V}$ on $\CC^\times$ is equivariant for $L_0$.
This induces, by exponentiation an action of $\ZZ$ on the factorization algebra by the automorphism $q^{L_0}$. 
By Proposition \ref{prop discrete equivariance} this implies that $\Obs^{\q,sm}_{V}$ descends along $p_q : \CC^\times \to E_q$ to a factorization algebra on $E_q = \CC^\times / q^\ZZ$ for any $q$.
The factorization homology along $E_q$ is quasi-isomorphic to the global sections 
\ben
\int_{E_q} \Obs^{\q,sm}_{V} \simeq \left(\Sym(\Omega^{1,*} (E_q) \otimes V [1] \oplus \Omega^{0,*}(E_q) \otimes V[1])[[\hbar]], \dbar + \hbar \Delta \right). 
\een

Next, we consider our locally constant factorization algebra on $\RR_{>0}$.
Consider the quotient map $p : \RR_{>0} \to S_q^1 = \RR_{>0} / |q|^\ZZ$. 
Here, we identify $r \in \RR_{>0}$ with $|q| r$. 
For each $n$, the connection $\nabla^{(n)} = \d - \frac{n}{r} \d r$ is left invariant by this action of $\ZZ$, as is the BV Laplacian $\Delta$.
Thus, the factorization algebra $\Obs^{\q,sm}_{LV}$ is $\ZZ$-equivariant and hence determines a factorization algebra on $S^1_q$. 

Furthermore, the map of factorization algebras (\ref{dense1}) induces a map of factorization homologies
\ben
\iota : \int_{S^1_q} \Obs^{\q, sm}_{LV} \to \int_{E_q} \Obs^{\q, sm}_{V} .
\een

\brian{should this go here?}
\begin{lem} 
This map is a quasi-isomorphism.
\end{lem}

\subsubsection{Twisted Hochschild homology}

The next piece of input we need to formulate the local chiral algebraic index theorem is to relate the factorization homology of the chiral theory along an elliptic curve to a twisted form of the Hochschild homology of the Weyl algebra on the the loop space.

First, we recall the Hochschild cochain complex computing Hochschild homology, that we will denote $\Hoch(A,M)$. 
Let $A$ be any (ordinary) algebra and $M$ an $A$-bimodule. 
Denote by $\cdot$ both the associative product on $A$ and each of the bimodule structures.
The underlying graded vector space of $\Hoch(A, M)$ is
\ben
\oplus_k A^{\tensor k} \tensor M [k] .
\een
Thus, $A^{\tensor k} \tensor M$ is in degree $-k$. 
The differential splits into three pieces
\ben
\d_{\Hoch} = \d_A + \d_L + \d_R
\een
where, for $a_1,\ldots a_k \in A$, $m \in M$, we have:
\begin{enumerate}
\item $\d_A (a_1 \tensor \cdots \tensor a_k \tensor m = \sum_{i=1}^{k-1} (-1)^{j-1} a_1 \tensor \cdots \tensor (a_j \cdot a_{j+1}) \tensor \cdots a_k \tensor m$;
\item $\d_L (a_1\tensor \cdots \tensor a_k \tensor m) = (-1)^{k-1} a_1\tensor \cdots \tensor a_{k-1} \tensor (a_k \cdot m)$;
\item $\d_R (a_1\tensor \cdots \tensor a_k \tensor m) = (-1)^k a_2 \tensor \cdots \tensor a_k \tensor (m \cdot a_1)$.
\end{enumerate}
This is the standard complex computing the Hochschild homology $HH_*(A, M) = H^*(\Hoch(A,M))$.

An example of a bimodule that will be important for us comes from the data of an algebra automorphism. 
The underlying vector space of this bimodule is $M = A$ itself.
Now, if $\sigma : A \to A$ is an automorphism then we can define the following left-module structure on $M = A$ by $a \cdot a' = \sigma(a) \cdot a'$. 
We will write $\Hoch(A, \sigma)$ for the Hocschild complex above with coefficients in the $A$-bimodule where the left-module structure is given by this $\sigma$-twisted action, and the right-module structure is the standard one.

If $\sF$ is any locally constant factorization algebra on $S^1$, then $p^* \sF$ is a $\ZZ$-equivariant factorization algebra on $\RR_{>0}$ where $p : \RR_{>0} \to S^1$ is the quotient map.
In particular, the element $1 \in \ZZ$ defines an equivalence $\rho_1 : 1^* p^* \sF \xto{\simeq} p^* \sF$. 
In addition, since $\sF$ is locally constant this implies that $1 \in \ZZ$ defines an automorphism $\rho_1 : p^* \sF \to p^* \sF$. 
This, in turn induces an automorphism of the underlying associative algebra defined by $p^*\sF$. 
Thus, to prescribe a locally constant factorization algebra $\sF$ on $S^1$, it is equivalent to give the data of an associative algebra $A_{\sF}$ (really an $E_1$-algebra) together with an algebra automorphism $\sigma$ (really an $E_1$-automorphism). 
It is a well-known result of \cite{lurie, john} that the factorization homology along $S^1$ of an associative algebra is equivalent to its Hochschild complex. 
The following is true about this twisted case. 

\begin{prop}[\cite{Lurie} 5.3.3] 
Let $\sF$ be a locally constant factorization algebra on $S^1$ and $(A_{\sF}, \sigma)$ the corresponding pair of an $E_1$ algebra and an $E_1$ algebra automorphism.
Then there is a quasi-isomorphism
\ben
\int_{S^1} \sF \simeq {\rm Hoch}_*(A_{\sF}, \sigma)
\een
where ${\rm Hoch}_*(A_{\sF}, \sigma)$ is the Hochschild cochain complex twisted by $\sigma$.
\end{prop}

\begin{eg}
Suppose $\fg$ is a Lie algebra and consider its universal enveloping algebra $U\fg$.
There is a convenient model for $U\fg$ as a locally constant factorization algebra on $\RR$.
One starts with the cosheaf $\Omega^*_{c} \tensor \fg$, where $\Omega^{*}_c$ is the cosheaf of compactly supported sections on $\RR$.
This is a cosheaf of Lie algebras, hence we may consider its Chevalley-Eilenberg chains $\clie^*(\Omega^*_{c} \tensor \fg)$.
In \cite{CG} it is shown that this is a locally constant factorization algebra on $\RR$ which is equivalent, as an associative algebra, to $U\fg$.

Now, suppose we fix a derivation $D$ of $\fg$.
Then, $D$ induces a factorization algebra derivation of $\clie^*(\Omega^*_{c} \tensor \fg)$.
We can deform the factorization algebra by the derivation $D$ in the following way: for every open set $U$ the deformed factorization algebra assigns the cochain complex
\be
\left(\Sym(\Omega^*_c(U) \tensor \fg[1]), \d_{dR} + \d_{CE} + \d t D\right) .
\ee
The operator $\d_{dR} + \d_{CE}$ is the ordinary differential for the cochain complex $\clie^*(\Omega^*_c(U)\tensor \fg)$.
Since $(\d t)^2 = 0$ and $D$ commutes with $\d_{dR}, \d_{CE}$, the operator $\d_{dR} + \d_{CE} + \d t D$ still defines a differential.
Moreover, this deformed factorization algebra is still translation invariant so we may consider its global sections along $S^1$.

Note that $D$ determines a derivation of the associative algebra $U\fg$. 
If $D$ is nilpotent, it defines an automorphism $\sigma_D = e^D = 1 + D + \frac{1}{2} D^2 + \cdots$. 
In this case, the global sections of the deformed factorization algebra provide a nice model for the twisted Hochschild homology of the universal enveloping algebra:
\ben
{\rm Hoch}_*(U\fg, \sigma_D) \simeq \left(\Sym(\Omega^*(S^1) \tensor \fg[1]), \d_{dR} + \d_{CE} + \d t D\right) .
\een
Both of these are equivalent to the factorization homology of the factorization algebra on $S^1$ determined by $U\fg$ together with the automorphism $\varphi_D$.
\end{eg} 

\subsubsection{}

In our situation we will exhibit an explicit quasi-isomorphism from the twisted Hochschild complex of the Weyl algebra on the loop space to the global sections of the $\beta\gamma$ system along an elliptic curve. 

First, let's consider the zero mode part of quantum mechanics on the loop space.
The global observables along $S^1$ for this sector are defined by
\ben
\Obs^{\q, sm}_{z^0 V} (S^1) = \left(\Sym (\Omega^*_c(S^1) \tensor (V[1] \oplus V^*[1])), \d_{dR} + \hbar \Delta\right) .
\een

We have already seen that for any interval $I \subset \RR_{>0}$ there is a quasi-isomorphism of cochain complexes
\ben
\Obs^{\q,sm}_{z^0V}(I)) \xto{\simeq} D_{z^0 V}^\hbar = D_V^\hbar
\een
given by taking $H^0$. 
It is convenient to choose explicit sections of this map.
To do this, we choose a bump function for $I$. 
This is a compactly supported function $f \in C^\infty_c(I)$ such that $\int f \d r = 0$. 
Given a linear element $v + v^* \in \sH(V) \subset D^\hbar_V$ we define the linear observable
\ben
f(r) \d r \tensor (v^* + v) \in \Omega^1_c(I)\tensor (V \oplus V^*) \subset \Obs^{\q,sm}_{z^0V}(I)) .
\een
\brian{extend to non-linear elements}

If $t \in S^1$ is a fixed point on the circle, then we will let $f_{t}$ be a bump function supported in an infinitesimal open set around $t \in S^1$.
Similarly, for each $k \geq 1$ distinct points $t_1,\ldots, t_k$ in $S^1$ pick bump functions $f_{t_1},\ldots, f_{t_{k}}$ whose supports are mutually disjoint. 
Define the following map
\ben
\Phi : \sH(V)^{\tensor k} \to \Obs^{\q, sm}_{z^0 V} (S^1)
\een
by sending $(v_1^* + v_1) \tensor \cdots \tensor (v_k^* + v_k)$ to the smoothed observable
\ben
\left(f_1(t) \d t \tensor (v_1^* +  v_1)\right) \cdot (f_2(t) \tensor (v_2^* + v_2)) \cdots (f_k(t) \tensor (v_k^* + v_k)) \in \Sym^k(\Omega^*(S^1) \tensor (V^* \oplus V)[1]) \subset \Obs^{\q,sm}_{z^0 V}(S^1) .
\een


%We first recount a construction that's implicit in \cite{GLL} relating ordinary Hocschild homology to quantum mechanics on $S^1$.


\begin{prop}
The map $\Phi_q : {\rm Hoch}_*(D^{\hbar}_{LV}) \xto{\simeq} \int_{S^1} \Obs^{\q,sm}_{LV}$ is a quasi-isomorphism.
\end{prop}

\subsubsection{BV integration}

To the vector space $V$ we can associate the $(-1)$-shifted symplectic vector space $T^*[-1] V = V \oplus V^*[-1]$. 
Then, the space of functions $\sO(T^*[-1]V)$ has the structure of a $P_0$-algebra. 
In this case, there is a natural quantization of this $P_0$-algebra to a BV algebra, defined over $\CC[[\hbar]]$. 
Choose a basis $\{x_i\}$ for $V$ and a dual basis $\{\xi_i\}$ for $V^*$. 
Note that in $T^*[-1] V$, the variables $\xi_i$ are odd.
The BV operator is defined by
\ben
\Delta = \sum_i \frac{\partial}{\partial x_i} \frac{\partial}{\partial \xi_i} .
\een 
The BV algebra is $(\sO(T^*[-1]V)[[\hbar]], \hbar \Delta, \cdot)$ where $\cdot$ is the ordinary commutative product of functions. 

One motivation for considering BV structures is that it gives us a setting to do integration. \brian{ref schwarz, kevin, etc.}
There is a pairing between homology classes of Lagrangian super-submanifolds (in this setting, these are submanifolds of the $(-1)$-shifted symplectic vector space $T^*[-1] V$) and the corresponding BV cohomology. 

In this case, there is essentially a unique super-submanifold of $T^*[-1] V$:
\ben
V^*[-1] \subset T^*[-1] V = V \oplus V^*[-1] .
\een 
The corresponding ``integration map" $\int_{BV} : \sO(T^*[-1]V)[[\hbar]] \to \CC[[\hbar]]$ is the Berezin integral over $V^*[-1]$ as a supermanifold.
This is the map that simply picks out the ``top fermion" and is normalized by the condition
\ben
\int_{BV} \xi_1 \cdots \xi_{\dim V} = 1 .
\een

Ordinarily, this toy model does not work for BV algebras coming from quantum field theory, as the BV algebra we obtain is not finite dimensional. 
One solution is to perform regularization, but there another approach that allows us to directly apply the discussion above. 

The BV algebra we are interested in is the global observables of the chiral theory on the elliptic curve $E_q$.
Let us fix the flat metric on $E_q$, which is descended from the one we used on $\CC$ to find a gauge fixing condition in order to regularize the theory. 
Inside of the Dolbeault complex $\Omega^{0,*}(E_q)$ we have the subspace of harmonic forms
\ben
\sH(E_q) \subset \Omega^{0,*}(E_q) .
\een 
This induces an inclusion of cochain complexes
\be\label{harmonics}
\sH(E_q) \tensor (V \oplus \frac{\d z}{z} V^*) \hookrightarrow \Omega^{0,*}(E_q) \tensor (V \oplus \frac{\d z}{z} V^*),
\ee
where the right-hand side is an expression for the fields $\sE_V(E_q)$ of the chiral theory on $E_q$ after using the frame $\frac{\d z}{z}$ for the canonical bundle. 
At the level of functions we then obtain a projection
\ben
\xymatrix{
\sO\left(\sH(E_q) \tensor (V \oplus \frac{\d z}{z} V^*)\right) & \ar[l]_-{\pi}  \sO(\sE_V(E_q)) .
}
\een
On the right-hand side, we mean the continuous functions, so this is inherently built from distributional differential forms on $E_q$. 
Using this projection, we obtain the following relationship to the smoothed global observables.

\begin{lem} Homotopy RG flow induces the quasi-isomorphism $\pi_\infty$ from the smoothed global observables on $E_q$ to the harmonic observables equipped with the scale $\infty$ BV Laplacian $\Delta_\infty$:
\ben
\xymatrix{
\Obs^{\q,sm}_V(E_q) \ar@{.>}_-{\pi_\infty} [dd]^-{\simeq}  \ar[r]^-{W_0^L}_-{\cong} & \Obs^{\q,sm}_V(E_q)[L] \ar[d]^-{i[L]}_-{\simeq} \\
& \Obs^{\q}(E_q)[L] \ar[d]^-{W_L^\infty}_-{\cong} \\
\left(\sO\left(\sH(E_q) \tensor (V \oplus \frac{\d z}{z} V^*)\right)[[\hbar]] , \hbar \Delta_\infty\right) & \ar[l]^-{\pi}  \Obs^{\q}_V [\infty].
}
\een 
\end{lem}
\begin{proof}
Observe that homotopy RG flow from scale zero to an arbitrary scale $L$ is well defined on the smoothed observables, which defines the top horizontal isomorphism $W_0^L$. 
The map $i[L]$ is inclusion of the smoothed observables at scale $L$ inside of all observables at scale $L$. 
The space of all observables at scale $L$ is isomorphic through $W_L^\infty$ to the observables at scale $\infty$. 
The scale $\infty$ observables $\Obs_V^\q[\infty]$ is equipped with the differential $\Delta_\infty$, which we observe restricts to functions on the harmonic fields. 
This is because the kernel $K_\infty$ defining the scale $\infty$ BV Laplacian actually lies in the subspace $\sH(E_q) \tensor \sH(E_q)$. 
\end{proof}

We observe that $\sH(E_q) = \CC[\delta]$, where $\delta$ is a formal parameter of degree $+1$ representing the $(0,1)$ form $\frac{\d \zbar}{\zbar}$ generating $H^1(E_q , \sO) \cong \CC$. 
We will use the natural $(-1)$-shifted symplectic structure on this polynomial algebra given by 
\ben
(f(\delta), g(\delta)) \mapsto \frac{\partial}{\partial \delta} \left(f(\delta)g(\delta)\right) .
\een 
Using the usual symplectic pairing between $V$ and $V^*$ we see that $\sH(E_q) \tensor (V \oplus \frac{\d z}{z} V^*)$ is a finite dimensional $(-1)$-shifted symplectic vector space. 
This is clearly compatible with the map (\ref{harmonics}) using the $(-1)$-shifted symplectic structure defining the chiral theory. 

Choosing the odd super-submanifold $\delta (V \oplus \frac{\d z}{z} V^*) \subset (V \oplus \frac{\d z}{z} V^*)[\delta] \cong \sH(E_q) \tensor (V \oplus \frac{\d z}{z} V^*)$ we obtain the BV integration map 
\ben
\int_{BV} : \left(\sO\left((V \oplus \frac{\d z}{z} V^*) [\delta] \right)[[\hbar]] , \hbar \Delta_\infty\right) \to \CC[[\hbar]] .
\een
For any vector space $W$ we can make the identification $\sO(W[\delta]) = \Omega^{-*}(W)$, the negatively graded de Rham forms on $W$. 
For instance, the constant coefficient one-forms are identified with $\delta^* W^*$, which sits in degree $-1$. 
It is convenient to identify $V \oplus \frac{\d z}{z} V^* \cong T^*V$ so that
\ben
\sO\left((V \oplus \frac{\d z}{z} V^*)[\delta]\right) \cong \Omega^{-*}(T^*V)
\een
 
The BV Laplacian $\Delta_\infty$ is identified with the Lie derivative $L_\pi$ with respect to the natural Poisson bivector on $T^*V$. 
Making these identifications, the BV integration map
\ben
\int_{BV} : \left(\Omega^{-*}(T^*V) [[\hbar]] , \hbar L_\pi\right) \to \CC[[\hbar]]
\een
simply picks out the top de Rham form sitting in degree $-2n$, where $n = \dim_\CC V$. 

\begin{rmk} The $\left(\Omega^{-*}(T^*V) [[\hbar]] , \hbar L_\pi\right)$ ...Poisson homology \brian{finish}.
\end{rmk}

\subsection{The $q$-twisted FSS cocycle}

The last piece of the local chiral algebraic index theorem is to define a $q$-twisted analog of the cocycle of Feigin-Felder-Shoikhet. 
It is a theorem of \cite{FeiginTsyganJacobi} that the Hochschild homology of the Weyl algebra on $n$ generators is one-dimensional.

\begin{thm}[\cite{FeiginTsyganJacobi}]\label{theorem FT}
There is an isomorphism $HH_*(D^\hbar_V, D^\hbar_V) \cong \CC[2n]$. 
Moreover, a cycle representing this homology is given by the ``top fermion"
\ben
c_{V} = \sum_{\sigma \in S_{2n}} 1 \tensor t_{\sigma(1)} \tensor \cdots \tensor t_{\sigma(n)} \tensor \frac{\partial}{\partial t_{\sigma(n+1)}} \tensor \cdots \tensor \frac{\partial}{\partial t_{\sigma(2n)}} \in (D^\hbar_{V})^{\tensor (k+1)} .
\een 
\end{thm}

The Feigin-Felder-Shoikhet cocycle is a representative for the dual of this element in Hochschild {\em cohomology}. 
We summarize their result as follows. 
The part about equivariance for $\Vect$ will be relevant later when we discuss how Gelfand-Kazhdan descent may be applied to formulate the global index theorem.

\begin{thm}[\cite{FFS}] There is a $\Vect$-equivariant cocycle 
\ben
\tau_{V} : \Hoch (D^\hbar_V, D^\hbar_V) \to \CC[2 \dim (V)]
\een
satisfying $\tau_{V} (c_{V}) = 1$. 
\end{thm}

\begin{rmk}
In \cite{FFS} they work in the symplectic case, but we are only interested in the case where the symplectic structure is a cotangent bundle.
\end{rmk}

The complex we are interested in is the $q$-twisted Hochschild homology of differential operators on $LV$. 
We note that there is a map of algebras $\pi^0 : D^\hbar_{LV} \to D^\hbar_{z^0V} = D^\hbar_V$ that simply projects onto the zero mode $z^0V \subset LV$. 
We consider the induced map of Hochschild complexes $\pi^0 : {\rm Hoch}(D^\hbar_{LV}, q) \to {\rm Hoch}(D^\hbar_{z^0 V}, 0)$ where the right-hand side is just the usual Hochschild complex of the Weyl algebra $D^\hbar_V$. 

\begin{prop}
For any $q \in D(0,1)^\times$ the composition
\ben
\tau_{LV}^q : {\rm Hoch}(D^\hbar_{LV}, q) \xto{\pi^0} {\rm Hoch}(D^\hbar_{z^0 V}, 0) \xto{\tau_{V}} \CC[2 \dim V] .
\een
is a quasi-isomorphism.
\end{prop}

This will follow from the following general lemma.
Suppose $A = \oplus_N A^{(N)}$ is any $\ZZ$-graded unital algebra over $\CC$ such that there exists an element $H \in A$ with the property $[H,a] = N a$ if $a \in A^{(N)}$.
For each $q \in D(0,1)^\times$ we define an automorphism that we abusively denote $q^{H} : A \to A$, sending $a \in A^{(N)}$ to $q^N a$. 
As above, we give $A$ an interesting $A$-bimodule structure by the formula
\ben
a \cdot a' \cdot a'' = q(a) a a'',
\een
where we use the ordinary multiplication on the right hand side. 
The resulting twisted Hochschild complex is denoted $\Hoch(A, q)$. 

\begin{lem}
The inclusion map $A^{(0)}\hookrightarrow A$ determines a canonicaly quasi-isomorphism
\ben
\Hoch (A^{(0)}, A^{(0)}) \xto{\simeq} \Hoch_*(A; q) .
\een
\end{lem}

\begin{proof}
The retraction $A^{(0)} \leftrightarrow A$ determines a splitting
\[
\Hoch_*(A,q) \cong \Hoch_*(A^{(0)},0) \oplus C,
\]
where $C$ is the subcomplex whose terms always have at least one entry not from $A^{(0)}$:
\[
C^\sharp = \bigoplus_{n \geq 0} \bigoplus_{i = 0}^n A \otimes \cdots \otimes A^{(\neq 0)} \otimes \cdots \otimes A[n],
\]
with $A^{(\neq 0)}$ in the $i$th slot. 
We need to show $C$ is acyclic.

We will do this by exhibiting an explicit chain homotopy.
The differential is of the form $\d_{\Hoch} : C^i \to C^{i-1}$. 
We define $\eta : C^{i-1} \to C^i$ by the following formula
\ben
\eta (a_1 \tensor \cdots \tensor a_i) = \sum_{j=1}^{i+1} a_1 \tensor \cdots \tensor 1_j \tensor \cdots \tensor a_i
\een
where $1_j$ means we put the unit in the $j$th slot of the tensor product. 

We now show that $\eta$ is a null homotopy.
Suppose that $a_1 \in \ne A^{(0)}$. 
Then, we compute
\ben
\d_{\Hoch} \eta (a_1 \tensor \cdots \tensor a_i) 
\een
\end{proof}


\begin{thm} (Local chiral algebraic index theorem) For any $q \in D(0,1)^\times$ the following diagram commutes
\ben
\xymatrix{
\displaystyle \int_{E_q} \Obs^{\q,sm}_{V} \ar[r]^-{\pi_\infty}_-{\simeq} & \left( \Omega^{-*}(T^*V) [[\hbar]], \hbar L_\pi \right) \ar[d]^-{\int_{BV}}_-{\simeq} \\
\Hoch\left(\Weyl_{LV}^\hbar ; q \right) \ar[u]^-{\Phi}_-{\simeq}  \ar[r]^-{\tau^q}_-{\simeq} & \CC[[\hbar]] [2n] .
}
\een
\end{thm}

\subsection{The global chiral algebraic index theorem}

\end{document}